% !TeX program = XeLaTeX
% !TeX root = gitabook.tex
\sect{तृतीयोऽध्यायः}

\uvacha{अर्जुन उवाच}
\twolineshloka
{ज्यायसी चेत् कर्मणस्ते मता बुद्धिर्जनार्दन}
{तत् किं कर्मणि घोरे मां नियोजयसि केशव}% .. 3-1

\twolineshloka
{व्यामिश्रेणेव वाक्येन बुद्धिं मोहयसीव मे}
{तदेकं वद निश्चित्य येन श्रेयोऽहमाप्नुयाम्}% .. 3-2

\uvacha{श्रीभगवानुवाच}
\twolineshloka
{लोकेऽस्मिन् द्विविधा निष्ठा पुरा प्रोक्ता मयाऽनघ}
{ज्ञानयोगेन साङ्ख्यानां कर्मयोगेन योगिनाम्}% .. 3-3

\twolineshloka
{न कर्मणामनारम्भान्नैष्कर्म्यं पुरुषोऽश्नुते}
{न च सन्न्यसनादेव सिद्धिं समधिगच्छति}% .. 3-4

\twolineshloka
{न हि कश्चित् क्षणमपि जातु तिष्ठत्यकर्मकृत्}
{कार्यते ह्यवशः कर्म सर्वः प्रकृतिजैर्गुणैः}% .. 3-5

\twolineshloka
{कर्मेन्द्रियाणि संयम्य य आस्ते मनसा स्मरन्}
{इन्द्रियार्थान् विमूढात्मा मिथ्याचारः स उच्यते}% .. 3-6

\twolineshloka
{यस्त्विन्द्रियाणि मनसा नियम्यऽऽरभतेऽर्जुन}
{कर्मेन्द्रियैः कर्मयोगमसक्तः स विशिष्यते}% .. 3-7

\twolineshloka
{नियतं कुरु कर्म त्वं कर्म ज्यायो ह्यकर्मणः}
{शरीरयात्राऽपि च ते न प्रसिद्‌ध्येदकर्मणः}% .. 3-8

\twolineshloka
{यज्ञार्थात् कर्मणोऽन्यत्र लोकोऽयं कर्मबन्धनः}
{तदर्थं कर्म कौन्तेय मुक्तसङ्गः समाचर}% .. 3-9

\twolineshloka
{सहयज्ञाः प्रजाः सृष्ट्वा पुरोवाच प्रजापतिः}
{अनेन प्रसविष्यध्वमेष वोऽस्त्विष्टकामधुक्}% .. 3-10

\twolineshloka
{देवान् भावयताऽनेन ते देवा भावयन्तु वः}
{परस्परं भावयन्तः श्रेयः परमवाप्स्यथ}% .. 3-11

\twolineshloka
{इष्टान् भोगान् हि वो देवा दास्यन्ते यज्ञभाविताः}
{तैर्दत्तानप्रदायैभ्यो यो भुङ्क्ते स्तेन एव सः}% .. 3-12

\twolineshloka
{यज्ञशिष्टाशिनः सन्तो मुच्यन्ते सर्वकिल्बिषैः}
{भुञ्जते ते त्वघं पापा ये पचन्त्यात्मकारणात्}% .. 3-13

\twolineshloka
{अन्नाद्भवन्ति भूतानि पर्जन्यादन्नसम्भवः}
{यज्ञाद्भवति पर्जन्यो यज्ञः कर्मसमुद्भवः}% .. 3-14

\twolineshloka
{कर्म ब्रह्मोद्भवं विद्धि ब्रह्माक्षरसमुद्भवम्}
{तस्मात् सर्वगतं ब्रह्म नित्यं यज्ञे प्रतिष्ठितम्}% .. 3-15

\twolineshloka
{एवं प्रवर्तितं चक्रं नानुवर्तयतीह यः}
{अघायुरिन्द्रियारामो मोघं पार्थ स जीवति}% .. 3-16

\twolineshloka
{यस्त्वात्मरतिरेव स्यादात्मतृप्तश्च मानवः}
{आत्मन्येव च सन्तुष्टस्तस्य कार्यं न विद्यते}% .. 3-17

\twolineshloka
{नैव तस्य कृतेनार्थो नाकृतेनेह कश्चन}
{न चास्य सर्वभूतेषु कश्चिदर्थव्यपाश्रयः}% .. 3-18

\twolineshloka
{तस्मादसक्तः सततं कार्यं कर्म समाचर}
{असक्तो ह्याचरन् कर्म परमाप्नोति पूरुषः}% .. 3-19

\twolineshloka
{कर्मणैव हि संसिद्धिम् आस्थिता जनकादयः}
{लोकसङ्ग्रहमेवापि सम्पश्यन् कर्तुमर्हसि}% .. 3-20

\twolineshloka
{यद्यदाचरति श्रेष्ठस्तत्तदेवेतरो जनः}
{स यत् प्रमाणं कुरुते लोकस्तदनुवर्तते}% .. 3-21

\twolineshloka
{न मे पार्थास्ति कर्तव्यं त्रिषु लोकेषु किञ्चन}
{नानवाप्तमवाप्तव्यं वर्त एव च कर्मणि}% .. 3-22

\twolineshloka
{यदि ह्यहं न वर्तेयं जातु कर्मण्यतन्द्रितः}
{मम वर्त्मानुवर्तन्ते मनुष्याः पार्थ सर्वशः}% .. 3-23

\twolineshloka
{उत्सीदेयुरिमे लोका न कुर्यां कर्म चेदहम्}
{सङ्करस्य च कर्ता स्यामुपहन्यामिमाः प्रजाः}% .. 3-24

\twolineshloka
{सक्ताः कर्मण्यविद्वांसो यथा कुर्वन्ति भारत}
{कुर्याद्विद्वांस्तथाऽसक्तश्चिकीर्षुर्लोकसङ्ग्रहम्}% .. 3-25

\twolineshloka
{न बुद्धिभेदं जनयेदज्ञानां कर्मसङ्गिनाम्}
{जोषयेत् सर्वकर्माणि विद्वान् युक्तः समाचरन्}% .. 3-26

\twolineshloka
{प्रकृतेः क्रियमाणानि गुणैः कर्माणि सर्वशः}
{अहङ्कारविमूढात्मा कर्ताऽहमिति मन्यते}% .. 3-27

\twolineshloka
{तत्त्ववित् तु महाबाहो गुणकर्मविभागयोः}
{गुणा गुणेषु वर्तन्त इति मत्वा न सज्जते}% .. 3-28

\twolineshloka
{प्रकृतेर्गुणसम्मूढाः सज्जन्ते गुणकर्मसु}
{तानकृत्स्नविदो मन्दान् कृत्स्नविन्न विचालयेत्}% .. 3-29

\twolineshloka
{मयि सर्वाणि कर्माणि सन्न्यस्याध्यात्मचेतसा}
{निराशीर्निर्ममो भूत्वा युध्यस्व विगतज्वरः}% .. 3-30

\twolineshloka
{ये मे मतमिदं नित्यम् अनुतिष्ठन्ति मानवाः}
{श्रद्धावन्तोऽनसूयन्तो मुच्यन्ते तेऽपि कर्मभिः}% .. 3-31

\twolineshloka
{ये त्वेतदभ्यसूयन्तो नानुतिष्ठन्ति मे मतम्}
{सर्वज्ञानविमूढांस्तान् विद्धि नष्टानचेतसः}% .. 3-32

\twolineshloka
{सदृशं चेष्टते स्वस्याः प्रकृतेर्ज्ञानवानपि}
{प्रकृतिं यान्ति भूतानि निग्रहः किं करिष्यति}% .. 3-33

\twolineshloka
{इन्द्रियस्येन्द्रियस्यार्थे रागद्वेषौ व्यवस्थितौ}
{तयोर्न वशमागच्छेत् तौ ह्यस्य परिपन्थिनौ}% .. 3-34

\twolineshloka
{श्रेयान् स्वधर्मो विगुणः परधर्मात् स्वनुष्ठितात्}
{स्वधर्मे निधनं श्रेयः परधर्मो भयावहः}% .. 3-35

\uvacha{अर्जुन उवाच}
\twolineshloka
{अथ केन प्रयुक्तोऽयं पापं चरति पूरुषः}
{अनिच्छन्नपि वार्ष्णेय बलादिव नियोजितः}% .. 3-36

\uvacha{श्रीभगवानुवाच}
\twolineshloka
{काम एष क्रोध एष रजोगुणसमुद्भवः}
{महाशनो महापाप्मा विद्‌ध्येनमिह वैरिणम्}% .. 3-37

\twolineshloka
{धूमेनऽऽव्रियते वह्निर्यथाऽदर्शो मलेन च}
{यथोल्बेनऽऽवृतो गर्भस्तथा तेनेदमावृतम्}% .. 3-38

\twolineshloka
{आवृतं ज्ञानमेतेन ज्ञानिनो नित्यवैरिणा}
{कामरूपेण कौन्तेय दुष्पूरेणानलेन च}% .. 3-39

\twolineshloka
{इन्द्रियाणि मनो बुद्धिरस्याधिष्ठानमुच्यते}
{एतैर्विमोहयत्येष ज्ञानमावृत्य देहिनम्}% .. 3-40

\twolineshloka
{तस्मात् त्वमिन्द्रियाण्यादौ नियम्य भरतर्षभ}
{पाप्मानं प्रजहि ह्येनं ज्ञानविज्ञाननाशनम्}% .. 3-41

\twolineshloka
{इन्द्रियाणि पराण्याहुरिन्द्रियेभ्यः परं मनः}
{मनसस्तु परा बुद्धिर्यो बुद्धेः परतस्तु सः}% .. 3-42

\twolineshloka
{एवं बुद्धेः परं बुद्‍ध्वा संस्तभ्यऽऽत्मानमात्मना}
{जहि शत्रुं महाबाहो कामरूपं दुरासदम्}% .. 3-43
{॥ॐ तत्सदिति श्रीमद्भगवद्गीतासूपनिषत्सु ब्रह्मविद्यायां योगशास्त्रे श्रीकृष्णार्जुनसंवादे कर्मयोगो नाम तृतीयोऽध्यायः॥}