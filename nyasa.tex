% !TeX program = XeLaTeX
% !TeX root = gitabook.tex
\centerline{\font\x="Siddhanta:script=Devanagari,mapping=tex-text" at 40pt \x ॥श्रीमद्भगवद्गीता॥}
\bigskip
\centerline{\Large ॥न्यासः॥}\addcontentsline{toc}{chapter}{न्यासः}
\medskip
\dnsub{करन्यासः}
ॐ अस्य श्रीमद्‌भगवद्‌गीतामालामन्त्रस्य।\\ 
भगवान्वेदव्यास ऋषिः। अनुष्टुप्‌ छन्दः।\\
श्रीकृष्ण परमात्मा देवता।\\
अशोच्यानन्वशोचस्त्वं प्रज्ञावादांश्च भाषसे इति बीजम्‌।\\
सर्वधर्मान्‌ परित्यज्य मामेकं शरणं व्रज इति शक्तिः।\\
अहं त्वा सर्वपापेभ्यो मोक्षयिष्यामि मा शुच इति कीलकम्‌।\\
नैनं छिन्दन्ति शस्त्राणि नैनं दहति पावक इत्यङ्गुष्ठाभ्यां नमः।\\
न चैनं क्लेदयन्त्यापो न शोषयति मारुत इति तर्जनीभ्यां नमः।\\
अच्छेद्योऽयमदाह्योऽयमक्लेद्योऽशोष्य एव च इति मध्यमाभ्यां नमः। नित्यः सर्वगतः स्थाणुरचलोऽयं सनातन इत्यनामिकाभ्यां नमः। पश्य मे पार्थ रूपाणि शतशोऽथ सहस्रश इति कनिष्ठिकाभ्यां नमः। नानाविधानि~दिव्यानि~नानावर्णाकृतीनि च~इति~करतलकरपृष्ठाभ्यां~नमः।\\
{॥इति करन्यासः॥}
\medskip
\newpage
\dnsub{हृदयादि न्यासः}
नैनं छिन्दन्ति शस्त्राणि नैनं दहति पावक इति हृदयाय नमः।\\
न चैनं क्लेदयन्त्यापो न शोषयति मारुत इति शिरसे स्वाहा।\\
अच्छेद्योऽयमदाह्योऽयमक्लेद्योऽशोष्य एव चेति शिखायै वषट्‌।\\
नित्यः सर्वगतः स्थाणुरचलोऽयं सनातन इति कवचाय हुम्‌।\\
पश्य मे पार्थ् रूपाणि शतशोऽथ सहस्रश इति नेत्रत्रयाय वौषट्‌।\\
नानाविधानि दिव्यानि नानावर्णाकृतीनि चेति अस्त्राय फट्‌।\\
{॥श्रीकृष्णप्रीत्यर्थे पाठे विनियोगः॥}
\vfill
\centerline{\Large ॥ध्यानम्॥}\addcontentsline{toc}{chapter}{ध्यानम्}
\fourlineindentedshloka
{ॐ पार्थाय प्रतिबोधितां भगवता नारायणेन स्वयम्}
{व्यासेन ग्रथितां पुराणमुनिना मध्ये महाभारतम्}
{अद्वैतामृतवर्षिणीं भगवतीमष्टादशाध्यायिनीम्}
{अम्ब त्वामनुसन्दधामि भगवद्गीते भवेद्वेषिणीम्}%.. 1..

\twolineshloka
{नमोऽस्तु ते व्यास विशालबुद्धे फुल्लारविन्दायतपत्रनेत्र}
{येन त्वया भारततैलपूर्णः प्रज्वालितो ज्ञानमयः प्रदीपः}% .. 2..

\twolineshloka
{प्रपन्नपारिजाताय तोत्रवेत्रैकपाणये}
{ज्ञानमुद्राय कृष्णाय गीतामृतदुहे नमः}% .. 3..

\twolineshloka
{सर्वोपनिषदो गावो दोग्धा गोपाल-नन्दनः}
{पार्थो वत्सः सुधीर्भोक्ता दुग्धं गीतामृतं महत्}%.. 4..

\twolineshloka
{वसुदेवसुतं देवं कंसचाणूरमर्दनम्}
{देवकीपरमानन्दं कृष्णं वन्दे जगद्गुरुम्}%.. 5..

\fourlineindentedshloka
{भीष्मद्रोणतटा जयद्रथजला गान्धारनीलोत्पला}
{शल्यग्राहवती कृपेण वहनी कर्णेन वेलाकुला}
{अश्वत्थामविकर्णघोरमकरा दुर्योधनावर्तिनी}
{सोत्तीर्णा खलु पाण्डवै रणनदी कैवर्तकः केशवः}% .. 6..

\fourlineindentedshloka
{पाराशर्यवचः सरोजममलं गीतार्थगन्धोत्कटम्}
{नानाख्यानककेसरं हरिकथासम्बोधनाबोधितम्}
{लोके सज्जनषट्‌पदैरहरहः पेपीयमानं मुदा}
{भूयाद्भारतपङ्कजं कलिमलप्रध्वंसि नः श्रेयसे}% .. 7..

\twolineshloka
{मूकं करोति वाचालं पङ्गुं लङ्घयते गिरिम्}
{यत्कृपा तमहं वन्दे परमानन्दमाधवम्}% .. 8..

\fourlineindentedshloka
{यं ब्रह्मा वरुणेन्द्र-रुद्र-मरुतः स्तुवन्ति दिव्यैः स्तवैः}
{वेदैः साङ्ग-पद-क्रमोपनिषदैर्गायन्ति यं सामगाः}
{ध्यानावस्थित-तद्गतेन मनसा पश्यन्ति यं योगिनः}
{यस्यान्तं न विदुः सुरासुर-गणा देवाय तस्मै नमः}%.. 9..

\twolineshloka
{नारायणं नमस्कृत्य नरं चैव नरोत्तमम्}
{देवीं सरस्वतीं व्यासं ततो जयमुदीरयेत्}%.. 10..

\resetShloka