\begingroup
\fontspec[Script=Devanagari]{Adobe Devanagari}
\fontsize{16pt}{20pt}\selectfont
\centerline{\large{ॐ}}
\centerline{॥श्री-गणेशाय नमः॥}
\centerline{॥श्री-गुरुभ्यो नमः॥}
\centerline{॥कृष्णं वन्दे जगद्गुरुम्॥}

\thispagestyle{empty}

\begin{center}
\chapter*{\LARGE{\bfseries प्रस्तावना}}

\twolineshloka*
{सदाशिवसमारम्भां शङ्कराचार्यमध्यमाम्}
{अस्मदाचार्यपर्यन्तां वन्दे गुरुपरम्पराम्}

\noindent\twolineshloka*
{सर्वोपनिषदो गावः दोग्धा गोपालनन्दनः}
{पार्थो वत्सः सुधीर्भोक्ता दुग्धं गीतामृतं महत्}

\end{center}

% कृष्णावतारः लीलावतराः इति प्रसिद्धः। रामावतारः गुणावतारः। रामात् धर्माचारणं ज्ञायन्ते। परन्तु रामस्य वचनानि रामायणे दुर्लभानि। यद्यपि रामः किमपि वदति, सः आत्मानं मानुषं मत्वा अतिविनयेन ऋषि-महात्मनां वचनानि प्रकटीकरोति। उपदेश-रूपेण यत्किञ्चिदेव लभ्यते। परन्तु कृष्णावतारे जगद्गुरुः कृष्णः पुनः पुनः उपदेशं करोति। भगवान् व्यासरूपेण अस्मभ्यं महाभारतम् अनुग्रहीतावान्। तत्र उपनिषद्-रूपेण भगवद्गीता वर्तते। 

% शङ्कराचार्यः अपि ``मोह-मुद्गरे (भज-गोविनन्दम् इति लोकप्रसिद्धं)'' इत्यस्मिन् श्लोके भगवद्गीतायाः महत्त्वं प्रदिशन्ति---

% \fourlineindentedshloka*
% {भगवद्गीता किञ्चिदधीता}
% {गङ्गाजललव-कणिका पीता}
% {सकृदपि येन मुरारि समर्चा}
% {क्रियते तस्य यमेन न चर्चा}


% महाभारतेऽपि भगवद्गीतायाः महत्त्वं दृश्यते। ``यत्र योगेश्वरः कृष्णः'' गीतायाः प्रसिद्धः अन्तिमः श्लोकः। तत्पश्चाद् अग्रे अध्याये---

% \annotwolineshloka
% {गीता सुगीता कर्तव्या किमन्यैः शास्त्र-विस्तरैः}
% {या स्वयं पद्मनाभस्य मुख-पद्माद्‌ विनिःसृता}{६-४३-१}

% \annotwolineshloka
% {सर्वशास्त्रमयी गीता सर्वदेवमयो हरिः}
% {सर्वतीर्थमयी गङ्गा सर्वदेवमयो मनुः}{६-४३-२}

% \annotwolineshloka
% {गीता गङ्गा च गायत्री गोविन्देति हृदि स्थिते}
% {चतुर्गकारसंयुक्ते पुनर्जन्म न विद्यते}{६-४३-३}

% इति श्लोकत्रयेण सङ्क्षिप्तेन गीतामाहात्म्यं वर्णितम्।

% यथा भगवद्गीतायां अर्जुनः वदति, ``भूयः कथय तृप्तिर्हि शृण्वतो नास्ति मेऽमृतम्॥१०-१८॥''! भगवद्गीतां पुनः पनः पठित्वा आनन्दम् लभामहे। भगवद्गीतायाः माहात्म्यम् अन्यपुराणेषु कथ्यन्ते। वराहपुराणतः भगवद्गीतायाः माहात्म्यम् अत्र प्रस्तुतः। पद्मपुराणेऽपि भगवद्गीतायाः प्रति अध्यायस्य पठनास्य माहात्म्यं वर्णितम्। विशेषतः भगवद्गीतायाः पारायणम् वैशाख/कार्तिक-मासयोः अन्तिम-त्रय-तिथि-पुण्यकालेषु विशेषतः अश्वमेधफलदायकम् इति प्रसिद्धम्---

% \twolineshloka*
% {गीतापाठं तु यः कुर्यादन्तिमे च दिनत्रये}
% {दिनेदिनेऽश्वमेधानां फलमेति न संशयः}
% ---स्कान्द-महापुराणे वैष्णवखण्डे वैशाखमाहात्म्ये पञ्चविंशोऽध्याये श्लोकः २०/कार्तिकमाहात्म्ये षड्त्रिंशोऽध्याये श्लोकः ९

% शिष्टाचारे मार्ग-शुक्ल-एकादशी गीता-जयन्ती इति अनुस्रियते। अस्यां तिथावपि भगवद्गीतायाः पारायणं कुर्मः।

% \annotwolineshloka
% {अध्येष्यते च य इमं धर्म्यं संवादमावयोः}
% {ज्ञानयज्ञेन तेनाहमिष्टः स्यामिति मे मतिः}{१८-७०}


% \centerline{सर्वम् श्री-कृष्णार्पणमस्तु॥}

कृष्णावतारः लीलावतारः इति प्रसिद्धः। रामावतारः मर्यादावतारः (गुणावतारः)। रामावतारात् धर्माचरणं ज्ञायते। किन्तु रामस्य वचनानि रामायणे (उपदेश\-रूपेण) विरलानि। यद्यपि रामः तत्र तत्र वदति, तथाऽपि सः आत्मानं मानुषं मत्वा अतिविनयेन ऋषि-महात्मनां वचनानि एव प्रकटीकरोति। अतः उपदेश-रूपेण यत्किञ्चिदेव लभ्यते। परन्तु कृष्णावतारे जगद्गुरुः कृष्णः स्वयमेव पुनः पुनः उपदेशं करोति। भगवान् व्यासः अस्मभ्यं महाभारतम् अनुगृहीतवान्। तत्र उपनिषद्-रूपेण भगवद्गीता वर्तते। 

श्रीमदाचार्यः शङ्करः अपि ``मोह-मुद्गरे (भज-गोविन्दम्)'' इत्यस्मिन् स्तोत्रे भगवद्गीतायाः महत्त्वं निर्दिशति---

\fourlineindentedshloka*
{भगवद्गीता किञ्चिदधीता}
{गङ्गाजललव-कणिका पीता}
{सकृदपि येन मुरारि समर्चा}
{क्रियते तस्य यमेन न चर्चा}


महाभारतेऽपि भगवद्गीतायाः महत्त्वं दृश्यते। ``यत्र योगेश्वरः कृष्णः'' इति गीतायाः प्रसिद्धः अन्तिमः श्लोकः। तत्पश्चाद् अग्रे (गीता-पर्वणि)---

\annotwolineshloka
{गीता सुगीता कर्तव्या किमन्यैः शास्त्र-विस्तरैः}
{या स्वयं पद्मनाभस्य मुख-पद्माद्‌ विनिःसृता}{६-४३-१}

\annotwolineshloka
{सर्वशास्त्रमयी गीता सर्वदेवमयो हरिः}
{सर्वतीर्थमयी गङ्गा सर्वदेवमयो मनुः}{६-४३-२}

\annotwolineshloka
{गीता गङ्गा च गायत्री गोविन्देति हृदि स्थिते}
{चतुर्गकारसंयुक्ते पुनर्जन्म न विद्यते}{६-४३-३}

इति श्लोकत्रयेण सङ्क्षिप्तेन गीतामाहात्म्यं वर्णितम्।

यथा भगवद्गीतायां अर्जुनः वदति, ``भूयः कथय तृप्तिर्हि शृण्वतो नास्ति मेऽमृतम्॥१०-१८॥''! तथैव वयमपि भगवद्गीतां पुनः पुनः पठित्वा आनन्दम् लभामहे। भगवद्गीतायाः माहात्म्यम् अन्यपुराणेषु अपि वर्ण्यते। वराहपुराणतः भगवद्गीतायाः माहात्म्यम् अत्र प्रस्तुतम्। पद्मपुराणेऽपि भगवद्गीतायाः प्रत्येकस्य अध्यायस्य पठनस्य माहात्म्यं वर्णितम्। विशेषतः भगवद्गीतायाः पारायणं वैशाख-कार्तिक-मासयोः अन्तिम-त्रय-तिथि-पुण्यकालेषु अश्वमेधफलदायकम् इति प्रसिद्धम्---

\twolineshloka*
{गीतापाठं तु यः कुर्यादन्तिमे च दिनत्रये}
{दिनेदिनेऽश्वमेधानां फलमेति न संशयः}
---स्कान्द-महापुराणे वैष्णवखण्डे वैशाखमाहात्म्ये पञ्चविंशोऽध्याये श्लोकः २०/कार्तिकमाहात्म्ये षड्त्रिंशोऽध्याये श्लोकः ९

शिष्टाचारे मार्गशीर्ष-शुक्ल-एकादशी `गीता-जयन्ती' इति अनुस्रियते। अस्यां तिथावपि भगवद्गीतायाः पारायणं कुर्मः।

\annotwolineshloka
{अध्येष्यते च य इमं धर्म्यं संवादमावयोः}
{ज्ञानयज्ञेन तेनाहमिष्टः स्यामिति मे मतिः}{१८-७०}

इति भगवतः योगेश्वरस्य श्रीकृष्णस्य वचनं स्मृत्वा तस्य पादारविन्दयोः प्रणमामः॥


\centerline{सर्वम् श्री-कृष्णार्पणमस्तु॥}

\medskip
\noindent{पौष-शुक्ल-पूर्णिमा} \hfill कार्तिकः रामसूनूः\\
विश्वावसु-संवत्सरः ५१२७ धनुः १८ \hfill सर्वज्ञात्म-प्रतिष्ठानम्\\
January 2, 2026

% \vfill
\begingroup
\centering\markboth{\fontspec{Adobe Devanagari} Colophon}{\thepage}
\chapter*{\fontspec{Adobe Devanagari} Colophon}
\endgroup

This document is organised into four parts: (1)~\textbf{मूलम्} (\textit{Bhagavadgītā Mūlam}), (2)~\textbf{पदच्छेदः} (Word Splits), (3)~\textbf{श्लोकानुक्रमणिका} (Index of Verses), and (4)~\textbf{पदानुक्रमणिका} (Word Index). The indices (3) and (4) were generated using a custom Python script co-developed with Gemini 3.0 Pro. All source materials are available at \url{http://github.com/Stotrasamhita/gita/)} for adaptation and re-use. Key features include:

\begin{enumerate}
\item \textbf{Bidirectional Linking:} The \textbf{मूलम्} and \textbf{पदच्छेदः} are hyperlinked to each other. Clicking on any shloka \textbf{number} in the \textbf{मूलम्} takes you `forward' to the corresponding split in the \textbf{पदच्छेदः}, while clicking the number in the \textbf{पदच्छेदः} brings you back to the \textbf{मूलम्}.
\item \textbf{Visual Continuity:} The two parts are typeset in sync to minimise visual disturbance during this back-and-forth navigation, allowing the splits to \emph{appear} seamlessly!
\item \textbf{Verse Index:} The \textbf{श्लोकानुक्रमणिका} features a lexically ordered list of \textit{shloka-pāda-s} (similar to the format seen in certain \textit{Purāṇa-s} published by Nag Publishers). Chapter and verse numbers are listed adjacently, with direct hyperlinks to the verse in the \textbf{मूलम्}.
\item \textbf{Word Index:} The \textbf{पदानुक्रमणिका} presents a lexically ordered concordance of individual \textit{pada-s} (words) across all eighteen chapters, comprising over 3,800 unique entries. Listed against each \textit{pada} are the corresponding verse numbers, which are also hyperlinked.
\item \textbf{Editorial Choices:} I have worked to ensure consistent hyphenation in the \textbf{मूलम्} and logical word splits to facilitate easier reading. I have also employed \textit{avagraha-s} (ऽ) carefully—following the convention of several older texts—to indicate \textit{akāra-s} in \textit{dīrgha sandhi-s}.
\end{enumerate}

The Bhagavad Gītā is exceptional in many ways, not least for its simple, accessible language. It is a boon for learners too—I hope that this rendition will help students of \textit{Saṃskṛtam} delve deeper, both into the language and into the depths of \textit{Vedānta} in the Bhagavad Gītā!

\vfill

\centerline{\large \textbf{Acknowledgments}}

\small Really grateful to all the selfless volunteers who have proofread various texts of Gita, and some excellent resources on archive.org, including Shankara Bhashyam, Gita Press Padachchheda and so on. Special thanks to \href{https://github.com/arindamsaha1507/Gita/}{Arindam Saha} for a high quality CSV with pada splits, which I have been correcting and improving. Ever grateful to H.~L.~Prasad for the core shloka typesetting macros, a gift that keeps on giving! After all these years, I still cannot fathom the brilliance of Knuth in conceiving \TeX, which seems to have gone way beyond its original mandate and delivering high-quality unicode documents! Krishna Krishna Krishna!

\bigskip

\centerline{\textit{\small Last updated: \textbf{\today}}}
\endgroup