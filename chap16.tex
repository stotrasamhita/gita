% !TeX program = XeLaTeX
% !TeX root = gitabook.tex
\chapt{षोडशोऽध्यायः}
\uvacha{श्रीभगवानुवाच}
\twolineshloka
{अभयं सत्त्वसंशुद्धिर्ज्ञानयोगव्यवस्थितिः}
{दानं दमश्च यज्ञश्च स्वाध्यायस्तप आर्जवम्}% .. 16-1

\twolineshloka
{अहिंसा सत्यमक्रोधस्त्यागः शान्तिरपैशुनम्}
{दया भूतेष्वलोलुप्त्वं मार्दवं ह्रीरचापलम्}% .. 16-2

\twolineshloka
{तेजः क्षमा धृतिः शौचमद्रोहो नातिमानिता}
{भवन्ति सम्पदं दैवीमभिजातस्य भारत}% .. 16-3

\twolineshloka
{दम्भो दर्पोऽभिमानश्च क्रोधः पारुष्यमेव च}
{अज्ञानं चाभिजातस्य पार्थ सम्पदमासुरीम्}% .. 16-4

\twolineshloka
{दैवी सम्पद्विमोक्षाय निबन्धायाऽऽसुरी मता}
{मा शुचः सम्पदं दैवीमभिजातोऽसि पाण्डव}% .. 16-5

\twolineshloka
{द्वौ भूतसर्गौ लोकेऽस्मिन् दैव आसुर एव च}
{दैवो विस्तरशः प्रोक्त आसुरं पार्थ मे शृणु}% .. 16-6

\twolineshloka
{प्रवृत्तिं च निवृत्तिं च जना न विदुरासुराः}
{न शौचं नापि चाऽऽचारो न सत्यं तेषु विद्यते}% .. 16-7

\twolineshloka
{असत्यमप्रतिष्ठं ते जगदाहुरनीश्वरम्}
{अपरस्परसम्भूतं किमन्यत् कामहैतुकम्}% .. 16-8

\twolineshloka
{एतां दृष्टिमवष्टभ्य नष्टात्मानोऽल्पबुद्धयः}
{प्रभवन्त्युग्रकर्माणः क्षयाय जगतोऽहिताः}% .. 16-9

\twolineshloka
{काममाश्रित्य दुष्पूरं दम्भमानमदान्विताः}
{मोहाद्गृहीत्वाऽसद्ग्राहान् प्रवर्तन्तेऽशुचिव्रताः}% .. 16-10

\twolineshloka
{चिन्तामपरिमेयां च प्रलयान्तामुपाश्रिताः}
{कामोपभोगपरमा एतावदिति निश्चिताः}% .. 16-11

\twolineshloka
{आशापाशशतैर्बद्धाः कामक्रोधपरायणाः}
{ईहन्ते कामभोगार्थम् अन्यायेनार्थसञ्चयान्}% .. 16-12

\twolineshloka
{इदमद्य मया लब्धमिमं प्राप्स्ये मनोरथम्}
{इदमस्तीदमपि मे भविष्यति पुनर्धनम्}% .. 16-13

\twolineshloka
{असौ मया हतः शत्रुर्हनिष्ये चापरानपि}
{ईश्वरोऽहमहं भोगी सिद्धोऽहं बलवान् सुखी}% .. 16-14

\twolineshloka
{आढ्योऽभिजनवानस्मि कोऽन्योऽस्ति सदृशो मया}
{यक्ष्ये दास्यामि मोदिष्य इत्यज्ञानविमोहिताः}% .. 16-15

\twolineshloka
{अनेकचित्तविभ्रान्ता मोहजालसमावृताः}
{प्रसक्ताः कामभोगेषु पतन्ति नरकेऽशुचौ}% .. 16-16

\twolineshloka
{आत्मसम्भाविताः स्तब्धा धनमानमदान्विताः}
{यजन्ते नामयज्ञैस्ते दम्भेनाविधिपूर्वकम्}% .. 16-17

\twolineshloka
{अहङ्कारं बलं दर्पं कामं क्रोधं च संश्रिताः}
{मामात्मपरदेहेषु प्रद्विषन्तोऽभ्यसूयकाः}% .. 16-18

\twolineshloka
{तानहं द्विषतः क्रूरान् संसारेषु नराधमान्}
{क्षिपाम्यजस्रमशुभान् आसुरीष्वेव योनिषु}% .. 16-19

\twolineshloka
{आसुरीं योनिमापन्ना मूढा जन्मनि जन्मनि}
{मामप्राप्यैव कौन्तेय ततो यान्त्यधमां गतिम्}% .. 16-20

\twolineshloka
{त्रिविधं नरकस्येदं द्वारं नाशनमात्मनः}
{कामः क्रोधस्तथा लोभस्तस्मादेतत्त्रयं त्यजेत्}% .. 16-21

\twolineshloka
{एतैर्विमुक्तः कौन्तेय तमोद्वारैस्त्रिभिर्नरः}
{आचरत्यात्मनः श्रेयस्ततो याति परां गतिम्}% .. 16-22

\twolineshloka
{यः शास्त्रविधिमुत्सृज्य वर्तते कामकारतः}
{न स सिद्धिमवाप्नोति न सुखं न परां गतिम्}% .. 16-23

\twolineshloka
{तस्माच्छास्त्रं प्रमाणं ते कार्याकार्यव्यवस्थितौ}
{ज्ञात्वा शास्त्रविधानोक्तं कर्म कर्तुमिहार्हसि}% .. 16-24
{॥ॐ तत्सदिति श्रीमद्भगवद्गीतासूपनिषत्सु ब्रह्मविद्यायां योगशास्त्रे श्रीकृष्णार्जुनसंवादे दैवासुरसम्पद्विभागयोगो नाम षोडशोऽध्यायः॥}