% !TeX program = XeLaTeX
% !TeX root = gitabook.tex
\chapt{नवमोऽध्यायः}
\uvacha{श्रीभगवानुवाच}
\twolineshloka
{इदं तु ते गुह्यतमं प्रवक्ष्याम्यनसूयवे}
{ज्ञानं विज्ञानसहितं यज्ज्ञात्वा मोक्ष्यसेऽशुभात्}% .. 9-1

\twolineshloka
{राजविद्या राजगुह्यं पवित्रमिदमुत्तमम्}
{प्रत्यक्षावगमं धर्म्यं सुसुखं कर्तुमव्ययम्}% .. 9-2

\twolineshloka
{अश्रद्दधानाः पुरुषा धर्मस्यास्य परन्तप}
{अप्राप्य मां निवर्तन्ते मृत्युसंसारवर्त्मनि}% .. 9-3

\twolineshloka
{मया ततमिदं सर्वं जगदव्यक्तमूर्तिना}
{मत्स्थानि सर्वभूतानि न चाहं तेष्ववस्थितः}% .. 9-4

\twolineshloka
{न च मत्स्थानि भूतानि पश्य मे योगमैश्वरम्}
{भूतभृन्न च भूतस्थो ममऽऽत्मा भूतभावनः}% .. 9-5

\twolineshloka
{यथाऽऽकाशस्थितो नित्यं वायुः सर्वत्रगो महान्}
{तथा सर्वाणि भूतानि मत्स्थानीत्युपधारय}% .. 9-6

\twolineshloka
{सर्वभूतानि कौन्तेय प्रकृतिं यान्ति मामिकाम्}
{कल्पक्षये पुनस्तानि कल्पादौ विसृजाम्यहम्}% .. 9-7

\twolineshloka
{प्रकृतिं स्वामवष्टभ्य विसृजामि पुनः पुनः}
{भूतग्राममिमं कृत्स्नमवशं प्रकृतेर्वशात्}% .. 9-8

\twolineshloka
{न च मां तानि कर्माणि निबध्नन्ति धनञ्जय}
{उदासीनवदासीनमसक्तं तेषु कर्मसु}% .. 9-9

\twolineshloka
{मयाऽध्यक्षेण प्रकृतिः सूयते सचराचरम्}
{हेतुनाऽनेन कौन्तेय जगद्विपरिवर्तते}% .. 9-10

\twolineshloka
{अवजानन्ति मां मूढा मानुषीं तनुमाश्रितम्}
{परं भावमजानन्तो मम भूतमहेश्वरम्}% .. 9-11

\twolineshloka
{मोघाशा मोघकर्माणो मोघज्ञाना विचेतसः}
{राक्षसीमासुरीं चैव प्रकृतिं मोहिनीं श्रिताः}% .. 9-12

\twolineshloka
{महात्मानस्तु मां पार्थ दैवीं प्रकृतिमाश्रिताः}
{भजन्त्यनन्यमनसो ज्ञात्वा भूतादिमव्ययम्}% .. 9-13

\twolineshloka
{सततं कीर्तयन्तो मां यतन्तश्च दृढव्रताः}
{नमस्यन्तश्च मां भक्त्या नित्ययुक्ता उपासते}% .. 9-14

\twolineshloka
{ज्ञानयज्ञेन चाप्यन्ये यजन्तो मामुपासते}
{एकत्वेन पृथक्त्वेन बहुधा विश्वतोमुखम्}% .. 9-15

\twolineshloka
{अहं क्रतुरहं यज्ञः स्वधाऽहमहमौषधम्}
{मन्त्रोऽहमहमेवऽऽज्यमहमग्निरहं हुतम्}% .. 9-16

\twolineshloka
{पिताऽहमस्य जगतो माता धाता पितामहः}
{वेद्यं पवित्रमोङ्कार ऋक्साम यजुरेव च}% .. 9-17

\twolineshloka
{गतिर्भर्ता प्रभुः साक्षी निवासः शरणं सुहृत्}
{प्रभवः प्रलयः स्थानं निधानं बीजमव्ययम्}% .. 9-18

\twolineshloka
{तपाम्यहमहं वर्षं निगृण्हाम्युत्सृजामि च}
{अमृतं चैव मृत्युश्च सदसच्चाहमर्जुन}% .. 9-19

\fourlineindentedshloka
{त्रैविद्या मां सोमपाः पूतपापाः}
{यज्ञैरिष्ट्वा स्वर्गतिं प्रार्थयन्ते}
{ते पुण्यमासाद्य सुरेन्द्रलोकम्}
{अश्नन्ति दिव्यान् दिवि देवभोगान्}% .. 9-20

\fourlineindentedshloka
{ते तं भुक्त्वा स्वर्गलोकं विशालम्}
{क्षीणे पुण्ये मर्त्यलोकं विशन्ति}
{एवं त्रयीधर्ममनुप्रपन्नाः}
{गतागतं कामकामा लभन्ते}% .. 9-21

\twolineshloka
{अनन्याश्चिन्तयन्तो मां ये जनाः पर्युपासते}
{तेषां नित्याभियुक्तानां योगक्षेमं वहाम्यहम्}% .. 9-22

\twolineshloka
{येऽप्यन्यदेवताभक्ता यजन्ते श्रद्धयान्विताः}
{तेऽपि मामेव कौन्तेय यजन्त्यविधिपूर्वकम्}% .. 9-23

\twolineshloka
{अहं हि सर्वयज्ञानां भोक्ता च प्रभुरेव च}
{न तु मामभिजानन्ति तत्त्वेनातश्च्यवन्ति ते}% .. 9-24

\twolineshloka
{यान्ति देवव्रता देवान् पितॄन् यान्ति पितृव्रताः}
{भूतानि यान्ति भूतेज्या यान्ति मद्याजिनोऽपि माम्}% .. 9-25

\twolineshloka
{पत्रं पुष्पं फलं तोयं यो मे भक्त्या प्रयच्छति}
{तदहं भक्त्युपहृतमश्नामि प्रयतात्मनः}% .. 9-26

\twolineshloka
{यत् करोषि यदश्नासि यज्जुहोषि ददासि यत्}
{यत् तपस्यसि कौन्तेय तत् कुरुष्व मदर्पणम्}% .. 9-27

\twolineshloka
{शुभाशुभफलैरेवं मोक्ष्यसे कर्मबन्धनैः}
{सन्न्यासयोगयुक्तात्मा विमुक्तो मामुपैष्यसि}% .. 9-28

\twolineshloka
{समोऽहं सर्वभूतेषु न मे द्वेष्योऽस्ति न प्रियः}
{ये भजन्ति तु मां भक्त्या मयि ते तेषु चाप्यहम्}% .. 9-29

\twolineshloka
{अपि चेत् सुदुराचारो भजते मामनन्यभाक्}
{साधुरेव स मन्तव्यः सम्यग्व्यवसितो हि सः}% .. 9-30

\twolineshloka
{क्षिप्रं भवति धर्मात्मा शश्वच्छान्तिं निगच्छति}
{कौन्तेय प्रतिजानीहि न मे भक्तः प्रणश्यति}% .. 9-31

\twolineshloka
{मां हि पार्थ व्यपाश्रित्य येऽपि स्युः पापयोनयः}
{स्त्रियो वैश्यास्तथा शूद्रास्तेऽपि यान्ति परां गतिम्}% .. 9-32

\twolineshloka
{किं पुनर्ब्राह्मणाः पुण्या भक्ता राजर्षयस्तथा}
{अनित्यमसुखं लोकमिमं प्राप्य भजस्व माम्}% .. 9-33

\twolineshloka
{मन्मना भव मद्भक्तो मद्याजी मां नमस्कुरु}
{मामेवैष्यसि युक्त्वैवमात्मानं मत्परायणः}% .. 9-34
{॥ॐ तत्सदिति श्रीमद्भगवद्गीतासूपनिषत्सु ब्रह्मविद्यायां योगशास्त्रे श्रीकृष्णार्जुनसंवादे राजविद्याराजगुह्ययोगो नाम नवमोऽध्यायः॥}