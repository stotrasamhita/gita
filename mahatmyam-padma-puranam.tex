\chapt{गीतामाहात्म्यम्—पद्मपुराणम्}

\sect{{[१]} पञ्चसप्तत्यधिकशततमोऽध्यायः} %6.175

\uvacha{पार्वत्युवाच}

\twolineshloka
{भगवन्सर्वतत्वज्ञ श्रीविष्णोस्त्वत्प्रसादतः}
{श्रुता नानाविधा धर्मा लोकनिस्तारहेतवः} %॥१॥

\threelineshloka
{अधुना श्रोतुमिच्छमि गीतामाहात्म्यमप्यहम्}
{श्रुतेन येन देवेश हरौ भक्तिर्विवर्द्धते}
{तद्वदस्वाधुना देव यद्यहं तव वल्लभा} %॥२॥

\uvacha{ईश्वर उवाच}

\twolineshloka
{अतसीपुष्पसङ्काशं खगेन्द्रासनमच्युतम्}
{शयानं शेषशय्यायां महाविष्णुमुपास्महे} %॥३॥

\twolineshloka
{कदाचिदासने रम्ये सुखासीनं मुरद्विषम्}
{आनन्दयित्री लोकानां लक्ष्मीः पप्रच्छ सादरात्} %॥४॥

\uvacha{श्रीरुवाच}

\twolineshloka
{शयालुरसि दुग्धाब्धौ भगवन्केन हेतुना}
{उदासीन इवैश्वर्यं जगन्ति स्थापयन्निव} %॥५॥

\uvacha{ईश्वर उवाच}

\twolineshloka
{इति देव्या वचः श्रुत्वा मुरभिज्ज्ञानगर्वितम्}
{उवाच श्लक्ष्णया वाचा विस्मयस्मेरलोचनः} %॥६॥

\uvacha{श्रीभगवानुवाच}

\twolineshloka
{नाहं सुमुखि निद्रालुर्निजं माहेश्वरं वपुः}
{दृशा तत्वानुवर्त्तिन्या पश्याम्यन्तर्निमग्नया} %॥७॥

\twolineshloka
{कुशाग्रया धिया देवि यदन्तर्योगिनो हृदि}
{पश्यन्ति यच्च वेदानां सारं मीमांसते भृशम्} %॥८॥

\twolineshloka
{तदेवमक्षरं ज्योतिरात्मरूपमनामयम्}
{अखण्डानन्द सन्दोह निष्पादि द्वैतवर्जितम्} %॥९॥

\twolineshloka
{यदाश्रया जगद्वृत्तिर्यन्मया चानुभूयते}
{न येन रहितं किञ्चिज्जगत्तत्वं चराचरम्} %॥१०॥

\twolineshloka
{निर्मथ्य बहुधालोक्य वेदशास्त्राम्बुधिं सुधीः}
{द्वैपायनो यदासाद्य गीताशास्त्रं निसृष्टवान्} %॥११॥

\twolineshloka
{यदास्थाय महानन्दमानन्दीकृतमानसः}
{निद्रालुरिव देवेशि दुग्धाब्धौ प्रतिभामि वै} %॥१२॥

\twolineshloka
{इति तस्य मुरारातेर्मितमानन्दवद्वचः}
{सा हर्षोत्फुल्ललोलाक्षी लक्ष्मी श्रुत्वा विसिस्मिरे} %॥१३॥

\uvacha{श्रीरुवाच}

\twolineshloka
{भवानेव हृषीकेश ध्येयोऽसि यमिनां सदा}
{तस्मात्त्वत्तः परं यत्तच्छ्रोतुं कौतूहलं हि मे} %॥१४॥

\twolineshloka
{चराचराणां लोकानां कर्त्ता हर्त्ता स्वयं प्रभुः}
{यथास्थितस्ततोऽन्यत्वं यदि मां बोधयाच्युत} %॥१५॥

\uvacha{श्रीभगवानुवाच}

\twolineshloka
{मायामयमिदं देवि वपुर्मे न तु तात्विकम्}
{सृष्टिस्थित्योपसंहारक्रियाजालोपबृंहितम्} %॥१६॥

\twolineshloka
{अतोऽन्यदात्मनोरूपं द्वैताद्वैतविवर्जितम्}
{भावाभावविनिर्मुक्तमाद्यन्तरहितं प्रिये} %॥१७॥

\twolineshloka
{शुद्धसंवित्प्रभालाभं परानन्दैकसुन्दरम्}
{रूपमैश्वरमात्मैक्यगम्यं गीतासु कीर्तितम्} %॥१८॥

\twolineshloka
{इत्याकर्ण्य वचो देवि देवस्यामिततेजसः}
{शङ्कमाना ह वाक्येषु परस्परविरोधिषु} %॥१९॥

\twolineshloka
{स्वयं चेत्परमानन्दमवाङ्मनसगोचरम्}
{कथं गीता बोधयति इति मे च्छिन्धि संशयम्} %॥२०॥

\uvacha{ईश्वर उवाच}

\twolineshloka
{श्रियः श्रुत्वा वचोयुक्तमितिहासपुरःसरम्}
{आत्मानुगामिनीं दृष्टिं गीतां बोधितवान्प्रभुः} %॥२०॥

\twolineshloka
{अहमात्मा परेशानि परापरविभेदतः}
{द्विधा ततः परः साक्षी निर्गुणो निष्कलः शिवः} %॥२१॥

\twolineshloka
{अपरः पञ्चवक्त्रोऽहं द्विधा तस्यापि संस्थितिः}
{शब्दार्थभेदतो वाच्यो यथात्माहं महेश्वरः} %॥२२॥

\twolineshloka
{गीतानां वाक्यरूपेण यन्निरुच्छिद्यते दृढः}
{मदीयपाशबन्धोऽयं संसारविषयात्मकः} %॥२३॥

\twolineshloka
{यदाभ्यासपराधीनौ पञ्चवक्त्रमहेश्वरौ}
{इति तस्य वचः श्रुत्वा गीतासारमहोदधेः} %॥२४॥

\twolineshloka
{इदं परविभेदेन बुध्यते भवभीरुभिः}
{तमपृच्छदिदं लक्ष्मीरङ्गप्रत्यङ्गसंस्थितम्} %॥२५॥

\onelineshloka*
{माहात्म्यं सेतिहासं च सर्वं तस्यै न्यवेदयत्}

\uvacha{श्रीभगवानुवाच}

\onelineshloka
{शृणु सुश्रोणि वक्ष्यामि गीतासु स्थितिमात्मनः} %॥२६॥

\twolineshloka
{वक्त्राणि पञ्च जानीहि पञ्चाध्यायाननुक्रमात्}
{दशाध्याया भुजाश्चैक उदरं द्वौ पदाम्बुजे} %॥२७॥

\twolineshloka
{एवमष्टादशाध्याया वाङ्मयी मूर्तिरैश्वरी}
{विज्ञेया ज्ञानमात्रेण महापातकनाशिनी} %॥२८॥

\twolineshloka
{अतोध्यायं तदर्धं वा श्लोकमर्द्धं तदर्धकम्}
{अभ्यस्यति सुमेधा यः सुशर्मेव समुच्यते} %॥२९॥

\uvacha{श्रीरुवाच}

\twolineshloka
{सुशर्मा नाम को देव किं जातीयः किमात्मकः}
{कुतस्तस्य च वै मुक्तिः केनाजायत हेतुना} %॥३०॥

\uvacha{श्रीभगवानुवाच}

\twolineshloka
{सुशर्मा नाम दुर्मेधाः सीमा पापात्मनामभूत्}
{जातो नात्मविदां वंशे विप्राणां क्रूरकर्मणाम्} %॥३१॥

\twolineshloka
{न ध्यानं न जपो होमो न चैवातिथिपूजनम्}
{केवलं विषयेष्वेव बलाढ्येनाभिवर्त्तते} %॥३२॥

\twolineshloka
{कृषिकर्मरतो नित्यं पर्णजीवी सुराप्रियः}
{मांसोपहारी सुचिरं कालमेवं निनाय सः} %॥३३॥

\twolineshloka
{आनेतुकामः पर्णानि पर्यटनृषिवाटिकाम्}
{ततः स तत्र दष्टोऽभूत्कालसर्पेण मूढधीः} %॥३४॥

\twolineshloka
{कालधर्मं समासाद्य गत्वा च निरयान्बहून्}
{पुनरागत्य मर्त्येषु बलीवर्दत्वमीयिवान्} %॥३५॥

\twolineshloka
{पङ्गुना केन विक्रीतः स स्वजीवनहेतवे}
{नयन्पृष्ठेन शरदः सप्ताष्टौ कष्टतोनयत्} %॥३६॥

\twolineshloka
{कदाचित्पङ्गुनासोऽपि चिरमावर्तितो जवात्}
{पपात तरसा भूमौ मूर्च्छां च प्रतिपेदिवान्} %॥३७॥

\twolineshloka
{विकलाङ्गो विवृत्ताक्षः फेनसन्ततिमुद्गिरन्}
{न जीवति न मृत्युं वा प्रतिपेदे स्वकर्मणा} %॥३८॥

\twolineshloka
{कौतुकाकृष्टलोकेऽस्मिंस्तस्मिन्जनसमागमे}
{श्रेयसे तस्य सुकृती कश्चित्पुण्यं वितीर्णवान्} %॥३९॥

\twolineshloka
{कर्माणि स्वान्यनुस्मृत्य ददुरन्ये च केचन}
{गणिका कापि तत्रस्था लोकयात्रानुवर्तिनी} %॥४०॥

\twolineshloka
{अज्ञात निजपुण्या सा किञ्चिदुत्सृष्टवत्यभूत्}
{परेतनगरीमादौ स नीतः कालकिङ्करैः} %॥४१॥

\twolineshloka
{गणिकादत्तपुण्येन पुण्यवानिति मोचितः}
{पुनरागत्य भूर्लोकं कुलशीलवतां गृहे} %॥४२॥

\twolineshloka
{द्विजन्मनामसौ जज्ञे जातिं स्वामनुसंस्मरन्}
{काले महति जिज्ञासुः श्रेयः स्वाज्ञाननोदनम्} %॥४३॥

\twolineshloka
{उपेत्य गणिकां दत्तं ख्यापयित्वा स पृष्टवान्}
{आचष्ट मां शुको नित्यं पञ्जरस्थः पठत्यसौ} %॥४४॥

\twolineshloka
{तेन पूतान्तरात्माहं तत्पुण्यं पर्यकल्पयम्}
{ताभ्यां शुकस्तु पृष्टोऽसौ व्याख्यातुमुपचक्रमे} %॥४५॥

\onelineshloka*
{आख्यायिकां पुरावृत्तां स्मृत्वा जातिं निजामपि}

\uvacha{शुक उवाच}

\onelineshloka
{पुरा विद्वानहं भूत्वा वैदुष्य स्मयमोहितः} %॥४६॥

\twolineshloka
{रागद्वेषेण विद्वत्सु गुणवत्स्वपि मत्सरी}
{कालेनाहं ततः प्रेत्य प्राप्य लोकाञ्जुगुप्सितान्} %॥४७॥

\twolineshloka
{सोऽहं कीरकुलेऽभूवं सद्गुरावतिनिन्दकः}
{काले धर्मणि दुष्कर्मा पितृभ्यां च वियोजितः} %॥४८॥

\twolineshloka
{निदाघेऽध्वनिसन्तप्ते आनीतो ऋषिपुङ्गवैः}
{पातितः पञ्जरस्थोऽहं माश्रमे महदाश्रये} %॥४९॥

\twolineshloka
{आवर्तयद्भ्यो गीतानामाद्यमध्यायमादरात्}
{श्रुत्वा ऋषिकुमारेभ्यः पाठं चाकरवं मुहुः} %॥५०॥

\twolineshloka
{एतस्मिन्नन्तरे कश्चिद्वागुरिश्चौरकर्मकृत्}
{मामाहृत्य तदाक्रीणादिति वृत्तमुदाहृतम्} %॥५१॥

\uvacha{श्रीभगवानुवाच}

\twolineshloka
{अध्यायोऽयं पुराम्नातो येन पापमनोदयम्}
{पूतान्तरात्मा येनासौ मोचितश्च द्विजोत्तमः} %॥५२॥

\twolineshloka
{एवमन्योन्यमाभाष्य तन्माहात्म्यं प्रशस्य च}
{ते जपन्तोनिशं धीरा मुक्तिं गेहे प्रपेदिरे} %॥५३॥

\twolineshloka
{तस्मादध्यायमाद्यं यः पठते शृणुते स्मरेत्}
{अभ्यसेत्तस्य न भवेद्भवाम्भोधिर्दुरुत्तरः} %॥५४॥

॥इति श्रीपाद्मे महापुराणे पञ्चपञ्चाशत्सहस्रसंहितायामुत्तरखण्डे सतीश्वरसंवादे गीतामाहात्म्ये पञ्चसप्तत्यधिकशततमोऽध्यायः॥१७५॥

\sect{{[२]} षट्सप्तत्यधिकशततमोऽध्यायः} %6.176

\uvacha{श्रीभगवानुवाच}

\twolineshloka
{आदिमस्यैवमाख्यानमुदीरितमनुत्तमम्}
{शृणु माहात्म्यमन्येषामध्यायानामपीन्दिरे} %॥१॥

\twolineshloka
{दक्षिणस्यां दिशि श्रीमानासीदाम्नायवादिनाम्}
{पुरे पुरन्दराह्वाने देवशर्मेति विश्रुतः} %॥२॥

\twolineshloka
{अर्चितातिथिराम्नातो वेदशास्त्रविशारदः}
{आहर्ता क्रतुसङ्घानां तापसानां प्रियः सदा} %॥३॥

\twolineshloka
{देवान्सन्तर्पयामास हव्यैर्हुतवहं चिरम्}
{नचोपलेभे धर्मात्मा शान्तिमेकान्तिकीं ततः} %॥४॥

\twolineshloka
{निःश्रेयसं स जिज्ञासुस्तापसाननुवासरम्}
{सिषेवे सत्यसङ्कल्पाननल्पैरेव कल्पकैः} %॥५॥

\twolineshloka
{एवमाचरतस्तस्य काले महति गच्छति}
{मुक्तकर्मा ततः कश्चित्प्रादुरासीत्पुरा भुवि} %॥६॥

\twolineshloka
{अनुभूतनिराकाङ्क्षी नासाग्रन्यस्तलोचनः}
{शान्तचेताः परं ब्रह्म ध्यायन्नानन्दनिर्भरः} %॥७॥

\twolineshloka
{पादौ तस्योपसङ्गृह्य प्रणतेनान्तरात्मना}
{चकार विधिवत्तस्मै विद्वानतिथिसत्क्रियाम्} %॥८॥

\twolineshloka
{तं च शुद्धेन भावेन परितुष्टं तपस्विनम्}
{प्रणतः परिपप्रच्छ निर्वाणस्थितिमात्मनः} %॥९॥

\twolineshloka
{स तस्मै कथयामास पुरेऽसौ पुरनामनि}
{मित्रवन्तमजापालमुपदेष्टारमात्मवित्} %॥१०॥

\twolineshloka
{स चाभिवन्द्य तत्पादावेत्यसौ पुरमूर्जितम्}
{तस्योत्तरदिशोभागे ददर्श विपुलं वनम्} %॥११॥

\twolineshloka
{मारुतान्दोलितानेक कुसुमामोदसुन्दरम्}
{उन्मत्तभ्रमरोद्गीत नादापूरितदिङ्मुखम्} %॥१२॥

\twolineshloka
{तस्मिन्वने सरित्तीरे निषीदन्तं शिलातले}
{मित्रवन्तं ददर्शाथ सानन्दस्तिमितेक्षणम्} %॥१३॥

\twolineshloka
{अपि स्वाभाविकं वैरं हित्वान्योन्यं विरोधिभिः}
{सत्वैरावृतमुद्याने मन्दस्यन्दनभास्वति} %॥१४॥

\twolineshloka
{शान्तेषु मृगयूथेषु दशानन्दमनोज्ञया}
{कृपानुविद्धया भूमिं निषिञ्चन्तमिवामृतम्} %॥१५॥

\twolineshloka
{उपेत्य विनयेनामुमुन्मनाः प्रीतमानसः}
{किञ्चिदानम्रशिरसा तेनापि स तु सत्कृतः} %॥१६॥

\twolineshloka
{उपतस्थे ततो विद्वान्मित्रवन्तमनन्यधीः}
{समाप्तध्यानकालं स पर्यपृच्छत्समाहितः} %॥१७॥

\uvacha{देवशर्मोवाच}

\twolineshloka
{आत्मानं वेत्तुमिच्छामि तदमुष्मिन्मनोरथे}
{लब्धसिद्धिमुपायं मामुपदेष्टुं त्वमर्हसि} %॥१८॥

\uvacha{श्रीभगवानुवाच}

\onelineshloka*
{परामृश्य क्षणं सोऽपि मित्रवानिदमब्रवीत्}

\uvacha{मित्रवानुवाच}

\onelineshloka
{विद्वन्विद्धि पुरावृत्तमुच्यमानमिदं मया} %॥१९॥

\twolineshloka
{अस्ति गोदावरीतीरे प्रतिष्ठानाभिधं पुरम्}
{तत्र दुर्दमनामासीदन्वये च मनीषिणाम्} %॥२०॥

\twolineshloka
{तत्रास्ति विक्रमो नाम सेव्यमानो महीपतिः}
{दानानि प्रत्यहं गृह्णन्वर्त्तते उदरम्भरः} %॥२१॥

\twolineshloka
{कालेन कालपाशेन बद्धानीतो यमालयम्}
{निरयेषु समग्रेषु यातना अनुभूय च} %॥२२॥

\twolineshloka
{कस्मिंश्चित्स कुले जातो दुर्वृत्तानां द्विजन्मनाम्}
{भवान्तरानुवर्तिन्या विद्यया स पुरस्कृतः} %॥२३॥

\twolineshloka
{उपयेमे दुराधर्षां कन्यकामधमे कुले}
{कालेन सा वयो हित्वा शैशवं यौवनं ययौ} %॥२४॥

\twolineshloka
{पीनस्तनी च सुश्रोणी मदविह्वललोचना}
{पतिं न सेहे दुर्वृत्तं चकमे स्वपतीन्परान्} %॥२५॥

\twolineshloka
{वृत्तिमाहर्तुकामस्मिन्निर्गता सा पुराद्बहिः}
{सङ्गता कामुकेनासौ चिरं चाण्डालजन्मना} %॥२६॥

\twolineshloka
{दधे गर्भमसौ तस्मात्सा च कन्योपपद्यते}
{सैव भार्यापि तस्यासीत्पूर्वपापप्रसङ्गतः} %॥२७॥

\twolineshloka
{सैव वृद्धा ततः काले डाकिनी समजायत}
{कुसङ्गात्कुमतिर्जाता दुष्टनारीप्रसङ्गतः} %॥२८॥

\twolineshloka
{चखाद व्याधितं व्याधमसृगास्वादलालसा}
{भ्रमन्ती विपिने घोरे जनैर्दृष्ट्वा बहिष्कृता} %॥२९॥

\twolineshloka
{परेतलोकमासाद्य व्याधौ व्याघ्रोऽभ्यवर्त्तत}
{नरकान्दारुणान्भुक्त्वा जीवहिंसा प्रभावतः} %॥३०॥

\twolineshloka
{सापि कालेन दुष्टात्मा मृत्युवेगमुपागता}
{निरयानेत्य दुर्द्धर्षानजाजायत मद्गृहे} %॥३१॥

\twolineshloka
{तामन्या अप्यहं विद्वन्पालयन्काननान्तरे}
{अपश्यन्द्वीपिनं घोरं जिघांसन्तमिवाखिलम्} %॥३२॥

\twolineshloka
{समालोक्य तमायान्तं भयेन प्रपलायितम्}
{अजायूथं परित्यज्य मया मरणभीरुणा} %॥३३॥

\twolineshloka
{उपदुद्राव स द्वीपी पूर्ववैरमनुस्मरन्}
{अजा तु तत्समीपेऽगात्सत्वरं सरिदन्तिके} %॥३४॥

\twolineshloka
{तत्र सा भयमुत्सृज्य हित्वा वैरमनर्गला}
{अवतस्थे स च द्वीपी तूष्णीमासीदमत्सरः} %॥३५॥

\twolineshloka
{तं तथाविधमालोक्य सा वक्तुमुपचक्रमे}
{द्वीपिन्नभीप्सितं भुङ्क्ष्व मांसमुद्धृत्य सादरः} %॥३६॥

\twolineshloka
{नेयं भवति ते बुद्धिः कथं वैरमतिं त्यजः}
{इत्याकर्ण्य तदा वाक्यं प्राह द्वीपी विमत्सरः} %॥३७॥

\twolineshloka
{स्थानेऽस्मिन्मे गतो द्वेषः क्षुत्पिपासा च निर्ययौ}
{न प्रार्थयामि तेन त्वां समीपे समुपस्थिताम्} %॥३८॥

\twolineshloka
{सैवमुक्ता पुनः प्राह जाताहं निर्भया कथम्}
{किमत्र कारणं वेत्सि यदि मे वक्तुमर्हसि} %॥३९॥

\twolineshloka
{एवमुक्तः पुनर्द्वीपी तामाहाजां न वेद्म्यहम्}
{पुरोगतमिमं प्रष्टुं महान्तमिति निर्गतौ} %॥४०॥

\twolineshloka
{ताभ्यामुभाभ्यामागत्य पृष्टोऽहं बहुविस्मयः}
{अहं च सहितस्ताभ्यामपृच्छं वानरेश्वरम्} %॥४१॥

\twolineshloka
{मया पृष्टः स विप्रेदमब्रवीत्सादरं कपिः}
{शृणु वक्ष्याम्यजापाल वृत्तमत्र पुरातनम्} %॥४२॥

\twolineshloka
{इदमायतनं पश्य पुरोवनगतं महत्}
{अत्र त्र्यम्बकलिङ्गं हि द्रुहिणेन प्रतिष्ठितम्} %॥४३॥

\twolineshloka
{सुकर्मानाम मेधावी पर्युपास्ते तपश्चरन्}
{वनपुष्पाण्यपाहृत्य सुरपूज्यं पुरोभवम्} %॥४४॥

\twolineshloka
{संस्नाप्य सरिदम्भोभिः केवलं कर्मणा वसन्}
{काले महति तस्यागादतिथिः कश्चिदन्तिकम्} %॥४५॥

\twolineshloka
{उपाहृत्य फलाहारं स तस्मै पर्यकल्पयत्}
{तेनातिथ्येन सम्प्रीतः सुकर्माणमभाषत} %॥४६॥

\twolineshloka
{किमिदं कर्मणो मूलं फलं भुक्त्वा तु तिष्ठसि}
{गतानुगतया वृत्त्या किं वा केवलमीहसे} %॥४७॥

\twolineshloka
{स एवमुक्तः प्रायेण प्रीतेनात्मविदा तदा}
{प्रत्युवाच वचः स्पष्टमात्मनो हितमुत्तमम्} %॥४८॥

\twolineshloka
{विद्वन्न वेद्मि तत्वेन फलमेतस्य कर्मणः}
{बुभुत्सया परः शम्भुः सेव्यते केवलं यया} %॥४९॥

\twolineshloka
{फलमेतस्य सेवायाः परिपाकं कपर्दिनः}
{यन्मांसमनुगृह्णासि संस्पृश्यात्ममनोरथम्} %॥५०॥

\twolineshloka
{तस्यैवं सूनृतं वाक्यं श्रुत्वा प्रीतस्तपोधनः}
{द्वितीयमालिलेखाऽसौ गीताध्यायं शिलातले} %॥५१॥

\twolineshloka
{आदिदेशऽऽशु तं विप्रं पठनाभ्यसनाय च}
{फलिष्यत्यात्मनः स्वैरं परितस्ते मनोरथः} %॥५२॥

\twolineshloka
{इत्युक्त्वान्तर्दधे धीमान्पुरतस्तस्य पश्यतः}
{विस्मितस्तस्य चादेशात्सोऽन्वतिष्ठदनारतम्} %॥५३॥

\twolineshloka
{ततः कालेन महता भावितात्मा प्रसन्नधीः}
{यत्रयत्र चचारासौ शान्तं तत्तत्तपोवनम्} %॥५४॥

\twolineshloka
{न द्वन्द्वबाधा नैव क्षुत्पिपासा न च वा भयम्}
{तपसा तस्य जानीहि द्वितीयाध्यायजापिनः} %॥५५॥

\uvacha{मित्रवानुवाच}

\twolineshloka
{एवमुक्तश्च तेनाऽहं ख्यापयित्वा परां कथाम्}
{अनुज्ञातप्रसन्नेन च्छागीव्याघ्रयुतोगमम्} %॥५६॥

\twolineshloka
{गत्वा शिलातले पश्यमध्यायं लिखितं पठेत्}
{तस्यैवावर्त्तनादाप्तं तपसः पारमुत्तमम्} %॥५७॥

\twolineshloka
{तेन त्वमपि कल्याण नित्यमाहर्तुमर्हसि}
{अध्यायं तेन ते मुक्तिरदूरस्था भविष्यति} %॥५८॥

\twolineshloka
{देवशर्मा समादिष्टस्तेन मित्रवता स्वयम्}
{अभ्यर्च्य प्रणतो भूत्वा पुरन्दरपुरं ययौ} %॥५९॥

\twolineshloka
{तत्रात्मविदमासाद्य देवतायतने क्वचित्}
{वृत्तमेतन्निवेद्यादावध्यायमपठत्ततः} %॥६०॥

\twolineshloka
{शिक्षितस्तेन पूतात्मा पठन्नध्यायमादरात्}
{द्वितीयमाससादोच्चैर्निरवद्यं परं पदम्} %॥६१॥

\twolineshloka
{द्वितीयस्येदमाख्यानं कथितं शृणु साम्प्रतम्}
{तृतीयस्याथ वक्ष्यामि माहात्म्यमपि चेन्दिरे} %॥६२॥

॥इति श्रीपाद्मे महापुराणे पञ्चपञ्चाशत्सहस्रसंहितायामुत्तरखण्डे सतीश्वरसंवादे गीतामाहात्म्ये षट्सप्तत्यधिकशततमोऽध्यायः॥१७६॥

\sect{{[३]} सप्तसप्तत्यधिकशततमोऽध्यायः} %6.177

\uvacha{श्रीभगवानुवाच}

\twolineshloka
{जनस्थाने जडो नाम द्विजन्मा कौशिकान्वयी}
{हित्वा जात्युचितं धर्मं वणिग्वृत्त्यां मनो दधे} %॥१॥

\twolineshloka
{व्यसनी परदारेषु दीव्यन्नक्षैः पिबन्मधु}
{मृगया निरतो नित्यं कालमेवं निनाय सः} %॥२॥

\twolineshloka
{क्षीणे वित्ते ततो रात्रौ चौर्यामारब्धवांस्ततः}
{प्रतिपेदे धनं तेन यज्वनां यष्टुमर्थिनाम्} %॥३॥

\twolineshloka
{स दूरमगमत्तेन वाणिज्यायोत्तरां दिशम्}
{कस्तूरिमगुरुं कृष्णं चामरांश्चन्द्रिकोज्ज्वलान्} %॥४॥

\twolineshloka
{गृहीत्वा वृत्य चानिन्ये पञ्चषादध्वयोजनात्}
{अथापरस्मिन्नहनि प्रियादर्शनदोहनि} %॥५॥

\twolineshloka
{दूरमध्वानमुल्लङ्घ्य रवावस्तमिते सति}
{ध्वान्ते प्रसर्पति स्वैरं दिशो दश तरोस्तले} %॥६॥

\twolineshloka
{गतो वशं सदस्यूनां निजघ्ने तैश्च सत्वरम्}
{धर्मलोपादसौ जज्ञे घोरश्चोग्रतरो ग्रहः} %॥७॥

\twolineshloka
{पिपासितो बुभुक्षार्तो लेलिहानश्च सृक्किणी}
{उर्ध्वकेशोऽतिजङ्घालुः पृष्टलग्नोदरो महत्} %॥८॥

\twolineshloka
{अस्थिमात्रशरीरोऽभूद्दुर्वृत्तनयनो भृशम्}
{अत्रान्तरे सुतस्तस्य धर्मात्मा वेदकोविदः} %॥९॥

\twolineshloka
{पर्यपालयदत्यर्थं दिदृक्षुस्तं तदागमत्}
{नित्यमन्वेषयन्वार्त्तां पान्थेभ्यो नोपलब्धवान्} %॥१०॥

\twolineshloka
{ततः कदाचिदायाते सहायिनि च मानवे}
{तस्माद्विदितवृत्तान्तः शुशोच पितरं बहु} %॥११॥

\twolineshloka
{ततो विमृश्य मेधावी चिकीर्षुः पारलौकिकम्}
{वाराणसीं ससम्भारः स गन्तुमुपचक्रमे} %॥१२॥

\twolineshloka
{मार्गे निवासान्सप्ताष्टौ नीत्वा तस्य तरोस्तले}
{सन्ध्यां प्रचक्रमे कर्तुं यत्रास्य निहतः पिता} %॥१३॥

\twolineshloka
{तत्राध्यायं स गीतानां तृतीयं सञ्जजाप ह}
{ततो घोरस्वरस्तत्र व्योममध्ये परामृशत्} %॥१४॥

\twolineshloka
{ददर्श घोरमाकाशात्पतन्तं पितरं ततः}
{विस्मयेन भयेनापि विकलीकृतचेतनः} %॥१५॥

\twolineshloka
{तेजसा भूयसा व्याप्तमालुलोके पुरोऽम्बरे}
{किङ्किणीकोटिसङ्कीर्णं तेजसा व्याप्तदिङ्मुखम्} %॥१६॥

\twolineshloka
{विमानमग्रतोऽपश्यद्दिव्यमध्यग्रचेतनः}
{तत्रापश्यत्समारूढं दिव्याभिः स्त्रीभिरावृतम्} %॥१७॥

\twolineshloka
{संस्तूयमानं मुनिभिः पितरं पीतवाससम्}
{प्रणतस्तं समालोक्य युयुजे तेन चाशिषा} %॥१८॥

\twolineshloka
{ततोऽपृच्छदिदं वृत्तं स च तस्मै न्यवेदयत्}
{दुस्त्यजात्कर्मणो वत्स वपुषो पुण्यकारणात्} %॥१९॥

\twolineshloka
{मोचितोऽस्मि त्वया दैवादध्यायं जपतान्तिके}
{तन्निवर्त्तस्व जपतः साम्प्रतं त्वामुपस्थितम्} %॥२०॥

\onelineshloka*
{वाराणसीं यदर्थं यत्तदनुष्ठितमात्मनः}

\uvacha{श्रीभगवानुवाच}

\onelineshloka
{एवमुक्तः स च प्राह पितरं दीप्ततेजसम्} %॥२१॥

\uvacha{सुत उवाच}

\onelineshloka*
{हितं ममानुशाधि त्वं कार्यमन्यन्मयानु किम्}

\uvacha{श्रीभगवानुवाच}

\onelineshloka
{ततः प्राह पिता पुत्रं कार्यमेतत्त्वयानघ} %॥२२॥

\twolineshloka
{यन्मयाचरितं कर्म भ्रात्रा मम तु तत्कृतम्}
{स यातो नरकं घोरं तं मोचयितुमर्हसि} %॥२३॥

\twolineshloka
{अन्ये मदन्वये ये वै निरयं प्रतिपेदिरे}
{ते च मोचयितव्यास्ते इति मेऽस्ति मनोरथः} %॥२४॥

\twolineshloka
{इत्येवमुक्तः पुत्रस्तं पुनः प्राह कृताञ्जलि}
{कर्मणा केन तान्सर्वान्मोचयामि तदादिश} %॥२५॥

\onelineshloka*
{एवं निवेदितो वाक्यं पिता सुतमुवाच ह}

\uvacha{पितोवाच}

\onelineshloka
{येनाहं मोचितो वत्स तदनुष्ठातुमर्हसि} %॥२६॥

\twolineshloka
{अनुष्ठाय तदुत्पन्नं तेभ्यः पुण्यं समुत्सृज}
{ततोऽहमिव ते सर्वे पूर्वे सन्त्यज्य यातनाम्} %॥२७॥

\twolineshloka
{गमिष्यन्त्यचिरेणैव तद्विष्णोः परमं पदम्}
{स सन्दिष्टोवदत्पुत्रो यद्येवं तात नारकान्} %॥२८॥

\twolineshloka
{सर्वानपि विमोक्ष्यामि यदि ते रोचते वचः}
{एवमस्तु शिवं भूयादुपपन्नं महत्प्रियम्} %॥२९॥

\twolineshloka
{इत्यादिश्य पिता पुत्रं ययौ विष्णोः परं पदम्}
{सोऽपि तस्मात्परावृत्त्य जनस्थानं प्रपद्य च} %॥३०॥

\twolineshloka
{सुन्दरस्य पुरः शौरेश्चालये कालमभ्यगात्}
{स कुर्वाणोःसमादीनि पित्रा च यदुदीरितम्} %॥३१॥

\twolineshloka
{उत्ससर्ज कृतं पुण्यं मोचयिष्यन्स नारकान्}
{अत्रान्तरे पदे विष्णोर्यातनापदमीयुषः} %॥३२॥

\twolineshloka
{नारकान्मोचयिष्यन्तः किङ्करा यममभ्ययुः}
{तेन ते पूजिताः सर्वे सत्क्रियाभिरनेकधा} %॥३३॥

\twolineshloka
{कुशलं परिपृष्टास्ते सर्वतः सुखमूचिरे}
{एमं सत्कृत्य मेधावी पितृलोक महेश्वरः} %॥३४॥

\twolineshloka
{हेतुमागमने पृच्छत्ते च तस्मै न्यवेदयन्}
{विद्धि कीनाशनाथ त्वं शेषपर्यङ्कशायिना} %॥३५॥

\twolineshloka
{शौरिणा प्रहितानस्मान्समादेष्टुं त्वदन्तिके}
{अस्मन्मुखेन देवस्त्वां कुशलं परिपृच्छति} %॥३६॥

\twolineshloka
{नारकान्प्राणिनः सर्वान्विमोक्तुं च नियच्छति}
{इत्याकर्ण्य समादिष्टं विष्णोरमिततेजसः} %॥३७॥

\twolineshloka
{न तेन मूर्ध्ना सम्भाव्य दध्यौ किञ्चन चेतसा}
{विमुक्तान्निरयात्सर्वांस्तान्विलोक्य मदोत्कटान्} %॥३८॥

\twolineshloka
{स तैरनुगतः सर्वैर्विष्णोरायतनं ततः}
{ययौ स वरयानेन यत्रास्ते दुग्धवारिधिः} %॥३९॥

\twolineshloka
{तदन्त उदितानेकसूर्यकोटिसमप्रभम्}
{इन्दीवरदलश्याममालुलोक जगद्गुरुम्} %॥४०॥

\twolineshloka
{शय्याफणिफणारत्नमरीच्यामिश्रतेजसम्}
{विलोक्यमानमानन्दनिर्भरं प्रीतमानसम्} %॥४१॥

\twolineshloka
{भावानुगैर्दृगालीकैः श्रिया प्रेम्णेक्षितं मुहुः}
{योगिभिः परितो जुष्टं ध्याननिष्पन्दतारकैः} %॥४२॥

\twolineshloka
{स्तूयमानं महेन्द्रेण पराजेतुं विरोधिनः}
{आम्नायवचसामन्ते ब्रह्मणो निःसृतैर्मुखात्} %॥४३॥

\twolineshloka
{मूर्तिमद्भिर्वचोभिश्च गीयमानं गुणोत्करम्}
{सम्प्रीतं चाप्युदासीनमपि सर्वासु योनिषु} %॥४४॥

\twolineshloka
{योगसञ्चितपुण्यानां यौगपद्येन जन्तुषु}
{विलोकमानमात्मानमखिलं सचराचरम्} %॥४५॥

\twolineshloka
{आमोद्यन्नमालोकैरात्मानं दीप्तिपूरितैः}
{आबिभ्राणं वपुर्व्यापि द्योतितं भोगिनस्विषा} %॥४६॥

\twolineshloka
{इन्दीवरदलश्यामं ज्योत्स्नयेव नभस्तलम्}
{विलोक्य तं स तुष्टाव धिया बहुलयानतः} %॥४७॥

\uvacha{यम उवाच}

\twolineshloka
{नमः समस्तनिर्माणनिर्मलीभूतचेतसे}
{वद नोद्गीर्णवेदाय विश्वरूपाय वेधसे} %॥४८॥

\twolineshloka
{बलवेगसुदुर्द्धर्ष दानवेन्द्र मदद्रुहे}
{नमः स्थितौ च सत्वाय विश्वाधाराय विष्णवे} %॥४९॥

\twolineshloka
{नमः पातकसङ्घात जिष्णवेऽखिलदेहिनाम्}
{ईषदुन्मीलल्लालाटनेत्राग्निप्रभवार्चिषे} %॥५०॥

\twolineshloka
{त्वं हि सर्वस्य लोकस्य गुरुरात्मा महेश्वरः}
{विसृज्य वैष्णवान्सर्वानतस्त्वमनुकम्पसे} %॥५१॥

\twolineshloka
{व्यापन्नखिलं लोकं मायया परिबृंहितम्}
{न तया परिभूतोऽसि न च तत्प्रभवैर्गुणैः} %॥५२॥

\twolineshloka
{अन्तरावर्तमानोऽपि न ताभ्यामभिभूयसे}
{दृशा विषयवर्तिन्या निग्रहीतमना अपि} %॥५३॥

\twolineshloka
{तया फलाभिगामिन्या आत्मन्येवाभिलीयसे}
{न तवास्ति महिम्नोन्तो यथा निरवधिः स्वयम्} %॥५४॥

\twolineshloka
{मौनमेवात्रयुक्तं मे विषयोऽसि कथं गिराम्}
{इति स्तुत्वा ततो वाक्यमिदमाह कृताञ्जलि} %॥५५॥

\twolineshloka
{विनियोगादमी युक्ता देहिनो निर्गुणा मया}
{समादिश यदन्यन्मे कार्यमस्ति जगद्गुरो} %॥५६॥

\twolineshloka
{इति विज्ञापितस्तेन तमाह मधुसूदनः}
{मेघगम्भीरया वाचा सिञ्चन्निव सुधारसैः} %॥५७॥

\twolineshloka
{पापादुद्धार्य ते लोको मया समयवर्तिना}
{त्वयि विन्यस्तभारोऽहं नानुशोचामि देहिनः} %॥५८॥

\onelineshloka*
{तदाचर निजं कर्म प्रयाहि स्वं निकेतनम्}

\uvacha{श्रीभगवानुवाच}

\onelineshloka
{इत्युक्त्वान्तर्दधे देवः सोऽपि स्वपुरमाययौ} %॥५९॥

\twolineshloka
{सोऽपि स्वजातिजान्सर्वान्निरयस्थाननेकशः}
{उद्धृत्य वरयानेन विष्णुलोकं ययौ स्वयम्} %॥६०॥

॥इति श्रीपाद्मे महापुराणे पञ्चपञ्चाशत्सहस्रसंहितायामुत्तरखण्डे सतीश्वरसंवादे गीतामाहात्म्ये सप्तसप्तत्यधिकशततमोऽध्यायः॥१७७॥

\sect{{[४]} अष्टसप्तत्यधिकशततमोऽध्यायः} %6.178

\uvacha{श्रीभगवानुवाच}

\twolineshloka
{चतुर्थस्यापि माहात्म्यमाख्यास्याम्यधुना शृणु}
{बदरीत्वं समुत्सृज्य येन कन्ये दिवं गते} %॥१॥

\uvacha{श्रीरुवाच}

\twolineshloka
{कथं कन्ये दिवं याते बदरीत्वं विसृज्य वै}
{ते कचास्तां पुरा देव कथं प्राप्ते तु मुख्यताम्} %॥२॥

\twolineshloka
{एतद्वेदितुमिच्छामि नाथ वक्तुं त्वमर्हसि}
{न तु तृप्यामि शृण्वन्ती परमां च कथामिमाम्} %॥३॥

\uvacha{श्रीभगवानुवाच}

\twolineshloka
{अस्ति भागीरथी तीरे नाम्ना वाराणसी पुरी}
{भरतो नाम युक्तात्मा तत्र विश्वेश्वरालये} %॥४॥

\twolineshloka
{नित्यमात्मरतस्तुर्यं जपत्यध्यायमादरात्}
{तदभ्यासाददुष्टात्मा न द्वन्द्वैरभिभूयते} %॥५॥

\twolineshloka
{काले कदाचन क्रीडन्ययौ स नगराद्बहिः}
{उपान्त्यवर्तिनो देवान्दिदृक्षुः स तपोधनः} %॥६॥

\twolineshloka
{विशश्राम तयोर्मूले बदर्योर्न्यपतत्फले}
{उपधाय तयोरेकामन्यामालम्ब्य चाङ्घ्रिणा} %॥७॥

\twolineshloka
{तपस्विनि ततो याते बदर्योश्च तथार्द्वयम्}
{शुष्कं निष्पत्रशाखं च दिवसैः पञ्चषैरभूत्} %॥८॥

\twolineshloka
{गृहे क्वचन विप्राणां जज्ञाते कन्यके ततः}
{वर्द्धमानं तयोर्युग्मं सप्तभिः परिवत्सरैः} %॥९॥

\twolineshloka
{विहृत्य दूरदेशेभ्यो यतिमायान्त मैक्ष्यत}
{गृहीत्वा चरणौ तस्य वचः सूनृतमब्रवीत्} %॥१०॥

\twolineshloka
{त्वत्प्रसादादेव मुने मोचितं द्वन्द्वमावयोः}
{उत्सृज्य बदरीभावं मानुष्यं प्रतिपद्यते} %॥११॥

\twolineshloka
{एवमुक्तो मुनिस्ताभ्यां विस्मृतः प्रत्युवाच सः}
{कदा वत्से युवां केन हेतुना मोचिते मया} %॥१२॥

\twolineshloka
{युवयोर्बदरीत्वे च हेतुं ब्रूतां न वेद्म्यहम्}
{ऊचतुः कन्यके तस्मै बादर्ये हेतुमात्मनः} %॥१३॥

\twolineshloka
{आदौ विमोचने तस्माद्दुस्त्यजादपि कारणम्}
{अस्ति गोदावरी तीरे तीर्थं पुण्यप्रदं नृणाम्} %॥१४॥

\twolineshloka
{च्छिन्नपापमिति ख्यातं परां कोटिमवापयत्}
{तत्र सत्यतपा नाम तपस्तेपे सुदारुणम्} %॥१५॥

\twolineshloka
{ग्रीष्मे महति दीप्तानां मध्यगो जातवेदसाम्}
{वर्षासु जलधाराभिर्नित्यमासिक्तमूर्द्धजः} %॥१६॥

\twolineshloka
{शिशिरे च वसन्नप्सु बिभ्रत्कण्टकितां तनुम्}
{विशुद्धः सततं काले तपस्तप्त्वा स संयमी} %॥१७॥

\twolineshloka
{आत्मन्येव मतिं चक्रे परां प्राप्य सुनिर्वृतिम्}
{सदा फलानि बिभ्रत्सु सान्द्रच्छायेषु शाखिषु} %॥१८॥

\twolineshloka
{निर्मत्सरेषु सत्वेषु बध्वा प्रीतिपरामपि}
{तपःफलानुसन्धाने वैदुष्येनोपपादितम्} %॥१९॥

\twolineshloka
{ब्रह्माप्येनं स्वयं पृच्छन्नुपतस्थे तमन्वहम्}
{तेन सङ्कोचहीनत्वाद्ब्रह्यण्यनुगतेन्वहम्} %॥२०॥

\twolineshloka
{तद्ध्यानानुगतव्यक्ति ववृधे तस्य तत्तपः}
{विमुक्तकल्पं मन्वानः समृद्धादात्मनः पदात्} %॥२१॥

\twolineshloka
{अन्तरायशतं चक्रे ततो भीतः पुरन्दरः}
{आहूयाप्सरसां मध्यादावां तुल्यं समादिशत्} %॥२२॥

\twolineshloka
{कुरुतं तत्तपोविघ्नमनेनाचरितं युवाम्}
{यो मां पदादवष्टभ्य स्वराज्यं भोक्तुमिच्छति} %॥२३॥

\twolineshloka
{इति सन्देशमापन्ने पुरस्ताच्च बिडौजसः}
{गोदावरीमगच्छावः स मुनिर्यत्र वर्तते} %॥२४॥

\twolineshloka
{मृदङ्गैर्मन्दगम्भीरैर्वेणुभिः कलवादिभिः}
{कलं गीतं समारब्धं तन्वङ्गीभिः समन्वितम्} %॥२५॥

\twolineshloka
{उद्वहन्त्यौ पृथुश्रोणीं घनपीनपयोधरे}
{स्मयस्मेरमुखाम्भोजे किञ्चिदाकुञ्चितालके} %॥२६॥

\twolineshloka
{मणिकुण्डलदृष्टांसे पुण्डरीकोज्ज्वलेक्षणे}
{तनुमध्ये सुवृत्तोरु वहन्त्यौ च समे पदे} %॥२७॥

\twolineshloka
{आनर्तां यो वनस्यार्थे स्वरताललयानुगाम्}
{दर्शयन्त्यौ स्वतः कृत्स्नां गतिं भावानुगामिनीम्} %॥२८॥

\twolineshloka
{मृदूपक्रममुत्पन्नं मन्दं मन्दं विवर्द्धनम्}
{गर्जयामास दिक्चक्रं तत्तयोर्नृत्यमानयोः} %॥२९॥

\twolineshloka
{ततोङ्गहारवेगेन वायुर्वासः सुशीतलः}
{ईषदुच्छ्वसिते चैले दर्शयन्त्यौ पयोधरौ} %॥३०॥

\twolineshloka
{उद्वर्धयन्ती कन्दर्प्पमुल्बणा गतिरावयोः}
{कोपमुत्पादयामासु मुने रविकृतात्मनः} %॥३१॥

\twolineshloka
{ततः शापं ददौ कोपाज्जलमुत्सृज्य पाणिना}
{बदरीत्वं प्रपद्येथां जाह्नवी रोधसीति नौ} %॥३२॥

\twolineshloka
{आवाभ्यां पारतन्त्र्येण यद्दुश्चरितमास्थितम्}
{तत्क्षमस्व विनम्राभ्यां मुनिः पश्चात्प्रसादितः} %॥३३॥

\twolineshloka
{ततः शापविमोक्षं नौ कल्पयामास पुण्यधीः}
{भरतागमनान्तोयमिति सत्यतया मुनिः} %॥३४॥

\twolineshloka
{मर्त्येषु जन्मलाभश्च स्मृतिश्चातीतजन्मनाम्}
{आवयोरन्तिकं गत्वा बदरीभूतयोस्ततः} %॥३५॥

\twolineshloka
{स्मरता तुर्यमध्यायं भवता निष्कृतिः कृता}
{तत्तावत्प्रणमावस्त्वां शापादेव न केवलात} %॥३६॥

\onelineshloka*
{घोरादेव तु संसारात्त्वयैतेन विमोचिते}

\uvacha{श्रीभगवानुवाच}

\onelineshloka
{एवमुक्तो मुनिस्ताभ्यामतिप्रीतमनास्ततः} %॥३७॥

\twolineshloka
{पूजितस्ते समामन्त्र्य यथागतमसौ ययौ}
{कन्ये चतुर्थमध्यायं जपेतां नित्यमादरात्} %॥३८॥

॥इति श्रीपाद्मे महापुराणे पञ्चपञ्चाशत्सहस्रसंहितायामुत्तरखण्डे सतीश्वरसंवादे गीतामाहात्म्ये अष्टसप्तत्यधिकशततमोऽध्यायः॥१७८॥

\sect{{[५]} एकोनाशीत्यधिकशततमोऽध्यायः} %6.179

\uvacha{श्रीभगवानुवाच}

\twolineshloka
{पञ्चमस्याधुना देवि माहात्म्यं लोकपूजितम्}
{कथयामि समासेन सावधाना शृणु प्रिये} %॥१॥

\twolineshloka
{पिङ्गलो नाम भद्रेषु पुरुकुत्सपुरे द्विजः}
{अवदाते कुले जातो विश्रुते वेदवादिनाम्} %॥२॥

\twolineshloka
{कुलोचितानि शास्त्राणि तथा वेदान्विसृज्य सः}
{तौर्यत्रिके मतिं चक्रे वादयन्मुरजादिकम्} %॥३॥

\twolineshloka
{कृतश्रमस्ततस्तत्र गीते नृत्ये च वादने}
{परां प्रसिद्धिमासाद्य नृपसद्म विवेश सः} %॥४॥

\twolineshloka
{समातस्थे स तेनासौ पुरा भूमिभुजा सह}
{परदारानुपाहृत्य बुभुजे ता अनन्यधीः} %॥५॥

\twolineshloka
{तत उत्सिक्तगर्वोऽयं सूचमानो निरङ्कुशः}
{परच्छिद्राणि चामुष्मै विविक्ते स निरन्तरम्} %॥६॥

\onelineshloka
{तस्यासीदरुणा नाम भार्या हीनकुलोद्भवा} %॥७॥

\twolineshloka
{भ्रमत्यन्वेषयन्ती सा कामुकेन विहारिणी}
{तमन्तरायं मन्वाना निशीथिन्यां निजालये} %॥८॥

\twolineshloka
{निजघान शिरश्छित्त्वा निचखान महीतले}
{वियोजितस्तः प्राणैरुपेत्य यमसादनम्} %॥९॥

\twolineshloka
{दुर्जयान्नरकान्भुङ्क्त्वा गृध्रोऽभूद्विजने वने}
{भगन्दरेण रोगेण सापि हित्वा वरां तनुम्} %॥१०॥

\twolineshloka
{उपेत्य नरकान्घोरान्जज्ञे तत्र वने शुकी}
{कणानादातुकामां तां सञ्चरन्तीमितस्ततः} %॥११॥

\twolineshloka
{विददार नखैस्तीक्ष्णैर्गृध्रो वैरमनुस्मरन्}
{नृकपाले पयः पूर्णे निपतन्तीं ततः शुकीम्} %॥१२॥

\twolineshloka
{अभिदुद्राव गृध्रोऽपि निजघ्ने स च जालिकैः}
{पत्नीवियोजिता प्राणैर्नृकपालजले ततः} %॥१३॥

\threelineshloka
{तत्रैव निममज्जा सा वेत्यक्रूरतरः खगः}
{पितृलोकं प्रपेदाते नीतौ तौ यमकिङ्करैः}
{प्राक्कृतं दुष्कृतं कर्म स्मरन्तौ भयभागिनौ} %॥१४॥

\twolineshloka
{ततो यमः समालोक्य तयोः कर्म जुगुप्सितम्}
{अकस्मादेवतत्स्नानान्मरणे सुकृतं महत्} %॥१५॥

\twolineshloka
{अनुजज्ञे ततो लोकमीप्सितं गन्तुमेतयोः}
{महापातकसङ्घातैरपि दुर्द्धर्षमानसौ} %॥१६॥

\twolineshloka
{ततोविस्मयमापन्नौ स्मृत्वा तौ दुष्कृतं निजम्}
{उपेत्य प्रणतौ भूत्वा वैवस्वतमपृच्छताम्} %॥१७॥

\twolineshloka
{सञ्चितं दुष्कृतं पूर्वमावाभ्यामपि गर्हितम्}
{लोकानामीप्सितानां तु को हेतुस्तद्वदस्व नौ} %॥१८॥

\twolineshloka
{एवमुक्तस्ततस्ताभ्यामाह वैवस्वतो वचः}
{आसीद्गङ्गातटे नाम्ना बुद्ध्वा ब्रह्म सनातनम्} %॥१९॥

\twolineshloka
{एकाकी निर्ममः शान्तो वीतरागो विमत्सरः}
{गीतानां पञ्चमाध्यायमावर्त्तयति सर्वदा} %॥२०॥

\twolineshloka
{तेन पुण्येन पूतात्मा बुद्ध्वा ब्रह्म सनातनम्}
{पापीयानपि यं श्रुत्वा तनुमुत्सृष्टवानसौ} %॥२१॥

\twolineshloka
{निर्मलीकृतदेहस्य गीताभिर्भावितात्मनः}
{तत्कपालजलं प्राप्य युवां यातौ पवित्रताम्} %॥२२॥

\twolineshloka
{तद्गच्छतं युवां लोकान्मनोरथपथिस्थितान्}
{गीतानां पञ्चमाध्यायमाहात्म्येन पवित्रितौ} %॥२३॥

\twolineshloka
{एवं तौ बोधितौ तेन मुदितौ समवर्तिना}
{व्योमयानं समारुह्य जग्मतुर्वैष्णवं पदम्} %॥२४॥

॥इति श्रीपाद्मे महापुराणे पञ्चपञ्चाशत्सहस्रसंहितायामुत्तरखण्डे सतीश्वरसंवादे गीतामाहात्म्ये एकोनाशीत्यधिकशततमोऽध्यायः॥१७९॥

\sect{{[६]} अशीत्यधिकशततमोऽध्यायः} %6.180

\uvacha{श्रीभगवानुवाच}

\twolineshloka
{षष्ठाध्यायस्य माहात्म्यं प्रवक्ष्यामि वरानने}
{यदाकर्णयतां नॄणां मुक्तिः करतले स्थिता} %॥१॥

\twolineshloka
{अस्ति गोदावरी तीरे प्रतिष्ठानं पुरं महत्}
{पिप्पलेशाभिधानोऽहं यत्रास्मि स्मेरलोचने} %॥२॥

\twolineshloka
{यत्र गोदावरी तीर शीकरैरवशीतलैः}
{हंसाः पक्षपुटैः कीर्णैर्हरन्ति यमिनां श्रमम्} %॥३॥

\twolineshloka
{स्फुरत्पद्मावलीकोशपरागसुरभीकृतम्}
{श्लाघ्यं गोदावरीतोयं येन ते निर्जरा नराः} %॥४॥

\twolineshloka
{धिक्सुधामौषधीशस्य विकृत्क्षयविधायिनीम्}
{महाराष्ट्रवधूकानां मज्जन्तीनां मुनीश्वराः} %॥५॥

\twolineshloka
{स्पृशन्ति यत्र वक्त्राणि फुल्लपङ्कजशङ्कया}
{यत्रखेलन्महाराष्ट्राः क्वणत्कङ्कणसुन्दराः} %॥६॥

\twolineshloka
{हरन्ति ध्वनयो लीनां मनांस्यपि तपस्विनाम्}
{अत्युच्चसौधशिखरविहारि वनितामुखम्} %॥७॥

\twolineshloka
{पश्यन्ननुदिनं यत्र क्षीयते मृगलाञ्च्छनः}
{अत्युच्चसौधवलभी महामणिमरीचिभिः} %॥८॥

\twolineshloka
{चुम्ब्यन्ते मुनिगन्धर्वैर्दूर्वाचन्दनचञ्चलैः}
{यस्मिन्नाधूयमानानां पताकानां समीरणैः} %॥९॥

\twolineshloka
{गतश्रमा रवेर्याने भवन्ति रथवाजिनः}
{राशीकृतैर्मलयजैरसङ्ख्यातैर्वणिग्गणैः} %॥१०॥

\twolineshloka
{यस्मिन्नुपलशेषोऽसौ लक्ष्यते मलयाचलः}
{पुञ्जीकृतानि दृश्यन्ते यत्र मुक्ताफलान्यपि} %॥११॥

\twolineshloka
{नगरीदेवता हास्य स्तबका इव सर्वतः}
{तत्र ज्ञानश्रुतिर्नाम्ना मेदिनीवल्लभोऽभवत्} %॥१२॥

\twolineshloka
{यस्मिन्नुद्धरति क्षोणीं शेषोऽयं मणिसन्निभाम्}
{अपि प्रतापमार्तण्डमण्डली तीव्रतेजसि} %॥१३॥

\twolineshloka
{नित्यमध्वरधूमेन श्यामलाः कल्पशाखिनः}
{असाधारणदातृत्वं पश्यन्त इव लज्जया} %॥१४॥

\twolineshloka
{यदध्वरपुरोडाश चर्वणास्वादलम्पटाः}
{न तत्यजुः सुपर्वाणः प्रतिष्ठानपुरं मनाक्} %॥१५॥

\twolineshloka
{यस्यदानाम्बुधाराभिः प्रतापज्योत्स्नयानिशम्}
{मखधूमैश्च सम्पुष्टा ववृषुः समये घनाः} %॥१६॥

\twolineshloka
{स्वल्पमात्रमपि क्वापि न पदं प्रापुरीतयः}
{नीतयः प्रसरन्ति स्म यस्मिन्शासति मेदिनीम्} %॥१७॥

\twolineshloka
{वापिकूपतडागानां च्छद्मना योऽनुवासरम्}
{हृदयस्थानि मेदिन्या निधानानि व्यलोकयत्} %॥१८॥

\twolineshloka
{पाण्डुराभिः पताकाभिः प्रासादो यस्य राजते}
{वियद्गङ्गातरङ्गोघैर्हिमाद्रिरिव सानुमान्} %॥१९॥

\twolineshloka
{दानैस्तपोभिर्यज्ञैश्च प्रजानां पालनेन च}
{तुष्टाः स्वर्गौकसस्तस्मै वरं दातुं समागमन्} %॥२०॥

\twolineshloka
{ततोन्तरिक्षमार्गेण धुन्वानाः पक्षसंहतीः}
{मृणालधवला देवी देवहंसा विनिर्गताः} %॥२१॥

\twolineshloka
{त्वरया गच्छतां तेषामन्योन्यं तत्र भाषिणाम्}
{भद्राश्वप्रमुखा द्वित्राः पुरस्तान्निर्ययुर्जवात्} %॥२२॥

\twolineshloka
{सर्वैर्मिलद्भिरूचुस्ते पुरस्ताद्गच्छतो जवात्}
{कथं वेगेन निर्याता भवन्तः पुरतः स्थिताः} %॥२३॥

\twolineshloka
{सर्वैर्मिलित्वा गन्तव्यमस्मिन्नध्वनि दुर्गमे}
{प्रकाशमानं पुरतस्तेजःपुञ्जं न पश्यथ} %॥२४॥

\twolineshloka
{ज्ञानश्रुतेर्महीभर्तुः पुण्यमूर्त्तेरतिस्फुटम्}
{निशम्येति वचः सम्यक्पाश्चात्याना पुरः स्थिताः} %॥२५॥

\twolineshloka
{हंसा हसित्वा सावज्ञमूचुर्वचनमुच्चकैः}
{रैक्याभिधस्य दुर्द्धर्ष तेजसो ब्रह्मवादिनः} %॥२६॥

\twolineshloka
{किं नु ज्ञानश्रुतेरस्य राज्ञस्तीव्रतरं महः}
{इति शुश्राव हंसानां गिरो ज्ञानश्रुतिर्नृपः} %॥२७॥

\twolineshloka
{अत्युच्चसौधभवनमारुह्य च सुखं स्थितः}
{ततः सारथिमाहूय भूपालो विस्मयान्वितः} %॥२८॥

\twolineshloka
{सन्दिदेश महात्माऽसौ रैक्य आनीयतामिति}
{ततोऽवधार्य भूपालवचः पीयूषगर्भितम्} %॥२९॥

\twolineshloka
{निर्जगाममहो नाम्ना सारथिः प्रथयन्मुदम्}
{यत्र वाराणसीनाम नगरी मुक्तिदायिनी} %॥३०॥

\twolineshloka
{यत्र विश्वेश्वरो नाम ह्युपदेष्टा जगत्पतिः}
{ततो गयाभिधे क्षेत्रे यत्र देवो गदाधरः} %॥३१॥

\twolineshloka
{उद्धर्तुमखिलान्लोकान्वसत्युत्फुल्ललोचनः}
{ततो गौरीगुरोः पार्श्वे सर्वैस्तीर्थैरनेकधा} %॥३२॥

\twolineshloka
{पर्यटन्गतवान्यत्र केदारः पापदारणः}
{यमालोक्य सकृन्मर्त्या मुक्ताः स्युर्नात्र संशयः} %॥३३॥

\twolineshloka
{महापापविर्निमुक्ता भुक्त्वा भोगान्यथेप्सितान्}
{ततो गौडेषु निर्यातो यत्रास्ते पुरुषोत्तमः} %॥३४॥

\twolineshloka
{यस्यावलोकनादेव नराः स्वर्लोकगामिनः}
{ततो द्वारावतीं प्रागान्नगरीं मुक्तिदायिनीम्} %॥३५॥

\twolineshloka
{यत्रास्ते गोमती तीरे रुक्मिणीवल्लभो हरिः}
{स्नात्वा च गोमतीतीर्थे पञ्चकृष्णान्विलोक्य च} %॥३६॥

\twolineshloka
{मर्त्यो मुक्तिमवाप्नोति भुक्त्वा भोगान्यथेप्सितान्}
{ततः समुद्रमासाद्य सोमनाथं विलोक्य च} %॥३७॥

\twolineshloka
{भुक्तिमुक्तिप्रदं देवं ततो निरगमन्सुधीः}
{अवन्तिकां पुरीं प्राप्तो भुक्तिमुक्तिप्रदायिनीम्} %॥३८॥

\threelineshloka
{यत्रोमया सुखं क्रीडन्महाकालोस्ति शङ्करः}
{अथोङ्कारंसमासाद्य शर्मदं नर्मदातटे}
{भुक्तिमुक्तिप्रदातारं त्वरया निर्गतस्ततः} %॥३९॥

\twolineshloka
{अश्वमेधकरं नाम्ना नगरं पर्यटंस्ततः}
{यत्र शार्ङ्गधरः साक्षादास्तेलक्ष्मीपतिः स्वयम्} %॥४०॥

\twolineshloka
{ततो विष्णुगयां प्राप्तः कुण्डं लोणारसंज्ञितम्}
{यत्र स्नात्वा च पीत्वा च मुच्यते बन्धनान्नरः} %॥४१॥

\twolineshloka
{ततः कोल्हापुरं नाम गतो रुद्रगयां प्रति}
{आस्ते भगवती यत्र लक्ष्मीर्भक्तिप्रदायिनी} %॥४२॥

\twolineshloka
{पञ्चनद्यां नरः स्नात्वा महालक्ष्मीं विलोक्य च}
{भुक्त्वा भोगान्यथाकामं भक्तिं च प्रतिपद्यते} %॥४३॥

\twolineshloka
{ततोऽमलगिरिं नाम नगरीं प्रतिपद्य च}
{नन्दिकेश्वरमारुह्य सोमनाथोऽस्ति यत्र तु} %॥४४॥

\twolineshloka
{दृष्ट्वा चतुर्भुजं देवं वरदानोद्यतं शिवम्}
{सोमनाथं नृणां मुक्तिर्भवत्येव न संशयः} %॥४५॥

\twolineshloka
{तुङ्गभद्रानदी तीरे दृष्ट्वा हरिहरं ततः}
{युगेयुगे भुजा यस्य पतन्त्यवनि मण्डले} %॥४६॥

\twolineshloka
{यद्विलोक्य नराः सर्वे रम्यं हरिहरं वपुः}
{भुक्त्वा भोगान्यथाकामं मुच्यन्ते बन्धनान्नराः} %॥४७॥

\twolineshloka
{स्वर्गे कल्पशतं स्थित्वा मुक्तसंसारबन्धनाः}
{ततः स्वामिनमालोक्य लोकानां स्वामिनं विभुम्} %॥४८॥

\twolineshloka
{यमालोक्य न पश्यन्ति निरयं जातुचिन्नराः}
{स्वर्गे कल्पशतं स्थित्वा मुक्तसंसारवासनाः} %॥४९॥

\twolineshloka
{मुक्तिं च प्रतिदद्यन्ते नात्र कार्या विचारणा}
{ततः श्रीशैलमासाद्य सिद्धगन्धर्वसेवितम्} %॥५०॥

\twolineshloka
{गिरिजावल्लभो यत्र मल्लिनाथोऽभिधानतः}
{उद्धर्तुमखिलाँल्लोकान्संसाराम्भोधिमध्यतः} %॥५१॥

\twolineshloka
{स्वर्गे कल्पशतं स्थित्वा मुक्तसंसारबन्धनाः}
{मुक्तिं च प्रतिपद्यन्ते नात्र कार्या विचारणा} %॥५२॥

\twolineshloka
{ततः श्रीशैलमासाद्य सिद्धगन्धर्वसेवितम्}
{गिरिजावल्लभो यत्र मल्लिनाथोऽभिधानतः} %॥५३॥

\twolineshloka
{उद्धर्तुमखिलाँल्लोकान्संसाराम्बुधिमध्यतः}
{कालेकाले परं ज्योतिर्यः सन्दर्शयते स्वयम्} %॥५४॥

\twolineshloka
{अवलोकयतां नॄणां यमनुस्मरतामपि}
{दूरे तिष्ठन्ति सन्त्रस्ता दूरं निरययातनाः} %॥५५॥

\twolineshloka
{स्वर्गे लोके सुखं भुक्त्वा मुक्तसंसारबन्धनाः}
{मुक्तिं च प्रतिपद्यन्ते मानवा नात्र संशयः} %॥५६॥

\twolineshloka
{रामोस्ति सानुजः सार्द्धं जानक्यापि ततो गतः}
{तत्र स्नात्वा च पीत्वा च मुच्यते नरकाद्ध्रुवम्} %॥५७॥

\twolineshloka
{कल्पकोटिशतं भुक्त्वा स्वर्गलोकसुखं नराः}
{मुक्तसंसारवर्त्मानो मुक्तिं यान्ति न संशयः} %॥५८॥

\twolineshloka
{ततो निवृत्य आयातः पश्यन्भीमरथी तटे}
{द्विभुजं विठ्ठलं देवं भुक्तिमुक्तिप्रदायकम्} %॥५९॥

\twolineshloka
{यत्र गोदावरीजन्मस्थानं ब्रह्मगिरिर्महान्}
{गौतमालयमासाद्य यत्रास्ते त्र्यम्बको हरः} %॥६०॥

\twolineshloka
{अरुणावरुणयोर्मध्ये यत्र गोदावरी नदी}
{तत्र स्नात्वा च पीत्वा च ब्रह्महत्या विलीयते} %॥६१॥

\twolineshloka
{असङ्ख्यतीर्थसम्पन्नं दृष्ट्वा ब्रह्मगिरिं नराः}
{मुक्तिमेव प्रपद्यन्ते मुक्ताः संसारदुःखतः} %॥६२॥

\twolineshloka
{गौतम्युभयतीरस्थ तीर्थान्वेषणकौतुकी}
{ततो जगाम सूतस्तु मथुरां पापनाशिनीम्} %॥६३॥

\twolineshloka
{यत्र स्वायम्भुवं देवं भजन्ति सुरमानवाः}
{आद्यं भगवतः स्थानं महन्मुक्तिप्रदायकम्} %॥६४॥

\twolineshloka
{त्रैलोक्येश जनिस्थानं विख्यातं वेदशास्त्रयोः}
{नानादेवगणैर्जुष्टं द्विजर्षिगणसेवितम्} %॥६५॥

\twolineshloka
{कालिन्दीकूलसंशोभि ह्यर्द्धचन्द्र प्रभाकृति}
{सर्वतीर्थनिवासैक पूर्णमानन्दसुन्दरम्} %॥६६॥

\twolineshloka
{गोवर्द्धनगिरिप्रख्यं पुण्यद्रुमलतावृतम्}
{द्विषड्वनं महापुण्यं विश्रान्ति श्रुतिसारभृत्} %॥६७॥

\twolineshloka
{ततः काश्मीरनगरमपश्यत्प्रत्यगुत्तरम्}
{दृष्ट्वा धर्मधुरं क्षेत्रं कुरुक्षेत्रं समन्ततः} %॥६८॥

\twolineshloka
{यत्राभ्रंलिहगेहानां पङ्क्तयः शङ्खपाण्डुराः}
{ता जाता धूर्जटेः स्पष्ट अट्टहासदशा इव} %॥६९॥

\twolineshloka
{भक्तिप्रसादमालानां सुवर्णकलशैर्वृतम्}
{स्वः सिन्धोः पतितानीव हेमपद्मानि मारुतैः} %॥७०॥

\twolineshloka
{यत्र प्रासादशिखरे नीलपट्टपताकिकाः}
{शैवालवलयाभान्ति स्वः सिन्धोर्लतिका इव} %॥७१॥

\twolineshloka
{यत्र काश्मीरमाश्रित्य नित्यं वसति भारती}
{नौचेद्युगपदेवेदं कथं लिखति वाङ्मयम्} %॥७२॥

\twolineshloka
{विश्रामन्त्याः सरस्वत्याश्चिरं यत्र मदालसाः}
{मृणालचञ्चवो हंसा वाहनानि चरन्त्यमी} %॥७३॥

\twolineshloka
{कलाविशेषं प्रहिता यत्र बोद्धुं विरञ्चिना}
{तारा इव विराजन्ते हंसा याताः समन्ततः} %॥७४॥

\twolineshloka
{स्थलपद्मानि दृश्यन्ते करस्पर्शसुखानि च}
{शयनाय नितम्बिन्या यस्मिन्दानववैरिणा} %॥७५॥

\twolineshloka
{उपन्यासैर्द्विजातीनां यत्र न श्रूयते स्फुटम्}
{मूकोऽपि निर्जरो वाचा पदकल्लोलडम्बरः} %॥७६॥

\twolineshloka
{यस्मिन्नध्वरधूमेन व्याप्तं गगनमण्डलम्}
{अपि च क्षालितं मेघैः कालिमानं न मुञ्चति} %॥७७॥

\twolineshloka
{गलितायाः सुधायास्तु यत्राध्वरमहार्चिषा}
{लाञ्छितं छद्मनास्थानं दृश्यते तुहिनत्विषि} %॥७८॥

\twolineshloka
{जन्माभ्यासवशादेव पठन्ति बटवः स्वयम्}
{यत्रोपाध्यायसान्निध्यमाश्रित्य सकलाः कलाः} %॥७९॥

\twolineshloka
{यत्र ब्राह्मणपत्नीनां कङ्कणध्वनि हुङ्कृतिः}
{लुम्पत्यनुदिनं भ्राम्यद्भ्रमराणां च गर्जितम्} %॥८०॥

\twolineshloka
{यत्र ब्राह्मणपत्नीनां कपोलफलकं मुहुः}
{स्पृशन्समीरणो मन्दं वाति शापभयादिव} %॥८१॥

\twolineshloka
{माणिक्येश्वरनामासौ यत्र शीतांशुशेखरः}
{वसत्यनुदिनं देवो वरदानाय देहिनाम्} %॥८२॥

\twolineshloka
{अर्चितो भूपतीन्जित्वा मणिकेशेन चादृतः}
{माणिक्येश्वर इत्याख्यां तदाप्रभृति यो दधौ} %॥८३॥

\twolineshloka
{राज्ञा काश्मीरदेवेश दिग्जयोत्सवकारिणा}
{असौ सुपूजितो यस्मान्माणिक्यैर्भूरिभूतिभिः} %॥८४॥

\twolineshloka
{संसेवमानं तद्वारि छायां शकटिकोपरि}
{कण्डूयमानमङ्गानि यन्तारैक्यमपश्यत} %॥८५॥

\twolineshloka
{राज्ञापि कथितैस्तैस्तैश्चिह्नैः परिचितं जवात्}
{प्रणतः सारथी रैक्यं प्रणम्य तमभाषत} %॥८६॥

\uvacha{सारथिरुवाच}

\twolineshloka
{कस्मिन्ब्रह्मन्किन्नामासि स्वछन्दोऽसि निरन्तरम्}
{किमर्थमत्र विश्रान्तः किं च कर्तुं चिकीर्षसि} %॥८७॥

\twolineshloka
{इत्याकर्ण्य च तद्वाक्यं परमानन्दनिर्भरः}
{स्मृत्वा सारथिमित्यूचे वयं पूर्णमनोरथाः} %॥८८॥

\twolineshloka
{परं केनापि बहुना परिचर्याविधायिना}
{भवितव्यं मनोवृत्तिं जानतास्माकमेव हि} %॥८९॥

\twolineshloka
{हृदयस्थितमादाय रैक्याभिप्रायमादरात्}
{शनैर्निरगमद्यन्ता यत्रास्ते वसुधाधिपः} %॥९०॥

\twolineshloka
{ततः प्रणम्य भूपालं यथावृत्तं न्यवेदयत्}
{बद्धाञ्जलिपुटो हृष्टः सारथिः स्वामिदर्शनात्} %॥९१॥

\twolineshloka
{ततो निशम्य तद्वाक्यं विस्मयस्मेरलोचनः}
{श्रद्धालुरभवद्भूपो रैक्यसम्भावनाविधौ} %॥९२॥

\twolineshloka
{आदायाश्वतरीयुग्मयुक्तां शकटिकामगात्}
{मुक्ताहारदुकूलानि सहस्रं च गवां नृपः} %॥९३॥

\twolineshloka
{गतोऽसौ तत्र यत्रास्ते योगी काश्मीरमण्डले}
{तन्निवेद्य पुरो राजा दण्डवत्पतितो भुवि} %॥९४॥

\twolineshloka
{आनम्य परया भक्त्या रैक्यो राज्ञे चुकोप ह}
{रे शूद्र मामकं वृत्तं न जानासि दुरीश्वर} %॥९५॥

\twolineshloka
{गृहाण शकटीमेतामुत्थाप्याश्वतरी युताम्}
{वस्त्राणि मुक्ताहारांश्च गाश्च दोग्ध्रीरपि स्वयम्} %॥९६॥

\twolineshloka
{इत्थमाज्ञतवान्भूपो रैक्यस्य भयमादधे}
{ततः शापभयाद्राजा तत्पदाम्भोरुहद्वयम्} %॥९७॥

\onelineshloka
{गृह्णन्भक्त्या प्रसीदेति ब्रह्मन्नित्यूचिवान्स्वयम्} %॥९८॥

\uvacha{राजोवाच}

\twolineshloka
{भगवंस्तव माहात्म्यमेतदत्युद्भुतं कुतः}
{प्रसन्नीभूय भगवन्नाख्याहि मम तत्त्वतः} %॥९९॥

\uvacha{रैक्य उवाच}

\twolineshloka
{गीतानां षष्ठमध्यायं जपामि प्रत्यहं नृप}
{तेनैव तेजोराशिर्मे सुराणामपि दुःसहः} %॥१००॥

\twolineshloka
{गीतानां षष्ठमध्यायं रैक्यादभ्यस्य यत्नतः}
{ज्ञानश्रुतिर्महीपालो मुक्तिमाप ततः सुधीः} %॥१०१॥

\twolineshloka
{रैक्योऽपि सुखमालेभे माणिक्येश्वरसन्निधौ}
{गीतानां षष्ठमध्यायं जपन्मोक्षप्रदायकम्} %॥१०२॥

\twolineshloka
{मरालवेषमास्थाय वरदानार्थमागताः}
{दिवौकसोऽपि निर्जग्मुः स्वैरं विस्मयकारिताः} %॥१०३॥

\twolineshloka
{इममध्यायमप्येकं यो जपेत्सततं नरः}
{सोऽपि तत्पदवीमेति विष्णोरेव न संशयः} %॥१०४॥

॥इति श्रीपाद्मे महापुराणे पञ्चपञ्चाशत्सहस्रसंहितायामुत्तरखण्डे सतीश्वरसंवादे गीतामाहात्म्ये अशीत्यधिकशततमोऽध्यायः॥१८०॥

\sect{{[७]} एकाशीत्यधिकशततमोऽध्यायः} %6.181

\uvacha{श्रीभगवानुवाच}

\twolineshloka
{अथ ते वर्णयिष्यामि सप्तमाध्यायगौरवम्}
{यदाकर्ण्य सुधापूर पूर्तिर्भवति कर्णयोः} %॥१॥

\twolineshloka
{अस्ति पाटलिपुत्राख्यं दुर्गमुत्तुङ्गगोपुरम्}
{तत्राभूद्ब्राह्मणो नाम शङ्कुकर्णो दयार्णवः} %॥२॥

\twolineshloka
{वैश्यवृत्तिं समासाद्य धनमर्जितवान्बहु}
{पितॄन्नतर्पयामास पूजयामास नो सुरान्} %॥३॥

\twolineshloka
{पार्थिवान्भोजयांश्चक्रे धनार्जनपरायणः}
{तुरीयपाणिग्रहणं मङ्गलार्थं गृहान्तरे} %॥४॥

\twolineshloka
{तनुजैर्बन्धुभिः सार्धं सम्प्रतस्थे कदाचन}
{रजन्यां धर्मकल्पायां निद्रालोस्तस्य दोस्तले} %॥५॥

\twolineshloka
{दशति स्म समागत्य दन्दशूकः कुतश्चन}
{स दष्टमात्रोऽसाध्यात्मा मणिमन्त्रौषधादिभिः} %॥६॥

\twolineshloka
{क्षणैः कतिपयैरेव गतासुरभवत्ततः}
{पिचुमन्ददलैर्नालैरवगुण्ठितविग्रहम्} %॥७॥

\twolineshloka
{तमारोप्य तरुस्कन्धे सूनवो गृहमाययुः}
{ततः कालेन बहुना ततो जातः सरीसृपः} %॥८॥

\twolineshloka
{तद्वासनानिबद्धात्मा जन्मपूर्वमनुस्मरन्}
{वञ्चयित्वा सुतानेतान्पूरयामि गृहाद्बहिः} %॥९॥

\twolineshloka
{आत्मनः कोटिसङ्ख्याकं यत्रास्ते स्थापितं वसु}
{ततो नारायणबलिश्रद्धया परयान्विताः} %॥१०॥

\twolineshloka
{कृतवन्तः परे तस्य सूनवो हि द्विजन्मनः}
{एकदा स्वप्नमागत्य पीडितः सर्पजन्मना} %॥११॥

\twolineshloka
{अभाषयन्मनोवृत्तं पुत्राणामग्रतः पिता}
{ततस्ते प्रातरुत्थाय परं विस्मयमोहिताः} %॥१२॥

\twolineshloka
{इतरेतरमाख्याय पर्यन्तस्ते निरङ्कुशाः}
{एकस्तत्रपितृस्नेहादुद्धर्तुमपि वाञ्छति} %॥१३॥

\twolineshloka
{अन्यो द्रविणलोभेन निहन्तुं सर्पमीहते}
{इतरस्तु पितृस्नेहरसमोहितमानसः} %॥१४॥

\twolineshloka
{किंवा अहिमयो न स्याच्छोचन्रोदिति केवलम्}
{मध्यमस्तु ततः पुत्रो वञ्चयित्वा सहोदरौ} %॥१५॥

\twolineshloka
{केनापि छद्मनोत्थाय जगाम निजमालयम्}
{ततः शनैः समाहूय गृहिणीं गुणशालिनीम्} %॥१६॥

\twolineshloka
{कुद्दालहस्तो निरगाद्यत्रास्ते पन्नगः पिता}
{तेनाविदितवित्तेन चिह्नैर्निश्चित्य तत्वतः} %॥१७॥

\twolineshloka
{स्थानमागत्य तं हन्तुं वल्मीकं लोभबुद्धितः}
{भार्ययोत्सार्य ते मृत्स्ना स्वयं तेन च खन्यते} %॥१८॥

\twolineshloka
{निखन्यमानादत्युग्रो वल्मीकादहिरुत्थितः}
{ततो गरलगण्डूषैर्निर्गतैरतिदुःसहैः} %॥१९॥

\onelineshloka
{गिरः स कथयाञ्चक्रे फणी फूत्कारमारुतैः} %॥२०॥

\uvacha{अहिरुवाच}

\twolineshloka
{कस्त्वं किमर्थमायातः कथं वा खन्यते बिलम्}
{केन वा प्रहितो मूढ तदाख्याहि ममाग्रतः} %॥२१॥

\uvacha{पुत्र उवाच}

\twolineshloka
{पुत्रस्तेऽहं शिवो नाम हेमग्रहणकौतुकी}
{आगतो रात्रलब्धस्य स्वप्नस्य तु सुविस्मितः} %॥२२॥

\uvacha{शिव उवाच}

\twolineshloka
{इत्थमाकर्ण्य पुत्रस्य गिरं लोकविगर्हिताम्}
{वक्तुमारभत स्पष्टं हसन्नुच्चैः फणी तदा} %॥२३॥

\uvacha{सर्प उवाच}

\twolineshloka
{यदिपुत्रोऽसि मे तूर्णं मामुन्मोचय बन्धनात्}
{निक्षेपार्थाय सञ्जातं पन्नगं पूर्वजन्मनः} %॥२४॥

\uvacha{पुत्र उवाच}

\twolineshloka
{पितः कथं ते मुक्तिः स्यादित्याचक्ष्व ममाग्रतः}
{परित्यक्त्वाखिलं लोकमागतोस्मि यथा निशि} %॥२५॥

\uvacha{पितोवाच}

\twolineshloka
{न तीर्थानि न दानानि न तपांसि न चाध्वराः}
{मामुन्मोचयितुं पुत्र प्रभवन्ति च सर्वथा} %॥२६॥

\twolineshloka
{गीतानां सप्तमाध्यायमन्तरेण सुधामयम्}
{जन्तोर्जरामृत्युदुःखनिराकरणकारणम्} %॥२७॥

\twolineshloka
{सप्तमाध्यायिनं विप्रं मदीये श्राद्धवासरे}
{भोजय श्रद्धया पुत्र तेन मुक्तिर्न संशयः} %॥२८॥

\twolineshloka
{अन्यानपि द्विजान्वत्स वेदविद्याविशारदान्}
{सम्भोजय यथाशक्ति परमश्रद्धयान्वितः} %॥२९॥

\twolineshloka
{इत्याकर्ण्य पितुर्वाक्यमुरगत्वमुपेयुषः}
{ते सर्वे सूनवो कुर्वन्यथादिष्टं ततोऽधिकम्} %॥३०॥

\twolineshloka
{शङ्कुकर्णस्ततः श्रीमानुत्सृज्य तनुमौरगीम्}
{कृत्वा विभागं पुत्राणां दिव्यन्देहमुपाददे} %॥३१॥

\twolineshloka
{विभज्य दत्तं पित्राय द्रव्यं तत्कोटिसङ्ख्यया}
{तेन ते सूनवः सर्वे मुमुदुः साधुवृतयः} %॥३२॥

\twolineshloka
{वापीकूपसरोयज्ञदेवप्रासादहेतवे}
{अन्नशालां ततः कुर्वन्पुत्रास्ते धर्मबुद्धयः} %॥३३॥

\twolineshloka
{सप्तमाध्याय जपतो मुक्तिभाजोभवंस्ततः}
{षष्ठमिष्टतमं ज्ञात्वा निर्वाणार्पितदृष्टयः} %॥३४॥

॥इति श्रीपाद्मे महापुराणे पञ्चपञ्चाशत्सहस्रसंहितायामुत्तरखण्डे सतीश्वरसंवादे गीतामाहात्म्ये एकाशीत्यधिकशततमोऽध्यायः॥१८१॥

\sect{{[८]} द्व्यशीत्यधिकशततमोऽध्यायः} %6.182

\uvacha{शिव उवाच}

\twolineshloka
{अष्टमाध्यायमाहात्म्यं शृणु वक्ष्यामि पार्वति}
{यस्य श्रवणमात्रेण परां मुदमवाप्स्यसि} %॥१॥

\twolineshloka
{आमर्दकं पुरं नाम्ना विश्रुतं दक्षिणापथि}
{द्विजन्मा भावशर्मेति तत्रासीद्गणिकापतिः} %॥२॥

\twolineshloka
{खादन्मांसं पिबन्मद्यं चोरयन्साधुसम्पदः}
{रममाणः परस्त्रीभिराखेटक कुतूहली} %॥३॥

\twolineshloka
{अत्यवाहयदत्युग्रो गरीयांसं मनोरथम्}
{सुहृदां विटगोष्ठ्यां च तालीफलसुधारसम्} %॥४॥

\twolineshloka
{निपीय कण्ठपर्यन्तमजीर्णेनातिपीडितः}
{मृतः कालेन पापात्मा जातस्तालीतरुर्महान्} %॥५॥

\twolineshloka
{तस्यच्छायामुपाश्रित्य निबिडामतिशीतलाम्}
{अभूतां दम्पती कौचिद्ब्रह्मराक्षसतां गतौ} %॥६॥

\uvacha{देव्युवाच}

\twolineshloka
{किं जातीयौ किमात्मानौ किंवृत्तावित्युदीरय}
{कर्मणा केन वा देव ब्रह्मराक्षसता तयोः} %॥७॥

\uvacha{शिव उवाच}

\twolineshloka
{वेदवेदाङ्गतत्वज्ञः सर्वशास्त्रार्थकोविदः}
{सदाचारोऽभवत्कश्चिद्दिवजो नाम कुशीवलः} %॥८॥

\twolineshloka
{जाया च तस्य कुमति नामधेया दुराशया}
{स सभार्यो महादानान्याददानोऽतिलोभवान्} %॥९॥

\twolineshloka
{महिषीं कालपुरुषं हयादीननुवासरम्}
{अप्रयच्छन्द्विजातिभ्यो दानलब्धां वराटिकाम्} %॥१०॥

\twolineshloka
{कालेन दम्पती प्रेतौ ब्रह्मराक्षसरूपिणौ}
{पर्यटन्तौ महीमेतां क्षुत्तृषाकुलविग्रहौ} %॥११॥

\twolineshloka
{विशश्रमतुरागत्य मूलं तालीतरोस्ततः}
{कथमेतन्महादुःखमावयोरपगच्छति} %॥१२॥

\twolineshloka
{कथं वा जायते मुक्तिर्ब्रह्मराक्षसयोनितः}
{इति पृष्टो गृहिण्याऽसौ ब्राह्मणः समभाषत} %॥१३॥

\twolineshloka
{ब्रह्मविद्योपदेशेन विनाध्यात्मविचारणात्}
{विनाकर्मविधिज्ञानात्कथं मुच्येत सङ्कटात्} %॥१४॥

\uvacha{भार्योवाच}

\twolineshloka
{किं तद् ब्रह्म किमध्यात्मं किं कर्म पुरुषोत्तम}
{एतावदुक्ते तत्पत्न्या यदाश्चर्यमभूच्छृणु} %॥१५॥

\twolineshloka
{अष्टमाध्यायश्लोकार्द्धश्रवणात्स तरुस्तदा}
{विहाय तालीरूपं तद्बभूव द्विजसत्तमः} %॥१६॥

\twolineshloka
{सद्योज्ञानविधूतात्मा विमुक्तः पापकञ्चुकात्}
{तन्माहात्म्याद्विनिर्मुक्तौ दम्पती तौ बभूवतुः} %॥१७॥

\twolineshloka
{एतावदेव मुक्तं च दैवान्निर्गत्य तन्मुखात्}
{ततोन्तरिक्षादायान्तं क्वणत्किङ्किणिकं शुभम्} %॥१८॥

\twolineshloka
{दिवि दिव्याङ्गनावक्त्रचन्द्रमण्डलमण्डितम्}
{अप्सरोवदनाम्भोज भ्राम्यद्भ्रमरसङ्कुलम्} %॥१९॥

\twolineshloka
{निर्मथ्यमानदुग्धाब्धि वेलाडिण्डिरपाण्डुरैः}
{गङ्गातरङ्गसुभगैश्चामरैरुपशोभितम्} %॥२०॥

\twolineshloka
{गायद्गन्धर्वसुभगं नृत्यत्सुरवधूशतम्}
{दिव्यं विमानमारूढौ दम्पती जग्मतुर्दिवम्} %॥२१॥

\twolineshloka
{अत्रत्यं वृत्तमखिलमेतद्विस्मयकारकम्}
{ततो लिलेख मेधावी श्लोकार्द्धमिदमादरात्} %॥२२॥

\twolineshloka
{ययौ वाराणसीं नाम नगरीं मुक्तिदायिनीम्}
{आराधयितुमन्विच्छन्देवदेवं जनार्दनम्} %॥२३॥

\twolineshloka
{स तत्र कर्तुमारेभे तपः परमुदारधीः}
{अत्रान्तरे जगन्नाथो देवदेवो जनार्दनः} %॥२४॥

\twolineshloka
{पृष्टो दुग्धाब्धिसुतया संयोज्य करसम्पुटम्}
{निद्रापथं विहायैव स्थीयते कथ्यतामिति} %॥२५॥

\uvacha{श्रीभगवानुवाच}

\twolineshloka
{काश्यां भागीरथी तीरे तपस्यतितरां द्विजः}
{भावशर्माति मेधावी मद्भक्तिरसपूरितः} %॥२६॥

\twolineshloka
{जपन्गीताष्टमाध्यायश्लोकार्द्धं नियतेन्द्रियः}
{सन्तुष्टवानहं देवि तदीयतपसा भृशम्} %॥२७॥

\twolineshloka
{चिरं विचारयन्नेव तत्तपः सदृशं फलम्}
{दातुमुत्कण्ठितमना वर्तेयं साम्प्रतं प्रिये} %॥२८॥

\uvacha{पार्वत्युवाच}

\twolineshloka
{हरिः प्रसन्नभूतोऽपि चिन्तां प्राप यदि प्रभो}
{भावशर्मा हरेर्भक्तः प्राप्तः किं तत्फलं पुनः} %॥२९॥

\uvacha{श्रीमहादेव उवाच}

\threelineshloka
{ततः प्रसादमासाद्य प्रसन्नस्य मुरद्विषः}
{सुखमात्यन्तिकं प्राप भावशर्मा द्विजोत्तमः}
{लेभिरे पदवीं सर्वे तदीया अपि वंशजाः} %॥३०॥

\twolineshloka
{तत्कर्मवशतो ये वै सम्प्राप्ता यातनां पुरा}
{एतदेवाष्टमाध्यायमाहात्म्यं किञ्चिदेव ते} %॥३१॥

\onelineshloka
{कथितं मृगशावाक्षि द्रष्टव्यं तु सदैव च} %॥३२॥

॥इति श्रीपाद्मे महापुराणे पञ्चपञ्चाशत्सहस्रसंहितायामुत्तरखण्डे सतीश्वरसंवादे गीतामाहात्म्ये द्व्यशीत्यधिकशततमोऽध्यायः॥१८२॥

\sect{{[९]} त्र्यशीत्यधिकशततमोऽध्यायः} %6.183

\uvacha{श्रीमहादेव उवाच}

\twolineshloka
{अतः परं प्रवक्ष्यामि नवमाध्यायमादरात्}
{संशृणुष्व स्थिरीभूय तुहिनाचल कन्यके} %॥१॥

\twolineshloka
{अस्ति माहिष्मतीनाम नगरी नर्मदातटे}
{तत्रासीन्माधवो नाम द्विजन्मा स शिवो द्विजः} %॥२॥

\twolineshloka
{वेदवेदाङ्गतत्वज्ञः कालेकालेऽतिथिप्रियः}
{अर्जयित्वा बहुधनं विद्ययैव विशुद्धधीः} %॥३॥

\twolineshloka
{महान्तमध्वरं कर्तुं समारेभे कदाचन}
{आलम्भनार्थमानीतश्छागः पूजितविग्रहः} %॥४॥

\twolineshloka
{वाचमूचे हसन्नुच्चैर्जगद्विस्मयकारकः}
{किमेतैर्बहुभिर्यागैर्विधिवद्विहितैरपि} %॥५॥

\twolineshloka
{विनश्वरफलैर्जन्म जरामरणहेतुभिः}
{एतावत्यपि मे विप्र दशेयं दृश्यतामिति} %॥६॥

\twolineshloka
{छागस्यैवं वचोतीव कुतूहलपरं जनाः}
{निशम्यविस्मयं याताः कृतमण्डपवासिनः} %॥७॥

\twolineshloka
{ततो बद्धाञ्जलिपुटो द्विजातिस्तिमितेक्षणः}
{प्रणम्य श्रद्दधानस्तमपृच्छच्छागमादरात्} %॥८॥

\uvacha{द्विज उवाच}

\twolineshloka
{किं जातीयः किमात्मा त्वं किं वृत्तमिति मे वद}
{केन वा कर्मणा वासीच्छागत्वमिति कारणम्} %॥९॥

\uvacha{छाग उवाच}

\twolineshloka
{आसं पुरा द्विजातीनामन्वये चातिनिर्मले}
{आहर्ता क्रतुसङ्घानां वेदविद्याविशारदः} %॥१०॥

\twolineshloka
{एकदा मम गेहिन्या पुत्ररोगप्रशान्तये}
{छागः प्रयाचितो मत्तः चण्डिकाभक्तिनम्रया} %॥११॥

\twolineshloka
{ततो निहन्यमानस्य चण्डिकामण्डपस्थले}
{छागस्य जननी मां तु शशाप ब्रह्मवादिनी} %॥१२॥

\twolineshloka
{अशास्त्रीयाध्वना पाप मत्सुतं यज्जिघांससि}
{द्विजात्यधमतेन त्वमजा योनिमवाप्स्यसि} %॥१३॥

\twolineshloka
{ततोऽहं प्रेत्यकालेन छागोऽभूवं द्विजोत्तम}
{निस्तीर्य चानेकविधा योनिसन्तापयातनाः} %॥१४॥

\uvacha{विप्र उवाच}

\twolineshloka
{जातिस्मरत्वमप्यस्ति पशुयोनिमुपेयुषः}
{त्वदीयजन्मशुश्रूषा कुतूहलरसोन्मुखम्} %॥१५॥

\uvacha{छाग उवाच}

\twolineshloka
{मनः सर्वान्द्विजानेतानपि तत्कथयाखिलम्}
{कदाचिन्मर्कटो भूवमाहितुण्डिकशिक्षया} %॥१६॥

\twolineshloka
{क्रीडद्भिर्वीक्षितो डिम्भैर्नृत्यन्प्रतिगृहाङ्गणे}
{उदारानात्मनो दारान्विलोक्य तनयानपि} %॥१७॥

\twolineshloka
{क्रियापराङ्मुखो जातस्त्यक्तनर्तनसम्भ्रमः}
{ततो वर्तुलदण्डैश्च दुःसहैराहितुण्डिकः} %॥१८॥

\twolineshloka
{मामुच्चैस्ताडयाञ्चक्रे रुषा लोहितलोचनः}
{ततोऽहं मूर्च्छितोऽभूवं क्षरत्क्षतजसन्ततिः} %॥१९॥

\twolineshloka
{आजिघ्रन्नन्नमुदकमगमं कालधर्मताम्}
{ततोऽहमासीच्छुनकः परिभ्राम्यन्गृहे गृहे} %॥२०॥

\twolineshloka
{कुक्षिं भरिरहं मार्गे त्यक्तोच्छिष्टान्नभक्षकः}
{कदाचिदाविशं स्वान्तरात्मवेश्ममहानसम्} %॥२१॥

\twolineshloka
{बुभुक्षितो भक्षयितुं स्थाली स्थापितमोदनम्}
{जिघ्रन्भूमितलं पश्यन्दिशो दश शनैर्भयात्} %॥२२॥

\twolineshloka
{शङ्कमानो जनरवात्पार्श्वे च विलिहन्निव}
{ततः कदाचिदागत्य वीक्षितस्तनुजैर्निजैः} %॥२३॥

\twolineshloka
{जायया च जरत्याहं ताडितो लकुटादिभिः}
{ततो भग्नकटिर्यातो बहुशोणितमुद्वहन्} %॥२४॥

\twolineshloka
{निर्जगाम बहिर्गेहात्कथञ्चिन्मूर्च्छयाकुलः}
{अङ्गेषु पूतिगन्धेषु क्रिमिगर्भेषु कालतः} %॥२५॥

\twolineshloka
{ततः कदश्वतां प्राप्तः शौण्डिकस्य च वेश्मनि}
{अश्वोभवमहं विद्वन्मृतः कालक्रमादिह} %॥२६॥

\twolineshloka
{कदाचिच्चत्वरे तेन समानीतो जनाकुले}
{विक्रयाय जरालीढ पतयालुरदावलि} %॥२७॥

\twolineshloka
{जायया द्वारकायात्रां कर्तुमुद्यतयासकृत्}
{मौल्येनाल्पीयसा क्रेतुं तुरङ्गं चेष्टमानया} %॥२८॥

\twolineshloka
{जगृहेहं तया दाम्ना अल्पेन वसुना जरन्}
{गन्तुं चारभतद्वित्रैः पुत्रैरारुह्य मां समम्} %॥२९॥

\twolineshloka
{शनैः शनै सरस्तीरे मग्नोऽहं गाढकर्दमे}
{तत्राहं कुटिलग्रीवश्चापतन्कर्दमान्तरे} %॥३०॥

\twolineshloka
{ताड्यमानो मुहुः पुत्रैः लकुटोपलपाणिभिः}
{उत्थाप्यमानो बहुधा प्राणान्मोचितवानहम्} %॥३१॥

\twolineshloka
{ततो निश्चित्य मां तत्र मृतं भग्नोद्यमाः सुताः}
{आक्रुश्य मातरं दीनां प्रावृत्य निर्ययुर्गृहम्} %॥३२॥

\twolineshloka
{ततः सम्प्रेत्य बहुना कालेन छागतां गतः}
{निस्तीर्णानेकहीनोच्च योनिसन्तापयातनः} %॥३३॥

\uvacha{द्विज उवाच}

\twolineshloka
{किमनेन महाछाग दुःखजातेन नित्यशः}
{यथावदञ्जसा मह्यं सुखमात्यन्तिकं भवेत्} %॥३४॥

\uvacha{छाग उवाच}

\twolineshloka
{आश्चर्यं कथयिष्यामि पुनरन्यदपि द्विज}
{स्वस्थमापृच्छमानस्य तवास्ति यदि कौतुकम्} %॥३५॥

\twolineshloka
{अस्ति नाम्ना कुरुक्षेत्रं नगरं मोक्षदायकम्}
{सूर्यवंशोऽभवत्तत्र चन्द्रशर्मा महीपतिः} %॥३६॥

\twolineshloka
{सूर्योपरागसमये श्रद्धया परयान्वितः}
{दानं स कालपुरुषं दातुं समुपचक्रमे} %॥३७॥

\twolineshloka
{समाहूय द्विजन्मानं वेदवेदाङ्गपारगम्}
{स्नातुं पुण्योदकैः पुण्यैर्ययौ सार्द्धं पुरोधसा} %॥३८॥

\twolineshloka
{अथोच्चैः कालपुरुषो वाचमूचे हसन्निव}
{अन्येनैव प्रगृह्णन्ति क्षेत्रे चाण्वपि किञ्चन} %॥३९॥

\twolineshloka
{सूर्योपरागसमये कुरुक्षेत्राभिधे स्थले}
{दानं च कालपुरुषं जिघृक्षसि कथं द्विज} %॥४०॥

\twolineshloka
{ज्ञात्वापि निश्चितं सर्वमेतत्पातककारकम्}
{प्रवर्तसे कथं कर्तुं धनलोभान्धया धिया} %॥४१॥

\twolineshloka
{इत्थमाकर्ण्य तद्वाक्यं जगद्विस्मयकारकम्}
{किमनेन महादानभयेनेत्यवदद्द्विजः} %॥४२॥

\twolineshloka
{एवंविध महादान पातकागाधवारिधिम्}
{जानामि तरितुं सम्यगुपायमहमेव हि} %॥४३॥

\twolineshloka
{ततः स्नात्वा महीपालः परिधाय च वाससी}
{शुचिः प्रसन्नहृदयः सितमाल्यानुलेपनः} %॥४४॥

\twolineshloka
{अवलम्ब्य कराम्भोजं पार्श्ववर्तिपुरोधसः}
{समाययौ सेव्यमानः स तत्कालोचितैर्जनैः} %॥४५॥

\twolineshloka
{समागत्य च भूपालः सम्प्रादात्कालपूरुषम्}
{यथोचितेन विधिना तस्मै भक्त्या द्विजन्मने} %॥४६॥

\twolineshloka
{निर्भिद्य कालपुरुषहृदयं निर्दयोदयः}
{पापात्मा निर्ययौ कश्चिच्चाण्डालो रक्तलोचनः} %॥४७॥

\twolineshloka
{किञ्च प्रापितकालस्य परनिन्दारसोत्सवे}
{निन्दा चाण्डालिका देहपार्श्वमागाद्द्विजन्मनः} %॥४८॥

\twolineshloka
{एतच्चाण्डालयुगलं निर्गत्यारुणलोचनम्}
{ततः सञ्चरितं चक्रे प्रसह्याङ्गे द्विजन्मनः} %॥४९॥

\twolineshloka
{गीतानां नवमाध्यायं जपन्नेव हृदिस्थितः}
{कम्पमानं द्विजं किञ्चित्तूष्णीं पश्यति भूपतौ} %॥५०॥

\twolineshloka
{अन्तर्निद्राणगोविन्दं कम्पमानमिवाम्बुधिम्}
{मरुदान्दोलनैर्विद्वान्द्विजन्मा पापसंश्रयम्} %॥५१॥

\twolineshloka
{ततो गताक्षरोद्भूतैर्वैष्णवैः परिपीडितम्}
{पलायमानं चाण्डालयुगलं निष्फलोद्यमम्} %॥५२॥

\twolineshloka
{तन्निश्चक्राम वेगेन द्विजातेः पार्श्ववर्ति यत्}
{शरीरे वर्त्तमानं च परनिन्दारसोत्सवे} %॥५३॥

\twolineshloka
{इत्थं कलितवृत्तान्तः प्रत्यक्षं क्षितिवल्लभः}
{पर्यपृच्छद्द्विजन्मानं विस्मयस्मेरलोचनः} %॥५४॥

\twolineshloka
{कथमापदियं घोरा निस्तीर्णा महती त्वया}
{कं मन्त्रं जपता विप्र कं वासं स्मरतासुरम्} %॥५५॥

\twolineshloka
{कः पुमान्का च सा योषित्कथमेतावुपस्थितौ}
{कथं च शान्तिमापन्नावित्युदीरय मे द्विज} %॥५६॥

\uvacha{द्विज उवाच}

\twolineshloka
{चाण्डालमूर्तिमासाद्य मूर्तिकिल्बिषमुल्बणम्}
{योषिन्मूर्त्तिमयीं निन्दाद्वयमेतदवैम्यहम्} %॥५७॥

\twolineshloka
{गीताया नवमाध्यायमन्त्रमाला मया स्मृता}
{तन्माहात्म्यमिदं सर्वं त्वमवेहि महीपते} %॥५८॥

\twolineshloka
{गीताया नवमाध्यायं जपामि प्रत्यहं नृप}
{निस्तीर्णाश्चापदस्तेन कुप्रतिग्रहसम्भवाः} %॥५९॥

\twolineshloka
{अभ्यस्य नवामाध्यायं राजा तस्माद्द्विजन्मनः}
{तावुभावपि लेभाते परां निर्वृतिमुत्तमाम्} %॥६०॥

॥इति श्रीपाद्मे महापुराणे पञ्चपञ्चाशत्सहस्रसंहितायामुत्तरखण्डे सतीश्वरसंवादे गीतामाहात्म्ये त्र्यशीत्यधिकशततमोऽध्यायः॥१८३॥

\sect{{[१०]} चतुरशीत्यधिकशततमोऽध्यायः} %6.184

\uvacha{देव्युवाच}

\twolineshloka
{सर्वज्ञ सर्वचैतन्य सर्वेश्वर गिरां गुरो}
{धन्यास्मि शिवमान्येन दृश्यमानेन यत्त्वया} %॥१॥

\twolineshloka
{निरूपितमिदं पुण्यं नवमाध्यायवैभवम्}
{अनेकविस्मयस्वादु कथानकमयं मधु} %॥२॥

\twolineshloka
{शृण्वन्त्या मम देवेश न तृप्तिर्जातु जायते}
{अकुण्ठाश्रवणोत्कण्ठा वर्द्धते वृषभध्वज} %॥३॥

\twolineshloka
{आसीन्ममहिमाम्भोधे गीतानां श्रुतिजीवितम्}
{तत्रापि दशमाध्यायं प्रधानं मुनयो जगुः} %॥४॥

\onelineshloka*
{तमुद्दिश्य महाध्यायमभिधेहि कथानकम्}

\uvacha{शिव उवाच}

\onelineshloka
{शृणु सुश्रोणि निश्रेणीं स्वर्गदुर्गस्य दुर्ल्लभाम्} %॥५॥

\twolineshloka
{सीमामिव प्रभावानां पावनीं परमां कथाम्}
{आसीत्काशीपुरे विप्रः पुण्यकीर्त्तिपरायणः} %॥६॥

\twolineshloka
{प्रशान्तचेता निर्मुक्त हिंसाकार्कश्यसाहसः}
{निवृत्तिनिरतो नित्यं जितेन्द्रियतया तथा} %॥७॥

\twolineshloka
{धीरधीरिति विख्यातो नन्दीव मयि भक्तिमान्}
{निस्तीर्णनिगमाम्भोधिः सर्वशास्त्रार्थकोविदः} %॥८॥

\twolineshloka
{तस्य ध्यानपराधीन चेतसः प्रतिगच्छतः}
{अन्तरात्मनि निर्मग्न मनसस्तत्त्वचक्षुषः} %॥९॥

\twolineshloka
{करावलम्बनं तस्य धावन्प्रीत्या ददाम्यहम्}
{कदाचन चमत्कारकारकं विमना मुनिः} %॥१०॥

\twolineshloka
{आचान्तः किञ्च नासाग्रपरमानन्दमेदुराम्}
{दृशमासाद्य निद्राणकरणोयमिवास्थितः} %॥११॥

\twolineshloka
{उपाधास्य विशालाक्षि विशालां द्वारदेहलीम्}
{अशेत निशि निःशङ्कं तावल्लम्बेक्षणः क्षणम्} %॥१२॥

\twolineshloka
{मामपृच्छद्भृङ्गिरिटिः प्रणम्यपादपङ्कजम्}
{अनेन विधिना केन विहितं तव दर्शनम्} %॥१३॥

\twolineshloka
{तपस्तप्तं हुतं जप्तं किमनेन महात्मना}
{दत्ते प्रतिपदं देवो यस्य हस्तावलम्बनम्} %॥१४॥

\twolineshloka
{अयं न लभते गन्तुं कस्मादस्मात्पुराद्बहिः}
{यदृच्छया यदा काशी सीमामुल्लङ्घ्य गच्छति} %॥१५॥

\twolineshloka
{न पश्यति तदा सर्वान्पार्श्वस्थानपि देहिनः}
{अत्र हेतुमहं ज्ञातुमिच्छामि स्वामिभाषिताम्} %॥१६॥

\twolineshloka
{अनुग्राह्योस्मि चेद्वक्तुं युक्तं चेत्तदुदीरय}
{भृङ्गिरिटेरिमं प्रश्नं समाकर्ण्याहमूचिवान्} %॥१७॥

\twolineshloka
{कदाचिदासं कैलासं पार्श्वे पुन्नागकानने}
{रणत्खेचरसुश्रोणि पूर्णस्तबककानने} %॥१८॥

\twolineshloka
{कलकण्ठकुलालाप कल्लोलितदिगन्तरे}
{गरुत्मदादिदात्यूह समूहस्वरसङ्कुले} %॥१९॥

\twolineshloka
{भ्रमद्दारुघटीयन्त्र प्रोल्लसद्बिन्दुदन्तुरे}
{प्रबुद्धसारणिप्रान्त कदलीकन्दलालसे} %॥२०॥

\twolineshloka
{कस्तूरीहरिणोपेते किन्नरस्वरमोहिते}
{रोमन्थमन्थरापाङ्गैर्मृगैः क्वापि निषेविते} %॥२१॥

\twolineshloka
{हंसैः कीरेषु पाण्डित्यं कुर्वाणैः सङ्कुले शुकैः}
{निर्हृदिक्वणिनीरन्ध्र समीरणविलोडिते} %॥२२॥

\twolineshloka
{माधवीपुष्पनिर्यासशीधुक्षीबमधुव्रते}
{उन्मीलित्त्रिवलीपुष्पगुच्छसौरभनिर्भरे} %॥२३॥

\twolineshloka
{प्रोत्फुल्लबकुलामोदमदमन्थरषट्पदे}
{सोमादुद्भूतपीयूषक्षालितक्षितिमण्डले} %॥२४॥

\twolineshloka
{अध्यास्य वेदिकामेकामहं क्षणमवस्थितः}
{उद्दण्डशाखिसङ्घट्टस्फुटन्मन्थामुखोत्करैः} %॥२५॥

\twolineshloka
{प्रकम्पिता चलच्छायो ववौ चण्डसमीरणः}
{पश्चादभून्महाघोषो निर्घोषित दरीतटः} %॥२६॥

\twolineshloka
{अवातरत्ततः कश्चित्पक्षी गगनगह्वरात्}
{शारदानरिदच्छायः कज्जलानामिवोच्चयः} %॥२७॥

\twolineshloka
{तमसामिव सङ्घातः पक्षच्छेदीव पर्वतः}
{अवष्टभ्य क्षितिं पद्मां पक्षी मां प्रणनाम सः} %॥२८॥

\twolineshloka
{आनीय पद्ममम्लानमसौ मत्पादयोर्व्यधात्}
{अथासौ स्पष्टया वाचा पक्षी स्तोत्रमुदीरयत्} %॥२९॥

\twolineshloka
{जयदेव चिदानन्द सुधासिन्धो जगत्पते}
{सदा सद्भावना सङ्ग कल्लोलानन्तविग्रह} %॥३०॥

\twolineshloka
{अद्वैतवासना मत्या मलत्रयविवर्जित}
{जितेन्द्रिय पराधीन समाधिप्राप्यविग्रह} %॥३१॥

\twolineshloka
{निरुपाधिविनिर्मुक्त निराकार निरामय}
{निःसीम निरहङ्कार निरावरणनिर्गुण} %॥३२॥

\twolineshloka
{शरणागतसन्त्राण प्रवीण चरणाम्बुज}
{भीममाल महाव्याल ज्वालादग्धमनोभव} %॥३३॥

\twolineshloka
{कुठारभिन्नदैत्येन्द्र गण्डूषित महाविभो}
{त्रिपुरप्रमदाभाल सिन्दूरोद्धूलिमार्जन} %॥३४॥

\twolineshloka
{कात्यायनीकुचाम्भोज वरकुङ्कुमचर्चित}
{नमः प्रमाणदूराय नमः प्रमतिरूपिणे} %॥३५॥

\twolineshloka
{नमश्चैतन्यनाथाय नमस्त्रैलोक्यरूपिणे}
{वन्दे तव पदाम्भोजं योगिप्रवरचुम्बितम्} %॥३६॥

\twolineshloka
{अपारभवपाथोऽधि पारावतरणाद्भुतम्}
{वाचस्पतिरपि स्तोत्रे भवतो न प्रगल्भते} %॥३७॥

\twolineshloka
{सहस्रवदनस्यापि फणीन्द्रस्य न चातुरी}
{त्वद्वर्णने महादेव कोऽहम्मल्पमतिः खगः} %॥३८॥

\twolineshloka
{स्तोत्रमेतत्समाकर्ण्य कृतं तेन पतत्रिणा}
{तमवोचमहं कोऽसि कुतस्त्योसि विहङ्गम} %॥३९॥

\twolineshloka
{हंसेन सदृशः कायो वर्णः काकेन सन्निभः}
{प्रयोजनं कमुद्दिश्य प्राप्तोसीह तदुच्यताम्} %॥४०॥

\twolineshloka
{इति पक्षी मया पृष्टः प्रश्रयानतकन्धरः}
{जगाद श्लक्ष्णया वाचा पक्षीवाक्यविदांवरः} %॥४१॥

\twolineshloka
{देवेश धूर्जटे विद्धि मां मरालं स्वयम्भुवः}
{कर्मणा येन मे कार्ष्ण्यं जातमाधुनिकं विभो} %॥४२॥

\twolineshloka
{तदाकर्णय सर्वज्ञ पृष्टं यदि तदुच्यते}
{मानसात्सरसः पृथ्वीं यातः प्राप्तोस्मि सङ्कटम्} %॥४३॥

\twolineshloka
{सौराष्ट्रनगरादारात्सरसि स्फुटदम्बुजे}
{बालेन्दुखण्डधवलान्मृणालकवलानहम्} %॥४४॥

\twolineshloka
{आदाय बलमाश्रित्य निरगां गगनं द्रुतम्}
{विहायसस्ततस्तस्मादकस्मादपतं भुवि} %॥४५॥

\twolineshloka
{अथ मोहपरीतात्मा सर्वथा विकलेन्द्रियः}
{वेपमानवपुर्मोहात्स्पृष्टः शीतैः समीरणैः} %॥४६॥

\twolineshloka
{प्रबुद्धः पतने हेतुमपश्यन्नात्मनस्तदा}
{अहो किमेतदापन्नमद्य पातः कथं मम} %॥४७॥

\twolineshloka
{कालिमा येन कायेस्मिन्पक्वकर्पूरपाण्डुरे}
{इत्यहं विस्मयाविष्टो यावत्कुर्वे विचारणम्} %॥४८॥

\twolineshloka
{तावदम्बुरुहाद्वाणीमश्रौषमहमीदृशीम्}
{उत्तिष्ठ हंस वक्ष्यामि कारणं पातकार्ष्ण्ययोः} %॥४९॥

\twolineshloka
{अथोत्थाय समागत्य मया मध्ये सरोवरे}
{दृष्टराजीविनीरम्या राजीवैः पञ्चभिर्युता} %॥५०॥

\twolineshloka
{कारणं प्रष्टुमारेभे कार्ष्ण्यस्य पतनस्य च}
{अथ तत्र घनश्यामान्स्वर्णवर्णाम्बरावृतान्} %॥५१॥

\twolineshloka
{चतुर्भुजान्गदाशङ्खचक्रपङ्केरुहायुधान्}
{किरीटहारकेयूरकुण्डलद्युतिचित्रितान्} %॥५२॥

\twolineshloka
{अद्राक्षमन्तरिक्षस्थान्पुरुषानयुतानि षट्}
{नत्वा प्रदक्षिणीकृत्य पञ्चपद्मां सरोजिनीम्} %॥५३॥

\onelineshloka*
{आत्मीयं पातमारभ्य पृष्टं तदखिलं मया}

\uvacha{पद्मिन्युवाच}

\onelineshloka
{कलहंस गतोऽसि त्वं मां विलङ्घ्य विहायसा} %॥५४॥

\twolineshloka
{तेन पातकयोगेन पतितोऽसि महीतले}
{तेनैव कालिमा काये पक्षिसत्तम लक्ष्यते} %॥५५॥

\twolineshloka
{भवन्तं पतितं वीक्ष्य कृपापूर्णेन चेतसा}
{मध्यमेनामुनाब्जेन वदता जातसौरभम्} %॥५६॥

\twolineshloka
{आघ्राय षट्पदाः षष्टिसहस्राणि दिवं ययुः}
{एते ये भवतां दृष्टाः नीलोत्पलसमत्विषः} %॥५७॥

\twolineshloka
{सर्वे ते सप्तमेऽतीते जन्मन्यासन्मुनेः सुताः}
{अस्यैव सरसस्तीरे तेपुस्ते परमं तपः} %॥५८॥

\twolineshloka
{कदाचित्कामिनी काचिच्चम्पकस्तबकस्तनी}
{चलापाङ्गकालाकान्ततरङ्गित रसालिनी} %॥५९॥

\twolineshloka
{नासा मुक्ताफलज्योत्स्ना चुम्बितस्मित दीधितिः}
{वीणां विन्यस्य कुचयोर्वनेऽस्मिन्मधुरं जगौ} %॥६०॥

\twolineshloka
{गायन्त्या स्वरमाकर्ण्य ब्राह्मणा हरिणा इव}
{तां समागत्य ते सर्वे सममेव व्यलोकयन्} %॥६१॥

\twolineshloka
{मया दृष्टा ममैवेयमित्यूचुस्ते परस्परम्}
{मुष्टामुष्टि ततस्तेषां भ्रातॄणामभवद्रणः} %॥६२॥

\twolineshloka
{अन्योन्यमुष्टिनिष्पिष्ट वक्षसस्त्यक्तजीविताः}
{ते भुक्त्वा निरयान्घोरान्बभूवुः सारसा भुवि} %॥६३॥

\twolineshloka
{तदा ते श्वापदान्जघ्नुर्दग्धावन्येन वह्निना}
{ततो मातङ्गतामेत्य पथिपान्थानघातयन्} %॥६४॥

\twolineshloka
{वने विषोदकं पीत्वा ते ययुर्यममन्दिरम्}
{खरोष्ट्रकपिमार्जार जन्मान्यासाद्य च क्रमात्} %॥६५॥

\twolineshloka
{ततो मधुव्रता जाता वर्तन्तेऽत्र सरोवरे}
{अद्य मे गन्धमाघ्राय प्रापुस्ते वैष्णवं पदम्} %॥६६॥

\twolineshloka
{शृणु पक्षीन्द्र वक्ष्यामि येन मय्यस्ति वैभवम्}
{एतस्माज्जन्मनः पूर्वे तृतीये जन्मनि क्षितौ} %॥६७॥

\twolineshloka
{सरोजवदनानाम द्विजातेः कन्यकाऽभवम्}
{पातिव्रत्यैकनिरता गुरुशुश्रूषणे रता} %॥६८॥

\twolineshloka
{कदाचित्सारिकामेकां पाठयन्त्याविलम्बितम्}
{सारिका भव पापे त्वं पत्या शप्तास्मि कुप्यता} %॥६९॥

\twolineshloka
{प्रेत्य सारित्वमासाद्य पातिव्रत्यप्रसादतः}
{मुनीनामेव सदने कन्या काचित्पुपोष माम्} %॥७०॥

\twolineshloka
{गीतानां दशमाध्यायं विभूतिरिति विश्रुतम्}
{प्रातः पठति विप्रोसावश्रौषं तमघापहम्} %॥७१॥

\twolineshloka
{कालेन सारिकादेहमहं हित्वा विहङ्गम}
{दशमाध्यायमाहात्म्यादप्सराश्चाभवं दिवि} %॥७२॥

\twolineshloka
{पद्मावतीति विख्याता पद्माया दयिता सुखी}
{कदाचन मया यान्त्या विमानेन विहायसा} %॥७३॥

\twolineshloka
{एतत्सरोवरं रम्यं विलोक्य विमलाम्बुजम्}
{अवतीर्य जलक्रीडायावदारभ्यते मया} %॥७४॥

\twolineshloka
{दुर्वासास्तावदायातो विवस्त्रा तेन वीक्षिता}
{तद्भयात्पद्मिनीरूपं धृतमेतन्मया स्वयम्} %॥७५॥

\twolineshloka
{पद्भ्यां पद्मद्वयं चैव द्वयं हस्तद्वयेन च}
{मुखेन पञ्चमाम्भोजमिति पञ्चाम्बुजास्म्यहम्} %॥७६॥

\twolineshloka
{दृष्टा तेन मुनीन्द्रेण कोपज्वलित चक्षुषा}
{अनेनैव स्वरूपेण तिष्ठ पापे शतं समाः} %॥७७॥

\twolineshloka
{इति शापं समुत्सृज्य स चैवान्तर्दधे क्षणात्}
{विभूत्यध्यायमाहात्म्याद्वाणी मे न विलीयते} %॥७८॥

\twolineshloka
{मद्विलङ्घनमात्रेण पतितोसि महीतले}
{अद्य शापनिवृत्तिर्मे तिष्ठतस्ते खगोत्तम} %॥७९॥

\twolineshloka
{निशामय मया गीयमानमध्यायमुत्तमम्}
{यस्याकर्णनमात्रेण त्वमद्यैव विमोक्ष्यसे} %॥८०॥

\twolineshloka
{इत्यसौ दशमाध्यायं पपाठ श्लक्ष्णया गिरा}
{तमाकर्ण्य तया दत्तमादाय च सरोरुहम्} %॥८१॥

\twolineshloka
{मया समर्पितं तुभ्यं पद्मिन्या पद्ममुत्तमम्}
{इत्युक्त्वा स जहौ देहं तदद्भुतमिवाभवत्} %॥८२॥

\uvacha{भृङ्गिरिटिरुवाच}

\twolineshloka
{पुरातने भवे कोऽयं ब्रह्म हंसोऽभवत्कथम्}
{तवाग्रतः कुतो हेतोरुत्ससर्ज कलेवरम्} %॥८३॥

\twolineshloka
{इति भृङ्गिरिटिर्वाक्यमाकर्ण्याहं तदाब्रुवम्}
{द्विजवेश्मनि पूर्वस्मिन्जन्मन्ययमजायत} %॥८४॥

\twolineshloka
{सुतपा इति विख्यातो ब्रह्मचारी जितेन्द्रियः}
{वसन्गुरुकुले कुर्वन्वेदाध्ययनमन्वहम्} %॥८५॥

\twolineshloka
{गुरुशुश्रूषणं सम्यग्विदधाति च भक्तितः}
{शयानस्य गुरोः शय्यां निद्रितः सपदा स्पृशत्} %॥८६॥

\twolineshloka
{तेन पापेन तिर्यक्त्वमयं स्वर्गेऽपि लब्धवान्}
{पद्मयोनिमरालानां मध्ये जातस्ततो द्विजः} %॥८७॥

\twolineshloka
{अस्मिञ्जन्मन्यमुष्येह पुरास्मल्लोकनावधि}
{गीतानां दशमाध्यायं नलिन्या कथितं ततः} %॥८८॥

\twolineshloka
{आकर्ण्य विहगो लेभे ब्रह्मज्ञानमनुत्तमम्}
{सोऽयं विप्रकुले जातो दशमाध्यायवैभवात्} %॥८९॥

\twolineshloka
{जन्माभ्यासवशादस्य शिशोरपि मुखाम्बुजात्}
{गीतानां दशमाध्यायः समुल्लसति सर्वदा} %॥९०॥

\twolineshloka
{तदर्थपरिणामेन सर्वभूतेष्ववस्थितम्}
{शङ्खचक्रधरं देवमयं पश्यति सर्वदा} %॥९१॥

\twolineshloka
{यस्मिन्यस्मिन्यदैवास्य दृष्टिः स्निग्धा शरीरिणः}
{स स मुक्तो भवेत्सर्वः सुरापो ब्रह्महाऽपि वा} %॥९२॥

\twolineshloka
{तद्विज्ञाय मया विप्रः परमात्मस्वरूपिणा}
{इदं नगरमानीतो मुक्तिक्षेत्रं स्वभावतः} %॥९३॥

\twolineshloka
{अत्रत्यानां मनुष्याणां मुक्तिः करतले स्थिता}
{तेनास्य दृष्टिपातेन विशेषोऽन्यो न जायते} %॥९४॥

\twolineshloka
{न ददामि बहिर्गन्तुमहमस्य पुरा कृतम्}
{दशमाध्यायमहात्म्यात्तत्वज्ञानं सुदुर्लभम्} %॥९५॥

\twolineshloka
{लब्धमेतेन मुनिना जीवन्मुक्तिरियं तथा}
{तेनास्य चलतो हस्तं ददामि पथि गच्छतः} %॥९६॥

\twolineshloka
{दशमाध्यायमहिमा भृङ्गिरिटे महानयम्}
{इति भृङ्गिरिटेरग्रे कथितं यत्कथानकम्} %॥९७॥

\twolineshloka
{तवेदमत्र कथितं सर्वपापप्रणाशनम्}
{नरो वाप्यथवा नारी यापि कोऽपि च वा पुनः} %॥९८॥

\onelineshloka
{अस्य श्रवणमात्रेण सर्वाश्रमफलं लभेत्} %॥९९॥

॥इति श्रीपाद्मे महापुराणे पञ्चपञ्चाशत्सहस्रसंहितायामुत्तरखण्डे सतीश्वरसंवादे गीतामाहात्म्ये चतुरशीत्यधिकशततमोध्यायः॥१८४॥

\sect{{[११]} पञ्चाशीत्यधिकशततमोऽध्यायः} %6.185

\uvacha{देव्युवाच}

\twolineshloka
{इतिहासोऽयमीशान श्रेयसां साधनं परम्}
{आकर्ण्य करुणापूर्ण ममकाङ्क्षा प्रवर्त्तते} %॥१॥

\twolineshloka
{एकादशस्य माहात्म्यमध्यायस्य कथाश्रयम्}
{व्यावर्णय विरूपाक्ष वक्तॄणां प्रथम प्रभो} %॥२॥

\uvacha{ईश्वर उवाच}

\twolineshloka
{आकर्णय कथां कान्ते गीतावर्णनसंश्रयाम्}
{विश्वरूपाभिधानस्य माहात्म्यमपि पावनम्} %॥३॥

\twolineshloka
{अध्यायस्य विशालाक्षि वक्तुं तावन्न शक्यते}
{सहस्राणि कथाः सन्ति तत्रैका कथ्यते मया} %॥४॥

\twolineshloka
{प्रणीतायास्तटे नद्या मेघङ्करमिति श्रुतम्}
{नगरं गरिमाधार तुङ्गप्राकारगोपुरम्} %॥५॥

\twolineshloka
{विशालाश्रमशालासु स्वर्णस्तम्भविभूषितम्}
{श्रीमद्भिः सुखिभिः शान्तैः सदाचारैर्जितेन्द्रियैः} %॥६॥

\twolineshloka
{अधिष्ठितं जवैश्चारुशृङ्गाटकमनोहरम्}
{मणिस्तम्भस्फुरत्स्वर्णापणचत्वरशोभितम्} %॥७॥

\twolineshloka
{पताकाकिङ्किणीक्वाण कदम्बक कलस्वरम्}
{वेदाध्ययननिर्घोषवाचालित दिगन्तरम्} %॥८॥

\twolineshloka
{तूर्यसङ्घोषणाकीर्ण विशालव्योममण्डलम्}
{पताकापल्लोद्भूतवातनिर्जितविग्रहम्} %॥९॥

\twolineshloka
{राजामार्गवरद्वारनारीमञ्जीरसिञ्जितैः}
{वल्लकीवेणुकैर्गीतेर्भाति वाजीन्द्र हेषितैः} %॥१०॥

\twolineshloka
{प्रेक्षमाणमिवाभीक्ष्णं दिक्पालानां पुरैः समम्}
{आस्ते जगत्पतिर्यत्र शार्ङ्गपाणिर्विराजितः} %॥११॥

\twolineshloka
{मूर्तिमत्परमं ब्रह्म जगल्लोचनजीवितम्}
{लक्ष्मीनयनराजीव पूजिताकारगौरवः} %॥१२॥

\twolineshloka
{त्रिविक्रमवपुर्मेघश्यामलः कोमलद्युति}
{श्रीवत्सवक्षाराजीव वनमालाविभूषितः} %॥१३॥

\twolineshloka
{अनेकभूषणोपेतः सरत्न इव वारिधिः}
{चलत्सौदामिनीदाम सान्द्रमेघसमद्युतिः} %॥१४॥

\twolineshloka
{तस्यास्ते मुकुटे साक्षात्शार्ङ्गपाणिः परः पुमान्}
{तं दृष्ट्वा मुच्यते जन्तुर्जन्मसंसारबन्धनात्} %॥१५॥

\twolineshloka
{यस्मिन्पुरे महातीर्थं विद्यते मेखलाभिधम्}
{यत्र स्नात्वा नरैर्नित्यं प्राप्यते वैष्णवं पदम्} %॥१६॥

\twolineshloka
{तत्र वीक्ष्य जगन्नाथं नरसिंहं कृपार्णवम्}
{सप्तजन्मार्जिताद्धोरान्मुच्यते दुष्कृतान्नरः} %॥१७॥

\threelineshloka
{मेखलायां गणाधीशं विलोकयति यो नरः}
{स निस्तरति विघ्नानि दुस्तराण्यपि सर्वदा}
{ब्रह्मचर्यपरो दान्तो निर्ममो निरहङ्कृतिः} %॥१८॥

\twolineshloka
{तस्मिन्मेघङ्करे कश्चिदभूद्ब्राह्मणसत्तमः}
{सुनन्द इति विख्यातो वेदशास्त्रविशारदः} %॥१९॥

\twolineshloka
{वशीकृतेन्न्द्रियग्रामो वासुदेवपरायणः}
{देवस्य शार्ङ्गिणः पार्श्वे गीताध्यायमिमं प्रिये} %॥२०॥

\twolineshloka
{एकादशं पठत्येष विश्वरूपप्रदर्शनम्}
{अध्यायस्य प्रभावेन ब्रह्मज्ञानमवाप सः} %॥२१॥

\twolineshloka
{परमानन्दसन्दोह श्लाघ्यसंवित्समाधिना}
{प्रत्यङ्मुखेन्द्रियतयानिश्चलां स्थितिमीयुषा} %॥२२॥

\twolineshloka
{सततं स्थीयते तेन जीवन्मुक्तेन योगिना}
{एकदा स महायोगी सिंहराशिस्थिते गुरौ} %॥२३॥

\twolineshloka
{गोदावरीतीर्थयात्रां विधातुमुपचक्रमे}
{प्रथमेऽह्नि समागत्य विरजं तीर्थमुत्तमम्} %॥२४॥

\twolineshloka
{नाभिमारभ्य तीर्थेषु स समभ्यर्च्य देवताम्}
{मर्ज्जन्मज्जन्जगद्धात्रीं कमलां स व्यलोकयत्} %॥२५॥

\twolineshloka
{तां सम्पूज्य महामायां सर्वकामफलप्रदाम्}
{तारातीर्थे ततः स्नात्वा कपिलासङ्गमे ततः} %॥२६॥

\twolineshloka
{अष्टतीर्थमसौ चक्रे विधाय पितृतर्पणम्}
{कुमारीशं शिवं नत्वा कपिलाद्वारमाययौ} %॥२७॥

\twolineshloka
{तत्र निर्मज्ज्य निर्धुतप्राग्जन्मान्तरदुष्कृतः}
{सम्पूज्य नत्वा श्रुत्वा च देवं वै मधुसूदनम्} %॥२८॥

\threelineshloka
{उषित्वा तत्र तां रात्रिं प्रागात्प्रातः सह द्विजैः}
{नरसिंहवने तत्र तीर्थे रामस्य दीर्घिका}
{प्रह्लादपूजितः साक्षादास्ते यत्र नृकेसरी} %॥२९॥

\twolineshloka
{तं दृष्ट्वा देवदेवेशं पूजयित्वा तु भक्तितः}
{तत्र तं दिवसं नीत्वा स ययावम्बिकापुरम्} %॥३०॥

\twolineshloka
{अनुग्रहेण भक्तानामम्बिका तत्र तिष्ठति}
{पूरयन्ति मनुष्याणां वाञ्च्छितान्यखिलान्यपि} %॥३१॥

\twolineshloka
{पूजयित्वाम्बिकां भक्त्या पुष्पगन्धानुलेपनैः}
{उपहारैश्च विविधैः स्तौत्रैः प्रणमनैरपि} %॥३२॥

\twolineshloka
{विप्रस्तस्मात्पुरा प्राप्तः कण्ठस्थानाभिधं पुरम्}
{यत्रास्ते परमा शक्तिर्महालक्ष्मीर्महाद्युतिः} %॥३३॥

\twolineshloka
{तामवेक्ष्य सुधाभानुभास्करद्युतिमण्डलाम्}
{संसारतापविच्छेदपद्मपीयूषवाहिनीम्} %॥३४॥

\twolineshloka
{योगिराजहृदम्भोज राजहंसनिषेविताम्}
{अनाहतमहानादमयीमद्वयरूपिणीम्} %॥३५॥

\twolineshloka
{महालक्ष्मीं भगवतीं वाञ्छितार्थप्रदायिनीम्}
{आराध्य भक्तिभावेन चेतसा स मुनीश्वरः} %॥३६॥

\twolineshloka
{विवाहमण्डपं प्राप पुरं विप्रैः समन्वितः}
{पुरे तत्र प्रतिगृहं वासस्थानमयाचत} %॥३७॥

\twolineshloka
{न लेभे वसतिं स्थातुं गेहे कस्मिन्नपि द्विजः}
{दर्शितं ग्रामपालेन विशालं वासमन्दिरम्} %॥३८॥

\twolineshloka
{प्रविश्य वशतिं चक्रे ब्राह्मणः सङ्गिभिः सह}
{ततः प्रभाते विमले सुनन्दोऽसौ द्विजोत्तमः} %॥३९॥

\twolineshloka
{बहिरालोकयाञ्चक्रे वासगेहान्निजं वपुः}
{अध्वन्यानखिलान्यत्र जातान्क्वापि यदृच्छया} %॥४०॥

\twolineshloka
{गम्यमानः समायान्तं ग्रामपालो ददर्श सः}
{तं बभाषे ग्रामपाल आयुष्मानसि सर्वशः} %॥४१॥

\twolineshloka
{भागधेयवतां पुंसां पुण्यः पुण्यवतामसि}
{प्रभावो विद्यते वत्स कोपि लोकोत्तरस्त्वयि} %॥४२॥

\twolineshloka
{क्व प्रयाताः सहायास्ते कथं तत्सदनाद्बहिः}
{तत्पश्य मुनिशार्दूल कथयामि तवाग्रतः} %॥४३॥

\twolineshloka
{किन्तु नान्यं त्वया तुल्यं पश्यामीह तपस्विनम्}
{किं जानासि महामन्त्रं कां विद्यामवलम्बसे} %॥४४॥

\twolineshloka
{कस्य देवस्य कारुण्याच्छक्तिर्लोकोत्तरा त्वयि}
{तत्कारुण्यवशात्तिष्ठ ग्रामेऽस्मिन्ब्राह्मणोत्तम} %॥४५॥

\twolineshloka
{शुश्रूषामखिलामेव भगवंस्ते करोम्यहम्}
{इति संवासयामास तस्मिन्ग्रामे मुनीश्वरम्} %॥४६॥

\twolineshloka
{परिचर्या च तस्यासौ भक्त्या चक्रे दिवानिशम्}
{दिवसेषु प्रयातेषु सप्ताष्टसु समेयिवान्} %॥४७॥

\twolineshloka
{प्रातरागत्य तस्याग्रे रुरोद भृशदुःखितः}
{अद्य मे भाग्यहीनस्य गुणवान्भक्तिमान्सुतः} %॥४८॥

\twolineshloka
{जाज्वल्यमानदंष्ट्रेण भक्षितो निशि रक्षसा}
{इत्येवं रक्षकेणोक्तं तं पप्रच्छ स संयमी} %॥४९॥

\onelineshloka*
{क्वास्ते स राक्षसः पुत्रो भक्षितस्ते कथं वद}

\uvacha{ग्रामपाल उवाच}

\onelineshloka
{वर्तते नगरे घोरः पुरुषादो निशाचरः} %॥५०॥

\twolineshloka
{स खादति नरानेत्य नित्यं नगरगोचरान्}
{स सर्वैर्नागरैरत्र प्रार्थितः पुरुषैः पुरा} %॥५१॥

\twolineshloka
{रक्ष राक्षस नः सर्वान्ग्रासं ते कल्पयामहे}
{पथिका निशि निद्रन्ति ये च तान्भुङ्क्ष्व राक्षस} %॥५२॥

\twolineshloka
{एतस्मिन्सदने पान्थान्ग्रामपालप्रवेशितान्}
{आहारं कल्पयाञ्चक्रुरात्मीयप्राणगुप्तये} %॥५३॥

\twolineshloka
{भवान्सुप्तोगृहेऽमुष्मिन्नधन्यैः संयुतः परैः}
{ते ग्रस्ताः किल चानेन न त्वं मुक्तोसि द्विजोत्तम} %॥५४॥

\twolineshloka
{प्रभावं भवतो वेत्ति भवानेव द्विजोत्तम}
{मदीय तनयस्याद्य मित्रमेकमुपागतम्} %॥५५॥

\twolineshloka
{अजानता मया सोऽपि तनयस्य प्रियः सखा}
{अन्यैः पान्थजनैः सार्द्धं तस्मिन्गेहे प्रवेशितः} %॥५६॥

\twolineshloka
{श्रुत्वा तत्र प्रविष्टं तं निशीथे तनयो मम}
{तमानेतुं गतः सोऽपि भक्षितस्तेन रक्षसा} %॥५७॥

\twolineshloka
{दुःखितेन मया प्रोक्तः प्रातः स पिशिताशनः}
{ममापि पुत्रो दुष्टात्मन्भवता निशि भक्षितः} %॥५८॥

\twolineshloka
{भवज्जठरनिर्मग्नः सुतोऽसौ येन जीवति}
{अस्ति चैवमुपायस्ते ब्रूहि मे त्वं निशाचर} %॥५९॥

\uvacha{राक्षस उवाच}

\twolineshloka
{अन्तःप्रविष्टा त्वत्पुत्रमज्ञात्वाहमभक्षयम्}
{अजानन्भक्षितः पान्थैः सहितोऽसौ सुतस्तव} %॥६०॥

\twolineshloka
{यथा जीवति मे कुक्षौ यथा भवति रक्षितः}
{तथा विहितमप्यस्ति दैवेन परमेष्ठिना} %॥६१॥

\twolineshloka
{गीतैकादशमध्यायं यः पठत्यनिशं द्विजः}
{तत्प्रभावेन मे मुक्तिर्मृतानां पुनरुद्भवः} %॥६२॥

\uvacha{ग्रामपाल उवाच}

\onelineshloka
{कथमेकादशाध्यायसामर्थ्यमिदमद्भुतम्} %॥६३॥

\onelineshloka
{इति पृष्टो मया विप्र स बभाषे निशाचरः} %॥६४॥

\twolineshloka
{पुरा गृध्रेण केनापि नभोमार्गेण गच्छता}
{अस्थिखण्डं स्वतुण्डाग्रात्पातितं क्वापि वारिणि} %॥६५॥

\twolineshloka
{तं जलाशयमागत्यकोऽपि ज्ञानीश्वरस्तदा}
{महातीर्थमिति ज्ञात्वा विदधे पितृतर्पणम्} %॥६६॥

\twolineshloka
{तमूचिरे जनाः सर्वे तीर्थमेतत्कथं वद}
{जपत्येकादशाध्यायं त्रिसन्ध्यं नियतेन्द्रियः} %॥६७॥

\twolineshloka
{कृतमौनस्तु विप्रोऽसौ चौरैर्व्यापादितः पथि}
{तस्यास्थिशकलं गृध्रवदनात्पतितं जले} %॥६८॥

\twolineshloka
{तेन तीर्थमिदं दिव्यं जातं पातकनाशनम्}
{ततस्ते मानवाः सर्वे सस्नुस्तत्र जलाशये} %॥६९॥

\twolineshloka
{निष्कल्मषतया चैवं प्रापुस्ते परमं पदम्}
{एकादशस्य सामर्थ्यादध्यायस्य भविष्यति} %॥७०॥

\twolineshloka
{ममापि मुक्तिः पान्थानां पुनरुत्थानमेव च}
{यो मया कश्चिदुद्गीर्णो ब्राह्मणोऽत्रैव तिष्ठति} %॥७१॥

\twolineshloka
{स च एकादशाध्यायं जपति स्म निरन्तरम्}
{स तेनाध्यायमन्त्रेण सप्तवाराभिमन्त्रितम्} %॥७२॥

\threelineshloka
{विधाय वारि विप्रेन्द्रः क्षिपेद्यदि ममोपरि}
{ततो मे शापनिर्मुक्तिर्भविष्यति न संशयः}
{इति तेनास्मि सन्दिष्टः समायातस्त्वदन्तिकम्} %॥७३॥

\uvacha{विप्र उवाच}

\twolineshloka
{राक्षसः केन पापेन जातोसौ वद रक्षक}
{यत्क्षपायां गृहे तस्मिन्नरान्खादति निद्रितान्} %॥७४॥

\uvacha{ग्रामपाल उवाच}

\twolineshloka
{अस्मिन्ग्रामे पुरा कश्चिदासीद्विप्रः कृषीवलः}
{एकदा शालिकेदाररक्षणे व्याकुलो द्विजः} %॥७५॥

\twolineshloka
{नातिदूरे महागृध्रः पान्थमेकमभक्षयत्}
{तं विमोचयितुं दूराद्दयाञ्चक्रेऽपि तापसः} %॥७६॥

\twolineshloka
{भुक्त्वा पान्थं खगस्तावन्निरगादम्बराध्वना}
{ततः स तापसः कोपात्तं बभाषे कृषीवलम्} %॥७७॥

\twolineshloka
{धिक्त्वां हालिक दुर्बुद्धे कठोरमतिनिर्घृण}
{कुक्षिभरं परत्राण विमुखं हतजीवितम्} %॥७८॥

\twolineshloka
{चोरैश्च दंष्ट्रिभिः सर्प्पैररिवह्निविषाम्बुभिः}
{गृध्रराक्षसभूतैश्च वेतालादिभिराहतान्} %॥७९॥

\twolineshloka
{जनानुपेक्षते शक्तः स तद्वधफलं लभेत्}
{न मोचयति यो विप्रं प्रभुश्चोरादिभिर्धृतम्} %॥८०॥

\twolineshloka
{स याति नरकं घोरं स पुनर्जायते वृकः}
{निहन्यमानं विपिने गृध्रव्याघ्रेण पीडितम्} %॥८१॥

\twolineshloka
{मुञ्चमुचेति यो वक्ति स याति परमां गतिम्}
{गवामर्थे हता व्याघ्रैर्व्याधैर्दुष्टैश्च राजभिः} %॥८२॥

\twolineshloka
{तेपि यान्ति पदं विष्णोर्दुष्प्राप्यं योगिनामपि}
{अश्वमेधसहस्राणि वाजपेयशतानि च} %॥८३॥

\twolineshloka
{शरणागतसन्त्राण कलां नार्हन्ति षोडशीम्}
{दीनस्योपेक्षणं कृत्वा भीतस्य च शरीरिणः} %॥८४॥

\twolineshloka
{पुण्यवानपि कालेन कुम्भीपाके स पच्यते}
{पश्यन्नपि भवान् पान्थं दुष्टगृध्रेण भक्षितम्} %॥८५॥

\twolineshloka
{निवारणसमर्थोपि न चक्रे यन्निवारणम्}
{निष्कृपोसि यतस्तस्माद्भविष्यसि निशाचरः} %॥८६॥

\twolineshloka
{इमां शापं मुने श्रुत्वा कम्पमानकलेवरः}
{प्रणम्य हालिको विप्रं बभाषे करुणं वचः} %॥८७॥

\twolineshloka
{अत्राहं क्षेत्ररक्षायां चिरं क्षिप्तेन चक्षुषा}
{न वेद्मि निकटं गृध्र हन्यमानमिमं नरम्} %॥८८॥

\onelineshloka*
{तेन मेऽनुग्रहं कर्तुं कृपणस्य त्वमर्हसि}

\uvacha{विप्र उवाच}

\onelineshloka
{यो वेत्त्येकादशाध्यायं जपत्यनुदिनं च यः} %॥८९॥

\twolineshloka
{तेनाभिमन्त्रितं वारि यदा शिरसि तावके}
{पतिष्यति तदा शापात्तव मुक्तिर्भविष्यति} %॥९०॥

\threelineshloka
{इत्युक्त्वा तापसो यातो हालिको राक्षसोऽभवत्}
{तदागच्छ द्विजश्रेष्ठ तेनाध्यायेन मन्त्रय}
{तीर्थोदकं स्वहस्तेन तस्य मूर्द्धनि निक्षिप} %॥९१॥

\uvacha{महादेव उवाच}

\twolineshloka
{इति तत्प्रार्थितं तस्य श्रुत्वा च करुणाप्लुतः}
{तथेति सहपालेन मुनी रक्षोन्तिकं ययौ} %॥९२॥

\twolineshloka
{एकादशेन तेनाम्बु विश्वरूपेण मन्त्रितम्}
{निक्षिप्तं तस्य शिरसि तेन विप्रेण योगिना} %॥९३॥

\twolineshloka
{गीताध्यायप्रभावेण शापमोक्षमवाप सः}
{विहाय राक्षसं देहं चतुर्बाहुस्ततोऽभवत्} %॥९४॥

\twolineshloka
{निगीर्णा ये जनास्तेन पान्था आसन्सहस्रशः}
{चतुर्भुजा बभूवुस्ते शङ्खचक्रगदाधराः} %॥९५॥

\twolineshloka
{ते विमानान्यारुरुहुस्तावदूचे स राक्षसम्}
{मदीयस्तनयः कस्तं दर्शयस्व निशाचर} %॥९६॥

\twolineshloka
{इत्युक्ते ग्रामपालेन दिव्यधीराह राक्षसः}
{एवं चतुर्भुजं विद्धि तमालश्यामलद्युतिम्} %॥९७॥

\twolineshloka
{माणिक्यमुकुटं दिव्यमणिकुण्डलमण्डितम्}
{हारहारिमहास्कन्धं स्वर्णकेयूरभूषितम्} %॥९८॥

\twolineshloka
{राजीवलोचनं स्निग्धं हस्ते कृतसरोरुहम्}
{दिव्यं विमानमारूढं देवत्वं प्राप्तमात्मजम्} %॥९९॥

\twolineshloka
{इति तस्य वचः श्रुत्वा सुतं दृष्ट्वा च तादृशम्}
{स्वगेहं नेतुमारेभे जहास स सुतस्ततः} %॥१००॥

\twolineshloka
{कति वाराणि जातोऽसि त्वं पुत्रो मम रक्षक}
{पूर्वपुत्रस्त्वदीयोऽस्मि अधुना विबुधोऽस्म्यहम्} %॥१०१॥

\twolineshloka
{यास्याम वैष्णवं धाम ब्राह्मणस्य प्रसादतः}
{निशाचरोऽपि प्राप्तोऽयं पश्य देहं चतुर्भजम्} %॥१०२॥

\twolineshloka
{एकादशस्य माहात्म्याद्याति स्वर्गं समं जनैः}
{विप्रादस्मात्तमध्यायमधीष्व त्वं जपानिशम्} %॥१०३॥

\twolineshloka
{भविष्यति न सन्देहस्तवापि गतिरीदृशी}
{तात तस्मात्सतां सङ्गं दुर्लभं सर्वथा जनैः} %॥१०४॥

\twolineshloka
{सोप्यद्य ते समुत्पन्नो ह्यात्मनः साधयेप्सितम्}
{किं धनैर्भोगदानैर्वा किं यज्ञैस्तपसा नु किम्} %॥१०५॥

\twolineshloka
{किं पूर्तैर्वापरं श्रेयो विश्वरूपस्य पाठतः}
{तद्विष्णोः परमं रूपमध्यायस्य श्रुतेन च} %॥१०६॥

\twolineshloka
{यत्पूर्णानन्दसन्दोह कृष्णब्रह्मास्यनिर्गतम्}
{कुरुक्षेत्रेऽर्जुने मित्रे तत्कैवल्यरसायनम्} %॥१०७॥

\twolineshloka*
{नृणां च भवभीतानामाधिव्याधिभयापहम्}
{अनेकजन्मदुःखघ्नं नान्यत्पश्यामि तत्स्मर}

\uvacha{शिव उवाच}

\onelineshloka
{इत्युक्त्वा सह तैः सर्वैर्ययौ विष्णोः परं पदम्} %॥१०८॥

\twolineshloka
{तमध्यायं ततो विप्राद्ग्रामपालः पपाठ सः}
{तावुभौ तस्य माहात्म्याज्जग्मतुर्वैष्णवं पदम्} %॥१०९॥

\twolineshloka
{इत्येकादशमाहात्म्य कथा तुभ्यं निरूपिता}
{यस्याः श्रवणमात्रेण महापातकसङ्क्षयः} %॥११०॥

॥इति श्रीपाद्मे महापुराणे पञ्चपञ्चाशत्सहस्रसंहितायामुत्तरखण्डे सतीश्वरसंवादे गीतामाहात्म्ये पञ्चाशीत्यधिकशततमोऽध्यायः॥१८५॥

\sect{{[१२]} षडशीत्यधिकशततमोऽध्यायः} %6.186

\uvacha{महादेव उवाच}

\twolineshloka
{अस्ति कोल्हापुरं नाम नगरं दक्षिणापथे}
{सुखानां सदनं साध्वि साधूनां सिद्धिसम्भवम्} %॥१॥

\twolineshloka
{परशक्तेः परम्पीठं सर्वदेवनिषेवितम्}
{पुराणेषु प्रसिद्धं यद्भुक्तिमुक्तिफलप्रदम्} %॥२॥

\twolineshloka
{कोटिशस्तत्र तीर्थानि शिवलिगानि कोटिशः}
{आस्ते रुद्रगया यत्र विशालं लोकविश्रुतम्} %॥३॥

\twolineshloka
{तुङ्गाचलमहावप्र गोपुरोल्लासितोरणम्}
{प्रासादशिखरे यत्र तुङ्गं च कनकध्वजम्} %॥४॥

\twolineshloka
{सोमकान्तिमहासौधवलभीपङ्क्तिशोभितम्}
{जालरन्ध्रोद्गिरद्धूपधूमाऽमोदितदिक्तटम्} %॥५॥

\twolineshloka
{चलत्पताकविस्तीर्णच्छायं देवालयान्वितम्}
{चतुरैः सुन्दरैः स्निग्धैः श्रीमद्भिः शुचिमानसैः} %॥६॥

\twolineshloka
{अधिष्ठितं सदाचारैः पुरुषेर्भूरिभूषणैः}
{कुरङ्गनयनाश्चन्द्रवदनाः कुटिलालकाः} %॥७॥

\twolineshloka
{उत्फुल्लचम्पकच्छायाः पीनतुङ्गपयोधराः}
{कृशमध्या निम्ननाभि वलित्रयविराजिताः} %॥८॥

\twolineshloka
{विशालजघनाश्चारुजङ्घा युग्मा वराङ्घ्रयः}
{वाचालमेखलादामनिक्वणन्मणिनूपुराः} %॥९॥

\twolineshloka
{रणत्कङ्कणहस्ताब्ज स्फुरत्करज रश्मयः}
{वसन्ति प्रमदा यत्र मादयन्त्यो मुनीनपि} %॥१०॥

\twolineshloka
{समस्तवस्तुसंयुक्तं सर्वभोगसमन्वितम्}
{मङ्गलैः सकलैर्युक्तं महालक्ष्मीसमन्वितम्} %॥११॥

\twolineshloka
{तत्रागच्छत्पुमान्कश्चिद्युवा गौरं सुलोचनः}
{कम्बुकण्ठः पृथुस्कन्धो महावक्षा महाभुजः} %॥१२॥

\twolineshloka
{समस्तलक्षणोपेतो गोचरासक्तमानसः}
{प्रविश्य नगरं पश्यन्शोभां सौधेषु सर्वतः} %॥१३॥

\twolineshloka
{उत्कण्ठितमना द्रष्टुम्महालक्ष्मीं सुरेश्वरीम्}
{मणिकुण्डे कृतः स्नानः सम्पन्न पितॄतर्पणः} %॥१४॥

\twolineshloka
{महालक्ष्मीं महामायां नत्वा तुष्टाव भक्तितः}
{जयत्यपारकरुणाशरण्या जगदम्बिका} %॥१५॥

\twolineshloka
{कुर्वाणा जगतो जन्म पालनं क्षपणं दृशा}
{यया शक्त्या कृतादिष्टः परमेष्ठी सृजत्यसौ} %॥१६॥

\twolineshloka
{अवष्टभ्य च तां शक्तिं पालयत्यच्युतो जगत्}
{यया शक्त्या कृतावेशः संहरत्यखिलं हरः} %॥१७॥

\twolineshloka
{तां भजे परमां शक्तिं सर्गस्थितिलयोर्जिताम्}
{योगिध्येयाङ्घ्रिकमले कमले कमलालये} %॥१८॥

\twolineshloka
{स्वभावानखिलान्नस्त्वं गृह्णासीन्द्रियगोचरान्}
{त्वमेव कल्पनाजालं तत्कल्पं कुरुषे मनः} %॥१९॥

\twolineshloka
{इच्छाज्ञानक्रियारूपा परसंवित्स्वरूपिणी}
{निष्फला निर्मला नित्या निराकारा निरञ्जना} %॥२०॥

\twolineshloka
{निरन्तरा निरातङ्का निरालम्बा निरामया}
{तवैवं महिमानं हि को वर्णयितुमीशते} %॥२१॥

\twolineshloka
{वन्दे निर्भिन्नषट्चक्रद्वादशान्तर्विहारिणीम्}
{अनाहतध्वनिमयीं बिन्दुनादकलात्मिकाम्} %॥२२॥

\twolineshloka
{मातस्त्वं पूर्णशीतांशु गलत्पीयूषवाहिनी}
{पुष्णासि वत्सले बालान्सनकादीन्दिगम्बरान्} %॥२३॥

\twolineshloka
{अनुस्यूता शिवा संविज्जाग्रत्स्वप्नसुषुप्तिषु}
{तुरीयायां वर्त्तमानां दयासूनृतसन्धिषु} %॥२४॥

\twolineshloka
{ददासि प्राणिनां सर्वाः सततं ब्रह्मसम्पदः}
{संहृत्य तत्वसङ्घातं तुरीयातीतया त्वया} %॥२५॥

\twolineshloka
{योगिनां बिम्बतादत्म्यं दीयते निर्विकल्पया}
{परां नमामि पश्यन्तीं मध्यमां वैखरीमपि} %॥२६॥

\twolineshloka
{रूपाणि देवि गृह्णासि जगत्सन्त्राणहेतवे}
{त्वं ब्राह्मी वैष्णवी च त्वं माहेशी च त्वमम्बिके} %॥२७॥

\twolineshloka
{वाराहि त्वं महालक्ष्मीर्नारसिंही त्वमैन्द्रिका}
{त्वं कौमारी चण्डिका त्वं लक्ष्मीस्त्वं विश्वपावनी} %॥२८॥

\twolineshloka
{सावित्री त्वं जगन्माता शशिनी त्वं च रोहिणी}
{त्वं स्वाहा त्वं स्वधा त्वं हि त्वं सुधा परमेश्वरी} %॥२९॥

\twolineshloka
{चण्डमुण्डभुजादण्डखण्डदोर्दण्डमण्डिते}
{रक्तबीजगलद्रक्तपानघूर्णितलोचने} %॥३०॥

\twolineshloka
{उन्मत्तमहिषग्रीवोन्मूलनप्रौढ दोर्युगे}
{शुम्भासुरमहादैत्यदारणायातविक्रमे} %॥३१॥

\twolineshloka
{अनन्तचरिते तुभ्यं नमस्त्रैलोक्यमातृके}
{भक्तकल्पलते मह्यं प्रसीद परमेश्वरि} %॥३२॥

\twolineshloka
{इति तेन स्तुता देवी महालक्ष्मीस्ततः स्वयम्}
{निजरूपं समास्थाय पुरुषं प्रत्युवाच तम्} %॥३३॥

\uvacha{श्रीलक्ष्मीरुवाच}

\onelineshloka*
{राजपुत्र प्रसन्नाऽहं वृणीष्व वरमुत्तमम्}

\uvacha{राजपुत्र उवाच}

\onelineshloka
{पिता मे धरणीपालो वाजिमेधं महाक्रतुम्} %॥३४॥

\twolineshloka
{कुर्वाणो दैवयोगेन रोगाक्रान्तो दिवं ययौ}
{तद्वपुस्तप्ततैलेन शोषयित्वा मया ततः} %॥३५॥

\twolineshloka
{स्थापितस्तत्र यागोऽसौ यथापूर्वमवर्तत}
{अथ क्रान्तमहीचक्रो यूपे यागतुरङ्गमः} %॥३६॥

\twolineshloka
{निशीथे बन्धनं हित्वा नीतः केनापि कुत्रचित्}
{अदृष्ट्वा तद्गतं क्वापि निवृत्तेषु जनेष्वहम्} %॥३७॥

\twolineshloka
{आमन्त्र्य ऋत्विजः सर्वाञ्छरणं त्वामुपागतः}
{प्रसन्ना यदि देवि त्वं तन्मे यागतुरङ्गमः} %॥३८॥

\twolineshloka
{दृष्टो भवतु यागोऽसौ सम्पूर्णो जायते यथा}
{आनृण्यं मम तातस्य तेन राज्ञो भविष्यति} %॥३९॥

\onelineshloka*
{तथा कुरु जगद्धात्रि शरणागतवत्सले}

\uvacha{देव्युवाच}

\onelineshloka
{ममद्वारं द्विजः सिद्धसमाधिरिति विश्रुतः} %॥४०॥

\twolineshloka
{ममाज्ञया स ते सर्वं कार्यं निष्पादयिष्यति}
{इत्युक्तः श्रीमहालक्ष्म्या ततो राजकुमारकः} %॥४१॥

\twolineshloka
{आजगाम मुनिः सिद्धसमाधिर्यत्र तिष्ठति}
{प्रणम्य तस्यपादाब्जं कृताञ्जलिरवस्थितः} %॥४२॥

\twolineshloka
{तमुवाच ततो विप्रः प्रहितोऽसि त्वमम्बया}
{त्वदीप्सितमिदं सर्वं साधयामि विलोकय} %॥४३॥

\twolineshloka
{इत्युक्त्वा त्रिदशान्सर्वानाचकर्ष स मान्त्रिकः}
{एक्षत क्षितिपालस्य तनयोऽसौ तदा सुरान्} %॥४४॥

\twolineshloka
{कृताञ्जलिपुटान्देवान्वेपमानकलेवरान्}
{अथ तानमरान्सर्वान्सम्बभाषे द्विजोत्तमः} %॥४५॥

\twolineshloka
{अमुष्य राजपुत्रस्य वाजी यज्ञाय कल्पितः}
{नीतोस्ति देवराजेन क्षपायामपहृत्य सः} %॥४६॥

\twolineshloka
{गीर्वाणा अश्वमेवास्य समानयत मा चिरम्}
{अथ तस्य मुनेर्वाक्याद्देवैर्यज्ञतुरङ्गमः} %॥४७॥

\twolineshloka
{समर्पितस्ततस्तेन तेऽनुज्ञात्वा दिवौकसः}
{आकृष्टानमरान्दृष्ट्वा गतं लब्ध्वा तुरङ्गमम्} %॥४८॥

\twolineshloka
{महीपतिसुतो नत्वा तं मुनिं वाक्यमब्रवीत्}
{आश्चर्यमिदमेतत्ते सामर्थ्यमृषिसत्तम} %॥४९॥

\twolineshloka
{कृतं त्वया चित्रमिदं त्रिदशाकर्षणं क्षणात्}
{हस्तादाकृष्य दत्तो मे यज्ञीयोऽयं तुरङ्गमः} %॥५०॥

\twolineshloka
{न किञ्चिदपरं यावद्दुष्करं यत्सुरैरपि}
{प्रभविष्यति तत्कर्तुं भवानेव न चापरः} %॥५१॥

\twolineshloka
{शृणु विप्र महीपाल पितासीन्मेबृहद्रथः}
{प्रारब्ध हयमेधोऽसौ दैवेन निधनं गतः} %॥५२॥

\twolineshloka
{अद्यापि तस्य देहोस्ति तप्ततैलेन शोषितः}
{तस्य सञ्जीवनं भूयः कर्त्तुमर्हसि सत्तम} %॥५३॥

\twolineshloka
{इत्याकर्ण्य स्मितं कृत्वा स जगाद महामुनिः}
{यामस्तत्र पिता यत्र तावको यागमण्डपः} %॥५४॥

\twolineshloka
{अथागत्य समं तेन तत्र सिद्धः समाधिना}
{पयोऽभिमन्त्र्य विदधे तस्य प्रेतस्य मूर्द्धनि} %॥५५॥

\twolineshloka
{ततः प्राप नृपः संज्ञामुत्तस्थे वै ददर्श च}
{स तं पप्रच्छ विप्रेन्द्रं कोऽसि धर्मेति भूपतिः} %॥५६॥

\twolineshloka
{ततो राजसुतः सर्वं भूपालाय न्यवेदयत्}
{स नत्वा ब्राह्मणं राजा तं पुनर्दत्तजीवितम्} %॥५७॥

\twolineshloka
{बभाषे केन पुण्येन त्वयि शक्तिरलौकिकी}
{यया मे जीवितं दत्तमाकृष्टाश्च दिवौकसः} %॥५८॥

\twolineshloka
{यागश्चोद्धरितो विप्र येनमेतन्निरूपय}
{इत्युक्तस्तेनविप्रोऽसौ जगाद श्लक्ष्णया गिरा} %॥५९॥

\twolineshloka
{गीतानां द्वादशाध्यायं जपाम्यहमतन्द्रितः}
{तेन शक्तिरियं राजन्यया प्राप्तोसि जीवितम्} %॥६०॥

\twolineshloka
{एतदाकर्ण्य राजाऽसौ द्वादशाध्यायमुत्तमम्}
{पपाठ तस्माद्विप्रर्षेः सकाशात्ब्राह्मणान्वितम्} %॥६१॥

\twolineshloka
{तस्याध्यायस्य माहत्म्यात्ते सर्वे सद्गतिं ययुः}
{अन्ये पठित्वा जीवाश्च मुक्तिमापुरहो पराम्} %॥६२॥

॥इति श्रीपाद्मे महापुराणे पञ्चपञ्चाशत्सहस्रसंहितायामुत्तरखण्डे सतीश्वरसंवादे गीतामाहात्म्ये षडशीत्यधिकशततमोऽध्यायः॥१८६॥

\sect{{[१३]} सप्ताशीत्यधिकशततमोऽध्यायः} %6.187

\uvacha{देव्युवाच}

\twolineshloka
{द्वादशाध्यायमाहात्म्यं भवता कथितं मम}
{ब्रूहि त्रयोदशाध्यायमाहात्म्यमतिसुन्दरम्} %॥१॥

\uvacha{ईश्वर उवाच}

\twolineshloka
{शृणु त्रयोदशाध्यायमहिमाम्भोनिधिं शिवे}
{यदाकर्णनमात्रेण परां मुदमवाप्स्यसि} %॥२॥

\twolineshloka
{अस्ति दक्षिणदिङ्मार्गे तुङ्गभद्रा महानदी}
{तत्तटे नगरं रम्यं नाम्ना हरिहरं पुरम्} %॥३॥

\twolineshloka
{यत्राऽस्ते भगवान्देवि देवो हरिहरः स्वयम्}
{यस्य दर्शनमात्रेण परं कल्याणमाप्यते} %॥४॥

\twolineshloka
{तस्मिन्पुरे द्विजन्मासीद्धरिदीक्षितसंज्ञिकः}
{तपः स्वाध्यायनिरतो श्रोत्रियो वेदपारगः} %॥५॥

\twolineshloka
{दुराचारेति तस्यासीद्भार्या नाम्ना च कर्मणा}
{न सुष्वाप समं पत्या दुरालापा कदाचन} %॥६॥

\twolineshloka
{क्षणमप्यात्मसदने न चास्ते स्वैरचारिणी}
{कण्ठदघ्नं द्विजद्वारे धयन्ती वारुणीरसम्} %॥७॥

\twolineshloka
{पतिसम्बधिनः सर्वान्तर्जयन्ती पुनः पुनः}
{विटैः सह सदोन्मत्ता रममाणा निरन्तरम्} %॥८॥

\twolineshloka
{कदचिद्व्याकुलं दृष्ट्वा पुरं पौरैरितस्ततः}
{सङ्केतगेहमकरोत्कान्तारे निर्जने स्वयम्} %॥९॥

\twolineshloka
{अथ तत्रैव सा धूर्ता रममाणा विटैः सह}
{निनाय सा बहून्कालान्निजयौवनगर्विता} %॥१०॥

\twolineshloka
{अथ तस्मिन्पुरे रम्ये निवसन्त्या निरङ्कुशम्}
{वसन्तकालं समभूत्परं चित्ते भुवः सखा} %॥११॥

\twolineshloka
{आमूलपल्लवाकीर्णं सहकारविकारिणा}
{पिकानां पञ्चमालापैः पुनः स जीवितः स्मरः} %॥१२॥

\twolineshloka
{स्फुरच्चम्पकसौरभ्यहारकैर्मलयानिलैः}
{मन्दं मन्दं प्रसर्पद्भिरान्दोलितवनद्रुमः} %॥१३॥

\twolineshloka
{उत्फुल्लमल्लिकामोद मदिरापारणावताम्}
{अलीनां कलटङ्कारैः समन्ताद्रावशोभितः} %॥१४॥

\twolineshloka
{प्रसन्नचारुभिः स्मेरः सरोवरसुगन्धिभिः}
{मीलन्मरालनिवहैः सरोभिः प्रकटीकृतः} %॥१५॥

\twolineshloka
{घनच्छायासुखासीन हरिणार्भकधारिभिः}
{नीरन्ध्रपल्लवैर्नानाशाखिभिः शोभितावनिः} %॥१६॥

\twolineshloka
{तस्मिन्वसन्तसमये मुदिता साभिसारिका}
{अपश्यज्जगदानन्ददायिनीं चन्द्रिकां निशि} %॥१७॥

\twolineshloka
{चञ्चच्चकोरचञ्च्वग्रगलत्पीयूषसीकराम्}
{द्रवदिन्दुशिलानिर्यत्सुधानिर्झरनिर्भराम्} %॥१८॥

\twolineshloka
{विकाशिकुसुमक्रोडसान्द्रीभूतकरोत्कराम्}
{उल्लासितपयोराशिकल्लोलालिङ्गिताम्बराम्} %॥१९॥

\twolineshloka
{मनोभवमहासिंहकुलटाकण्ठकर्तरीम्}
{घनान्धकारसन्दोहविदारणपटीयसीम्} %॥२०॥

\twolineshloka
{श्वेतीकृतसतीकारपरार्थहिमगर्भिणीम्}
{म्लानपङ्कजसङ्कोचा द्यूनामानन्ददायिनीम्} %॥२१॥

\twolineshloka
{चक्रवाकवधूवक्त्र करुणाक्रोशसाक्षिणीम्}
{मुक्ताश्रेणीविशुद्धांशु प्रभासितदिगन्तराम्} %॥२२॥

\twolineshloka
{अथ तस्यां प्रभूतायां पूरयन्त्या दिशो दश}
{कामान्धा कामिनीजाता पथिसौधविहारिणी} %॥२३॥

\twolineshloka
{अपश्यन्ती विटान्रात्रौ निर्भिद्य भवनार्गलम्}
{ययौ सङ्केतभवनं निर्गत्य नगराद्बहिः} %॥२४॥

\twolineshloka
{तत्र प्रियतमं कञ्चित्काममोहितमानसा}
{अन्वेषयन्ती नाद्राक्षीत्कुञ्जे कुञ्जे तरौ तरौ} %॥२५॥

\twolineshloka
{आकर्णयन्ती कान्तस्य मन्दालापान्पदे पदे}
{अभियाति ततः क्रीडन्यत्र संहारिनिस्वनः} %॥२६॥

\twolineshloka
{चक्रवाकरवान्श्रुत्वा कान्तालापभ्रमादसौ}
{सरोवराणि सर्वाणि पर्यटन्ती मुहुर्मुहुः} %॥२७॥

\twolineshloka
{कान्तभ्रान्त्या तरुतले प्रसुप्तान्हरिणोत्करान्}
{प्रबोधयन्ती सोच्छ्वासमागतास्मीति भाषिणी} %॥२८॥

\twolineshloka
{आलिङ्गन्ती वनस्थाणुं जीवनेश्वरशङ्कया}
{तदाननभ्रमाद्भूयं चुम्बन्ती विकचाम्बुजम्} %॥२९॥

\threelineshloka
{तत्रतत्र कृतव्यर्थश्रमादृष्टप्रियास्वयम्}
{विललाप वने तस्मिन्मूर्च्छती विविधोक्तिभिः}
{हा कान्त हा गुणाक्रान्त मच्चैतन्यस्य नायक} %॥३०॥

\twolineshloka
{हे मनोहरसौभाग्य भाग्यलावण्यशेवधे}
{हा पूर्णचन्द्रवदन हा सरोजायतेक्षण} %॥३१॥

\twolineshloka
{हा कान्ततत्वसौहित्य विश्रमाय सुरद्रुम}
{यदि कोपेन कुत्रापि गुह्यवेषोऽत्र तिष्ठसि} %॥३२॥

\twolineshloka
{प्रसादयामि त्वां कान्त दत्त्वा प्राणान्प्रियानपि}
{इत्युच्चैः सर्वतो दिक्षु विलपन्त्या वियोगतः} %॥३३॥

\twolineshloka
{तस्याः श्रुत्वा वचः कोऽपि सुप्तो व्याघ्रः प्रबुद्धवान्}
{कुर्वन्घुरघुरध्वानं पश्यन्प्रतिदिशं रुषा} %॥३४॥

\twolineshloka
{आस्फालयन्नखैर्भूमिं गर्जन्नाकाशगह्वरम्}
{पृष्ठनिर्भग्नलाङ्गूल द्रुतमुत्थाय वेगवान्} %॥३५॥

\twolineshloka
{गतो व्याघ्रः समुत्पत्य यत्रास्ते साभिसारिका}
{अथ सापि तमायान्तमालोक्य पतिशङ्कया} %॥३६॥

\twolineshloka
{निर्जगाम पुरः स्थातुं प्रेममनिर्भरमानसा}
{ततस्तस्यनखक्रीडाक्रूरतान्धीकृता सती} %॥३७॥

\twolineshloka
{जहौ प्रियवपुः शङ्कां श्रुत्वा गर्जितमूर्ज्जितम्}
{तथाविधापि सा नारी भ्रान्तिमुत्सृज्यसत्वरम्} %॥३८॥

\twolineshloka
{व्याघ्रत्वं तु कुतो हेतोर्मां निहन्तुमिहागतः}
{इदं सर्वं समाख्याहि यतस्त्वं हन्तुमिच्छसि} %॥३९॥

\twolineshloka
{इति तस्या वचः श्रुत्वा शार्दूलश्चण्डविक्रमः}
{क्षणं विहाय तद्ग्रासमुवाच प्रहसन्निव} %॥४०॥

\twolineshloka
{मलापहा नदी नाम्ना देशे तिष्ठति दक्षिणे}
{नगरी मुनिपर्णेति तस्या रोधसि वर्तते} %॥४१॥

\twolineshloka
{तत्रास्ते भगवान्साक्षात्पञ्चलिङ्गो महेश्वरः}
{तस्यां पुर्यामहं विप्र पुत्रो भूत्वा स्थितस्ततः} %॥४२॥

\twolineshloka
{अयाज्यान्याजयन्नश्नन्नेकोद्दिष्टे नदी तटे}
{वेदपाठफलं शश्वद्विक्रीणन्धनकाङ्क्षया} %॥४३॥

\twolineshloka
{भिक्षुकानपरान्लोभात्तिरस्कुर्वन्दुरुक्तिभिः}
{अदेयद्रविणं गृह्णन्नदत्तमनिशं दिनम्} %॥४४॥

\twolineshloka
{छलयन्सकलांल्लोकान्क्षणग्रहणकौतुकात्}
{ततः कतिपये काले जरठत्वमुपागतः} %॥४५॥

\twolineshloka
{वलीपलितवानन्धः प्रपतन्प्रस्खलद्गतिः}
{पतद्दन्तोभवं भूयः प्रतिग्रहपरायणः} %॥४६॥

\twolineshloka
{हस्ते गृहीत दर्भोऽहमगमं तीर्थसन्निधिम्}
{धनग्रहणलोभेन भ्रमन्पर्वसु पर्वसु} %॥४७॥

\twolineshloka
{ततोऽहं शिथिलाङ्गः सन्कञ्चिद्भूनिर्जरालयम्}
{गतवान्याचितं भोक्तुं दष्टो मध्ये पदे शुना} %॥४८॥

\twolineshloka
{अपतं मूर्छितो भूत्वा ततः क्षितितले क्षणात्}
{ततोऽहं गलितप्राणो व्याघ्रयोनिमुपागतः} %॥४९॥

\twolineshloka
{अत्र तिष्ठामि कान्तारे पूर्वपापमनुस्मरन्}
{न भक्षयामि धर्मिष्ठान्मुनीन्साधुजनान्सतीः} %॥५०॥

\twolineshloka
{किन्तु पापान्दुराचारनसतीर्भक्षयाम्यहम्}
{अतोऽसतित्वं तत्वेन ममैव कवलायसे} %॥५१॥

\twolineshloka
{इत्युकत्वा स्वैर्नखैः कूरैस्तां विभज्याङ्गखण्डशः}
{अथतां भक्षितां तेन पापदेहमुपाश्रिताम्} %॥५२॥

\twolineshloka
{यमस्य किङ्करा निन्युः सद्यः संयमनीं पुरीम्}
{यमादेशेन तत्रापि पातयामासुराशु ताम्} %॥५३॥

\twolineshloka
{विण्मूत्ररक्तपूर्णेषु घोरकुण्डेष्वनेकधा}
{कल्पकोटिषु पीतासु तस्मादानीय तां मुहुः} %॥५४॥

\twolineshloka
{रौरवे स्थापयामासुर्मन्वन्तरशतावधि}
{ततोप्याकृष्य तां दीनां रुदतीं सर्वतोमुखीम्} %॥५५॥

\twolineshloka
{मुक्तकेशां भग्नगात्रां चिक्षिपुर्दहनानने}
{एवं पापपरां घोरां भुक्त्वा नरकयातनाम्} %॥५६॥

\twolineshloka
{इह जाता महापापात्पुनः श्वपचयोनिषु}
{ततः श्वपचगेहेऽपि वर्द्धमाना दिनेदिने} %॥५७॥

\twolineshloka
{पूर्वजन्मवशेनैव तथैवासीद्यथा पुरा}
{ततः कतिपये काले पुनः स्वम्भवनंययौ} %॥५८॥

\twolineshloka
{यत्रास्ते जृम्भकादेवी शिवस्यान्तः पुरेश्वरी}
{तत्रापश्यद्द्विजन्मानं वासुदेवाभिधं शुचिम्} %॥५९॥

\twolineshloka
{गीतात्रयोदशाध्यायमुद्गिरन्तमनारतम्}
{ततस्तच्छ्रवणादेव मुक्ता श्वपचविग्रहात्} %॥६०॥

\onelineshloka
{दिव्यं देहं समासाद्य जगाम त्रिदशालयम्} %॥६१॥

॥इति श्रीपाद्मे महापुराणे पञ्चपञ्चाशत्सहस्रसंहितायामुत्तरखण्डे सतीश्वरसंवादे गीतामाहात्म्ये सप्ताशीत्यधिकशततमोऽध्यायः॥१८७॥

\sect{{[१४]} अष्टाशीत्यधिकशततमोऽध्यायः} %6.188

\uvacha{ईश्वर उवाच}

\twolineshloka
{अतः परं प्रवक्ष्यामि भवानि भवमुक्तये}
{गीताचतुर्दशाध्यायमवधारय सुस्मिते} %॥१॥

\twolineshloka
{मेदिन्यां यत्किलस्थूलमस्ति काश्मीरमण्डलम्}
{राजधानी सरस्वत्या आस्ते चैव मनोहरा} %॥२॥

\twolineshloka
{यामधिष्ठाय वाग्देवी ब्रह्मलोकं प्रयच्छति}
{हंसमारुह्यमानापि सावित्रीप्रहतैरपि} %॥३॥

\twolineshloka
{सरस्वतीपदाम्भोज सेवामाश्रित्य कुङ्कुमैः}
{यत्र गौरवयन्त्याशा हंसपक्षपुटोद्भवैः} %॥४॥

\twolineshloka
{निरन्तरं तया चैव नृणां संस्कृतभाषिणाम्}
{सुपर्वाणमयी भाषा निमेषेणोपलभ्यते} %॥५॥

\twolineshloka
{प्रातर्गृहाङ्गणोद्भूतै र्यत्रकुङ्कुमपांसुलैः}
{सर्वतोरुणितच्छाय शशाङ्करविमण्डलम्} %॥६॥

\twolineshloka
{तत्रासीत्तेजसां राशिः शौर्यवर्मा नरेश्वरः}
{उद्यदुज्ज्वलबाणौघ खण्डितारातिमण्डलः} %॥७॥

\twolineshloka
{अभूच्च सिंहलद्वीपे राजा सिंहपराक्रमः}
{नाम्ना विक्रमवेतालः कलानामपि शेवधिः} %॥८॥

\twolineshloka
{उभौ परस्परं मैत्रीं वर्द्धयाञ्चक्रतुः क्रमात्}
{तत्तद्देशसमुत्पन्नैरपूर्वैः प्रचुरोत्करैः} %॥९॥

\twolineshloka
{एकदा प्रहितं प्रेम्णा प्रभूतं शौर्यवर्मणा}
{राजा विक्रमवेतालो विलोक्य शुनकीद्वयम्} %॥१०॥

\twolineshloka
{मत्तमातङ्गतुरग मणिभूषणचामरम्}
{प्रेषयामास मित्राय प्रभूतं शौर्यवर्मणे} %॥११॥

\twolineshloka
{एकदा शिबिकारूढश्चारु चामरवीजितः}
{सुवर्णशृङ्खलारूढं वाद्यडिण्डिमडम्बरम्} %॥१२॥

\twolineshloka
{शुनीयुगलमादाय मृगया कौतुकोत्सुकः}
{राजा जगाम बाह्यालीं समं राजकुमारकैः} %॥१३॥

\twolineshloka
{पणबन्धविधानेन समुपेतं शशामिषम्}
{तत्र राजकुमाराणां महान्कोलाहलोऽभवत्} %॥१४॥

\twolineshloka
{ततः समानवयसा केनचिद्राजसूनुना}
{बहुमूल्यं पणं कृत्वा राजाचिक्रीडकौतुकी} %॥१५॥

\twolineshloka
{ततोवतार्य दोलाया विरुदावलिगर्विताम्}
{धावतः शशकस्योच्चैः पृष्ठे मुञ्चन्नृपं शुनीम्} %॥१६॥

\twolineshloka
{मुमोच राजपुत्रोऽपि प्रेमपात्रं महाभुजः}
{विरराम शुनीमुच्चैः सङ्कीर्त्य विरुदावलीम्} %॥१७॥

\twolineshloka
{अलक्ष्यमाणवेगेऽस्मिन्शुनीयुगलकेभृशम्}
{धावत्युत्थितमेवासीत्पश्यतां सर्वभूभृताम्} %॥१८॥

\twolineshloka
{पपात गर्ते महति शशकोऽतिश्रमादसौ}
{पतितोऽपि शुनी वश्यो नाभवच्छशशावकः} %॥१९॥

\twolineshloka
{ततः शनैः समुत्थाय धावन्नाक्रम्यरोषतः}
{जगृहे राजशुन्याऽसौ शशकः फेनमुद्वमन्} %॥२०॥

\twolineshloka
{ततः कथञ्चिदुप्लुत्य गच्छन्विस्खलयञ्छशः}
{राजपुत्रशुनक्यासौ गृहीतः कन्धरातटे} %॥२१॥

\twolineshloka
{जितमस्माभिरत्यर्थमिति सञ्जल्पतां नृणाम्}
{कोलाहले शङ्किताया शुन्या निर्गतवान्मुखात्} %॥२२॥

\twolineshloka
{ततो दंष्ट्राव्रणश्रेणी क्षरद्रुधिरधारकः}
{क्वापि र्ममरभूभागे निलीयस्थितवाञ्छशः} %॥२३॥

\twolineshloka
{जिघ्रन्त्या राजशुन्याऽसौ भूभागं धनरोषया}
{दृष्टमात्रः परित्रस्तो हस्तमात्रं ततोऽगमत्} %॥२४॥

\twolineshloka
{यत्र कर्पूरकदली क्रोडव्याघ्रदरीतलः}
{चोली कपोलफलकान्चुबन्वाति समीरणः} %॥२५॥

\twolineshloka
{उद्भिन्न केतकीकोशरजोमुकुलितेक्षणः}
{विस्रब्धाहरणा यत्रच्छायां तां परितन्वतः} %॥२६॥

\twolineshloka
{नारिकेलफलैर्यत्र स्वयं निपतितैरधः}
{अपि चूतफलैस्तृप्ताः पक्वैः शाखामृगा अपि} %॥२७॥

\twolineshloka
{अपि केसरिणो यत्र खेलन्ति कलभैः समम्}
{फणिनः केकिबर्हेषु निर्विशङ्कं विशन्ति च} %॥२८॥

\twolineshloka
{यत्राश्रमान्तरे विप्रो वत्सनामा जितेन्द्रियः}
{शान्तश्चतुर्दशाध्यायं जपन्नास्ते निरन्तरम्} %॥२९॥

\twolineshloka
{तत्र तच्छिष्यपादाब्ज प्रक्षालनजलैः कृते}
{कर्दमे न्यपतद्गत्वा जीवशेषो मुहुः श्वसन्} %॥३०॥

\twolineshloka
{ततः कर्दमसंस्पर्शमात्रनिस्तीर्ण संसृतिः}
{दिव्यं विमानमारुह्य निर्ययौ शशको दिवम्} %॥३१॥

\twolineshloka
{ततः शुन्यपि लिप्ताङ्गीस्तोकैः कर्दमबिन्दुभिः}
{क्षुत्पिपासार्तिरहिता शुनीरूपं विहाय सा} %॥३२॥

\twolineshloka
{ततो दिव्याङ्गनारम्यं गन्धर्वैरुपशोभितम्}
{दिव्यं विमानमारुह्य शुन्यपि त्रिदिवं ययौ} %॥३३॥

\twolineshloka
{ततो जहास मेधावी शिष्यो नाम्ना स्वकन्धरः}
{विचार्य विस्मितः पूर्वजन्मवैरस्य कारणम्} %॥३४॥

\twolineshloka
{राजापि पर्यपृच्छत्तं विस्मयस्मेरलोचनः}
{प्रणम्य परया भक्त्या विनयैकपयोनिधिः} %॥३५॥

\twolineshloka
{कथां कथय मे विप्र हीनयोनि निषेवितौ}
{अज्ञौयौ जग्मतुः स्वर्गे शुनी शशकशावकौ} %॥३६॥

\uvacha{शिष्य उवाच}

\twolineshloka
{वत्सनामा द्विजन्मास्ते वनेऽमुष्मिन्जितेन्द्रियः}
{चतुर्दशं तु ह्यध्यायं गीतानां सर्वदा जपन्} %॥३७॥

\twolineshloka
{शिष्योऽहं तस्य भूपाल ब्रह्मविद्याविशारदः}
{चतुर्दश तु अध्यायं जपामि प्रत्यहं नृप} %॥३८॥

\twolineshloka
{मदीयचरणाम्भोजप्रक्षालनजले लुठन्}
{शशस्त्रिदिवमापन्नः शुनक्या सह भूपते} %॥३९॥

\uvacha{राजोवाच}

\twolineshloka
{हेतुना केन कथय हसितं च द्विजोत्तम}
{अतः किमपि साकूतं मन्यमानेन सादरम्} %॥४०॥

\uvacha{शिष्य उवाच}

\twolineshloka
{महाराष्ट्रेति नगरं नाम्ना प्रत्युदकं महत्}
{तत्रासीद्ब्राह्मणो नाम्ना केशवः कितवाग्रणीः} %॥४१॥

\twolineshloka
{विलोभनाभवत्तस्य जाया स्वैरविहारिणी}
{तेन सा हन्यते क्रोधाद्वैरं सञ्चिन्त्य जन्मनः} %॥४२॥

\twolineshloka
{ततः स्त्रीवधपापेन शशको जायते द्विजः}
{किल्बिषाच्छुनकी सापि जाता वञ्चनजन्मनः} %॥४३॥

\twolineshloka
{पूर्वेण जन्मनाभ्यस्तं वैरं विस्मरतो नहि}
{आसेदिवद्भ्यां बहुधा योन्यन्तरमपि क्वचित्} %॥४४॥

\twolineshloka
{इत्याकलय्य सकलं भूपालः श्रद्धयान्वितः}
{गीतामभ्यस्य सकलामवाप परमां गतिम्} %॥४५॥

॥इति श्रीपाद्मे महापुराणे पञ्चपञ्चाशत्सहस्रसंहितायामुत्तरखण्डे सतीश्वरसंवादे गीतामाहात्म्ये अष्टाशीत्यधिकशततमोऽध्यायः॥१८८॥

\sect{{[१५]} एकोननवत्यधिकशततमोऽध्यायः} %6.189

\uvacha{ईश्वर उवाच}

\twolineshloka
{प्रवक्ष्यामि विशालाक्षि तुहिनाचलकन्यके}
{गीता पञ्चदशाध्यायमाहात्म्यमवधारय} %॥१॥

\twolineshloka
{कृपालुर्नरसिंहोऽभून्नाम्ना गौडेषु भूपतिः}
{यस्यासिधारया सङ्ख्ये देवासङ्घा वधीकृताः} %॥२॥

\twolineshloka
{यदीय मत्तमातङ्ग दानधाराजलैरिला}
{निदाघेपि च सेहेतां रविसन्तापवेदनाम्} %॥३॥

\twolineshloka
{सङ्क्रन्दनपरित्रस्ता यदीय शरणं गताः}
{रेजिरे करिणो मत्ताश्चलन्तः पर्वता इव} %॥४॥

\twolineshloka
{मत्तमातङ्गचीत्कारप्रतिस्वनमिवादरात्}
{यस्य गोपायतः शैला व्याहरन्ति कृपावतः} %॥५॥

\twolineshloka
{यदीय धावत्तुरगक्षुरसङ्घातजर्जरम्}
{नाभूच्चित्रं कथङ्कारं गतखण्डं धरातलम्} %॥६॥

\twolineshloka
{यस्मिन्वृत्रहणो मित्रे समुद्धरति मेदिनीम् 3}
{पुनरुज्वलयाञ्चक्रे महाभाष्यं फणीश्वरः} %॥७॥

\twolineshloka
{तस्यासीत्सैनिको धीमाञ्छस्त्रशास्त्रकलानिधिः}
{नाम्ना सरभभेरुण्डः प्रचण्डभुजमण्डलः} %॥८॥

\twolineshloka
{भाण्डारेण तुरङ्गैश्च भटैर्वीररसोद्भवैः}
{समान एव भूपालदुर्गैरत्यन्तदुर्गमैः} %॥९॥

\twolineshloka
{स कदाचित्स्वयं राज्यं कर्तुं पापो दधे मनः}
{निहत्य वसुधापालं बलात्साकं कुमारकै} %॥१०॥

\twolineshloka
{कर्तुं व्यवस्य दिवसैः स्वल्पैरित्थं चिकीर्षया}
{विषूचिकामयादाशु परासुः समजायत} %॥११॥

\twolineshloka
{कालेनाल्पीयसा प्रेत्य पापात्मा तेन कर्मणा}
{तेजस्वी तुरगो जातः सिन्धुदेशे कृशोदरिः} %॥१२॥

\twolineshloka
{मूलेन बहुना क्रीत्वा हयतत्वविदा ततः}
{बहुयत्नवता नीतः केनचिद्वैश्यसूनुना} %॥१३॥

\twolineshloka
{राजापि पौत्रनप्त्त्राद्यैस्तस्यैव मरणात्परम्}
{कालेन वृद्धतां प्राप्तः स्वराज्यं चापि पालयन्} %॥१४॥

\twolineshloka
{स वैश्यसूनुस्तं चाश्वं राज्ञे दातुं समागतः}
{राज्ञोद्वारिस्थतस्तत्र प्रतीक्षंस्तत्समागमम्} %॥१५॥

\twolineshloka
{ज्ञातपूर्वोऽपि वैश्योऽसौ प्रतीहारेण दर्शितः}
{किमर्थं ब्रूहि राज्ञेति पृष्टः स्पष्टमभाषत} %॥१६॥

\twolineshloka
{देव त्रिजगतीरत्नमिति मत्वा तुरङ्गमः}
{मया नियुतमूल्येन बहुना साधुलक्षणः} %॥१७॥

\twolineshloka
{ततोऽवलोक्य वक्त्राणि भूपालः पार्श्ववर्तिनाम्}
{समादिदेश वणिजमश्वोऽत्र नीयतामिति} %॥१८॥

\twolineshloka
{शिरांसि धूनयन्नृणामश्वलक्षणवेदिनाम्}
{शूराणामपि चेतांसि मुहुरुत्साहयन्महान्} %॥१९॥

\twolineshloka
{अखण्डमेदिनीवेगबहुसङ्क्रमणार्जितम्}
{लालाफेनछलेनासौ वमन्शुभ्रतरं यशः} %॥२०॥

\twolineshloka
{उच्चैःश्रवस्तुलां भेजे गुणसाम्येन तत्त्वतः}
{विवृण्वन्नति तेजस्वी ह्रियेवनतकन्धरः} %॥२१॥

\twolineshloka
{चामरैरिन्दुधवलैर्वीज्यमानो निरन्तरम्}
{दुग्धाम्भोनिधिलोलैस्तैः श्वासैरुच्चैःश्रवा इव} %॥२२॥

\twolineshloka
{नीलातपत्रयुगलं घनच्छायतुलश्रिया}
{बिभ्राणो वारिदालीढ हिमाद्रि शिखरश्रियम्} %॥२३॥

\twolineshloka
{मेदिनीमण्डलस्पर्श सङ्क्रान्तमिव पावकम्}
{मुहुरुद्धरयन्धुन्वन्बन्धुरं कन्धरातटम्} %॥२४॥

\twolineshloka
{दारयन्वैरिणः सर्वान्व्याहरंश्च जयश्रियम्}
{हेषारवेण गुरुणा दिक्षुप्रख्यापयन्यशः} %॥२५॥

\twolineshloka
{सत्त्वस्य राशिरत्युच्चैर्गतीनामिव शेवधिः}
{रूपस्य निलयं साक्षाल्लक्षणानां पयोनिधिः} %॥२६॥

\twolineshloka
{आनीतो वणिजा वाजी राज्ञा च समदृश्यत}
{बहुधा वर्णितोऽमात्यैरश्वलक्षणवेदिभिः} %॥२७॥

\twolineshloka
{यथेच्छं वणिजोदीर्णं स्वर्णं दत्त्वा महीपतिः}
{जग्राह तुरगं वेगादसीमानन्दनिर्भरः} %॥२८॥

\twolineshloka
{ततोऽश्वपालमाहूय यत्नतस्तं निरूप्य च}
{विसर्जितसभालोको गृहान्तरमगान्नृपः} %॥२९॥

\twolineshloka
{अनेकधा समापृष्टो महीपालं रण्णाङ्गणे}
{शस्त्रव्रणकिणश्रेणीभूषणं सत्वसन्निभम्} %॥३०॥

\twolineshloka
{एकदा मृगयां खेलन्कुतूहलरसात्मना}
{तमारुह्य महीपालो वनं प्रति विवेश ह} %॥३१॥

\twolineshloka
{विसृज्य सैनिकान्पृष्टे धावतः परितोऽखिलान्}
{आकृष्यमाणो हरिणैः पिपासाकुलितोऽभवत्} %॥३२॥

\twolineshloka
{तत उत्तीर्य तुरगाज्जलमन्वेषयन् नृपः}
{बध्वाश्वं तरुशाखायामारुरोह शिलां नृपः} %॥३३॥

\twolineshloka
{गीतापञ्चदशाध्यायश्लोकार्द्धलिखितं ततः}
{पातितं मरुता तत्र यत्र खण्डे विलोकयत्} %॥३४॥

\twolineshloka
{पत्रं वाचयतो राज्ञः श्रुत्वा गीताक्षरावलीम्}
{ततो मुक्तिपदं लेभे तुरगस्त्वरयाऽपतत्} %॥३५॥

\twolineshloka
{ततो ग्रन्थिं समाच्छिद्य पल्याणवतार्य च}
{उत्थापमानस्तुरगो राज्ञा नोत्थितवान्मृतः} %॥३६॥

\twolineshloka
{ततः सरभभेरुण्डो नृपमाभाष्य सुस्वरम्}
{दिव्यं विमानमारुह्य जगाम त्रिदशालयम्} %॥३७॥

\twolineshloka
{ततो गिरिं समारुह्य ददर्शाश्रममुत्तमम्}
{पुन्नागकदलीचूतनारिकेलसमन्वितम्} %॥३८॥

\twolineshloka
{द्राक्षेक्षुवाटिकापूगनागकेसरचम्पकम्}
{खेलत्करभसारङ्गं नृत्यत्केकिकुलं नृपः} %॥३९॥

\twolineshloka
{प्रणिपत्य द्विजन्मानमुटजाभ्यन्तरस्थितम्}
{पप्रच्छ परया भक्त्या मुक्तसंसारवासनात्} %॥४०॥

\twolineshloka
{तुरगो निरगात्स्वर्गं हेतुना केन मे वद}
{इत्याकर्ण्य वचो राज्ञो द्विजन्मा वाचमूचिवान्} %॥४१॥

\twolineshloka
{कालेन बहुना प्रेत्य तत्पापात्तुरगोऽभवत्}
{अथ पञ्चदशाध्यायश्लोकार्द्धं लिखितं क्वचित्} %॥४२॥

\twolineshloka
{ततो वाचयतः श्रुत्वा निरगात्तुरगो दिवम्}
{ततः समागतैस्तत्र परिवारजनैर्वृतः} %॥४३॥

\twolineshloka
{प्रणिपत्य द्विजन्मानं हृष्टरोमा विनिर्गतः}
{पत्रं तदेव लिखितं गीतापञ्चदशाक्षरम्} %॥४४॥

\twolineshloka
{वाचयन्स महीपालो हर्षसम्फुल्ललोचनः}
{अभिषिच्य निजं पुत्रं मन्त्रविन्मन्त्रिभिः समम्} %॥४५॥

\onelineshloka
{सिंहासने सिंहबलं मुक्तिमाप विशुद्धधीः} %॥४६॥

॥इति श्रीपाद्मे महापुराणे पञ्चपञ्चाशत्सहस्रसंहितायामुत्तरखण्डे सतीश्वरसंवादे गीतामाहात्म्ये एकोननवत्यधिकशततमोऽध्यायः॥१८९॥

\sect{{[१६]} नवत्यधिकशततमोऽध्यायः} %6.190

\uvacha{ईश्वर उवाच}

\twolineshloka
{अतः परं प्रवक्ष्यामि षोडशाध्यायगौरवम्}
{आकर्णय कुरङ्गाक्षि हर्षोत्कण्ठप्रवर्षिणि} %॥१॥

\twolineshloka
{अस्ति सौराष्ट्रिकं नाम्ना पुरं गुर्जरमण्डले}
{तत्रासीत्खङ्गबाहुश्च राजा चन्द्र इवापरः} %॥२॥

\twolineshloka
{यदीय कुसुमामोदमालासुरभितोदरे}
{वारान्निधौ हरिः स्वास्थ्यादशेत सह पद्मया} %॥३॥

\twolineshloka
{यदीय कीर्तिकर्पूरकणाभान्तिनभोङ्गणे}
{कीर्णा वैरिकृतश्वास मारुतैस्तारकाछलात्} %॥४॥

\twolineshloka
{यस्यासिधारा तीर्थेषु स्नाता वै रिपुभूभुजः}
{व्यावर्त्तं ते दिवोऽद्यापि स्वर्गस्त्रीवाग्विमोहिताः} %॥५॥

\twolineshloka
{तस्यारिमर्दनोनाम मदहस्ती मदोद्धुरः}
{मदाम्बुधारा सलिले गुञ्जद्भ्रमरमण्डलः} %॥६॥

\twolineshloka
{कपोलफलकोत्तीर्ण मदधाराजलाविलः}
{बभौ यो निर्झरोद्गोरैरञ्जनाद्रिरिवाच्चकैः} %॥७॥

\twolineshloka
{यस्याङ्गेषु व्यराजन्त चामराश्चन्द्रिकोज्ज्वलाः}
{किरणाइव शीतांशोः पतिताः काननोदरे} %॥८॥

\twolineshloka
{सिन्दूरपांसुपटली राजत्कुम्भस्थलो बभौ}
{यः सन्ध्यावारिदव्याप्तं वियत्खण्डमिव स्थितम्} %॥९॥

\twolineshloka
{स कदाचिन्मोचयित्वा शृङ्खलान्निगडानपि}
{भङ्क्त्वा लौहदृढस्तम्भं प्रसह्य निशि निर्गतः} %॥१०॥

\twolineshloka
{आधोरणगणान्सर्वान्पार्ष्णिविस्फूर्जदङ्कुशान्}
{क्रोधादवगणय्यैव निजशालां बभञ्ज सः} %॥११॥

\twolineshloka
{तीक्ष्णाङ्कुशमुखैर्विष्वक्हन्यमानोऽपि वैणवैः}
{दण्डैस्तु त्रासयामासुः सादिनो न मनागपि} %॥१२॥

\twolineshloka
{ततो राजा समागत्य निशम्येदं कुतूहलम्}
{तत्र हस्तिकलाभिज्ञैः समं राजकुमारकैः} %॥१३॥

\twolineshloka
{अदृश्यत समागत्य राज्ञा दन्तावलो बली}
{मोहयन्नुद्भटाटोपो हृताट्टालिकमालिकः} %॥१४॥

\twolineshloka
{ददृशुस्तं महाभीमं पौरा दूरतरं स्थिताः}
{गोपायन्तः शिशून्भीत्या निवृत्तान्यकुतूहलाः} %॥१५॥

\twolineshloka
{रुद्धेषु तत्र मार्गेषु पलायनपरैर्जनैः}
{वासितेषु तदीयोग्रदानधाराम्बुसीकरैः} %॥१६॥

\twolineshloka
{स्नात्वा तेनाध्वना यातः सरसः कश्चन द्विजः}
{गीतानां षोडशाध्यायश्लोकान्कतिपयाञ्जपन्} %॥१७॥

\onelineshloka
{निषिद्ध्यमानो बहुधा पौरैराधोरणैरपि} %॥१८॥

\twolineshloka
{अमन्यमानः करिणो भीतान्चलितवांस्ततः}
{फूत्कारेण स आवृण्वञ्जनान्विपरिमर्दयन्} %॥१९॥

\twolineshloka
{स्पृशन्दानाम्बुजं तस्य स्वस्तिमान्निर्गतो द्विजः}
{ततो महानभूत्तत्र विस्मयो वागगोचरः} %॥२०॥

\twolineshloka
{मानसे भूमिपालस्य पौराणामपि पश्यताम्}
{समाहूय ततो राजा फुल्लराजीवलोचनः} %॥२१॥

\onelineshloka
{तमापृच्छद्द्विजं वाहादवतीर्य प्रणम्य च} %॥२२॥

\uvacha{राजोवाच}

\twolineshloka
{अलौकिकमिदं विप्र त्वयाद्याचरितं महत्}
{कृतान्तकल्पादेतस्मात्कथं निर्यातवान्गजात्} %॥२३॥

\twolineshloka
{कमर्चयसि गीर्वाणं कं मन्त्रं जपसि प्रभो}
{का च सिद्धिस्तवास्तीति द्विजन्मन्समुदीरय} %॥२४॥

\uvacha{द्विज उवाच}

\twolineshloka
{गीतायाः षोडशोध्यायः श्लोकान्कतिपयानहम्}
{जपामि प्रत्यहं भूप तेनैताः सर्वसिद्धयः} %॥२५॥

\twolineshloka
{ततो विहाय द्विरदं कौतूहलरसं नृपः}
{आजगाम द्विजन्मानमादाय निजमन्दिरम्} %॥२६॥

\onelineshloka
{शुभं मुहूर्तमन्वीक्ष्य तोषयित्वा द्विजोत्तमम्} %॥२७॥

\twolineshloka
{सुवर्णैर्लक्ष्यसङ्ख्याकैर्गीतामन्त्रमुपाददे}
{गीतायाः षोडशाध्यायश्लोकान्कतिपयानपि} %॥२८॥

\twolineshloka
{समभ्यस्य ततो राजा सत्कारेण सकौतुकः}
{अथैकदा विनिर्गत्य बाह्यालीं सह सैनिकैः} %॥२९॥

\twolineshloka
{तमेवामोचयद्राजा मत्तमाधोरणाद्गजम्}
{विस्पष्टमिति वाक्यानि राज्यसौख्यममानयन्} %॥३०॥

\twolineshloka
{तृणवज्जीवितं राजा गजस्याग्रेविशत्ततः}
{आदाय गण्डफलकं मदपङ्क्तिनिरङ्कुशम्} %॥३१॥

\twolineshloka
{आययौ मन्त्रिविश्वासान्नृपः साहसिकाग्रणीः}
{राहोरिव मुखादिन्दुः कालास्यादिव धार्मिकः} %॥३२॥

\twolineshloka
{साधुः खलस्य वदनान्नृयोनिरगमद्गजात्}
{आगत्य नगरं राजा त्वभिषिच्य कुमारकम्} %॥३३॥

\onelineshloka
{गीतायाः षोडशाध्यायादवाप परमां गतिम्} %॥३४॥

॥इति श्रीपाद्मे महापुराणे पञ्चपञ्चाशत्सहस्रसंहितायामुत्तरखण्डे सतीश्वरसंवादे गीतामाहात्म्ये नवत्यधिकशततमोऽध्यायः॥१९०॥

\sect{{[१७]} एकनवत्यधिकशततमोऽध्यायः} %6.191

\uvacha{ईश्वर उवाच}

\twolineshloka
{षोडशाध्यायसामर्थ्यं कथितं शृणु साम्प्रतम्}
{स्पष्टं सप्तदशाध्यायमहिमाम्भो निधिं शिवे} %॥१॥

\twolineshloka
{खड्गबाहोः सुतस्यैव भृत्यो दुःशासनोऽभवत्}
{तं गजं धर्तुमागत्य गजात्प्राप्तो यमक्षयम्} %॥२॥

\twolineshloka
{तद्वासनानिबद्धात्मा गजयोनिमवाप्य च}
{गीता सप्तदशाध्यायं श्रुत्वा प्राप्तः परं पदम्} %॥३॥

\uvacha{देव्युवाच}

\twolineshloka
{दुःशासनो गजत्वं च प्राप्य मुक्त इति श्रुतम्}
{तदेव वद कल्याण विस्तरेण मम प्रभो} %॥४॥

\uvacha{ईश्वर उवाच}

\twolineshloka
{स्थितः कश्चन दुर्मेधा मण्डलीककुमारकैः}
{बहुमूल्यं पणं कृत्वा गजमारूढवांस्ततः} %॥५॥

\twolineshloka
{गत्वा कतिपयान्येव पदानि जनवारितः}
{नाम्ना दुःशासनो मूढः प्रौढवाक्यमुदीरयन्} %॥६॥

\twolineshloka
{ततो निशम्य तद्वाक्यं क्रोधान्धः सिन्धुरोऽभवत्}
{न्यपतच्च स्खलत्पादः कम्पमानकुमारकैः} %॥७॥

\twolineshloka
{ततो निपतितं किञ्चिदुच्छ्वसन्तं गजो रुषा}
{ऊर्द्ध्वमुन्मूलयाञ्चक्रे कृतान्तकनिरङ्कुशः} %॥८॥

\twolineshloka
{गतासोरपि रोषेण तस्यास्थ्न्यां च गणं गजः}
{विकीर्णवान्पृथक्कृत्वा मत्तो दन्तावलस्ततः} %॥९॥

\twolineshloka
{ततः कालेन सम्प्रेत्य गजयोनिमवाप सः}
{महान्तं कालमनयत्सिंहलद्वीपभूपतेः} %॥१०॥

\twolineshloka
{मैत्री गरीयसी सार्द्धं खड्गबाहुमहीभुजा}
{ततोऽयं जलमार्गेण प्रापितो वारुणो मतः} %॥११॥

\twolineshloka
{जयदेवेन खड्गबाहोः प्रीत्या नीतो महीभुजा}
{जातिं स्मरन्स्वकीयां स पश्यन्बन्धून्सहोदरान्} %॥१२॥

\twolineshloka
{दुःखेन महतास्तोकान्दिवसानत्यवाहयत्}
{उवास खड्गबाहोश्च गृहे तूष्णीमनिर्दिशन्} %॥१३॥

\twolineshloka
{स कदाचित्तु सन्तुष्टः समस्याश्लोकपूरणे}
{कस्मैचित्कवये प्रादात्तमुपायन हस्तिनम्} %॥१४॥

\twolineshloka
{शतेन तेन कविना रोगोपद्रवभीरुणा}
{मालवक्षोणिपालस्य विक्रीतश्चैत्यकुञ्जरः} %॥१५॥

\twolineshloka
{कियत्यपि गते काले पाल्यमानोऽपि यत्नतः}
{मुमूर्षुरभवत्तत्र कुञ्जरो दुर्जरज्वरः} %॥१६॥

\threelineshloka
{न जिघ्रति पयः शीतं नादत्ते कवलं गजः}
{स्वपित्यपि न सौख्येन मुञ्चत्यश्रूणि केवलम्}
{ततो हस्तिपकाख्यातं वृत्तान्तमवनीपतिः} %॥१७॥

\twolineshloka
{आकर्ण्य स समायातो यत्रास्ते ज्वरितो गजः}
{स चावलोक्य भूपालं जगद्विस्मयकारिणीम्} %॥१८॥

\twolineshloka
{वाचमूचे गजः सम्यग्विसृष्टज्वरवेदनः}
{राजन्नशेषशास्त्रज्ञ राजनीतिपयोनिधे} %॥१९॥

\twolineshloka
{निर्जितारातिसङ्घात मुरारिचरणप्रिय}
{किमौषधैरलं वैद्यैः किं धानैः किं नु जापकैः} %॥२०॥

\twolineshloka
{गीतासप्तदशाध्यायजापकं द्विजमानय}
{तेनायं मामको रोगः शाम्यत्यत्र न संशयः} %॥२१॥

\twolineshloka
{यथादिष्टं गजेनासौ तथा चक्रे नराधिपः}
{ततो गजत्वमुत्सृज्य मुक्तो दुःशासनोऽभवत्} %॥२२॥

\twolineshloka
{तेन विप्रेणाभिमन्त्र्य जले क्षिप्ते तदुत्तमे}
{अथ दिव्यं समारूढं विमानमवनीपतिः} %॥२३॥

\onelineshloka
{तं दुःशासनमद्राक्षीत्पाकशासनतेजसम्} %॥२४॥

\uvacha{राजोवाच}

\twolineshloka
{किं जातीयः किमात्मा त्वं किं वृत्त इति मे वद}
{केन वा कर्मणा जातो गजः कथमिहागतः} %॥२५॥

\twolineshloka
{पृष्टो राज्ञा विमुक्तोऽसौ विमानस्थः स्थिराक्षरम्}
{वृत्तं यथा यदाचख्यौ निजं दुःशासनः क्रमात्} %॥२६॥

\twolineshloka
{गीतासप्तदशाध्यायं जपन्मालवभूपतिः}
{नरवर्माभवन्मुक्तः कालेनाल्पीयसा ततः} %॥२७॥

॥इति श्रीपाद्मे महापुराणे पञ्चपञ्चाशत्सहस्रसंहितायामुत्तरखण्डे सतीश्वरसंवादे गीतामाहात्म्ये एकनवत्यधिकशततमोऽध्यायः॥१९१॥

\sect{{[१८]} द्विनवत्यधिकशततमोऽध्यायः} %6.192

\uvacha{पावर्त्युवाच}

\twolineshloka
{उक्तं सप्तदशाध्यायगौरवं भवता शिव}
{स त्वमष्टादशाध्यायमहिमानमुदीरय} %॥१॥

\uvacha{ईश्वर उवाच}

\twolineshloka
{आकर्णय चिदानन्द निष्पादि निगमोत्तरम्}
{पुण्यमष्टादशाध्यायमाहात्म्यं गिरिनन्दिनि} %॥२॥

\twolineshloka
{समस्तशास्त्रसर्वस्वं श्रोत्रप्राप्तं रसायनम्}
{संसारयातनाजालविदारणपरायणम्} %॥३॥

\twolineshloka
{परं रहस्यं सिद्धानामविद्योन्मूलनक्षमम्}
{चैतन्यं कैटभारातेरग्रगण्यं परं पदम्} %॥४॥

\twolineshloka
{विवेकवल्लरीमूलं कामक्रोधमलापहम्}
{पुरन्दरादिगीर्वाणचित्तविश्रामकारकम्} %॥५॥

\twolineshloka
{सनकादिमहायोगिमनोरञ्जनकारणम्}
{पाठमात्रपराभूतकालकिङ्करगर्जितम्} %॥६॥

\twolineshloka
{अष्टोत्तरशतव्याधिमूलोन्मूलनकारकम्}
{नातः परतरं किञ्चिद्रहस्यं हंसगामिनि} %॥७॥

\twolineshloka
{उपतापत्रयहरं महापातकनाशनम्}
{यथाकालेष्वहं नित्यो यथा पशुषु कामधुक्} %॥८॥

\twolineshloka
{यथा व्यासो मुनीन्द्रेषु व्यासेषु ब्रह्मवित्तमः}
{यथा दिवौकसां शक्रो गुरुः शक्राद्यथा वरः} %॥९॥

\twolineshloka
{यथा रसानां पीयूषमुत्तमं विश्वविश्रुतम्}
{तथा गिरीणां कैलासो यथा चेन्द्रो दिवौकसाम्} %॥१०॥

\twolineshloka
{तीर्थानां पुष्करं तीर्थं यथा पुष्पेषु पङ्कजम्}
{पतिव्रतासु नारीषु यथा लोकेष्वरुन्धती} %॥११॥

\twolineshloka
{यथा मखेष्वश्वमेधो यथोद्यानेषु नन्दनम्}
{यथा रुद्रेषु सर्वेषु वीरभद्रो ममानुगः} %॥१२॥

\twolineshloka
{यथा दानेषु भूदानं यथा सिन्धुषु गौतमी}
{यथा लोके हरिक्षेत्रं प्रशस्तं धर्मकर्मसु} %॥१३॥

\twolineshloka
{तथैवाष्टादशाध्यायमाहात्म्यं भुवनोत्तरम्}
{तत्राऽऽख्यानमिदं पुण्यं भक्त्याऽऽकर्णय पार्वति} %॥१४॥

\twolineshloka
{यथाकर्णनमात्रेण जन्तुः पापैः प्रमुच्यते}
{अस्ति मेरुगिरेः शृङ्गे पुरी रम्याऽमरावती} %॥१५॥

\twolineshloka
{पुरा मम विनोदाय निर्मिता विश्वकर्मणा}
{निरन्तरं गुणयुता कोटिगीर्वाणसेविता} %॥१६॥

\twolineshloka
{तेजःपुञ्जवती साक्षाद् ब्रह्मविद्येव विश्रुता}
{चिन्तामणिशिलाबद्धाः प्रासादा यत्र कामदाः} %॥१७॥

\twolineshloka
{जयन्ति मेरुशिखरे चतुर्मुखपुरावधि}
{यत्र कल्पद्रुमच्छाया सुखासीना पुलोमजा} %॥१८॥

\twolineshloka
{शृणोति श्यामला गीर्भिर्गीतं गन्धर्वयोषिताम्}
{यत्राऽऽस्ते सबलारातिर्देवैश्च दलितायुषाम्} %॥१९॥

\twolineshloka
{दैत्यानां रक्तकल्लोलैः स्वर्धुनी शोणतां गता}
{पुरातनसुधास्वादु स्मारं स्मारं दिवौकसः} %॥२०॥

\twolineshloka
{धयन्ति यत्र क्षुत्क्षामा कलां प्रत्यहमैन्दवीम्}
{तस्यां कैवल्यकल्पायामासीत्पूर्वं शतक्रतुः} %॥२१॥

\twolineshloka
{शची समन्वितः श्रीमान्सर्वगीर्वाणसेवितः}
{स कदाचित्सुखासीनो विष्णुदूतैरधिश्रितम्} %॥२२॥

\twolineshloka
{सहस्रनेत्रमायान्तमपश्यत्पुरुषं परम्}
{ततस्तदीय तेजोभिरभिभूतः पुरन्दरः} %॥२३॥

\twolineshloka
{मणिसिंहासनात्तूर्णं पतितः स्थानमण्डपे}
{सिंहासनात्प्रयातस्य ततस्तस्य हरेर्भटाः} %॥२४॥

\twolineshloka
{गीर्वाणगणसाम्राज्यपट्टबन्धं वितेनिरे}
{अथ तस्याभिषिक्तस्य महेन्द्रस्य पुलोमजा} %॥२५॥

\twolineshloka
{वामाङ्कमारुरोहाऽऽशु दिव्यदुन्दुम्भिनिःस्वनैः}
{अथ त्रिदिवसङ्गीतगीर्वाणाः प्रमदान्विताः} %॥२६॥

\twolineshloka
{सुरा नीराजयामासुर्दिव्यरत्नैः सुरस्त्रियः}
{ततो वै ऋषयो वेदैराशीर्वादास्तदा ददुः} %॥२७॥

\twolineshloka
{रम्भाद्या ननृतुस्तस्य पुरस्तादप्सरोगणाः}
{गन्धर्वा ललितालापाञ्जगुर्मङ्गलकौतुकम्} %॥२८॥

\twolineshloka
{एवं नूतनमिन्द्रं तु जुष्टं बहुभिरुत्सवैः}
{विना शतक्रतुं दृष्ट्वा शक्रो विस्मयमाययौ} %॥२९॥

\twolineshloka
{न मया विहिता मार्गे तडागा न विनिर्मिताः}
{न रोपिता महावृक्षाः पान्थविश्रान्तिकारकः} %॥३०॥

\twolineshloka
{न कदाचिदहो दृष्टो देवस्त्रिपुरभैरवः}
{निधिवासे स्थिता देवी पूजिता न मदालसा} %॥३१॥

\twolineshloka
{मेघङ्करस्थितः शार्ङ्गधरो नैव निरीक्षितः}
{न कृतं विरजे स्नानं नैवं काशीं पुरीं गतः} %॥३२॥

\twolineshloka
{न देववाटिकावासी दृष्टो नरहरिः स्वयम्}
{एरण्डविष्णुर्हेरम्बो न जातु परिशीलितः} %॥३३॥

\twolineshloka
{रेणुका नेक्षिता जातु माता पुरनिवासिनी}
{न भक्त्या पूजिता देवी दाना पुरनिवासिनी} %॥३४॥

\twolineshloka
{न भक्त्या त्रिपुरे दृष्टस्त्रिलिङ्गस्त्र्यम्बकः स्वयम्}
{न शार्दूलतडागस्थो वीक्षितः सोमनाथकः} %॥३५॥

\twolineshloka
{रेवापुरस्थितो देवो घुसृणेशो न वीक्षितः}
{नागदन्तपुरे ख्यातो नागनाथो न वेक्षितः} %॥३६॥

\twolineshloka
{पर्णग्रामस्थितो दृष्टो न महानमृतेश्वरः}
{न तुङ्गभद्रा तीरस्थो दृष्टो हरिहरः स्वयम्} %॥३७॥

\twolineshloka
{न वेङ्कटाद्रिनिलयः श्रीनिवासः सुलक्षितः}
{कावेरी कर्णिकातीरे श्रीरङ्गो नैव वीक्षितः} %॥३८॥

\twolineshloka
{दीनास्त्वनाथाः क्रोशन्तः कारागारान्न मोचिताः}
{अन्नदानेन दौर्भिक्षे प्राणिनो नैव पूजिताः} %॥३९॥

\onelineshloka
{रात्रौ रात्रौ कृता क्वापि निर्जले नोदकप्रभा} %॥४०॥

\twolineshloka
{न गौतम्यां कृतं स्नानं न दृष्टो हरिणेश्वरः}
{कृष्णवेण्यां च न कृतं स्नानं कन्यागते गुरौ} %॥४१॥

\twolineshloka
{दत्तं नो भूखण्डमपि कवयो नैव पूजिताः}
{न तीर्थेषु कृतं सत्रं न ग्रामेषु कृता मखाः} %॥४२॥

\twolineshloka
{पुष्करिण्यो न विहिता मध्ये मार्गं बहूदकाः}
{न प्रासादाः कृताः क्वापि ब्रह्मविष्णुपिनाकिनाम्} %॥४३॥

\twolineshloka
{न जातुचिद्भयाक्रान्ता रक्षिताः शरणागताः}
{कथमेकेन पुण्येन देवदत्तमिहार्जितम्} %॥४४॥

\twolineshloka
{इति चिन्ताकुलो भूत्वा हरिं प्रष्टुं पुरन्दरः}
{ययौ सरभसङ्खिन्नः क्षीरकूपारगह्वरम्} %॥४५॥

\twolineshloka
{तत्र प्रविश्य गोविन्दं कृतनिद्रं स वासवः}
{अकस्मान्निजसाम्राज्यभ्रंशदुःखमुवाच ह} %॥४६॥

\uvacha{इन्द्र उवाच}

\twolineshloka
{रमाकान्त भवत्प्रीत्यै कृतं क्रतुशतं पुरा}
{तेन पुण्येन सम्प्राप्तं मया पौरन्दरं पदम्} %॥४७॥

\twolineshloka
{इदानीं नूतनः कोऽपि जातो दिवि पुरन्दरः}
{न तेन धर्मो विहितो न तेन क्रतवः कृताः} %॥४८॥

\onelineshloka
{मम सिंहासनं दिव्यं कथमाक्रान्तमच्युत} %॥४९॥

\uvacha{महादेव उवाच}

\twolineshloka
{इत्येवं वदतस्तस्य श्रुत्वा वाचं रमापतिः}
{उन्मीलितस्मिताक्षोऽसावुवाच मधुरं वचः} %॥५०॥

\uvacha{श्रीभगवानुवाच}

\twolineshloka
{किं दानैरल्पफलदैः किं तपोभिः किमध्वरैः}
{वर्तमानः क्षितितले सत्वं प्रीणितवान्पुरा} %॥५१॥

\uvacha{इन्द्र उवाच}

\twolineshloka
{भगवन्कर्मणा केन स त्वां प्रीणितवान्द्विजः}
{यत्प्रीत्या भगवान्प्रादात्तस्मै पौरन्दरं पदम्} %॥५२॥

\uvacha{श्रीभगवानुवाच}

\twolineshloka
{जपत्यष्टादशाध्याये गीतानां श्लोकपञ्चकम्}
{यत्पुण्येन च सम्प्राप्तं तव साम्राज्यमुत्तमम्} %॥५३॥

\twolineshloka
{सर्वपुण्यशिरोरत्नभूतेन त्वं स्थिरो भव}
{इति विष्णोर्वचः श्रुत्वा ज्ञातोपायं पुरन्दरः} %॥५४॥

\twolineshloka
{विप्रवेषधरो भूत्वा गतो गोदावरीतटम्}
{तत्रापश्यत्पुरं पुण्यं कालिकाग्राममुत्तमम्} %॥५५॥

\twolineshloka
{यत्र कालेश्वरो देवो वर्तते कालमर्दनः}
{तत्र गोदावरी तीरे स्थितं परमधार्मिकम्} %॥५६॥

\twolineshloka
{अपश्यत्करुणावन्तं ब्राह्मणं वेदपारगम्}
{नित्यमष्टादशाध्यायं जपन्तं दान्तचेतसम्} %॥५७॥

\twolineshloka
{ततस्तच्चरणद्वन्द्वे लुठित्वा परया मुदा}
{सत्वमष्टादशाध्यायमपठत्तेन शिक्षितम्} %॥५८॥

\twolineshloka
{अथ पुण्येन तेनाऽसौ विष्णोः सायुज्यमाययौ}
{हित्वा पुरन्दरादीनां देवानां पदमल्पकम्} %॥५९॥

\twolineshloka
{ज्ञात्वाऽतीव मुदायुक्तो वैकुण्ठमगमत्पुरम्}
{अत एवं परं तत्त्वं मुनीनामिदमुत्तमम्} %॥६०॥

\twolineshloka
{दिव्यमष्टादशाध्यायमाहात्म्यं कथितं मया}
{यस्य श्रवणमात्रेण सर्वपापैः प्रमुच्यते} %॥६१॥

\twolineshloka
{इत्येवं गीतामाहात्म्यं कथितं पापनाशनम्}
{पुण्यं पवित्रमायुष्यं स्वर्ग्यं स्वस्त्ययनं महत्} %॥६२॥

\twolineshloka
{यः शृणोति महाभागे श्रद्धया संयुतः पुमान्}
{सर्वयज्ञफलं प्राप्य विष्णोः सायुज्यमाप्नुयात्} %॥६३॥

॥इति श्रीपाद्मे महापुराणे पञ्चपञ्चाशत्सहस्रसंहितायामुत्तरखण्डे सतीश्वरसंवादे गीतामाहात्म्ये द्विनवत्यधिकशततमोऽध्यायः॥१९२॥
