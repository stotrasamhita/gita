% !TeX program = XeLaTeX
% !TeX root = gitabook.tex
\chapt{प्रथमोऽध्यायः\textsf{---}अर्जुनविषादयोगः}

\uvacha{धृतराष्ट्र उवाच}

\twolineshloka
{धर्मक्षेत्रे कुरुक्षेत्रे समवेताः युयुत्सवः}
{मामकाः पाण्डवाः च एव किम् अकुर्वत सञ्जय}% .. 1-1

\uvacha{सञ्जय उवाच}

\twolineshloka
{दृष्ट्वा तु पाण्डवानीकं व्यूढं दुर्योधनः तदा}
{आचार्यम् उपसङ्गम्य राजा वचनम् अब्रवीत्}% .. 1-2

\twolineshloka
{पश्य एतां पाण्डुपुत्राणाम् आचार्य महतीं चमूम्}
{व्यूढां द्रुपदपुत्रेण तव शिष्येण धीमता}% .. 1-3

\twolineshloka
{अत्र शूराः महेष्वासाः भीमार्जुनसमाः युधि}
{युयुधानः विराटः च द्रुपदः च महारथः}% .. 1-4

\twolineshloka
{धृष्टकेतुः चेकितानः काशिराजः च वीर्यवान्}
{पुरुजित् कुन्तिभोजः च शैब्यः च नरपुङ्गवः}% .. 1-5

\twolineshloka
{युधामन्युः च विक्रान्तः उत्तमौजाः च वीर्यवान्}
{सौभद्रः द्रौपदेयाः च सर्वे एव महारथाः}% .. 1-6

\twolineshloka
{अस्माकं तु विशिष्टाः ये तान् निबोध द्विजोत्तम}
{नायकाः मम सैन्यस्य संज्ञार्थं तान् ब्रवीमि ते}% .. 1-7

\twolineshloka
{भवान् भीष्मः च कर्णः च कृपः च समितिञ्जयः}
{अश्वत्थामा विकर्णः च सौमदत्तिः तथा एव च}% .. 1-8

\twolineshloka
{अन्ये च बहवः शूराः मदर्थे त्यक्तजीविताः}
{नानाशस्त्रप्रहरणाः सर्वे युद्धविशारदाः}% .. 1-9

\twolineshloka
{अपर्याप्तं तत् अस्माकं बलं भीष्माभिरक्षितम्}
{पर्याप्तं तु इदम् एतेषां बलं भीमाभिरक्षितम्}% .. 1-10

\twolineshloka
{अयनेषु च सर्वेषु यथाभागमवस्थिताः}
{भीष्मम् एव अभिरक्षन्तु भवन्तः सर्वे एव हि}% .. 1-11

\twolineshloka
{तस्य सञ्जनयन् हर्षं कुरुवृद्धः पितामहः}
{सिंहनादं विनद्य उच्चैः शङ्खं दध्मौ प्रतापवान्}% .. 1-12

\twolineshloka
{ततः शङ्खाः च भेर्यः च पणवानकगोमुखाः}
{सहसा एव अभ्यहन्यन्त सः शब्दस्तुमुलः अभवत्}% .. 1-13

\twolineshloka
{ततः श्वेतैः हयैः युक्ते महति स्यन्दने स्थितौ}
{माधवः पाण्डवः च एव दिव्यौ शङ्खौ प्रदध्मतुः}% .. 1-14

\twolineshloka
{पाञ्चजन्यं हृषीकेशः देवदत्तं धनञ्जयः}
{पौण्ड्रं दध्मौ महाशङ्खं भीमकर्मा वृकोदरः}% .. 1-15

\twolineshloka
{अनन्तविजयं राजा कुन्तीपुत्रः युधिष्ठिरः}
{नकुलः सहदेवः च सुघोषमणिपुष्पकौ}% .. 1-16

\twolineshloka
{काश्यः च परमेष्वासः शिखण्डी च महारथः}
{धृष्टद्युम्नः विराटः च सात्यकिः च अपराजितः}% .. 1-17

\twolineshloka
{द्रुपदः द्रौपदेयाः च सर्वशः पृथिवीपते}
{सौभद्रः च महाबाहुः शङ्खान् दध्मुः पृथक्पृथक्}% .. 1-18

\twolineshloka
{सः घोषः धार्तराष्ट्राणां हृदयानि व्यदारयत्}
{नभः च पृथिवीं च एव तुमुलः अभ्यनुनादयन्}% .. 1-19

\twolineshloka
{अथ व्यवस्थितान् दृष्ट्वा धार्तराष्ट्रान् कपिध्वजः}
{प्रवृत्ते शस्त्रसम्पाते धनुः उद्यम्य पाण्डवः}% .. 1-20

\onelineshloka*
{हृषीकेशं तदा वाक्यम् इदम् आह महीपते}

\uvacha{अर्जुन उवाच}

\onelineshloka
{सेनयोः उभयोः मध्ये रथं स्थापय मे अच्युत}% .. 1-21

\twolineshloka
{यावत् एतान् निरीक्षे अहं योद्धुकामान् अवस्थितान्}
{कैः मया सह योद्धव्यम् अस्मिन् रणसमुद्यमे}% .. 1-22

\twolineshloka
{योत्स्यमानान् अवेक्षे अहं ये एते अत्र समागताः}
{धार्तराष्ट्रस्य दुर्बुद्धेः युद्धे प्रियचिकीर्षवः}% .. 1-23

\uvacha{सञ्जय उवाच}

\twolineshloka
{एवम् उक्तः हृषीकेशः गुडाकेशेन भारत}
{सेनयोः उभयोर्मध्ये स्थापयित्वा रथोत्तमम्}% .. 1-24

\twolineshloka
{भीष्मद्रोणप्रमुखतः सर्वेषां च महीक्षिताम्}
{उवाच पार्थ पश्य एतान् समवेतान् कुरून् इति}% .. 1-25

\twolineshloka
{तत्र अपश्यत् स्थितान् पार्थः पितॄन् अथ पितामहान्}
{आचार्यान् मातुलान् भ्रातॄन् पुत्रान् पौत्रान् सखीन् तथा}% .. 1-26

\twolineshloka
{श्वशुरान् सुहृदः च एव सेनयोः उभयोः अपि}
{तान् समीक्ष्य सः कौन्तेयः सर्वान् बन्धून् अवस्थितान्}% .. 1-27

\onelineshloka*
{कृपया परया आविष्टः विषीदन् इदम् अब्रवीत्}

\uvacha{अर्जुन उवाच}

\onelineshloka
{दृष्ट्वा इमं स्वजनं कृष्ण युयुत्सुं समुपस्थितम्}% .. 1-28

\twolineshloka
{सीदन्ति मम गात्राणि मुखं च परिशुष्यति}
{वेपथुः च शरीरे मे रोमहर्षः च जायते}% .. 1-29

\twolineshloka
{गाण्डीवं स्रंसते हस्तात् त्वक् च एव परिदह्यते}
{न च शक्नोमि अवस्थातुं भ्रमति इव च मे मनः}% .. 1-30

\twolineshloka
{निमित्तानि च पश्यामि विपरीतानि केशव}
{न च श्रेयः अनुपश्यामि हत्वा स्वजनम् आहवे}% .. 1-31

\twolineshloka
{न काङ्क्षे विजयं कृष्ण न च राज्यं सुखानि च}
{किं नः राज्येन गोविन्द किं भोगैः जीवितेन वा}% .. 1-32

\twolineshloka
{येषाम् अर्थे काङ्क्षितं नः राज्यं भोगाः सुखानि च}
{ते इमे अवस्थिता युद्धे प्राणान् त्यक्त्वा धनानि च}% .. 1-33

\twolineshloka
{आचार्याः पितरः पुत्राः तथा एव च पितामहाः}
{मातुलाः श्वशुराः पौत्राः श्यालाः सम्बन्धिनः तथा}% .. 1-34

\twolineshloka
{एतान् न हन्तुम् इच्छामि घ्नतः अपि मधुसूदन}
{अपि त्रैलोक्यराज्यस्य हेतोः किं नु महीकृते}% .. 1-35

\twolineshloka
{निहत्य धार्तराष्ट्रान् नः का प्रीतिः स्यात् जनार्दन}
{पापम् एव आश्रयेत् अस्मान् हत्वा एतान् आततायिनः}% .. 1-36

\twolineshloka
{तस्मान् न अर्हाः वयं हन्तुं धार्तराष्ट्रान् स्वबान्धवान्}
{स्वजनं हि कथं हत्वा सुखिनः स्याम माधव}% .. 1-37

\twolineshloka
{यदि अपि एते न पश्यन्ति लोभोपहतचेतसः}
{कुलक्षयकृतं दोषं मित्रद्रोहे च पातकम्}% .. 1-38

\twolineshloka
{कथं न ज्ञेयम् अस्माभिः पापात् अस्मान् निवर्तितुम्}
{कुलक्षयकृतं दोषं प्रपश्यद्भिः जनार्दन}% .. 1-39

\twolineshloka
{कुलक्षये प्रणश्यन्ति कुलधर्माः सनातनाः}
{धर्मे नष्टे कुलं कृत्स्नम् अधर्मः अभिभवति उत}% .. 1-40

\twolineshloka
{अधर्माभिभवात् कृष्ण प्रदुष्यन्ति कुलस्त्रियः}
{स्त्रीषु दुष्टासु वार्ष्णेय जायते वर्णसङ्करः}% .. 1-41

\twolineshloka
{सङ्करः नरकाय एव कुलघ्नानां कुलस्य च}
{पतन्ति पितरः हि एषां लुप्तपिण्डोदकक्रियाः}% .. 1-42

\twolineshloka
{दोषैः एतैः कुलघ्नानां वर्णसङ्करकारकैः}
{उत्साद्यन्ते जातिधर्माः कुलधर्माः च शाश्वताः}% .. 1-43

\twolineshloka
{उत्सन्नकुलधर्माणां मनुष्याणां जनार्दन}
{नरके अनियतं वासः भवति इति अनुशुश्रुम}% .. 1-44

\twolineshloka
{अहो बत महत्पापं कर्तुं व्यवसिताः वयम्}
{यत् राज्यसुखलोभेन हन्तुं स्वजनम् उद्यताः}% .. 1-45

\twolineshloka
{यदि माम् अप्रतीकारम् अशस्त्रं शस्त्रपाणयः}
{धार्तराष्ट्राः रणे हन्युः तत् मे क्षेमतरं भवेत्}% .. 1-46

\uvacha{सञ्जय उवाच}

\twolineshloka
{एवम् उक्त्वा अर्जुनः सङ्ख्ये रथोपस्थे उपाविशत्}
{विसृज्य सशरं चापं शोकसंविग्नमानसः}% .. 1-47

{॥ॐ तत्सदिति श्रीमद्भगवद्गीतासूपनिषत्सु ब्रह्मविद्यायां योगशास्त्रे श्रीकृष्णार्जुनसंवादे अर्जुनविषादयोगो नाम प्रथमोऽध्यायः॥}

\chapt{द्वितीयोऽध्यायः\textsf{---}साङ्ख्ययोगः}

\uvacha{सञ्जय उवाच}

\twolineshloka
{तं तथा कृपया आविष्टम् अश्रुपूर्णाकुलेक्षणम्}
{विषीदन्तम् इदं वाक्यम् उवाच मधुसूदनः}% .. 2-1

\uvacha{श्रीभगवानुवाच}

\twolineshloka
{कुतः त्वा कश्मलम् इदं विषमे समुपस्थितम्}
{अनार्यजुष्टम् अस्वर्ग्यम् अकीर्तिकरम् अर्जुन}% .. 2-2

\twolineshloka
{क्लैब्यं मा स्म गमः पार्थ न एतत् त्वयि उपपद्यते}
{क्षुद्रं हृदयदौर्बल्यं त्यक्त्वा उत्तिष्ठ परन्तप}% .. 2-3

\uvacha{अर्जुन उवाच}

\twolineshloka
{कथं भीष्मम् अहं सङ्ख्ये द्रोणं च मधुसूदन}
{इषुभिः प्रतियोत्स्यामि पूजार्हौ अरिसूदन}% .. 2-4

\fourlineindentedshloka
{गुरून् अहत्वा हि महानुभावान्}
{श्रेयः भोक्तुं भैक्ष्यम् अपि इह लोके}
{हत्वा अर्थकामान् तु गुरून् इह एव}
{भुञ्जीय भोगान् रुधिरप्रदिग्धान्}% .. 2-5

\fourlineindentedshloka
{न च एतत् विद्मः कतरत् नः गरीयः}
{यत् वा जयेम यदि वा नः जयेयुः}
{यान् एव हत्वा न जिजीविषामः}
{ते अवस्थिताः प्रमुखे धार्तराष्ट्राः}% .. 2-6

\fourlineindentedshloka
{कार्पण्यदोषोपहतस्वभावः}
{पृच्छामि त्वां धर्मसम्मूढचेताः}
{यत् श्रेयः स्यात् निश्चितं ब्रूहि तत् मे}
{शिष्यः ते अहं शाधि मां त्वां प्रपन्नम्}% .. 2-7

\fourlineindentedshloka
{न हि प्रपश्यामि मम अपनुद्यात्}
{यत् शोकम् उच्छोषणम् इन्द्रियाणाम्}
{अवाप्य भूमौ असपत्नम् ऋद्धं}
{राज्यं सुराणाम् अपि च आधिपत्यम्}% .. 2-8

\uvacha{सञ्जय उवाच}

\twolineshloka
{एवम् उक्त्वा हृषीकेशं गुडाकेशः परन्तपः}
{न योत्स्ये इति गोविन्दम् उक्त्वा तूष्णीं बभूव ह}% .. 2-9

\twolineshloka
{तम् उवाच हृषीकेशः प्रहसन् इव भारत}
{सेनयोः उभयोः मध्ये विषीदन्तम् इदं वचः}% .. 2-10

\uvacha{श्रीभगवानुवाच}

\twolineshloka
{अशोच्यान् अन्वशोचः त्वं प्रज्ञावादान् च भाषसे}
{गतासून् अगतासून् च न अनुशोचन्ति पण्डिताः}% .. 2-11

\twolineshloka
{न तु एव अहं जातु न आसं न त्वं न इमे जनाधिपाः}
{न च एव न भविष्यामः सर्वे वयम् अतः परम्}% .. 2-12

\twolineshloka
{देहिनः अस्मिन् यथा देहे कौमारं यौवनं जरा}
{तथा देहान्तरप्राप्तिः धीरः तत्र न मुह्यति}% .. 2-13

\twolineshloka
{मात्रास्पर्शाः तु कौन्तेय शीतोष्णसुखदुःखदाः}
{आगमापायिनः अनित्याः तान् तितिक्षस्व भारत}% .. 2-14

\twolineshloka
{यं हि न व्यथयन्ति एते पुरुषं पुरुषर्षभ}
{समदुःखसुखं धीरं सः अमृतत्वाय कल्पते}% .. 2-15

\twolineshloka
{न असतः विद्यते भावः न अभावः विद्यते सतः}
{उभयोः अपि दृष्टः अन्तः तु अनयोः तत्त्वदर्शिभिः}% .. 2-16

\twolineshloka
{अविनाशि तु तत् विद्धि येन सर्वम् इदं ततम्}
{विनाशम् अव्ययस्य अस्य न कश्चित् कर्तुम् अर्हति}% .. 2-17

\twolineshloka
{अन्तवन्तः इमे देहाः नित्यस्य उक्ताः शरीरिणः}
{अनाशिनः अप्रमेयस्य तस्मात् युध्यस्व भारत}% .. 2-18

\twolineshloka
{यः एनं वेत्ति हन्तारं यः च एनं मन्यते हतम्}
{उभौ तौ न विजानीतः न अयं हन्ति न हन्यते}% .. 2-19

\fourlineindentedshloka
{न जायते म्रियते वा कदाचित्}
{न अयं भूत्वा भविता वा न भूयः}
{अजः नित्यः शाश्वतः अयं पुराणः}
{न हन्यते हन्यमाने शरीरे}% .. 2-20

\twolineshloka
{वेद अविनाशिनं नित्यं यः एनम् अजम् अव्ययम्}
{कथं सः पुरुषः पार्थ कं घातयति हन्ति कम्}% .. 2-21

\fourlineindentedshloka
{वासांसि जीर्णानि यथा विहाय}
{नवानि गृह्णाति नरः अपराणि}
{तथा शरीराणि विहाय जीर्णानि}
{अन्यानि संयाति नवानि देही}% .. 2-22

\twolineshloka
{न एनं छिन्दन्ति शस्त्राणि न एनं दहति पावकः}
{न च एनं क्लेदयन्ति आपः न शोषयति मारुतः}% .. 2-23

\twolineshloka
{अच्छेद्यः अयम् अदाह्यः अयम् अक्लेद्यः अशोष्यः एव च}
{नित्यः सर्वगतः स्थाणुः अचलः अयं सनातनः}% .. 2-24

\twolineshloka
{अव्यक्तः अयम् अचिन्त्यः अयम् अविकार्यः अयम् उच्यते}
{तस्मात् एवं विदित्वा एनं न अनुशोचितुम् अर्हसि}% .. 2-25

\twolineshloka
{अथ च एनं नित्यजातं नित्यं वा मन्यसे मृतम्}
{तथा अपि त्वं महाबाहो न एवं शोचितुमर्हसि}% .. 2-26

\twolineshloka
{जातस्य हि ध्रुवः मृत्युः ध्रुवं जन्म मृतस्य च}
{तस्मात् अपरिहार्ये अर्थे न त्वं शोचितुम् अर्हसि}% .. 2-27

\twolineshloka
{अव्यक्तादीनि भूतानि व्यक्तमध्यानि भारत}
{अव्यक्तनिधनानि एव तत्र का परिदेवना}% .. 2-28

\fourlineindentedshloka
{आश्चर्यवत् पश्यति कश्चित् एनम्}
{आश्चर्यवत् वदति तथा एव च अन्यः}
{आश्चर्यवत् च एनम् अन्यः शृणोति}
{श्रुत्वा अपि एनं वेद न च एव कश्चित्}% .. 2-29

\twolineshloka
{देही नित्यम् अवध्यः अयं देहे सर्वस्य भारत}
{तस्मात् सर्वाणि भूतानि न त्वं शोचितुम् अर्हसि}% .. 2-30

\twolineshloka
{स्वधर्मम् अपि च अवेक्ष्य न विकम्पितुम् अर्हसि}
{धर्म्यात् हि युद्धात् श्रेयः अन्यत् क्षत्रियस्य न विद्यते}% .. 2-31

\twolineshloka
{यदृच्छया च उपपन्नं स्वर्गद्वारम् अपावृतम्}
{सुखिनः क्षत्रियाः पार्थ लभन्ते युद्धम् ईदृशम्}% .. 2-32

\twolineshloka
{अथ चेत् त्वम् इमं धर्म्यं सङ्ग्रामं न करिष्यसि}
{ततः स्वधर्मं कीर्तिं च हित्वा पापम् अवाप्स्यसि}% .. 2-33

\twolineshloka
{अकीर्तिं च अपि भूतानि कथयिष्यन्ति ते अव्ययाम्}
{सम्भावितस्य च अकीर्तिः मरणात् अतिरिच्यते}% .. 2-34

\twolineshloka
{भयात् रणात् उपरतं मंस्यन्ते त्वां महारथाः}
{येषां च त्वं बहुमतः भूत्वा यास्यसि लाघवम्}% .. 2-35

\twolineshloka
{अवाच्यवादान् च बहून् वदिष्यन्ति तव अहिताः}
{निन्दन्तः तव सामर्थ्यं ततः दुःखतरं नु किम्}% .. 2-36

\twolineshloka
{हतः वा प्राप्स्यसि स्वर्गं जित्वा वा भोक्ष्यसे महीम्}
{तस्मात् उत्तिष्ठ कौन्तेय युद्धाय कृतनिश्चयः}% .. 2-37

\twolineshloka
{सुखदुःखे समे कृत्वा लाभालाभौ जयाजयौ}
{ततः युद्धाय युज्यस्व न एवं पापम् अवाप्स्यसि}% .. 2-38

\twolineshloka
{एषा ते अभिहिता साङ्ख्ये बुद्धिः योगे तु इमां शृणु}
{बुद्ध्या युक्तः यया पार्थ कर्मबन्धं प्रहास्यसि}% .. 2-39

\twolineshloka
{न इह अभिक्रमनाशः अस्ति प्रत्यवायः न विद्यते}
{स्वल्पम् अपि अस्य धर्मस्य त्रायते महतः भयात्}% .. 2-40

\twolineshloka
{व्यवसायात्मिका बुद्धिः एक इह कुरुनन्दन}
{बहुशाखाः हि अनन्ताः च बुद्धयः अव्यवसायिनाम्}% .. 2-41

\twolineshloka
{याम् इमां पुष्पितां वाचं प्रवदन्ति अविपश्चितः}
{वेदवादरताः पार्थ न अन्यत् अस्ति इति वादिनः}% .. 2-42

\twolineshloka
{कामात्मानः स्वर्गपराः जन्मकर्मफलप्रदाम्}
{क्रियाविशेषबहुलां भोगैश्वर्यगतिं प्रति}% .. 2-43

\twolineshloka
{भोगैश्वर्यप्रसक्तानां तया अपहृतचेतसाम्}
{व्यवसायात्मिका बुद्धिः समाधौ न विधीयते}% .. 2-44

\twolineshloka
{त्रैगुण्यविषयाः वेदाः निस्त्रैगुण्यः भव अर्जुन}
{निर्द्वन्द्वः नित्यसत्त्वस्थः निर्योगक्षेमः आत्मवान्}% .. 2-45

\twolineshloka
{यावान् अर्थः उदपाने सर्वतः सम्प्लुतोदके}
{तावान् सर्वेषु वेदेषु ब्राह्मणस्य विजानतः}% .. 2-46

\twolineshloka
{कर्मणि एव अधिकारः ते मा फलेषु कदाचन}
{मा कर्मफलहेतुः भूः मा ते सङ्गः अस्तु अकर्मणि}% .. 2-47

\twolineshloka
{योगस्थः कुरु कर्माणि सङ्गं त्यक्त्वा धनञ्जय}
{सिद्ध्यसिद्ध्योः समः भूत्वा समत्वं योगः उच्यते}% .. 2-48

\twolineshloka
{दूरेण हि अवरं कर्म बुद्धियोगात् धनञ्जय}
{बुद्धौ शरणम् अन्विच्छ कृपणाः फलहेतवः}% .. 2-49

\twolineshloka
{बुद्धियुक्तः जहाति इह उभे सुकृतदुष्कृते}
{तस्मात् योगाय युज्यस्व योगः कर्मसु कौशलम्}% .. 2-50

\twolineshloka
{कर्मजं बुद्धियुक्ताः हि फलं त्यक्त्वा मनीषिणः}
{जन्मबन्धविनिर्मुक्ताः पदं गच्छन्ति अनामयम्}% .. 2-51

\twolineshloka
{यदा ते मोहकलिलं बुद्धिः व्यतितरिष्यति}
{तदा गन्तासि निर्वेदं श्रोतव्यस्य श्रुतस्य च}% .. 2-52

\twolineshloka
{श्रुतिविप्रतिपन्ना ते यदा स्थास्यति निश्चला}
{समाधौ अचला बुद्धिः तदा योगम् अवाप्स्यसि}% .. 2-53

\uvacha{अर्जुन उवाच}

\twolineshloka
{स्थितप्रज्ञस्य का भाषा समाधिस्थस्य केशव}
{स्थितधीः किं प्रभाषेत किम् आसीत व्रजेत किम्}% .. 2-54

\uvacha{श्रीभगवानुवाच}

\twolineshloka
{प्रजहाति यदा कामान् सर्वान् पार्थ मनोगतान्}
{आत्मनि एव आत्मना तुष्टः स्थितप्रज्ञः तदा उच्यते}% .. 2-55

\twolineshloka
{दुःखेषु अनुद्विग्नमनाः सुखेषु विगतस्पृहः}
{वीतरागभयक्रोधः स्थितधीः मुनिः उच्यते}% .. 2-56

\twolineshloka
{यः सर्वत्र अनभिस्नेहः तत् तत् प्राप्य शुभाशुभम्}
{न अभिनन्दति न द्वेष्टि तस्य प्रज्ञा प्रतिष्ठिता}% .. 2-57

\twolineshloka
{यदा संहरते च अयं कूर्मः अङ्गानि इव सर्वशः}
{इन्द्रियाणि इन्द्रियार्थेभ्यः तस्य प्रज्ञा प्रतिष्ठिता}% .. 2-58

\twolineshloka
{विषयाः विनिवर्तन्ते निराहारस्य देहिनः}
{रसवर्जं रसः अप्यस्य परं दृष्ट्वा निवर्तते}% .. 2-59

\twolineshloka
{यततः हि अपि कौन्तेय पुरुषस्य विपश्चितः}
{इन्द्रियाणि प्रमाथीनि हरन्ति प्रसभं मनः}% .. 2-60

\twolineshloka
{तानि सर्वाणि संयम्य युक्तः आसीत मत्परः}
{वशे हि यस्य इन्द्रियाणि तस्य प्रज्ञा प्रतिष्ठिता}% .. 2-61

\twolineshloka
{ध्यायतः विषयान् पुंसः सङ्गः तेषु उपजायते}
{सङ्गात् सञ्जायते कामः कामात् क्रोधः अभिजायते}% .. 2-62

\twolineshloka
{क्रोधात् भवति सम्मोहः सम्मोहात् स्मृतिविभ्रमः}
{स्मृतिभ्रंशात् बुद्धिनाशः बुद्धिनाशात् प्रणश्यति}% .. 2-63

\twolineshloka
{रागद्वेषविमुक्तैः तु विषयान् इन्द्रियैः चरन्}
{आत्मवश्यैः विधेयात्मा प्रसादम् अधिगच्छति}% .. 2-64

\twolineshloka
{प्रसादे सर्वदुःखानां हानिः अस्य उपजायते}
{प्रसन्नचेतसः हि आशु बुद्धिः पर्यवतिष्ठते}% .. 2-65

\twolineshloka
{न अस्ति बुद्धिः अयुक्तस्य न च अयुक्तस्य भावना}
{न च अभावयतः शान्तिः अशान्तस्य कुतः सुखम्}% .. 2-66

\twolineshloka
{इन्द्रियाणां हि चरतां यत् मनः अनुविधीयते}
{तत् अस्य हरति प्रज्ञां वायुः नावम् इव अम्भसि}% .. 2-67

\twolineshloka
{तस्मात् यस्य महाबाहो निगृहीतानि सर्वशः}
{इन्द्रियाणि इन्द्रियार्थेभ्यः तस्य प्रज्ञा प्रतिष्ठिता}% .. 2-68

\twolineshloka
{या निशा सर्वभूतानां तस्यां जागर्ति संयमी}
{यस्यां जाग्रति भूतानि सा निशा पश्यतः मुनेः}% .. 2-69

\fourlineindentedshloka
{आपूर्यमाणम् अचलप्रतिष्ठं}
{समुद्रम् आपः प्रविशन्ति यद्वत्}
{तद्वत् कामाः यं प्रविशन्ति सर्वे}
{सः शान्तिम् आप्नोति न कामकामी}% .. 2-70

\twolineshloka
{विहाय कामान् यः सर्वान् पुमान् चरति निःस्पृहः}
{निर्ममः निरहङ्कारः सः शान्तिम् अधिगच्छति}% .. 2-71

\twolineshloka
{एषा ब्राह्मी स्थितिः पार्थ न एनां प्राप्य विमुह्यति}
{स्थित्वा अस्याम् अन्तकाले अपि ब्रह्मनिर्वाणम् ऋच्छति}% .. 2-72

{॥ॐ तत्सदिति श्रीमद्भगवद्गीतासूपनिषत्सु ब्रह्मविद्यायां योगशास्त्रे श्रीकृष्णार्जुनसंवादे साङ्ख्ययोगो नाम द्वितीयोऽध्यायः॥}

\chapt{तृतीयोऽध्यायः\textsf{---}कर्मयोगः}

\uvacha{अर्जुन उवाच}

\twolineshloka
{ज्यायसी चेत् कर्मणः ते मता बुद्धिः जनार्दन}
{तत् किं कर्मणि घोरे मां नियोजयसि केशव}% .. 3-1

\twolineshloka
{व्यामिश्रेण इव वाक्येन बुद्धिं मोहयसि इव मे}
{तत् एकं वद निश्चित्य येन श्रेयः अहम् आप्नुयाम्}% .. 3-2

\uvacha{श्रीभगवानुवाच}

\twolineshloka
{लोके अस्मिन् द्विविधा निष्ठा पुरा प्रोक्ता मया अनघ}
{ज्ञानयोगेन साङ्ख्यानां कर्मयोगेन योगिनाम्}% .. 3-3

\twolineshloka
{न कर्मणाम् अनारम्भात् नैष्कर्म्यं पुरुषः अश्नुते}
{न च सन्न्यसनात् एव सिद्धिं समधिगच्छति}% .. 3-4

\twolineshloka
{न हि कश्चित् क्षणम् अपि जातु तिष्ठति अकर्मकृत्}
{कार्यते हि अवशः कर्म सर्वः प्रकृतिजैः गुणैः}% .. 3-5

\twolineshloka
{कर्मेन्द्रियाणि संयम्य यः आस्ते मनसा स्मरन्}
{इन्द्रियार्थान् विमूढात्मा मिथ्याचारः सः उच्यते}% .. 3-6

\twolineshloka
{यः तु इन्द्रियाणि मनसा नियम्य आरभते अर्जुन}
{कर्मेन्द्रियैः कर्मयोगम् असक्तः सः विशिष्यते}% .. 3-7

\twolineshloka
{नियतं कुरु कर्म त्वं कर्म ज्यायः हि अकर्मणः}
{शरीरयात्रा अपि च ते न प्रसिद्ध्येत् अकर्मणः}% .. 3-8

\twolineshloka
{यज्ञार्थात् कर्मणः अन्यत्र लोकः अयं कर्मबन्धनः}
{तत् अर्थं कर्म कौन्तेय मुक्तसङ्गः समाचर}% .. 3-9

\twolineshloka
{सहयज्ञाः प्रजाः सृष्ट्वा पुरा उवाच प्रजापतिः}
{अनेन प्रसविष्यध्वम् एषः वः अस्तु इष्टकामधुक्}% .. 3-10

\twolineshloka
{देवान् भावयत अनेन ते देवाः भावयन्तु वः}
{परस्परं भावयन्तः श्रेयः परम् अवाप्स्यथ}% .. 3-11

\twolineshloka
{इष्टान् भोगान् हि वः देवाः दास्यन्ते यज्ञभाविताः}
{तैः दत्तान् अप्रदाय एभ्यः यः भुङ्क्ते स्तेनः एव सः}% .. 3-12

\twolineshloka
{यज्ञशिष्टाशिनः सन्तः मुच्यन्ते सर्वकिल्बिषैः}
{भुञ्जते ते तु अघं पापाः ये पचन्ति आत्मकारणात्}% .. 3-13

\twolineshloka
{अन्नात् भवन्ति भूतानि पर्जन्यात् अन्नसम्भवः}
{यज्ञात् भवति पर्जन्यः यज्ञः कर्मसमुद्भवः}% .. 3-14

\twolineshloka
{कर्म ब्रह्मोद्भवं विद्धि ब्रह्म अक्षरसमुद्भवम्}
{तस्मात् सर्वगतं ब्रह्म नित्यं यज्ञे प्रतिष्ठितम्}% .. 3-15

\twolineshloka
{एवं प्रवर्तितं चक्रं न अनुवर्तयति इह यः}
{अघायुः इन्द्रियारामः मोघं पार्थ सः जीवति}% .. 3-16

\twolineshloka
{यः तु आत्मरतिः एव स्यात् आत्मतृप्तः च मानवः}
{आत्मनि एव च सन्तुष्टः तस्य कार्यं न विद्यते}% .. 3-17

\twolineshloka
{न एव तस्य कृतेन अर्थः न अकृतेन इह कश्चन}
{न च अस्य सर्वभूतेषु कश्चित् अर्थव्यपाश्रयः}% .. 3-18

\twolineshloka
{तस्मात् असक्तः सततं कार्यं कर्म समाचर}
{असक्तः हि आचरन् कर्म परम् आप्नोति पूरुषः}% .. 3-19

\twolineshloka
{कर्मणा एव हि संसिद्धिम् आस्थिताः जनकादयः}
{लोकसङ्ग्रहम् एव अपि सम्पश्यन् कर्तुम् अर्हसि}% .. 3-20

\twolineshloka
{यत् यत् आचरति श्रेष्ठः तत् तत् एव इतरः जनः}
{सः यत् प्रमाणं कुरुते लोकः तत् अनुवर्तते}% .. 3-21

\twolineshloka
{न मे पार्थ अस्ति कर्तव्यं त्रिषु लोकेषु किञ्चन}
{न अनवाप्तम् अवाप्तव्यं वर्ते एव च कर्मणि}% .. 3-22

\twolineshloka
{यदि हि अहं न वर्तेयं जातु कर्मणि अतन्द्रितः}
{मम वर्त्म अनुवर्तन्ते मनुष्याः पार्थ सर्वशः}% .. 3-23

\twolineshloka
{उत्सीदेयुः इमे लोकाः न कुर्यां कर्म चेत् अहम्}
{सङ्करस्य च कर्ता स्याम् उपहन्याम् इमाः प्रजाः}% .. 3-24

\twolineshloka
{सक्ताः कर्मणि अविद्वांसः यथा कुर्वन्ति भारत}
{कुर्यात् विद्वान् तथा असक्तः चिकीर्षुः लोकसङ्ग्रहम्}% .. 3-25

\twolineshloka
{न बुद्धिभेदं जनयेत् अज्ञानां कर्मसङ्गिनाम्}
{जोषयेत् सर्वकर्माणि विद्वान् युक्तः समाचरन्}% .. 3-26

\twolineshloka
{प्रकृतेः क्रियमाणानि गुणैः कर्माणि सर्वशः}
{अहङ्कारविमूढात्मा कर्ता अहम् इति मन्यते}% .. 3-27

\twolineshloka
{तत्त्ववित् तु महाबाहो गुणकर्मविभागयोः}
{गुणाः गुणेषु वर्तन्ते इति मत्वा न सज्जते}% .. 3-28

\twolineshloka
{प्रकृतेः गुणसम्मूढाः सज्जन्ते गुणकर्मसु}
{तान् अकृत्स्नविदः मन्दान् कृत्स्नवित् न विचालयेत्}% .. 3-29

\twolineshloka
{मयि सर्वाणि कर्माणि सन्न्यस्य अध्यात्मचेतसा}
{निराशीः निर्ममः भूत्वा युध्यस्व विगतज्वरः}% .. 3-30

\twolineshloka
{ये मे मतम् इदं नित्यम् अनुतिष्ठन्ति मानवाः}
{श्रद्धावन्तः अनसूयन्तः मुच्यन्ते ते अपि कर्मभिः}% .. 3-31

\twolineshloka
{ये तु एतत् अभ्यसूयन्तः न अनुतिष्ठन्ति मे मतम्}
{सर्वज्ञानविमूढान् तान् विद्धि नष्टान् अचेतसः}% .. 3-32

\twolineshloka
{सदृशं चेष्टते स्वस्याः प्रकृतेः ज्ञानवान् अपि}
{प्रकृतिं यान्ति भूतानि निग्रहः किं करिष्यति}% .. 3-33

\twolineshloka
{इन्द्रियस्य इन्द्रियस्यार्थे रागद्वेषौ व्यवस्थितौ}
{तयोः न वशम् आगच्छेत् तौ हि अस्य परिपन्थिनौ}% .. 3-34

\twolineshloka
{श्रेयान् स्वधर्मः विगुणः परधर्मात् स्वनुष्ठितात्}
{स्वधर्मे निधनं श्रेयः परधर्मः भयावहः}% .. 3-35

\uvacha{अर्जुन उवाच}

\twolineshloka
{अथ केन प्रयुक्तः अयं पापं चरति पूरुषः}
{अनिच्छन् अपि वार्ष्णेय बलात् इव नियोजितः}% .. 3-36

\uvacha{श्रीभगवानुवाच}

\twolineshloka
{कामः एषः क्रोधः एषः रजः गुणसमुद्भवः}
{महाशनः महापाप्मा विद्धि एनम् इह वैरिणम्}% .. 3-37

\twolineshloka
{धूमेन आव्रियते वह्निः यथा आदर्शः मलेन च}
{यथा उल्बेन आवृतः गर्भः तथा तेन इदम् आवृतम्}% .. 3-38

\twolineshloka
{आवृतं ज्ञानम् एतेन ज्ञानिनः नित्यवैरिणा}
{कामरूपेण कौन्तेय दुष्पूरेण अनलेन च}% .. 3-39

\twolineshloka
{इन्द्रियाणि मनः बुद्धिः अस्य अधिष्ठानम् उच्यते}
{एतैः विमोहयति एषः ज्ञानम् आवृत्य देहिनम्}% .. 3-40

\twolineshloka
{तस्मात् त्वम् इन्द्रियाणि आदौ नियम्य भरतर्षभ}
{पाप्मानं प्रजहि हि एनं ज्ञानविज्ञाननाशनम्}% .. 3-41

\twolineshloka
{इन्द्रियाणि पराणि आहुः इन्द्रियेभ्यः परं मनः}
{मनसः तु परा बुद्धिः यः बुद्धेः परतः तु सः}% .. 3-42

\twolineshloka
{एवं बुद्धेः परं बुद्ध्वा संस्तभ्य आत्मानम् आत्मना}
{जहि शत्रुं महाबाहो कामरूपं दुरासदम्}% .. 3-43

{॥ॐ तत्सदिति श्रीमद्भगवद्गीतासूपनिषत्सु ब्रह्मविद्यायां योगशास्त्रे श्रीकृष्णार्जुनसंवादे कर्मयोगो नाम तृतीयोऽध्यायः॥}

\chapt{चतुर्थोऽध्यायः\textsf{---}ज्ञानकर्मसन्न्यासयोगः}

\uvacha{श्रीभगवानुवाच}

\twolineshloka
{इमं विवस्वते योगं प्रोक्तवान् अहम् अव्ययम्}
{विवस्वान् मनवे प्राह मनुः इक्ष्वाकवे अब्रवीत्}% .. 4-1

\twolineshloka
{एवं परम्पराप्राप्तम् इमं राजर्षयः विदुः}
{सः कालेन इह महता योगः नष्टः परन्तप}% .. 4-2

\twolineshloka
{सः एव अयं मया ते अद्य योगः प्रोक्तः पुरातनः}
{भक्तः असि मे सखा च इति रहस्यं हि एतत् उत्तमम्}% .. 4-3

\uvacha{अर्जुन उवाच}

\twolineshloka
{अपरं भवतः जन्म परं जन्म विवस्वतः}
{कथम् एतत् विजानीयां त्वम् आदौ प्रोक्तवान् इति}% .. 4-4

\uvacha{श्रीभगवानुवाच}

\twolineshloka
{बहूनि मे व्यतीतानि जन्मानि तव च अर्जुन}
{तानि अहं वेद सर्वाणि न त्वं वेत्थ परन्तप}% .. 4-5

\twolineshloka
{अजः अपि सन् अव्ययात्मा भूतानाम् ईश्वरः अपि सन्}
{प्रकृतिं स्वाम् अधिष्ठाय सम्भवामि आत्ममायया}% .. 4-6

\twolineshloka
{यदा यदा हि धर्मस्य ग्लानिः भवति भारत}
{अभ्युत्थानम् अधर्मस्य तदा आत्मानं सृजामि अहम्}% .. 4-7

\twolineshloka
{परित्राणाय साधूनां विनाशाय च दुष्कृताम्}
{धर्मसंस्थापनार्थाय सम्भवामि युगे युगे}% .. 4-8

\twolineshloka
{जन्म कर्म च मे दिव्यम् एवं यः वेत्ति तत्त्वतः}
{त्यक्त्वा देहं पुनः जन्म न एति माम् एति सः अर्जुन}% .. 4-9

\twolineshloka
{वीतरागभयक्रोधाः मन्मयाः माम् उपाश्रिताः}
{बहवः ज्ञानतपसा पूताः मद्भावम् आगताः}% .. 4-10

\twolineshloka
{ये यथा मां प्रपद्यन्ते तान् तथा एव भजामि अहम्}
{मम वर्त्म अनुवर्तन्ते मनुष्याः पार्थ सर्वशः}% .. 4-11

\twolineshloka
{काङ्क्षन्तः कर्मणां सिद्धिं यजन्ते इह देवताः}
{क्षिप्रं हि मानुषे लोके सिद्धिः भवति कर्मजा}% .. 4-12

\twolineshloka
{चातुर्वर्ण्यं मया सृष्टं गुणकर्मविभागशः}
{तस्य कर्तारम् अपि मां विद्धि अकर्तारम् अव्ययम्}% .. 4-13

\twolineshloka
{न मां कर्माणि लिम्पन्ति न मे कर्मफले स्पृहा}
{इति मां यः अभिजानाति कर्मभिः न सः बध्यते}% .. 4-14

\twolineshloka
{एवं ज्ञात्वा कृतं कर्म पूर्वैः अपि मुमुक्षुभिः}
{कुरु कर्म एव तस्मात् त्वं पूर्वैः पूर्वतरं कृतम्}% .. 4-15

\twolineshloka
{किं कर्म किम् अकर्म इति कवयः अपि अत्र मोहिताः}
{तत् ते कर्म प्रवक्ष्यामि यत् ज्ञात्वा मोक्ष्यसे अशुभात्}% .. 4-16

\twolineshloka
{कर्मणः हि अपि बोद्धव्यं बोद्धव्यं च विकर्मणः}
{अकर्मणः च बोद्धव्यं गहना कर्मणः गतिः}% .. 4-17

\twolineshloka
{कर्मणि अकर्म यः पश्येत् अकर्मणि च कर्म यः}
{सः बुद्धिमान् मनुष्येषु सः युक्तः कृत्स्नकर्मकृत्}% .. 4-18

\twolineshloka
{यस्य सर्वे समारम्भाः कामसङ्कल्पवर्जिताः}
{ज्ञानाग्निदग्धकर्माणं तम् आहुः पण्डितं बुधाः}% .. 4-19

\twolineshloka
{त्यक्त्वा कर्मफलासङ्गं नित्यतृप्तः निराश्रयः}
{कर्मणि अभिप्रवृत्तः अपि न एव किञ्चित् करोति सः}% .. 4-20

\twolineshloka
{निराशीः यतचित्तात्मा त्यक्तसर्वपरिग्रहः}
{शारीरं केवलं कर्म कुर्वन् न आप्नोति किल्बिषम्}% .. 4-21

\twolineshloka
{यदृच्छालाभसन्तुष्टः द्वन्द्वातीतः विमत्सरः}
{समः सिद्धौ असिद्धौ च कृत्वा अपि न निबध्यते}% .. 4-22

\twolineshloka
{गतसङ्गस्य मुक्तस्य ज्ञानावस्थितचेतसः}
{यज्ञाय आचरतः कर्म समग्रं प्रविलीयते}% .. 4-23

\twolineshloka
{ब्रह्मार्पणं ब्रह्म हविः ब्रह्माग्नौ ब्रह्मणा हुतम्}
{ब्रह्म एव तेन गन्तव्यं ब्रह्मकर्मसमाधिना}% .. 4-24

\twolineshloka
{दैवम् एव अपरे यज्ञं योगिनः पर्युपासते}
{ब्रह्माग्नौ अपरे यज्ञं यज्ञेन एव उपजुह्वति}% .. 4-25

\twolineshloka
{श्रोत्रादीनीन्द्रियाणि अन्ये संयमाग्निषु जुह्वति}
{शब्दादीन् विषयान् अन्ये इन्द्रियाग्निषु जुह्वति}% .. 4-26

\twolineshloka
{सर्वाणि इन्द्रियकर्माणि प्राणकर्माणि च अपरे}
{आत्मसंयमयोगाग्नौ जुह्वति ज्ञानदीपिते}% .. 4-27

\twolineshloka
{द्रव्ययज्ञाः तपोयज्ञाः योगयज्ञाः तथा अपरे}
{स्वाध्यायज्ञानयज्ञाः च यतयः संशितव्रताः}% .. 4-28

\twolineshloka
{अपाने जुह्वति प्राणं प्राणे अपानं तथा अपरे}
{प्राणापानगती रुद्ध्वा प्राणायामपरायणाः}% .. 4-29

\twolineshloka
{अपरे नियताहाराः प्राणान् प्राणेषु जुह्वति}
{सर्वे अपि एते यज्ञविदः यज्ञक्षपितकल्मषाः}% .. 4-30

\twolineshloka
{यज्ञशिष्टामृतभुजः यान्ति ब्रह्म सनातनम्}
{नायं लोकः अस्ति अयज्ञस्य कुतः अन्यः कुरुसत्तम}% .. 4-31

\twolineshloka
{एवं बहुविधाः यज्ञाः वितताः ब्रह्मणः मुखे}
{कर्मजान् विद्धि तान् सर्वान् एवं ज्ञात्वा विमोक्ष्यसे}% .. 4-32

\twolineshloka
{श्रेयान् द्रव्यमयात् यज्ञात् ज्ञानयज्ञः परन्तप}
{सर्वं कर्माखिलं पार्थ ज्ञाने परिसमाप्यते}% .. 4-33

\twolineshloka
{तत् विद्धि प्रणिपातेन परिप्रश्नेन सेवया}
{उपदेक्ष्यन्ति ते ज्ञानं ज्ञानिनः तत्त्वदर्शिनः}% .. 4-34

\twolineshloka
{यत् ज्ञात्वा न पुनः मोहम् एवं यास्यसि पाण्डव}
{येन भूतानि अशेषेण द्रक्ष्यसि आत्मनि अथो मयि}% .. 4-35

\twolineshloka
{अपि चेत् असि पापेभ्यः सर्वेभ्यः पापकृत्तमः}
{सर्वं ज्ञानप्लवेन एव वृजिनं सन्तरिष्यसि}% .. 4-36

\twolineshloka
{यथा एधांसि समिद्धः अग्निः भस्मसात् कुरुते अर्जुन}
{ज्ञानाग्निः सर्वकर्माणि भस्मसात् कुरुते तथा}% .. 4-37

\twolineshloka
{न हि ज्ञानेन सदृशं पवित्रम् इह विद्यते}
{तत् स्वयं योगसंसिद्धः कालेन आत्मनि विन्दति}% .. 4-38

\twolineshloka
{श्रद्धावान् लभते ज्ञानं तत्परः संयतेन्द्रियः}
{ज्ञानं लब्ध्वा परां शान्तिम् अचिरेणाधिगच्छति}% .. 4-39

\twolineshloka
{अज्ञः च अश्रद्दधानः च संशयात्मा विनश्यति}
{न अयं लोकः अस्ति न परः न सुखं संशयात्मनः}% .. 4-40

\twolineshloka
{योगसन्न्यस्तकर्माणं ज्ञानसञ्छिन्नसंशयम्}
{आत्मवन्तं न कर्माणि निबध्नन्ति धनञ्जय}% .. 4-41

\twolineshloka
{तस्मात् अज्ञानसम्भूतं हृत्स्थं ज्ञानासिना आत्मनः}
{छित्त्वा एनं संशयं योगम् आतिष्ठ उत्तिष्ठ भारत}% .. 4-42

{॥ॐ तत्सदिति श्रीमद्भगवद्गीतासूपनिषत्सु ब्रह्मविद्यायां योगशास्त्रे श्रीकृष्णार्जुनसंवादे ज्ञानकर्मसन्न्यासयोगो नाम चतुर्थोऽध्यायः॥}

\chapt{पञ्चमोऽध्यायः\textsf{---}कर्मसन्न्यासयोगः}

\uvacha{अर्जुन उवाच}

\twolineshloka
{सन्न्यासं कर्मणां कृष्ण पुनः योगं च शंससि}
{यत् श्रेयः एतयोः एकं तत् मे ब्रूहि सुनिश्चितम्}% .. 5-1

\uvacha{श्रीभगवानुवाच}

\twolineshloka
{सन्न्यासः कर्मयोगः च निःश्रेयसकरौ उभौ}
{तयोः तु कर्मसन्न्यासात् कर्मयोगः विशिष्यते}% .. 5-2

\twolineshloka
{ज्ञेयः सः नित्यसन्न्यासी यः न द्वेष्टि न काङ्क्षति}
{निर्द्वन्द्वः हि महाबाहो सुखं बन्धात् प्रमुच्यते}% .. 5-3

\twolineshloka
{साङ्ख्ययोगौ पृथक् बालाः प्रवदन्ति न पण्डिताः}
{एकम् अपि आस्थितः सम्यक् उभयोः विन्दते फलम्}% .. 5-4

\twolineshloka
{यत् साङ्ख्यैः प्राप्यते स्थानं तत् योगैः अपि गम्यते}
{एकं साङ्ख्यं च योगं च यः पश्यति सः पश्यति}% .. 5-5

\twolineshloka
{सन्न्यासः तु महाबाहो दुःखम् आप्तुम् अयोगतः}
{योगयुक्तः मुनिः ब्रह्म नचिरेण अधिगच्छति}% .. 5-6

\twolineshloka
{योगयुक्तः विशुद्धात्मा विजितात्मा जितेन्द्रियः}
{सर्वभूतात्मभूतात्मा कुर्वन् अपि न लिप्यते}% .. 5-7

\twolineshloka
{न एव किञ्चित् करोमि इति युक्तः मन्येत तत्त्ववित्}
{पश्यन् शृण्वन् स्पृशन् जिघ्रन् अश्नन् गच्छन् स्वपन् श्वसन्}% .. 5-8

\twolineshloka
{प्रलपन् विसृजन् गृह्णन् उन्मिषन् निमिषन् अपि}
{इन्द्रियाणि इन्द्रियार्थेषु वर्तन्ते इति धारयन्}% .. 5-9

\twolineshloka
{ब्रह्मणि आधाय कर्माणि सङ्गं त्यक्त्वा करोति यः}
{लिप्यते न सः पापेन पद्मपत्रम् इव अम्भसा}% .. 5-10

\twolineshloka
{कायेन मनसा बुद्ध्या केवलैः इन्द्रियैः अपि}
{योगिनः कर्म कुर्वन्ति सङ्गं त्यक्त्वा आत्मशुद्धये}% .. 5-11

\twolineshloka
{युक्तः कर्मफलं त्यक्त्वा शान्तिम् आप्नोति नैष्ठिकीम्}
{अयुक्तः कामकारेण फले सक्तः निबध्यते}% .. 5-12

\twolineshloka
{सर्वकर्माणि मनसा सन्न्यस्य आस्ते सुखं वशी}
{नवद्वारे पुरे देही न एव कुर्वन् न कारयन्}% .. 5-13

\twolineshloka
{न कर्तृत्वं न कर्माणि लोकस्य सृजति प्रभुः}
{न कर्मफलसंयोगं स्वभावः तु प्रवर्तते}% .. 5-14

\twolineshloka
{न आदत्ते कस्यचित् पापं न च एव सुकृतं विभुः}
{अज्ञानेन आवृतं ज्ञानं तेन मुह्यन्ति जन्तवः}% .. 5-15

\twolineshloka
{ज्ञानेन तु तत् अज्ञानं येषां नाशितम् आत्मनः}
{तेषाम् आदित्यवत् ज्ञानं प्रकाशयति तत् परम्}% .. 5-16

\twolineshloka
{तत् बुद्धयः तत् आत्मानः तत् निष्ठाः तत् परायणाः}
{गच्छन्ति अपुनरावृत्तिं ज्ञाननिर्धूतकल्मषाः}% .. 5-17

\twolineshloka
{विद्याविनयसम्पन्ने ब्राह्मणे गवि हस्तिनि}
{शुनि च एव श्वपाके च पण्डिताः समदर्शिनः}% .. 5-18

\twolineshloka
{इह एव तैः जितः सर्गः येषां साम्ये स्थितं मनः}
{निर्दोषं हि समं ब्रह्म तस्मात् ब्रह्मणि ते स्थिताः}% .. 5-19

\twolineshloka
{न प्रहृष्येत् प्रियं प्राप्य न उद्विजेत् प्राप्य च अप्रियम्}
{स्थिरबुद्धिः असम्मूढः ब्रह्मवित् ब्रह्मणि स्थितः}% .. 5-20

\twolineshloka
{बाह्यस्पर्शेषु असक्तात्मा विन्दति आत्मनि यत् सुखम्}
{सः ब्रह्मयोगयुक्तात्मा सुखम् अक्षयम् अश्नुते}% .. 5-21

\twolineshloka
{ये हि संस्पर्शजाः भोगाः दुःखयोनयः एव ते}
{आदि अन्तवन्तः कौन्तेय न तेषु रमते बुधः}% .. 5-22

\twolineshloka
{शक्नोति इह एव यः सोढुं प्राक् शरीरविमोक्षणात्}
{कामक्रोधोद्भवं वेगं सः युक्तः सः सुखी नरः}% .. 5-23

\twolineshloka
{यः अन्तःसुखः अन्तरारामः तथा अन्तर्ज्योतिः एव यः}
{सः योगी ब्रह्मनिर्वाणं ब्रह्मभूतः अधिगच्छति}% .. 5-24

\twolineshloka
{लभन्ते ब्रह्मनिर्वाणम् ऋषयः क्षीणकल्मषाः}
{छिन्नद्वैधाः यतात्मानः सर्वभूतहिते रताः}% .. 5-25

\twolineshloka
{कामक्रोधवियुक्तानां यतीनां यतचेतसाम्}
{अभितः ब्रह्मनिर्वाणं वर्तते विदितात्मनाम्}% .. 5-26

\twolineshloka
{स्पर्शान् कृत्वा बहिः बाह्यान् चक्षुः च एव अन्तरे भ्रुवोः}
{प्राणापानौ समौ कृत्वा नासाभ्यन्तरचारिणौ}% .. 5-27

\twolineshloka
{यतेन्द्रियमनः बुद्धिः मुनिः मोक्षपरायणः}
{विगतेच्छाभयक्रोधः यः सदा मुक्तः एव सः}% .. 5-28

\twolineshloka
{भोक्तारं यज्ञतपसां सर्वलोकमहेश्वरम्}
{सुहृदं सर्वभूतानां ज्ञात्वा मां शान्तिम् ऋच्छति}% .. 5-29

{॥ॐ तत्सदिति श्रीमद्भगवद्गीतासूपनिषत्सु ब्रह्मविद्यायां योगशास्त्रे श्रीकृष्णार्जुनसंवादे कर्मसन्न्यासयोगो नाम पञ्चमोऽध्यायः॥}

\chapt{षष्ठोऽध्यायः\textsf{---}आत्मसंयमयोगः}

\uvacha{श्रीभगवानुवाच}

\twolineshloka
{अनाश्रितः कर्मफलं कार्यं कर्म करोति यः}
{सः सन्न्यासी च योगी च न निरग्निः न च अक्रियः}% .. 6-1

\twolineshloka
{यं सन्न्यासम् इति प्राहुः योगं तं विद्धि पाण्डव}
{न हि असन्न्यस्तसङ्कल्पः योगी भवति कश्चन}% .. 6-2

\twolineshloka
{आरुरुक्षोः मुनेः योगं कर्म कारणम् उच्यते}
{योगारूढस्य तस्य एव शमः कारणम् उच्यते}% .. 6-3

\twolineshloka
{यदा हि न इन्द्रियार्थेषु न कर्मसु अनुषज्जते}
{सर्वसङ्कल्पसन्न्यासी योगारूढः तदा उच्यते}% .. 6-4

\twolineshloka
{उद्धरेत् आत्मना आत्मानं न आत्मानम् अवसादयेत्}
{आत्मा एव हि आत्मनः बन्धुः आत्मा एव रिपुः आत्मनः}% .. 6-5

\twolineshloka
{बन्धुः आत्मा आत्मनः तस्य येन आत्मा एव आत्मना जितः}
{अनात्मनः तु शत्रुत्वे वर्तेत आत्मा एव शत्रुवत्}% .. 6-6

\twolineshloka
{जितात्मनः प्रशान्तस्य परमात्मा समाहितः}
{शीतोष्णसुखदुःखेषु तथा मानापमानयोः}% .. 6-7

\twolineshloka
{ज्ञानविज्ञानतृप्तात्मा कूटस्थः विजितेन्द्रियः}
{युक्तः इति उच्यते योगी समलोष्टाश्मकाञ्चनः}% .. 6-8

\twolineshloka
{सुहृत् मित्रार्युदासीनमध्यस्थद्वेष्यबन्धुषु}
{साधुषु अपि च पापेषु समबुद्धिः विशिष्यते}% .. 6-9

\twolineshloka
{योगी युञ्जीत सततम् आत्मानं रहसि स्थितः}
{एकाकी यतचित्तात्मा निराशीः अपरिग्रहः}% .. 6-10

\twolineshloka
{शुचौ देशे प्रतिष्ठाप्य स्थिरम् आसनम् आत्मनः}
{न अत्युच्छ्रितं न अतिनीचं चैलाजिनकुशोत्तरम्}% .. 6-11

\twolineshloka
{तत्र एकाग्रं मनः कृत्वा यतचित्तेन्द्रियक्रियः}
{उपविश्य आसने युञ्ज्यात् योगम् आत्मविशुद्धये}% .. 6-12

\twolineshloka
{समं कायशिरोग्रीवं धारयन् अचलं स्थिरः}
{सम्प्रेक्ष्य नासिकाग्रं स्वं दिशः च अनवलोकयन्}% .. 6-13

\twolineshloka
{प्रशान्तात्मा विगतभीः ब्रह्मचारिव्रते स्थितः}
{मनः संयम्य मच्चित्तः युक्तः आसीत मत्परः}% .. 6-14

\twolineshloka
{युञ्जन् एवं सदा आत्मानं योगी नियतमानसः}
{शान्तिं निर्वाणपरमां मत्संस्थाम् अधिगच्छति}% .. 6-15

\twolineshloka
{न अति अश्नतः तु योगः अस्ति न च एकान्तम् अनश्नतः}
{न च अतिस्वप्नशीलस्य जाग्रतः न एव च अर्जुन}% .. 6-16

\twolineshloka
{युक्ताहारविहारस्य युक्तचेष्टस्य कर्मसु}
{युक्तस्वप्नावबोधस्य योगः भवति दुःखहा}% .. 6-17

\twolineshloka
{यदा विनियतं चित्तम् आत्मनि एव अवतिष्ठते}
{निःस्पृहः सर्वकामेभ्यः युक्तः इति उच्यते तदा}% .. 6-18

\twolineshloka
{यथा दीपः निवातस्थः न इङ्गते सा उपमा स्मृता}
{योगिनः यतचित्तस्य युञ्जतः योगम् आत्मनः}% .. 6-19

\twolineshloka
{यत्र उपरमते चित्तं निरुद्धं योगसेवया}
{यत्र च एव आत्मना आत्मानं पश्यन् आत्मनि तुष्यति}% .. 6-20

\twolineshloka
{सुखम् आत्यन्तिकं यत् तत् बुद्धिग्राह्यम् अतीन्द्रियम्}
{वेत्ति यत्र न च एव अयं स्थितः चलति तत्त्वतः}% .. 6-21

\twolineshloka
{यं लब्ध्वा च अपरं लाभं मन्यते न अधिकं ततः}
{यस्मिन् स्थितः न दुःखेन गुरुणा अपि विचाल्यते}% .. 6-22

\twolineshloka
{तं विद्यात् दुःखसंयोगवियोगं योगसंज्ञितम्}
{सः निश्चयेन योक्तव्यः योगः अनिर्विण्णचेतसा}% .. 6-23

\twolineshloka
{सङ्कल्पप्रभवान् कामान् त्यक्त्वा सर्वान् अशेषतः}
{मनसा एव इन्द्रियग्रामं विनियम्य समन्ततः}% .. 6-24

\twolineshloka
{शनैः शनैः उपरमेत् बुद्ध्या धृतिगृहीतया}
{आत्मसंस्थं मनः कृत्वा न किञ्चित् अपि चिन्तयेत्}% .. 6-25

\twolineshloka
{यतः यतः निश्चरति मनः चञ्चलम् अस्थिरम्}
{ततः ततः नियम्य एतत् आत्मनि एव वशं नयेत्}% .. 6-26

\twolineshloka
{प्रशान्तमनसं हि एनं योगिनं सुखम् उत्तमम्}
{उपैति शान्तरजसं ब्रह्मभूतम् अकल्मषम्}% .. 6-27

\twolineshloka
{युञ्जन् एवं सदा आत्मानं योगी विगतकल्मषः}
{सुखेन ब्रह्मसंस्पर्शम् अत्यन्तं सुखम् अश्नुते}% .. 6-28

\twolineshloka
{सर्वभूतस्थम् आत्मानं सर्वभूतानि च आत्मनि}
{ईक्षते योगयुक्तात्मा सर्वत्र समदर्शनः}% .. 6-29

\twolineshloka
{यः मां पश्यति सर्वत्र सर्वं च मयि पश्यति}
{तस्य अहं न प्रणश्यामि सः च मे न प्रणश्यति}% .. 6-30

\twolineshloka
{सर्वभूतस्थितं यः मां भजति एकत्वम् आस्थितः}
{सर्वथा वर्तमानः अपि सः योगी मयि वर्तते}% .. 6-31

\twolineshloka
{आत्मौपम्येन सर्वत्र समं पश्यति यः अर्जुन}
{सुखं वा यदि वा दुःखं सः योगी परमः मतः}% .. 6-32

\uvacha{अर्जुन उवाच}

\twolineshloka
{यः अयं योगः त्वया प्रोक्तः साम्येन मधुसूदन}
{एतस्य अहं न पश्यामि चञ्चलत्वात् स्थितिं स्थिराम्}% .. 6-33

\twolineshloka
{चञ्चलं हि मनः कृष्ण प्रमाथि बलवत् दृढम्}
{तस्य अहं निग्रहं मन्ये वायोः इव सुदुष्करम्}% .. 6-34

\uvacha{श्रीभगवानुवाच}

\twolineshloka
{असंशयं महाबाहो मनः दुर्निग्रहं चलम्}
{अभ्यासेन तु कौन्तेय वैराग्येण च गृह्यते}% .. 6-35

\twolineshloka
{असंयतात्मना योगः दुष्प्रापः इति मे मतिः}
{वश्यात्मना तु यतता शक्यः अवाप्तुम् उपायतः}% .. 6-36

\uvacha{अर्जुन उवाच}

\twolineshloka
{अयतिः श्रद्धया उपेतः योगात् चलितमानसः}
{अप्राप्य योगसंसिद्धिं कां गतिं कृष्ण गच्छति}% .. 6-37

\twolineshloka
{कच्चित् न उभयविभ्रष्टः छिन्नाभ्रम् इव नश्यति}
{अप्रतिष्ठः महाबाहो विमूढः ब्रह्मणः पथि}% .. 6-38

\twolineshloka
{एतत् मे संशयं कृष्ण छेत्तुम् अर्हसि अशेषतः}
{त्वत् अन्यः संशयस्य अस्य छेत्ता न हि उपपद्यते}% .. 6-39

\uvacha{श्रीभगवानुवाच}

\twolineshloka
{पार्थ न एव इह न अमुत्र विनाशः तस्य विद्यते}
{न हि कल्याणकृत् कश्चित् दुर्गतिं तात गच्छति}% .. 6-40

\twolineshloka
{प्राप्य पुण्यकृतां लोकान् उषित्वा शाश्वतीः समाः}
{शुचीनां श्रीमतां गेहे योगभ्रष्टः अभिजायते}% .. 6-41

\twolineshloka
{अथवा योगिनाम् एव कुले भवति धीमताम्}
{एतत् हि दुर्लभतरं लोके जन्म यत् ईदृशम्}% .. 6-42

\twolineshloka
{तत्र तं बुद्धिसंयोगं लभते पौर्वदेहिकम्}
{यतते च ततः भूयः संसिद्धौ कुरुनन्दन}% .. 6-43

\twolineshloka
{पूर्वाभ्यासेन तेन एव ह्रियते हि अवशः अपि सः}
{जिज्ञासुः अपि योगस्य शब्दब्रह्म अतिवर्तते}% .. 6-44

\twolineshloka
{प्रयत्नात् यतमानः तु योगी संशुद्धकिल्बिषः}
{अनेकजन्मसंसिद्धः ततः याति परां गतिम्}% .. 6-45

\twolineshloka
{तपस्विभ्यः अधिकः योगी ज्ञानिभ्यः अपि मतः अधिकः}
{कर्मिभ्यः च अधिकः योगी तस्मात् योगी भव अर्जुन}% .. 6-46

\twolineshloka
{योगिनाम् अपि सर्वेषां मत् गतेन अन्तरात्मना}
{श्रद्धावान् भजते यः मां सः मे युक्ततमः मतः}% .. 6-47

{॥ॐ तत्सदिति श्रीमद्भगवद्गीतासूपनिषत्सु ब्रह्मविद्यायां योगशास्त्रे श्रीकृष्णार्जुनसंवादे आत्मसंयमयोगो नाम षष्ठोऽध्यायः॥}

\chapt{सप्तमोऽध्यायः\textsf{---}ज्ञानविज्ञानयोगः}

\uvacha{श्रीभगवानुवाच}

\twolineshloka
{मयि आसक्तमनाः पार्थ योगं युञ्जन् मत् आश्रयः}
{असंशयं समग्रं मां यथा ज्ञास्यसि तत् शृणु}% .. 7-1

\twolineshloka
{ज्ञानं ते अहं सविज्ञानम् इदं वक्ष्यामि अशेषतः}
{यत् ज्ञात्वा न इह भूयः अन्यत् ज्ञातव्यम् अवशिष्यते}% .. 7-2

\twolineshloka
{मनुष्याणां सहस्रेषु कश्चित् यतति सिद्धये}
{यतताम् अपि सिद्धानां कश्चित् मां वेत्ति तत्त्वतः}% .. 7-3

\twolineshloka
{भूमिः आपः अनलः वायुः खं मनः बुद्धिः एव च}
{अहङ्कारः इति इयं मे भिन्ना प्रकृतिः अष्टधा}% .. 7-4

\twolineshloka
{अपरा इयम् इतः तु अन्यां प्रकृतिं विद्धि मे पराम्}
{जीवभूतां महाबाहो यया इदं धार्यते जगत्}% .. 7-5

\twolineshloka
{एतत् योनीनि भूतानि सर्वाणि इति उपधारय}
{अहं कृत्स्नस्य जगतः प्रभवः प्रलयः तथा}% .. 7-6

\twolineshloka
{मत्तः परतरं न अन्यत् किञ्चित् अस्ति धनञ्जय}
{मयि सर्वम् इदं प्रोतं सूत्रे मणिगणाः इव}% .. 7-7

\twolineshloka
{रसः अहम् अप्सु कौन्तेय प्रभा अस्मि शशिसूर्ययोः}
{प्रणवः सर्ववेदेषु शब्दः खे पौरुषं नृषु}% .. 7-8

\twolineshloka
{पुण्यः गन्धः पृथिव्यां च तेजः च अस्मि विभावसौ}
{जीवनं सर्वभूतेषु तपः च अस्मि तपस्विषु}% .. 7-9

\twolineshloka
{बीजं मां सर्वभूतानां विद्धि पार्थ सनातनम्}
{बुद्धिः बुद्धिमताम् अस्मि तेजः तेजस्विनाम् अहम्}% .. 7-10

\twolineshloka
{बलं बलवतां च अहं कामरागविवर्जितम्}
{धर्माविरुद्धः भूतेषु कामः अस्मि भरतर्षभ}% .. 7-11

\twolineshloka
{ये च एव सात्त्विकाः भावाः राजसाः तामसाः च ये}
{मत्तः एव इति तान् विद्धि न तु अहं तेषु ते मयि}% .. 7-12

\twolineshloka
{त्रिभिः गुणमयैः भावैः एभिः सर्वम् इदं जगत्}
{मोहितं न अभिजानाति माम् एभ्यः परम् अव्ययम्}% .. 7-13

\twolineshloka
{दैवी हि एषा गुणमयी मम माया दुरत्यया}
{माम् एव ये प्रपद्यन्ते मायाम् एतां तरन्ति ते}% .. 7-14

\twolineshloka
{न मां दुष्कृतिनः मूढाः प्रपद्यन्ते नराधमाः}
{मायया अपहृतज्ञानाः आसुरं भावम् आश्रिताः}% .. 7-15

\twolineshloka
{चतुर्विधाः भजन्ते मां जनाः सुकृतिनः अर्जुन}
{आर्तः जिज्ञासुः अर्थार्थी ज्ञानी च भरतर्षभ}% .. 7-16

\twolineshloka
{तेषां ज्ञानी नित्ययुक्तः एकभक्तिः विशिष्यते}
{प्रियः हि ज्ञानिनः अत्यर्थम् अहं सः च मम प्रियः}% .. 7-17

\twolineshloka
{उदाराः सर्वे एव एते ज्ञानी तु आत्मा एव मे मतम्}
{आस्थितः सः हि युक्तात्मा माम् एव अनुत्तमां गतिम्}% .. 7-18

\twolineshloka
{बहूनां जन्मनाम् अन्ते ज्ञानवान् मां प्रपद्यते}
{वासुदेवः सर्वम् इति सः महात्मा सुदुर्लभः}% .. 7-19

\twolineshloka
{कामैः तैः तैः हृतज्ञानाः प्रपद्यन्ते अन्यदेवताः}
{तं तं नियमम् आस्थाय प्रकृत्या नियताः स्वया}% .. 7-20

\twolineshloka
{यः यः यां यां तनुं भक्तः श्रद्धया अर्चितुम् इच्छति}
{तस्य तस्य अचलां श्रद्धां ताम् एव विदधामि अहम्}% .. 7-21

\twolineshloka
{सः तया श्रद्धया युक्तः तस्याः राधनम् ईहते}
{लभते च ततः कामान् मया एव विहितान् हि तान्}% .. 7-22

\twolineshloka
{अन्तवत् तु फलं तेषां तत् भवति अल्पमेधसाम्}
{देवान् देवयजः यान्ति मत् भक्ताः यान्ति माम् अपि}% .. 7-23

\twolineshloka
{अव्यक्तं व्यक्तिम् आपन्नं मन्यन्ते माम् अबुद्धयः}
{परं भावम् अजानन्तः मम अव्ययम् अनुत्तमम्}% .. 7-24

\twolineshloka
{न अहं प्रकाशः सर्वस्य योगमायासमावृतः}
{मूढः अयं न अभिजानाति लोकः माम् अजम् अव्ययम्}% .. 7-25

\twolineshloka
{वेद अहं समतीतानि वर्तमानानि च अर्जुन}
{भविष्याणि च भूतानि मां तु वेद न कश्चन}% .. 7-26

\twolineshloka
{इच्छाद्वेषसमुत्थेन द्वन्द्वमोहेन भारत}
{सर्वभूतानि सम्मोहं सर्गे यान्ति परन्तप}% .. 7-27

\twolineshloka
{येषां तु अन्तगतं पापं जनानां पुण्यकर्मणाम्}
{ते द्वन्द्वमोहनिर्मुक्ताः भजन्ते मां दृढव्रताः}% .. 7-28

\twolineshloka
{जरामरणमोक्षाय माम् आश्रित्य यतन्ति ये}
{ते ब्रह्म तत् विदुः कृत्स्नम् अध्यात्मं कर्म च अखिलम्}% .. 7-29

\twolineshloka
{साधिभूताधिदैवं मां साधियज्ञं च ये विदुः}
{प्रयाणकाले अपि च मां ते विदुः युक्तचेतसः}% .. 7-30

{॥ॐ तत्सदिति श्रीमद्भगवद्गीतासूपनिषत्सु ब्रह्मविद्यायां योगशास्त्रे श्रीकृष्णार्जुनसंवादे ज्ञानविज्ञानयोगो नाम सप्तमोऽध्यायः॥}

\chapt{अष्टमोऽध्यायः\textsf{---}अक्षरब्रह्मयोगः}

\uvacha{अर्जुन उवाच}

\twolineshloka
{किं तत् ब्रह्म किम् अध्यात्मं किं कर्म पुरुषोत्तम}
{अधिभूतं च किं प्रोक्तम् अधिदैवं किम् उच्यते}% .. 8-1

\twolineshloka
{अधियज्ञः कथं कः अत्र देहे अस्मिन् मधुसूदन}
{प्रयाणकाले च कथं ज्ञेयः असि नियतात्मभिः}% .. 8-2

\uvacha{श्रीभगवानुवाच}

\twolineshloka
{अक्षरं ब्रह्म परमं स्वभावः अध्यात्मम् उच्यते}
{भूतभावोद्भवकरः विसर्गः कर्मसंज्ञितः}% .. 8-3

\twolineshloka
{अधिभूतं क्षरः भावः पुरुषः च अधिदैवतम्}
{अधियज्ञः अहम् एव अत्र देहे देहभृतां वर}% .. 8-4

\twolineshloka
{अन्तकाले च माम् एव स्मरन् मुक्त्वा कलेवरम्}
{यः प्रयाति सः मत् भावं याति न अस्ति अत्र संशयः}% .. 8-5

\twolineshloka
{यं यं वा अपि स्मरन् भावं त्यजति अन्ते कलेवरम्}
{तं तम् एव एति कौन्तेय सदा तत् भावभावितः}% .. 8-6

\twolineshloka
{तस्मात् सर्वेषु कालेषु माम् अनुस्मर युध्य च}
{मयि अर्पितमनोबुद्धिः माम् एव एष्यसि असंशयः}% .. 8-7

\twolineshloka
{अभ्यासयोगयुक्तेन चेतसा न अन्यगामिना}
{परमं पुरुषं दिव्यं याति पार्थ अनुचिन्तयन्}% .. 8-8

\fourlineindentedshloka
{कविं पुराणम् अनुशासितारम्}
{अणोः अणीयांसम् अनुस्मरेत् यः}
{सर्वस्य धातारम् अचिन्त्यरूपम्}
{आदित्यवर्णं तमसः परस्तात्}% .. 8-9

\fourlineindentedshloka
{प्रयाणकाले मनसा अचलेन}
{भक्त्या युक्तः योगबलेन च एव}
{भ्रुवोः मध्ये प्राणम् आवेश्य सम्यक्}
{सः तं परं पुरुषम् उपैति दिव्यम्}% .. 8-10

\fourlineindentedshloka
{यत् अक्षरं वेदविदः वदन्ति}
{विशन्ति यत् यतयः वीतरागाः}
{यत् इच्छन्तः ब्रह्मचर्यं चरन्ति}
{तत् ते पदं सङ्ग्रहेण प्रवक्ष्ये}% .. 8-11

\twolineshloka
{सर्वद्वाराणि संयम्य मनः हृदि निरुध्य च}
{मूर्ध्नि आधाय आत्मनः प्राणम् आस्थितः योगधारणाम्}% .. 8-12

\twolineshloka
{ओम् इति एकाक्षरं ब्रह्म व्याहरन् माम् अनुस्मरन्}
{यः प्रयाति त्यजन् देहं सः याति परमां गतिम्}% .. 8-13

\twolineshloka
{अनन्यचेताः सततं यः मां स्मरति नित्यशः}
{तस्य अहं सुलभः पार्थ नित्ययुक्तस्य योगिनः}% .. 8-14

\twolineshloka
{माम् उपेत्य पुनः जन्म दुःखालयम् अशाश्वतम्}
{न आप्नुवन्ति महात्मानः संसिद्धिं परमां गताः}% .. 8-15

\twolineshloka
{आब्रह्मभुवनात् लोकाः पुनरावर्तिनः अर्जुन}
{माम् उपेत्य तु कौन्तेय पुनः जन्म न विद्यते}% .. 8-16

\twolineshloka
{सहस्रयुगपर्यन्तम् अहः यत् ब्रह्मणः विदुः}
{रात्रिं युगसहस्रान्तां ते अहोरात्रविदः जनाः}% .. 8-17

\twolineshloka
{अव्यक्तात् व्यक्तयः सर्वाः प्रभवन्ति अहः आगमे}
{रात्रि आगमे प्रलीयन्ते तत्र एव अव्यक्तसंज्ञके}% .. 8-18

\twolineshloka
{भूतग्रामः सः एव अयं भूत्वा भूत्वा प्रलीयते}
{रात्रि आगमे अवशः पार्थ प्रभवति अहः आगमे}% .. 8-19

\twolineshloka
{परः तस्मात् तु भावः अन्यः अव्यक्तः अव्यक्तात् सनातनः}
{यः सः सर्वेषु भूतेषु नश्यत्सु न विनश्यति}% .. 8-20

\twolineshloka
{अव्यक्तः अक्षरः इति उक्तः तम् आहुः परमां गतिम्}
{यं प्राप्य न निवर्तन्ते तत् धाम परमं मम}% .. 8-21

\twolineshloka
{पुरुषः सः परः पार्थ भक्त्या लभ्यः तु अनन्यया}
{यस्य अन्तःस्थानि भूतानि येन सर्वम् इदं ततम्}% .. 8-22

\twolineshloka
{यत्र काले तु अनावृत्तिम् आवृत्तिं च एव योगिनः}
{प्रयाताः यान्ति तं कालं वक्ष्यामि भरतर्षभ}% .. 8-23

\twolineshloka
{अग्निः ज्योतिः अहः शुक्लः षण्मासाः उत्तरायणम्}
{तत्र प्रयाताः गच्छन्ति ब्रह्म ब्रह्मविदः जनाः}% .. 8-24

\twolineshloka
{धूमः रात्रिः तथा कृष्णः षण्मासाः दक्षिणायनम्}
{तत्र चान्द्रमसं ज्योतिः योगी प्राप्य निवर्तते}% .. 8-25

\twolineshloka
{शुक्लकृष्णे गती हि एते जगतः शाश्वते मते}
{एकया याति अनावृत्तिम् अन्यया आवर्तते पुनः}% .. 8-26

\twolineshloka
{न एते सृती पार्थ जानन् योगी मुह्यति कश्चन}
{तस्मात् सर्वेषु कालेषु योगयुक्तः भव अर्जुन}% .. 8-27

\fourlineindentedshloka
{वेदेषु यज्ञेषु तपःसु च एव}
{दानेषु यत् पुण्यफलं प्रदिष्टम्}
{अत्येति तत् सर्वम् इदं विदित्वा}
{योगी परं स्थानम् उपैति च आद्यम्}% .. 8-28

{॥ॐ तत्सदिति श्रीमद्भगवद्गीतासूपनिषत्सु ब्रह्मविद्यायां योगशास्त्रे श्रीकृष्णार्जुनसंवादे अक्षरब्रह्मयोगो नाम अष्टमोऽध्यायः॥}

\chapt{नवमोऽध्यायः\textsf{---}राजविद्याराजगुह्ययोगः}

\uvacha{श्रीभगवानुवाच}

\twolineshloka
{इदं तु ते गुह्यतमं प्रवक्ष्यामि अनसूयवे}
{ज्ञानं विज्ञानसहितं यत् ज्ञात्वा मोक्ष्यसे अशुभात्}% .. 9-1

\twolineshloka
{राजविद्या राजगुह्यं पवित्रम् इदम् उत्तमम्}
{प्रत्यक्षावगमं धर्म्यं सुसुखं कर्तुम् अव्ययम्}% .. 9-2

\twolineshloka
{अश्रद्दधानाः पुरुषाः धर्मस्य अस्य परन्तप}
{अप्राप्य मां निवर्तन्ते मृत्युसंसारवर्त्मनि}% .. 9-3

\twolineshloka
{मया ततम् इदं सर्वं जगत् अव्यक्तमूर्तिना}
{मत्स्थानि सर्वभूतानि न च अहं तेषु अवस्थितः}% .. 9-4

\twolineshloka
{न च मत्स्थानि भूतानि पश्य मे योगम् ऐश्वरम्}
{भूतभृत् न च भूतस्थः मम आत्मा भूतभावनः}% .. 9-5

\twolineshloka
{यथा आकाशस्थितः नित्यं वायुः सर्वत्रगः महान्}
{तथा सर्वाणि भूतानि मत्स्थानि इति उपधारय}% .. 9-6

\twolineshloka
{सर्वभूतानि कौन्तेय प्रकृतिं यान्ति मामिकाम्}
{कल्पक्षये पुनः तानि कल्पादौ विसृजामि अहम्}% .. 9-7

\twolineshloka
{प्रकृतिं स्वाम् अवष्टभ्य विसृजामि पुनः पुनः}
{भूतग्रामम् इमं कृत्स्नम् अवशं प्रकृतेः वशात्}% .. 9-8

\twolineshloka
{न च मां तानि कर्माणि निबध्नन्ति धनञ्जय}
{उदासीनवत् आसीनम् असक्तं तेषु कर्मसु}% .. 9-9

\twolineshloka
{मया अध्यक्षेण प्रकृतिः सूयते सचराचरम्}
{हेतुना अनेन कौन्तेय जगत् विपरिवर्तते}% .. 9-10

\twolineshloka
{अवजानन्ति मां मूढाः मानुषीं तनुम् आश्रितम्}
{परं भावम् अजानन्तः मम भूतमहेश्वरम्}% .. 9-11

\twolineshloka
{मोघाशाः मोघकर्माणः मोघज्ञानाः विचेतसः}
{राक्षसीम् आसुरीं च एव प्रकृतिं मोहिनीं श्रिताः}% .. 9-12

\twolineshloka
{महात्मानः तु मां पार्थ दैवीं प्रकृतिम् आश्रिताः}
{भजन्ति अनन्यमनसः ज्ञात्वा भूतादिम् अव्ययम्}% .. 9-13

\twolineshloka
{सततं कीर्तयन्तः मां यतन्तः च दृढव्रताः}
{नमस्यन्तः च मां भक्त्या नित्ययुक्ताः उपासते}% .. 9-14

\twolineshloka
{ज्ञानयज्ञेन च अपि अन्ये यजन्तः माम् उपासते}
{एकत्वेन पृथक्त्वेन बहुधा विश्वतोमुखम्}% .. 9-15

\twolineshloka
{अहं क्रतुः अहं यज्ञः स्वधा अहम् अहम् औषधम्}
{मन्त्रः अहम् अहम् एव आज्यम् अहम् अग्निः अहं हुतम्}% .. 9-16

\twolineshloka
{पिता अहम् अस्य जगतो माता धाता पितामहः}
{वेद्यं पवित्रम् ओङ्कारः ऋक्साम यजुः एव च}% .. 9-17

\twolineshloka
{गतिः भर्ता प्रभुः साक्षी निवासः शरणं सुहृत्}
{प्रभवः प्रलयः स्थानं निधानं बीजम् अव्ययम्}% .. 9-18

\twolineshloka
{तपामि अहम् अहं वर्षं निगृह्णामि उत्सृजामि च}
{अमृतं च एव मृत्युः च सत् असत् च अहम् अर्जुन}% .. 9-19

\fourlineindentedshloka
{त्रैविद्याः मां सोमपाः पूतपापाः}
{यज्ञैः इष्ट्वा स्वर्गतिं प्रार्थयन्ते}
{ते पुण्यम् आसाद्य सुरेन्द्रलोकम्}
{अश्नन्ति दिव्यान् दिवि देवभोगान्}% .. 9-20

\fourlineindentedshloka
{ते तं भुक्त्वा स्वर्गलोकं विशालं}
{क्षीणे पुण्ये मर्त्यलोकं विशन्ति}
{एवं त्रयीधर्मम् अनुप्रपन्नाः}
{गतागतं कामकामाः लभन्ते}% .. 9-21

\twolineshloka
{अनन्याः चिन्तयन्तः मां ये जनाः पर्युपासते}
{तेषां नित्याभियुक्तानां योगक्षेमं वहामि अहम्}% .. 9-22

\twolineshloka
{ये अपि अन्यदेवताभक्ताः यजन्ते श्रद्धया अन्विताः}
{ते अपि माम् एव कौन्तेय यजन्ति अविधिपूर्वकम्}% .. 9-23

\twolineshloka
{अहं हि सर्वयज्ञानां भोक्ता च प्रभुः एव च}
{न तु माम् अभिजानन्ति तत्त्वेन अतः च्यवन्ति ते}% .. 9-24

\twolineshloka
{यान्ति देवव्रताः देवान् पितॄन् यान्ति पितृव्रताः}
{भूतानि यान्ति भूतेज्याः यान्ति मत् याजिनः अपि माम्}% .. 9-25

\twolineshloka
{पत्रं पुष्पं फलं तोयं यः मे भक्त्या प्रयच्छति}
{तत् अहं भक्त्युपहृतम् अश्नामि प्रयतात्मनः}% .. 9-26

\twolineshloka
{यत् करोषि यत् अश्नासि यत् जुहोषि ददासि यत्}
{यत् तपस्यसि कौन्तेय तत् कुरुष्व मत् अर्पणम्}% .. 9-27

\twolineshloka
{शुभाशुभफलैः एवं मोक्ष्यसे कर्मबन्धनैः}
{सन्न्यासयोगयुक्तात्मा विमुक्तः माम् उपैष्यसि}% .. 9-28

\twolineshloka
{समः अहं सर्वभूतेषु न मे द्वेष्यः अस्ति न प्रियः}
{ये भजन्ति तु मां भक्त्या मयि ते तेषु च अपि अहम्}% .. 9-29

\twolineshloka
{अपि चेत् सुदुराचारः भजते माम् अनन्यभाक्}
{साधुः एव सः मन्तव्यः सम्यक् व्यवसितः हि सः}% .. 9-30

\twolineshloka
{क्षिप्रं भवति धर्मात्मा शश्वत् शान्तिं निगच्छति}
{कौन्तेय प्रतिजानीहि न मे भक्तः प्रणश्यति}% .. 9-31

\twolineshloka
{मां हि पार्थ व्यपाश्रित्य ये अपि स्युः पापयोनयः}
{स्त्रियः वैश्याः तथा शूद्राः ते अपि यान्ति परां गतिम्}% .. 9-32

\twolineshloka
{किं पुनः ब्राह्मणाः पुण्याः भक्ताः राजर्षयः तथा}
{अनित्यम् असुखं लोकम् इमं प्राप्य भजस्व माम्}% .. 9-33

\twolineshloka
{मन्मनाः भव मद्भक्तः मद्याजी मां नमस्कुरु}
{माम् एव एष्यसि युक्त्वा एवम् आत्मानं मत्परायणः}% .. 9-34

{॥ॐ तत्सदिति श्रीमद्भगवद्गीतासूपनिषत्सु ब्रह्मविद्यायां योगशास्त्रे श्रीकृष्णार्जुनसंवादे राजविद्याराजगुह्ययोगो नाम नवमोऽध्यायः॥}

\chapt{दशमोऽध्यायः\textsf{---}विभूतियोगः}

\uvacha{श्रीभगवानुवाच}

\twolineshloka
{भूयः एव महाबाहो शृणु मे परमं वचः}
{यत् ते अहं प्रीयमाणाय वक्ष्यामि हितकाम्यया}% .. 10-1

\twolineshloka
{न मे विदुः सुरगणाः प्रभवं न महर्षयः}
{अहम् आदिः हि देवानां महर्षीणां च सर्वशः}% .. 10-2

\twolineshloka
{यः माम् अजम् अनादिं च वेत्ति लोकमहेश्वरम्}
{असम्मूढः सः मर्त्येषु सर्वपापैः प्रमुच्यते}% .. 10-3

\twolineshloka
{बुद्धिः ज्ञानम् असम्मोहः क्षमा सत्यं दमः शमः}
{सुखं दुःखं भवः अभावः भयं च अभयम् एव च}% .. 10-4

\twolineshloka
{अहिंसा समता तुष्टिः तपः दानं यशः अयशः}
{भवन्ति भावाः भूतानां मत्तः एव पृथग्विधाः}% .. 10-5

\twolineshloka
{महर्षयः सप्त पूर्वे चत्वारः मनवः तथा}
{मत् भावाः मानसाः जाताः येषां लोके इमाः प्रजाः}% .. 10-6

\twolineshloka
{एतां विभूतिं योगं च मम यः वेत्ति तत्त्वतः}
{सः अविकम्पेन योगेन युज्यते न अत्र संशयः}% .. 10-7

\twolineshloka
{अहं सर्वस्य प्रभवः मत्तः सर्वं प्रवर्तते}
{इति मत्वा भजन्ते मां बुधाः भावसमन्विताः}% .. 10-8

\twolineshloka
{मत् चित्ताः मत् गतप्राणाः बोधयन्तः परस्परम्}
{कथयन्तः च मां नित्यं तुष्यन्ति च रमन्ति च}% .. 10-9

\twolineshloka
{तेषां सततयुक्तानां भजतां प्रीतिपूर्वकम्}
{ददामि बुद्धियोगं तं येन माम् उपयान्ति ते}% .. 10-10

\twolineshloka
{तेषाम् एव अनुकम्पार्थम् अहम् अज्ञानजं तमः}
{नाशयामि आत्मभावस्थः ज्ञानदीपेन भास्वता}% .. 10-11

\uvacha{अर्जुन उवाच}

\twolineshloka
{परं ब्रह्म परं धाम पवित्रं परमं भवान्}
{पुरुषं शाश्वतं दिव्यम् आदिदेवम् अजं विभुम्}% .. 10-12

\twolineshloka
{आहुः त्वाम् ऋषयः सर्वे देवर्षिः नारदः तथा}
{असितः देवलः व्यासः स्वयं च एव ब्रवीषि मे}% .. 10-13

\twolineshloka
{सर्वम् एतत् ऋतं मन्ये यत् मां वदसि केशव}
{न हि ते भगवन् व्यक्तिं विदुः देवाः न दानवाः}% .. 10-14

\twolineshloka
{स्वयम् एव आत्मना आत्मानं वेत्थ त्वं पुरुषोत्तम}
{भूतभावन भूतेश देवदेव जगत्पते}% .. 10-15

\twolineshloka
{वक्तुम् अर्हसि अशेषेण दिव्याः हि आत्मविभूतयः}
{याभिः विभूतिभिः लोकान् इमान् त्वं व्याप्य तिष्ठसि}% .. 10-16

\twolineshloka
{कथं विद्याम् अहं योगिन् त्वां सदा परिचिन्तयन्}
{केषु केषु च भावेषु चिन्त्यः असि भगवन् मया}% .. 10-17

\twolineshloka
{विस्तरेण आत्मनः योगं विभूतिं च जनार्दन}
{भूयः कथय तृप्तिः हि शृण्वतः न अस्ति मे अमृतम्}% .. 10-18

\uvacha{श्रीभगवानुवाच}

\twolineshloka
{हन्त ते कथयिष्यामि दिव्याः हि आत्मविभूतयः}
{प्राधान्यतः कुरुश्रेष्ठ न अस्ति अन्तः विस्तरस्य मे}% .. 10-19

\twolineshloka
{अहम् आत्मा गुडाकेश सर्वभूताशयस्थितः}
{अहम् आदिः च मध्यं च भूतानाम् अन्तः एव च}% .. 10-20

\twolineshloka
{आदित्यानाम् अहं विष्णुः ज्योतिषां रविः अंशुमान्}
{मरीचिः मरुताम् अस्मि नक्षत्राणाम् अहं शशी}% .. 10-21

\twolineshloka
{वेदानां सामवेदः अस्मि देवानाम् अस्मि वासवः}
{इन्द्रियाणां मनः च अस्मि भूतानाम् अस्मि चेतना}% .. 10-22

\twolineshloka
{रुद्राणां शङ्करः च अस्मि वित्तेशः यक्षरक्षसाम्}
{वसूनां पावकः च अस्मि मेरुः शिखरिणाम् अहम्}% .. 10-23

\twolineshloka
{पुरोधसां च मुख्यं मां विद्धि पार्थ बृहस्पतिम्}
{सेनानीनाम् अहं स्कन्दः सरसाम् अस्मि सागरः}% .. 10-24

\twolineshloka
{महर्षीणां भृगुः अहं गिराम् अस्मि एकम् अक्षरम्}
{यज्ञानां जपयज्ञः अस्मि स्थावराणां हिमालयः}% .. 10-25

\twolineshloka
{अश्वत्थः सर्ववृक्षाणां देवर्षीणां च नारदः}
{गन्धर्वाणां चित्ररथः सिद्धानां कपिलः मुनिः}% .. 10-26

\twolineshloka
{उच्चैःश्रवसम् अश्वानां विद्धि माम् अमृतोद्भवम्}
{ऐरावतं गजेन्द्राणां नराणां च नराधिपम्}% .. 10-27

\twolineshloka
{आयुधानाम् अहं वज्रं धेनूनाम् अस्मि कामधुक्}
{प्रजनः च अस्मि कन्दर्पः सर्पाणाम् अस्मि वासुकिः}% .. 10-28

\twolineshloka
{अनन्तः च अस्मि नागानां वरुणः यादसाम् अहम्}
{पितॄणाम् अर्यमा च अस्मि यमः संयमताम् अहम्}% .. 10-29

\twolineshloka
{प्रह्लादः च अस्मि दैत्यानां कालः कलयताम् अहम्}
{मृगाणां च मृगेन्द्रः अहं वैनतेयः च पक्षिणाम्}% .. 10-30

\twolineshloka
{पवनः पवताम् अस्मि रामः शस्त्रभृताम् अहम्}
{झषाणां मकरः च अस्मि स्रोतसाम् अस्मि जाह्नवी}% .. 10-31

\twolineshloka
{सर्गाणाम् आदिः अन्तः च मध्यं च एव अहम् अर्जुन}
{अध्यात्मविद्या विद्यानां वादः प्रवदताम् अहम्}% .. 10-32

\twolineshloka
{अक्षराणाम् अकारः अस्मि द्वन्द्वः सामासिकस्य च}
{अहम् एव अक्षयः कालः धाता अहं विश्वतोमुखः}% .. 10-33

\twolineshloka
{मृत्युः सर्वहरः च अहम् उद्भवः च भविष्यताम्}
{कीर्तिः श्रीः वाक् च नारीणां स्मृतिः मेधा धृतिः क्षमा}% .. 10-34

\twolineshloka
{बृहत्साम तथा साम्नां गायत्री छन्दसाम् अहम्}
{मासानां मार्गशीर्षः अहम् ऋतूनां कुसुमाकरः}% .. 10-35

\twolineshloka
{द्यूतं छलयताम् अस्मि तेजः तेजस्विनाम् अहम्}
{जयः अस्मि व्यवसायः अस्मि सत्त्वं सत्त्ववताम् अहम्}% .. 10-36

\twolineshloka
{वृष्णीनां वासुदेवः अस्मि पाण्डवानां धनञ्जयः}
{मुनीनाम् अपि अहं व्यासः कवीनाम् उशना कविः}% .. 10-37

\twolineshloka
{दण्डः दमयताम् अस्मि नीतिः अस्मि जिगीषताम्}
{मौनं च एव अस्मि गुह्यानां ज्ञानं ज्ञानवताम् अहम्}% .. 10-38

\twolineshloka
{यत् च अपि सर्वभूतानां बीजं तत् अहम् अर्जुन}
{न तत् अस्ति विना यत् स्यात् मया भूतं चराचरम्}% .. 10-39

\twolineshloka
{न अन्तः अस्ति मम दिव्यानां विभूतीनां परन्तप}
{एषः तु उद्देशतः प्रोक्तः विभूतेः विस्तरः मया}% .. 10-40

\twolineshloka
{यत् यत् विभूतिमत् सत्त्वं श्रीमत् ऊर्जितम् एव वा}
{तत् तत् एव अवगच्छ त्वं मम तेजः अंशसम्भवम्}% .. 10-41

\twolineshloka
{अथवा बहुना एतेन किं ज्ञातेन तव अर्जुन}
{विष्टभ्य अहम् इदं कृत्स्नम् एकांशेन स्थितः जगत्}% .. 10-42

{॥ॐ तत्सदिति श्रीमद्भगवद्गीतासूपनिषत्सु ब्रह्मविद्यायां योगशास्त्रे श्रीकृष्णार्जुनसंवादे विभूतियोगो नाम दशमोऽध्यायः॥}

\chapt{एकादशोऽध्यायः\textsf{---}विश्वरूपदर्शनयोगः}

\uvacha{अर्जुन उवाच}

\twolineshloka
{मत् अनुग्रहाय परमं गुह्यम् अध्यात्मसंज्ञितम्}
{यत् त्वया उक्तं वचः तेन मोहः अयं विगतः मम}% .. 11-1

\twolineshloka
{भव अपि अयौ हि भूतानां श्रुतौ विस्तरशः मया}
{त्वत्तः कमलपत्राक्ष माहात्म्यम् अपि च अव्ययम्}% .. 11-2

\twolineshloka
{एवम् एतत् यथा आत्थ त्वम् आत्मानं परमेश्वर}
{द्रष्टुम् इच्छामि ते रूपम् ऐश्वरं पुरुषोत्तम}% .. 11-3

\twolineshloka
{मन्यसे यदि तत् शक्यं मया द्रष्टुम् इति प्रभो}
{योगेश्वर ततः मे त्वं दर्शय आत्मानम् अव्ययम्}% .. 11-4

\uvacha{श्रीभगवानुवाच}

\twolineshloka
{पश्य मे पार्थ रूपाणि शतशः अथ सहस्रशः}
{नानाविधानि दिव्यानि नानावर्णाकृतीनि च}% .. 11-5

\twolineshloka
{पश्य आदित्यान् वसून् रुद्रान् अश्विनौ मरुतः तथा}
{बहूनि अदृष्टपूर्वाणि पश्य आश्चर्याणि भारत}% .. 11-6

\twolineshloka
{इह एकस्थं जगत् कृत्स्नं पश्य अद्य सचराचरम्}
{मम देहे गुडाकेश यत् च अन्यत् द्रष्टुम् इच्छसि}% .. 11-7

\twolineshloka
{न तु मां शक्यसे द्रष्टुम् अनेन एव स्वचक्षुषा}
{दिव्यं ददामि ते चक्षुः पश्य मे योगम् ऐश्वरम्}% .. 11-8

\uvacha{सञ्जय उवाच}

\twolineshloka
{एवम् उक्त्वा ततः राजन् महायोगेश्वरः हरिः}
{दर्शयामास पार्थाय परमं रूपम् ऐश्वरम्}% .. 11-9

\twolineshloka
{अनेकवक्त्रनयनम् अनेकाद्भुतदर्शनम्}
{अनेकदिव्यआभरणं दिव्यानेकोद्यतायुधम्}% .. 11-10

\twolineshloka
{दिव्यमाल्याम्बरधरं दिव्यगन्धानुलेपनम्}
{सर्वाश्चर्यमयं देवम् अनन्तं विश्वतोमुखम्}% .. 11-11

\twolineshloka
{दिवि सूर्यसहस्रस्य भवेत् युगपत् उत्थिता}
{यदि भाः सदृशी सा स्यात् भासः तस्य महात्मनः}% .. 11-12

\twolineshloka
{तत्र एकस्थं जगत् कृत्स्नं प्रविभक्तम् अनेकधा}
{अपश्यत् देवदेवस्य शरीरे पाण्डवः तदा}% .. 11-13

\twolineshloka
{ततः सः विस्मयाविष्टः हृष्टरोमा धनञ्जयः}
{प्रणम्य शिरसा देवं कृताञ्जलिः अभाषत}% .. 11-14

\uvacha{अर्जुन उवाच}

\fourlineindentedshloka
{पश्यामि देवान् तव देव देहे}
{सर्वान् तथा भूतविशेषसङ्घान्}
{ब्रह्माणम् ईशं कमलासनस्थम्}
{ऋषीन् च सर्वान् उरगान् च दिव्यान्}% .. 11-15

\fourlineindentedshloka
{अनेकबाहूदरवक्त्रनेत्रं}
{पश्यामि त्वां सर्वतः अनन्तरूपम्}
{न अन्तं न मध्यं न पुनः तव आदिं}
{पश्यामि विश्वेश्वर विश्वरूप}% .. 11-16

\fourlineindentedshloka
{किरीटिनं गदिनं चक्रिणं च}
{तेजोराशिं सर्वतः दीप्तिमन्तम्}
{पश्यामि त्वां दुर्निरीक्ष्यं समन्तात्}
{दीप्तानलार्कद्युतिम् अप्रमेयम्}% .. 11-17

\fourlineindentedshloka
{त्वम् अक्षरं परमं वेदितव्यं}
{त्वम् अस्य विश्वस्य परं निधानम्}
{त्वम् अव्ययः शाश्वतधर्मगोप्ता}
{सनातनः त्वं पुरुषः मतः मे}% .. 11-18

\fourlineindentedshloka
{अनादिमध्यान्तम् अनन्तवीर्यम्}
{अनन्तबाहुं शशिसूर्यनेत्रम्}
{पश्यामि त्वां दीप्तहुताशवक्त्रं}
{स्वतेजसा विश्वम् इदं तपन्तम्}% .. 11-19

\fourlineindentedshloka
{द्यावापृथिव्योः इदम् अन्तरं हि}
{व्याप्तं त्वया एकेन दिशः च सर्वाः}
{दृष्ट्वा अद्भुतं रूपम् उग्रं तव इदं}
{लोकत्रयं प्रव्यथितं महात्मन्}% .. 11-20

\fourlineindentedshloka
{अमी हि त्वां सुरसङ्घाः विशन्ति}
{केचित् भीताः प्राञ्जलयः गृणन्ति}
{स्वस्ति इति उक्त्वा महर्षिसिद्धसङ्घाः}
{स्तुवन्ति त्वां स्तुतिभिः पुष्कलाभिः}% .. 11-21

\fourlineindentedshloka
{रुद्रादित्याः वसवः ये च साध्याः}
{विश्वे अश्विनौ मरुतः च उष्मपाः च}
{गन्धर्वयक्षासुरसिद्धसङ्घाः}
{वीक्षन्ते त्वां विस्मिताः च एव सर्वे}% .. 11-22

\fourlineindentedshloka
{रूपं महत् ते बहुवक्त्रनेत्रं}
{महाबाहो बहुबाहूरुपादम्}
{बहूदरं बहुदंष्ट्राकरालं}
{दृष्ट्वा लोकाः प्रव्यथिताः तथा अहम्}% .. 11-23

\fourlineindentedshloka
{नभःस्पृशं दीप्तमनेकवर्णं}
{व्यात्ताननं दीप्तविशालनेत्रम्}
{दृष्ट्वा हि त्वां प्रव्यथितान्तरात्मा}
{धृतिं न विन्दामि शमं च विष्णो}% .. 11-24

\fourlineindentedshloka
{दंष्ट्राकरालानि च ते मुखानि}
{दृष्ट्वा एव कालानलसन्निभानि}
{दिशः न जाने न लभे च शर्म}
{प्रसीद देवेश जगन्निवास}% .. 11-25

\fourlineindentedshloka
{अमी च त्वां धृतराष्ट्रस्य पुत्राः}
{सर्वे सह एव अवनिपालसङ्घैः}
{भीष्मः द्रोणः सूतपुत्रः तथा असौ}
{सह अस्मदीयैः अपि योधमुख्यैः}% .. 11-26

\fourlineindentedshloka
{वक्त्राणि ते त्वरमाणाः विशन्ति}
{दंष्ट्राकरालानि भयानकानि}
{केचित् विलग्नाः दशनान्तरेषु}
{सन्दृश्यन्ते चूर्णितैः उत्तमाङ्गैः}% .. 11-27

\fourlineindentedshloka
{यथा नदीनां बहवः अम्बुवेगाः}
{समुद्रम् एव अभिमुखाः द्रवन्ति}
{तथा तव अमी नरलोकवीराः}
{विशन्ति वक्त्राणि अभिविज्वलन्ति}% .. 11-28

\fourlineindentedshloka
{यथा प्रदीप्तं ज्वलनं पतङ्गाः}
{विशन्ति नाशाय समृद्धवेगाः}
{तथा एव नाशाय विशन्ति लोकाः}
{तव अपि वक्त्राणि समृद्धवेगाः}% .. 11-29

\fourlineindentedshloka
{लेलिह्यसे ग्रसमानः समन्तात्}
{लोकान् समग्रान् वदनैः ज्वलद्भिः}
{तेजोभिः आपूर्य जगत् समग्रं}
{भासः तव उग्राः प्रतपन्ति विष्णो}% .. 11-30

\fourlineindentedshloka
{आख्याहि मे को भवानुग्ररूपः}
{नमः अस्तु ते देववर प्रसीद}
{विज्ञातुम् इच्छामि भवन्तम् आद्यं}
{न हि प्रजानामि तव प्रवृत्तिम्}% .. 11-31

\uvacha{श्रीभगवानुवाच}

\fourlineindentedshloka
{कालः अस्मि लोकक्षयकृत् प्रवृद्धः}
{लोकान् समाहर्तुम् इह प्रवृत्तः}
{ऋते अपि त्वां न भविष्यन्ति सर्वे}
{ये अवस्थिताः प्रत्यनीकेषु योधाः}% .. 11-32

\fourlineindentedshloka
{तस्मात् त्वम् उत्तिष्ठ यशः लभस्व}
{जित्वा शत्रून् भुङ्क्ष्व राज्यं समृद्धम्}
{मया एव एते निहताः पूर्वम् एव}
{निमित्तमात्रं भव सव्यसाचिन्}% .. 11-33

\fourlineindentedshloka
{द्रोणं च भीष्मं च जयद्रथं च}
{कर्णं तथा अन्यान् अपि योधवीरान्}
{मया हतान् त्वं जहि मा व्यथिष्ठाः}
{युध्यस्व जेता असि रणे सपत्नान्}% .. 11-34

\uvacha{सञ्जय उवाच}

\fourlineindentedshloka
{एतत् श्रुत्वा वचनं केशवस्य}
{कृताञ्जलिः वेपमानः किरीटी}
{नमस्कृत्वा भूयः एव आह कृष्णं}
{सगद्गदं भीतभीतः प्रणम्य}% .. 11-35

\uvacha{अर्जुन उवाच}

\fourlineindentedshloka
{स्थाने हृषीकेश तव प्रकीर्त्या}
{जगत् प्रहृष्यति अनुरज्यते च}
{रक्षांसि भीतानि दिशः द्रवन्ति}
{सर्वे नमस्यन्ति च सिद्धसङ्घाः}% .. 11-36

\fourlineindentedshloka
{कस्मात् च ते न नमेरन् महात्मन्}
{गरीयसे ब्रह्मणः अपि आदिकर्त्रे}
{अनन्त देवेश जगत् निवास}
{त्वम् अक्षरं सत् असत् तत् परं यत्}% .. 11-37

\fourlineindentedshloka
{त्वम् आदिदेवः पुरुषः पुराणः}
{त्वम् अस्य विश्वस्य परं निधानम्}
{वेत्ता असि वेद्यं च परं च धाम}
{त्वया ततं विश्वम् अनन्तरूप}% .. 11-38

\fourlineindentedshloka
{वायुः यमः अग्निः वरुणः शशाङ्कः}
{प्रजापतिः त्वं प्रपितामहः च}
{नमः नमः ते अस्तु सहस्रकृत्वः}
{पुनः च भूयः अपि नमः नमः ते}% .. 11-39

\fourlineindentedshloka
{नमः पुरस्तात् अथ पृष्ठतः ते}
{नमः अस्तु ते सर्वतः एव सर्व}
{अनन्तवीर्यामितविक्रमः त्वं}
{सर्वं समाप्नोषि ततः असि सर्वः}% .. 11-40

\fourlineindentedshloka
{सखा इति मत्वा प्रसभं यत् उक्तं}
{हे कृष्ण हे यादव हे सखा इति}
{अजानता महिमानं तव इदं}
{मया प्रमादात् प्रणयेन वा अपि}% .. 11-41

\fourlineindentedshloka
{यत् च अवहासार्थम् असत् कृतः असि}
{विहारशय्यासनभोजनेषु}
{एकः अथवा अपि अच्युत तत् समक्षं}
{तत् क्षामये त्वाम् अहम् अप्रमेयम्}% .. 11-42

\fourlineindentedshloka
{पिता असि लोकस्य चराचरस्य}
{त्वम् अस्य पूज्यः च गुरुः गरीयान्}
{न त्वत् समः अस्ति अभ्यधिकः कुतः अन्यः}
{लोकत्रये अपि अप्रतिमप्रभाव}% .. 11-43

\fourlineindentedshloka
{तस्मात् प्रणम्य प्रणिधाय कायं}
{प्रसादये त्वाम् अहम् ईशम् ईड्यम्}
{पिता इव पुत्रस्य सखा इव सख्युः}
{प्रियः प्रियायाः अर्हसि देव सोढुम्}% .. 11-44

\fourlineindentedshloka
{अदृष्टपूर्वं हृषितः अस्मि दृष्ट्वा}
{भयेन च प्रव्यथितं मनः मे}
{तत् एव मे दर्शय देव रूपं}
{प्रसीद देवेश जगन्निवास}% .. 11-45

\fourlineindentedshloka
{किरीटिनं गदिनं चक्रहस्तम्}
{इच्छामि त्वां द्रष्टुम् अहं तथा एव}
{तेन एव रूपेण चतुर्भुजेन}
{सहस्रबाहो भव विश्वमूर्ते}% .. 11-46

\uvacha{श्रीभगवानुवाच}

\fourlineindentedshloka
{मया प्रसन्नेन तव अर्जुन इदं}
{रूपं परं दर्शितम् आत्मयोगात्}
{तेजोमयं विश्वम् अनन्तम् आद्यं}
{यत् मे त्वत् अन्येन न दृष्टपूर्वम्}% .. 11-47

\fourlineindentedshloka
{न वेदयज्ञाध्ययनैः न दानैः}
{न च क्रियाभिः न तपोभिः उग्रैः}
{एवं रूपः शक्यः अहं नृलोके}
{द्रष्टुं त्वत् अन्येन कुरुप्रवीर}% .. 11-48

\fourlineindentedshloka
{मा ते व्यथा मा च विमूढभावः}
{दृष्ट्वा रूपं घोरम् ईदृक् मम इदम्}
{व्यपेतभीः प्रीतमनाः पुनः त्वं}
{तत् एव मे रूपम् इदं प्रपश्य}% .. 11-49

\uvacha{सञ्जय उवाच}

\fourlineindentedshloka
{इति अर्जुनं वासुदेवः तथा उक्त्वा}
{स्वकं रूपं दर्शयामास भूयः}
{आश्वासयामास च भीतम् एनं}
{भूत्वा पुनः सौम्यवपुः महात्मा}% .. 11-50

\uvacha{अर्जुन उवाच}

\twolineshloka
{दृष्ट्वा इदं मानुषं रूपं तव सौम्यं जनार्दन}
{इदानीम् अस्मि संवृत्तः सचेताः प्रकृतिं गतः}% .. 11-51

\uvacha{श्रीभगवानुवाच}

\twolineshloka
{सुदुर्दर्शम् इदं रूपं दृष्टवान् असि यत् मम}
{देवाः अपि अस्य रूपस्य नित्यं दर्शनकाङ्क्षिणः}% .. 11-52

\twolineshloka
{न अहं वेदैः न तपसा न दानेन न च इज्यया}
{शक्यः एवं विधः द्रष्टुं दृष्टवान् असि मां यथा}% .. 11-53

\twolineshloka
{भक्त्या तु अनन्यया शक्यः अहम् एवं विधः अर्जुन}
{ज्ञातुं द्रष्टुं च तत्त्वेन प्रवेष्टुं च परन्तप}% .. 11-54

\twolineshloka
{मत्कर्मकृत् मत्परमः मद्भक्तः सङ्गवर्जितः}
{निर्वैरः सर्वभूतेषु यः सः माम् एति पाण्डव}% .. 11-55

{॥ॐ तत्सदिति श्रीमद्भगवद्गीतासूपनिषत्सु ब्रह्मविद्यायां योगशास्त्रे श्रीकृष्णार्जुनसंवादे विश्वरूपदर्शनयोगो नाम एकादशोऽध्यायः॥}

\chapt{द्वादशोऽध्यायः\textsf{---}भक्तियोगः}

\uvacha{अर्जुन उवाच}

\twolineshloka
{एवं सततयुक्ताः ये भक्ताः त्वां पर्युपासते}
{ये च अपि अक्षरम् अव्यक्तं तेषां के योगवित्तमाः}% .. 12-1

\uvacha{श्रीभगवानुवाच}

\twolineshloka
{मयि आवेश्य मनः ये मां नित्ययुक्ताः उपासते}
{श्रद्धया परया उपेताः ते मे युक्ततमाः मताः}% .. 12-2

\twolineshloka
{ये तु अक्षरम् अनिर्देश्यम् अव्यक्तं पर्युपासते}
{सर्वत्रगम् अचिन्त्यं च कूटस्थम् अचलं ध्रुवम्}% .. 12-3

\twolineshloka
{सन्नियम्य इन्द्रियग्रामं सर्वत्र समबुद्धयः}
{ते प्राप्नुवन्ति माम् एव सर्वभूतहिते रताः}% .. 12-4

\twolineshloka
{क्लेशः अधिकतरः तेषाम् अव्यक्तासक्तचेतसाम्}
{अव्यक्ता हि गतिः दुःखं देहवद्भिः अवाप्यते}% .. 12-5

\twolineshloka
{ये तु सर्वाणि कर्माणि मयि सन्न्यस्य मत्पराः}
{अनन्येन एव योगेन मां ध्यायन्तः उपासते}% .. 12-6

\twolineshloka
{तेषाम् अहं समुद्धर्ता मृत्युसंसारसागरात्}
{भवामि न चिरात् पार्थ मयि आवेशितचेतसाम्}% .. 12-7

\twolineshloka
{मयि एव मनः आधत्स्व मयि बुद्धिं निवेशय}
{निवसिष्यसि मयि एव अतः ऊर्ध्वं न संशयः}% .. 12-8

\twolineshloka
{अथ चित्तं समाधातुं न शक्नोषि मयि स्थिरम्}
{अभ्यासयोगेन ततः माम् इच्छ आप्तुं धनञ्जय}% .. 12-9

\twolineshloka
{अभ्यासे अपि असमर्थः असि मत्कर्मपरमः भव}
{मदर्थम् अपि कर्माणि कुर्वन् सिद्धिम् अवाप्स्यसि}% .. 12-10

\twolineshloka
{अथ एतत् अपि अशक्तः असि कर्तुं मद्योगम् आश्रितः}
{सर्वकर्मफलत्यागं ततः कुरु यतात्मवान्}% .. 12-11

\twolineshloka
{श्रेयः हि ज्ञानम् अभ्यासात् ज्ञानात् ध्यानं विशिष्यते}
{ध्यानात् कर्मफलत्यागः त्यागात् शान्तिः अनन्तरम्}% .. 12-12

\twolineshloka
{अद्वेष्टा सर्वभूतानां मैत्रः करुणः एव च}
{निर्ममः निरहङ्कारः समदुःखसुखः क्षमी}% .. 12-13

\twolineshloka
{सन्तुष्टः सततं योगी यतात्मा दृढनिश्चयः}
{मयि अर्पितमनोबुद्धिः यः मद्भक्तः सः मे प्रियः}% .. 12-14

\twolineshloka
{यस्मात् न उद्विजते लोकः लोकात् न उद्विजते च यः}
{हर्षामर्षभयोद्वेगैः मुक्तः यः सः च मे प्रियः}% .. 12-15

\twolineshloka
{अनपेक्षः शुचिः दक्षः उदासीनः गतव्यथः}
{सर्वारम्भपरित्यागी यः मद्भक्तः सः मे प्रियः}% .. 12-16

\twolineshloka
{यः न हृष्यति न द्वेष्टि न शोचति न काङ्क्षति}
{शुभाशुभपरित्यागी भक्तिमान् यः सः मे प्रियः}% .. 12-17

\twolineshloka
{समः शत्रौ च मित्रे च तथा मानापमानयोः}
{शीतोष्णसुखदुःखेषु समः सङ्गविवर्जितः}% .. 12-18

\twolineshloka
{तुल्यनिन्दास्तुतिः मौनी सन्तुष्टः येन केनचित्}
{अनिकेतः स्थिरमतिः भक्तिमान् मे प्रियः नरः}% .. 12-19

\twolineshloka
{ये तु धर्म्यामृतम् इदं यथा उक्तं पर्युपासते}
{श्रद्दधानाः मत्परमाः भक्ताः ते अतीव मे प्रियाः}% .. 12-20

{॥ॐ तत्सदिति श्रीमद्भगवद्गीतासूपनिषत्सु ब्रह्मविद्यायां योगशास्त्रे श्रीकृष्णार्जुनसंवादे भक्तियोगो नाम द्वादशोऽध्यायः॥}

\chapt{त्रयोदशोऽध्यायः\textsf{---}क्षेत्रक्षेत्रज्ञविभागयोगः}

%\uvacha{अर्जुन उवाच}

%\twolineshloka
%{प्रकृतिं पुरुषं चैव क्षेत्रं क्षेत्रज्ञमेव च}
%{एतद्वेदितुमिच्छामि ज्ञानं ज्ञेयं च केशव}% .. 13-0

\uvacha{श्रीभगवानुवाच}

\twolineshloka
{इदं शरीरं कौन्तेय क्षेत्रम् इति अभिधीयते}
{एतत् यः वेत्ति तं प्राहुः क्षेत्रज्ञः इति तद्विदः}% .. 13-1

\twolineshloka
{क्षेत्रज्ञं च अपि मां विद्धि सर्वक्षेत्रेषु भारत}
{क्षेत्रक्षेत्रज्ञयोः ज्ञानं यत् तत् ज्ञानं मतं मम}% .. 13-2

\twolineshloka
{तत् क्षेत्रं यत् च यादृक् च यत् विकारि यतः च यत्}
{सः च यः यत् प्रभावः च तत् समासेन मे शृणु}% .. 13-3

\twolineshloka
{ऋषिभिः बहुधा गीतं छन्दोभिः विविधैः पृथक्}
{ब्रह्मसूत्रपदैः च एव हेतुमद्भिः विनिश्चितैः}% .. 13-4

\twolineshloka
{महाभूतानि अहङ्कारः बुद्धिः अव्यक्तम् एव च}
{इन्द्रियाणि दशैकं च पञ्च च इन्द्रियगोचराः}% .. 13-5

\twolineshloka
{इच्छा द्वेषः सुखं दुःखं सङ्घातः चेतना धृतिः}
{एतत् क्षेत्रं समासेन सविकारम् उदाहृतम्}% .. 13-6

\twolineshloka
{अमानित्वम् अदम्भित्वम् अहिंसा क्षान्तिः आर्जवम्}
{आचार्योपासनं शौचं स्थैर्यम् आत्मविनिग्रहः}% .. 13-7

\twolineshloka
{इन्द्रियार्थेषु वैराग्यम् अनहङ्कारः एव च}
{जन्ममृत्युजराव्याधिदुःखदोषानुदर्शनम्}% .. 13-8

\twolineshloka
{असक्तिः अनभिष्वङ्गः पुत्रदारगृहादिषु}
{नित्यं च समचित्तत्वम् इष्ट अनिष्टोपपत्तिषु}% .. 13-9

\twolineshloka
{मयि च अनन्ययोगेन भक्तिः अव्यभिचारिणी}
{विविक्तदेशसेवित्वम् अरतिः जनसंसदि}% .. 13-10

\twolineshloka
{अध्यात्मज्ञाननित्यत्वं तत्त्वज्ञानार्थदर्शनम्}
{एतत् ज्ञानम् इति प्रोक्तम् अज्ञानं यत् अतः अन्यथा}% .. 13-11

\twolineshloka
{ज्ञेयं यत् तत् प्रवक्ष्यामि यत् ज्ञात्वा अमृतम् अश्नुते}
{अनादिमत् परं ब्रह्म न सत् तत् न असत् उच्यते}% .. 13-12

\twolineshloka
{सर्वतः पाणिपादं तत् सर्वतः अक्षिशिरोमुखम्}
{सर्वतः श्रुतिमत् लोके सर्वम् आवृत्य तिष्ठति}% .. 13-13

\twolineshloka
{सर्वेन्द्रियगुणाभासं सर्वेन्द्रियविवर्जितम्}
{असक्तं सर्वभृत् च एव निर्गुणं गुणभोक्तृ च}% .. 13-14

\twolineshloka
{बहिरन्तः च भूतानाम् अचरं चरम् एव च}
{सूक्ष्मत्वात् तत् अविज्ञेयं दूरस्थं च अन्तिके च तत्}% .. 13-15

\twolineshloka
{अविभक्तं च भूतेषु विभक्तम् इव च स्थितम्}
{भूतभर्तृ च तत् ज्ञेयं ग्रसिष्णु प्रभविष्णु च}% .. 13-16

\twolineshloka
{ज्योतिषाम् अपि तत् ज्योतिः तमसः परम् उच्यते}
{ज्ञानं ज्ञेयं ज्ञानगम्यं हृदि सर्वस्य विष्ठितम्}% .. 13-17

\twolineshloka
{इति क्षेत्रं तथा ज्ञानं ज्ञेयं च उक्तं समासतः}
{मद्भक्तः एतत् विज्ञाय मद्भावाय उपपद्यते}% .. 13-18

\twolineshloka
{प्रकृतिं पुरुषं च एव विद्धि अनादी उभौ अपि}
{विकारान् च गुणान् च एव विद्धि प्रकृतिसम्भवान्}% .. 13-19

\twolineshloka
{कार्यकारणकर्तृत्वे हेतुः प्रकृतिः उच्यते}
{पुरुषः सुखदुःखानां भोक्तृत्वे हेतुः उच्यते}% .. 13-20

\twolineshloka
{पुरुषः प्रकृतिस्थः हि भुङ्क्ते प्रकृतिजान् गुणान्}
{कारणं गुणसङ्गः अस्य सत् असत् योनिजन्मसु}% .. 13-21

\twolineshloka
{उपद्रष्टा अनुमन्ता च भर्ता भोक्ता महेश्वरः}
{परमात्मा इति च अपि उक्तः देहे अस्मिन् पुरुषः परः}% .. 13-22

\twolineshloka
{यः एवं वेत्ति पुरुषं प्रकृतिं च गुणैः सह}
{सर्वथा वर्तमानः अपि न सः भूयः अभिजायते}% .. 13-23

\twolineshloka
{ध्यानेन आत्मनि पश्यन्ति केचित् आत्मानम् आत्मना}
{अन्ये साङ्ख्येन योगेन कर्मयोगेन च अपरे}% .. 13-24

\twolineshloka
{अन्ये तु एवम् अजानन्तः श्रुत्वा अन्येभ्यः उपासते}
{ते अपि च अतितरन्ति एव मृत्युं श्रुतिपरायणाः}% .. 13-25

\twolineshloka
{यावत् सञ्जायते किञ्चित् सत्त्वं स्थावरजङ्गमम्}
{क्षेत्रक्षेत्रज्ञसंयोगात् तत् विद्धि भरतर्षभ}% .. 13-26

\twolineshloka
{समं सर्वेषु भूतेषु तिष्ठन्तं परमेश्वरम्}
{विनश्यत्सु अविनश्यन्तं यः पश्यति सः पश्यति}% .. 13-27

\twolineshloka
{समं पश्यन् हि सर्वत्र समवस्थितम् ईश्वरम्}
{न हिनस्ति आत्मना आत्मानं ततः याति परां गतिम्}% .. 13-28

\twolineshloka
{प्रकृत्या एव च कर्माणि क्रियमाणानि सर्वशः}
{यः पश्यति तथा आत्मानम् अकर्तारं सः पश्यति}% .. 13-29

\twolineshloka
{यदा भूतपृथग्भावम् एकस्थम् अनुपश्यति}
{ततः एव च विस्तारं ब्रह्म सम्पद्यते तदा}% .. 13-30

\twolineshloka
{अनादित्वात् निर्गुणत्वात् परमात्मा अयम् अव्ययः}
{शरीरस्थः अपि कौन्तेय न करोति न लिप्यते}% .. 13-31

\twolineshloka
{यथा सर्वगतं सौक्ष्म्यात् आकाशं न उपलिप्यते}
{सर्वत्रावस्थितः देहे तथा आत्मा न उपलिप्यते}% .. 13-32

\twolineshloka
{यथा प्रकाशयति एकः कृत्स्नं लोकम् इमं रविः}
{क्षेत्रं क्षेत्री तथा कृत्स्नं प्रकाशयति भारत}% .. 13-33

\twolineshloka
{क्षेत्रक्षेत्रज्ञयोः एवम् अन्तरं ज्ञानचक्षुषा}
{भूतप्रकृतिमोक्षं च ये विदुः यान्ति ते परम्}% .. 13-34

{॥ॐ तत्सदिति श्रीमद्भगवद्गीतासूपनिषत्सु ब्रह्मविद्यायां योगशास्त्रे श्रीकृष्णार्जुनसंवादे क्षेत्रक्षेत्रज्ञविभागयोगो नाम त्रयोदशोऽध्यायः॥}

\chapt{चतुर्दशोऽध्यायः\textsf{---}गुणत्रयविभागयोगः}

\uvacha{श्रीभगवानुवाच}

\twolineshloka
{परं भूयः प्रवक्ष्यामि ज्ञानानां ज्ञानम् उत्तमम्}
{यत् ज्ञात्वा मुनयः सर्वे परां सिद्धिम् इतः गताः}% .. 14-1

\twolineshloka
{इदं ज्ञानम् उपाश्रित्य मम साधर्म्यम् आगताः}
{सर्गे अपि न उपजायन्ते प्रलये न व्यथन्ति च}% .. 14-2

\twolineshloka
{मम योनिः महत् ब्रह्म तस्मिन् गर्भं दधामि अहम्}
{सम्भवः सर्वभूतानां ततः भवति भारत}% .. 14-3

\twolineshloka
{सर्वयोनिषु कौन्तेय मूर्तयः सम्भवन्ति याः}
{तासां ब्रह्म महत् योनिः अहं बीजप्रदः पिता}% .. 14-4

\twolineshloka
{सत्त्वं रजः तमः इति गुणाः प्रकृतिसम्भवाः}
{निबध्नन्ति महाबाहो देहे देहिनम् अव्ययम्}% .. 14-5

\twolineshloka
{तत्र सत्त्वं निर्मलत्वात् प्रकाशकम् अनामयम्}
{सुखसङ्गेन बध्नाति ज्ञानसङ्गेन च अनघ}% .. 14-6

\twolineshloka
{रजः रागात्मकं विद्धि तृष्णासङ्गसमुद्भवम्}
{तत् निबध्नाति कौन्तेय कर्मसङ्गेन देहिनम्}% .. 14-7

\twolineshloka
{तमः तु अज्ञानजं विद्धि मोहनं सर्वदेहिनाम्}
{प्रमादालस्यनिद्राभिः तत् निबध्नाति भारत}% .. 14-8

\twolineshloka
{सत्त्वं सुखे सञ्जयति रजः कर्मणि भारत}
{ज्ञानम् आवृत्य तु तमः प्रमादे सञ्जयति उत}% .. 14-9

\twolineshloka
{रजः तमः च अभिभूय सत्त्वं भवति भारत}
{रजः सत्त्वं तमः च एव तमः सत्त्वं रजः तथा}% .. 14-10

\twolineshloka
{सर्वद्वारेषु देहे अस्मिन् प्रकाशः उपजायते}
{ज्ञानं यदा तदा विद्यात् विवृद्धं सत्त्वम् इति उत}% .. 14-11

\twolineshloka
{लोभः प्रवृत्तिः आरम्भः कर्मणाम् अशमः स्पृहा}
{रजसि एतानि जायन्ते विवृद्धे भरतर्षभ}% .. 14-12

\twolineshloka
{अप्रकाशः अप्रवृत्तिः च प्रमादः मोहः एव च}
{तमसि एतानि जायन्ते विवृद्धे कुरुनन्दन}% .. 14-13

\twolineshloka
{यदा सत्त्वे प्रवृद्धे तु प्रलयं याति देहभृत्}
{तदा उत्तमविदां लोकान् अमलान् प्रतिपद्यते}% .. 14-14

\twolineshloka
{रजसि प्रलयं गत्वा कर्मसङ्गिषु जायते}
{तथा प्रलीनः तमसि मूढयोनिषु जायते}% .. 14-15

\twolineshloka
{कर्मणः सुकृतस्य आहुः सात्त्विकं निर्मलं फलम्}
{रजसः तु फलं दुःखम् अज्ञानं तमसः फलम्}% .. 14-16

\twolineshloka
{सत्त्वात् सञ्जायते ज्ञानं रजसः लोभः एव च}
{प्रमादमोहौ तमसः भवतः अज्ञानम् एव च}% .. 14-17

\twolineshloka
{ऊर्ध्वं गच्छन्ति सत्त्वस्थाः मध्ये तिष्ठन्ति राजसाः}
{जघन्यगुणवृत्तिस्थाः अधः गच्छन्ति तामसाः}% .. 14-18

\twolineshloka
{न अन्यं गुणेभ्यः कर्तारं यदा द्रष्टा अनुपश्यति}
{गुणेभ्यः च परं वेत्ति मद्भावं सः अधिगच्छति}% .. 14-19

\twolineshloka
{गुणान् एतान् अतीत्य त्रीन् देही देहसमुद्भवान्}
{जन्ममृत्युजरादुःखैः विमुक्तः अमृतम् अश्नुते}% .. 14-20

\uvacha{अर्जुन उवाच}

\twolineshloka
{कैः लिङ्गैः त्रीन् गुणान् एतान् अतीतः भवति प्रभो}
{किम् आचारः कथं च एतान् त्रीन् गुणान् अतिवर्तते}% .. 14-21

\uvacha{श्रीभगवानुवाच}

\twolineshloka
{प्रकाशं च प्रवृत्तिं च मोहम् एव च पाण्डव}
{न द्वेष्टि सम्प्रवृत्तानि न निवृत्तानि काङ्क्षति}% .. 14-22

\twolineshloka
{उदासीनवत् आसीनः गुणैः यः न विचाल्यते}
{गुणाः वर्तन्ते इति एवं यः अवतिष्ठति न इङ्गते}% .. 14-23

\twolineshloka
{समदुःखसुखः स्वस्थः समलोष्टाश्मकाञ्चनः}
{तुल्यप्रियाप्रियः धीरः तुल्यनिन्दात्मसंस्तुतिः}% .. 14-24

\twolineshloka
{मानापमानयोः तुल्यः तुल्यः मित्रारिपक्षयोः}
{सर्वारम्भपरित्यागी गुणातीतः सः उच्यते}% .. 14-25

\twolineshloka
{मां च यः अव्यभिचारेण भक्तियोगेन सेवते}
{सः गुणान् समतीत्य एतान् ब्रह्मभूयाय कल्पते}% .. 14-26

\twolineshloka
{ब्रह्मणः हि प्रतिष्ठा अहम् अमृतस्य अव्ययस्य च}
{शाश्वतस्य च धर्मस्य सुखस्य एकान्तिकस्य च}% .. 14-27

{॥ॐ तत्सदिति श्रीमद्भगवद्गीतासूपनिषत्सु ब्रह्मविद्यायां योगशास्त्रे श्रीकृष्णार्जुनसंवादे गुणत्रयविभागयोगो नाम चतुर्दशोऽध्यायः॥}

\chapt{पञ्चदशोऽध्यायः\textsf{---}पुरुषोत्तमयोगः}

\uvacha{श्रीभगवानुवाच}

\twolineshloka
{ऊर्ध्वमूलम् अधःशाखम् अश्वत्थं प्राहुः अव्ययम्}
{छन्दांसि यस्य पर्णानि यः तं वेद सः वेदवित्}% .. 15-1

\fourlineindentedshloka
{अधः च ऊर्ध्वं प्रसृताः तस्य शाखाः}
{गुणप्रवृद्धाः विषयप्रवालाः}
{अधः च मूलानि अनुसन्ततानि}
{कर्मानुबन्धीनि मनुष्यलोके}% .. 15-2

\fourlineindentedshloka
{न रूपम् अस्य इह तथा उपलभ्यते}
{न अन्तः न च आदिः न च सम्प्रतिष्ठा}
{अश्वत्थम् एनं सुविरूढमूलम्}
{असङ्गशस्त्रेण दृढेन छित्त्वा}% .. 15-3

\fourlineindentedshloka
{ततः पदं तत् परिमार्गितव्यं}
{यस्मिन् गताः न निवर्तन्ति भूयः}
{तम् एव च आद्यं पुरुषं प्रपद्ये}
{यतः प्रवृत्तिः प्रसृता पुराणी}% .. 15-4

\fourlineindentedshloka
{निर्मानमोहाः जितसङ्गदोषाः}
{अध्यात्मनित्याः विनिवृत्तकामाः}
{द्वन्द्वैः विमुक्ताः सुखदुःखसंज्ञैः}
{गच्छन्ति अमूढाः पदम् अव्ययं तत्}% .. 15-5

\twolineshloka
{न तत् भासयते सूर्यः न शशाङ्कः न पावकः}
{यत् गत्वा न निवर्तन्ते तत् धाम परमं मम}% .. 15-6

\twolineshloka
{मम एव अंशः जीवलोके जीवभूतः सनातनः}
{मनःषष्ठानीन्द्रियाणि प्रकृतिस्थानि कर्षति}% .. 15-7

\twolineshloka
{शरीरं यत् अवाप्नोति यत् च अपि उत्क्रामति ईश्वरः}
{गृहीत्वा एतानि संयाति वायुः गन्धान् इव आशयात्}% .. 15-8

\twolineshloka
{श्रोत्रं चक्षुः स्पर्शनं च रसनं घ्राणम् एव च}
{अधिष्ठाय मनः च अयं विषयान् उपसेवते}% .. 15-9

\twolineshloka
{उत्क्रामन्तं स्थितं वा अपि भुञ्जानं वा गुणान्वितम्}
{विमूढाः न अनुपश्यन्ति पश्यन्ति ज्ञानचक्षुषः}% .. 15-10

\twolineshloka
{यतन्तः योगिनः च एनं पश्यन्ति आत्मनि अवस्थितम्}
{यतन्तः अपि अकृतात्मानः न एनं पश्यन्ति अचेतसः}% .. 15-11

\twolineshloka
{यत् आदित्यगतं तेजः जगत् भासयते अखिलम्}
{यत् चन्द्रमसि यत् च अग्नौ तत् तेजः विद्धि मामकम्}% .. 15-12

\twolineshloka
{गाम् आविश्य च भूतानि धारयामि अहम् ओजसा}
{पुष्णामि च ओषधीः सर्वाः सोमः भूत्वा रसात्मकः}% .. 15-13

\twolineshloka
{अहं वैश्वानरः भूत्वा प्राणिनां देहम् आश्रितः}
{प्राणापानसमायुक्तः पचामि अन्नं चतुर्विधम्}% .. 15-14

\fourlineindentedshloka
{सर्वस्य च अहं हृदि सन्निविष्टः}
{मत्तः स्मृतिः ज्ञानम् अपोहनं च}
{वेदैः च सर्वैः अहम् एव वेद्यः}
{वेदान्तकृत् वेदवित् एव च अहम्}% .. 15-15

\twolineshloka
{द्वौ इमौ पुरुषौ लोके क्षरः च अक्षरः एव च}
{क्षरः सर्वाणि भूतानि कूटस्थः अक्षरः उच्यते}% .. 15-16

\twolineshloka
{उत्तमः पुरुषः तु अन्यः परमात्मा इति उदाहृतः}
{यः लोकत्रयम् आविश्य बिभर्ति अव्ययः ईश्वरः}% .. 15-17

\twolineshloka
{यस्मात् क्षरम् अतीतः अहम् अक्षरात् अपि च उत्तमः}
{अतः अस्मि लोके वेदे च प्रथितः पुरुषोत्तमः}% .. 15-18

\twolineshloka
{यः माम् एवम् असम्मूढः जानाति पुरुषोत्तमम्}
{सः सर्ववित् भजति मां सर्वभावेन भारत}% .. 15-19

\twolineshloka
{इति गुह्यतमं शास्त्रम् इदम् उक्तं मया अनघ}
{एतत् बुद्ध्वा बुद्धिमान् स्यात् कृतकृत्यः च भारत}% .. 15-20

{॥ॐ तत्सदिति श्रीमद्भगवद्गीतासूपनिषत्सु ब्रह्मविद्यायां योगशास्त्रे श्रीकृष्णार्जुनसंवादे पुरुषोत्तमयोगो नाम पञ्चदशोऽध्यायः॥}

\chapt{षोडशोऽध्यायः\textsf{---}दैवासुरसम्पद्विभागयोगः}

\uvacha{श्रीभगवानुवाच}

\twolineshloka
{अभयं सत्त्वसंशुद्धिः ज्ञानयोगव्यवस्थितिः}
{दानं दमः च यज्ञः च स्वाध्यायः तपः आर्जवम्}% .. 16-1

\twolineshloka
{अहिंसा सत्यम् अक्रोधः त्यागः शान्तिः अपैशुनम्}
{दया भूतेषु अलोलुप्त्वं मार्दवं ह्रीः अचापलम्}% .. 16-2

\twolineshloka
{तेजः क्षमा धृतिः शौचम् अद्रोहः न अतिमानिता}
{भवन्ति सम्पदं दैवीम् अभिजातस्य भारत}% .. 16-3

\twolineshloka
{दम्भः दर्पः अभिमानः च क्रोधः पारुष्यम् एव च}
{अज्ञानं च अभिजातस्य पार्थ सम्पदम् आसुरीम्}% .. 16-4

\twolineshloka
{दैवी सम्पत् विमोक्षाय निबन्धाय आसुरी मता}
{मा शुचः सम्पदं दैवीम् अभिजातः असि पाण्डव}% .. 16-5

\twolineshloka
{द्वौ भूतसर्गौ लोके अस्मिन् दैवः आसुरः एव च}
{दैवः विस्तरशः प्रोक्तः आसुरं पार्थ मे शृणु}% .. 16-6

\twolineshloka
{प्रवृत्तिं च निवृत्तिं च जनाः न विदुः आसुराः}
{न शौचं न अपि च आचारः न सत्यं तेषु विद्यते}% .. 16-7

\twolineshloka
{असत्यम् अप्रतिष्ठं ते जगत् आहुः अनीश्वरम्}
{अपरस्परसम्भूतं किम् अन्यत् कामहैतुकम्}% .. 16-8

\twolineshloka
{एतां दृष्टिम् अवष्टभ्य नष्टात्मानः अल्पबुद्धयः}
{प्रभवन्ति उग्रकर्माणः क्षयाय जगतः अहिताः}% .. 16-9

\twolineshloka
{कामम् आश्रित्य दुष्पूरं दम्भमानमदान्विताः}
{मोहात् गृहीत्वा असत् ग्राहान् प्रवर्तन्ते अशुचिव्रताः}% .. 16-10

\twolineshloka
{चिन्ताम् अपरिमेयां च प्रलयान्ताम् उपाश्रिताः}
{कामोपभोगपरमाः एतावत् इति निश्चिताः}% .. 16-11

\twolineshloka
{आशापाशशतैः बद्धाः कामक्रोधपरायणाः}
{ईहन्ते कामभोगार्थम् अन्यायेन अर्थसञ्चयान्}% .. 16-12

\twolineshloka
{इदम् अद्य मया लब्धम् इमं प्राप्स्ये मनोरथम्}
{इदम् अस्ति इदम् अपि मे भविष्यति पुनः धनम्}% .. 16-13

\twolineshloka
{असौ मया हतः शत्रुः हनिष्ये च अपरान् अपि}
{ईश्वरः अहम् अहं भोगी सिद्धः अहं बलवान् सुखी}% .. 16-14

\twolineshloka
{आढ्यः अभिजनवान् अस्मि कः अन्यः अस्ति सदृशः मया}
{यक्ष्ये दास्यामि मोदिष्ये इति अज्ञानविमोहिताः}% .. 16-15

\twolineshloka
{अनेकचित्तविभ्रान्ताः मोहजालसमावृताः}
{प्रसक्ताः कामभोगेषु पतन्ति नरके अशुचौ}% .. 16-16

\twolineshloka
{आत्मसम्भाविताः स्तब्धाः धनमानमदान्विताः}
{यजन्ते नामयज्ञैः ते दम्भेन अविधिपूर्वकम्}% .. 16-17

\twolineshloka
{अहङ्कारं बलं दर्पं कामं क्रोधं च संश्रिताः}
{माम् आत्मपरदेहेषु प्रद्विषन्तः अभ्यसूयकाः}% .. 16-18

\twolineshloka
{तान् अहं द्विषतः क्रूरान् संसारेषु नराधमान्}
{क्षिपामि अजस्रम् अशुभान् आसुरीषु एव योनिषु}% .. 16-19

\twolineshloka
{आसुरीं योनिम् आपन्नाः मूढाः जन्मनि जन्मनि}
{माम् अप्राप्य एव कौन्तेय ततः यान्ति अधमां गतिम्}% .. 16-20

\twolineshloka
{त्रिविधं नरकस्य इदं द्वारं नाशनम् आत्मनः}
{कामः क्रोधः तथा लोभः तस्मात् एतत् त्रयं त्यजेत्}% .. 16-21

\twolineshloka
{एतैः विमुक्तः कौन्तेय तमोद्वारैः त्रिभिः नरः}
{आचरति आत्मनः श्रेयः ततः याति परां गतिम्}% .. 16-22

\twolineshloka
{यः शास्त्रविधिम् उत्सृज्य वर्तते कामकारतः}
{न सः सिद्धिम् अवाप्नोति न सुखं न परां गतिम्}% .. 16-23

\twolineshloka
{तस्मात् शास्त्रं प्रमाणं ते कार्याकार्यव्यवस्थितौ}
{ज्ञात्वा शास्त्रविधानोक्तं कर्म कर्तुम् इह अर्हसि}% .. 16-24

{॥ॐ तत्सदिति श्रीमद्भगवद्गीतासूपनिषत्सु ब्रह्मविद्यायां योगशास्त्रे श्रीकृष्णार्जुनसंवादे दैवासुरसम्पद्विभागयोगो नाम षोडशोऽध्यायः॥}

\chapt{सप्तदशोऽध्यायः\textsf{---}श्रद्धात्रयविभागयोगः}

\uvacha{अर्जुन उवाच}

\twolineshloka
{ये शास्त्रविधिम् उत्सृज्य यजन्ते श्रद्धया अन्विताः}
{तेषां निष्ठा तु का कृष्ण सत्त्वम् आहो रजः तमः}% .. 17-1

\uvacha{श्रीभगवानुवाच}

\twolineshloka
{त्रिविधा भवति श्रद्धा देहिनां सा स्वभावजा}
{सात्त्विकी राजसी च एव तामसी च इति तां शृणु}% .. 17-2

\twolineshloka
{सत्त्वानुरूपा सर्वस्य श्रद्धा भवति भारत}
{श्रद्धामयः अयं पुरुषः यः यत् श्रद्धः सः एव सः}% .. 17-3

\twolineshloka
{यजन्ते सात्त्विकाः देवान् यक्षरक्षांसि राजसाः}
{प्रेतान् भूतगणान् च अन्ये यजन्ते तामसाः जनाः}% .. 17-4

\twolineshloka
{अशास्त्रविहितं घोरं तप्यन्ते ये तपः जनाः}
{दम्भाहङ्कारसंयुक्ताः कामरागबलान्विताः}% .. 17-5

\twolineshloka
{कर्षयन्तः शरीरस्थं भूतग्रामम् अचेतसः}
{मां च एव अन्तःशरीरस्थं तान् विद्धि आसुरनिश्चयान्}% .. 17-6

\twolineshloka
{आहारः तु अपि सर्वस्य त्रिविधः भवति प्रियः}
{यज्ञः तपः तथा दानं तेषां भेदम् इमं शृणु}% .. 17-7

\twolineshloka
{आयुःसत्त्वबलारोग्यसुखप्रीतिविवर्धनाः}
{रस्याः स्निग्धाः स्थिराः हृद्याः आहाराः सात्त्विकप्रियाः}% .. 17-8

\twolineshloka
{कट्वम्ललवणात्युष्णतीक्ष्णरूक्षविदाहिनः}
{आहाराः राजसस्य इष्टाः दुःखशोकामयप्रदाः}% .. 17-9

\twolineshloka
{यातयामं गतरसं पूति पर्युषितं च यत्}
{उच्छिष्टम् अपि च अमेध्यं भोजनं तामसप्रियम्}% .. 17-10

\twolineshloka
{अफलाकाङ्क्षिभिः यज्ञः विधिदृष्टः यः इज्यते}
{यष्टव्यम् एव इति मनः समाधाय सः सात्त्विकः}% .. 17-11

\twolineshloka
{अभिसन्धाय तु फलं दम्भार्थम् अपि च एव यत्}
{इज्यते भरतश्रेष्ठ तं यज्ञं विद्धि राजसम्}% .. 17-12

\twolineshloka
{विधिहीनम् असृष्टान्नं मन्त्रहीनम् अदक्षिणम्}
{श्रद्धाविरहितं यज्ञं तामसं परिचक्षते}% .. 17-13

\twolineshloka
{देवद्विजगुरुप्राज्ञपूजनं शौचम् आर्जवम्}
{ब्रह्मचर्यम् अहिंसा च शारीरं तपः उच्यते}% .. 17-14

\twolineshloka
{अनुद्वेगकरं वाक्यं सत्यं प्रियहितं च यत्}
{स्वाध्यायाभ्यसनं च एव वाङ्मयं तपः उच्यते}% .. 17-15

\twolineshloka
{मनः प्रसादः सौम्यत्वं मौनम् आत्मविनिग्रहः}
{भावसंशुद्धिः इति एतत् तपः मानसम् उच्यते}% .. 17-16

\twolineshloka
{श्रद्धया परया तप्तं तपः तत् त्रिविधं नरैः}
{अफलाकाङ्क्षिभिः युक्तैः सात्त्विकं परिचक्षते}% .. 17-17

\twolineshloka
{सत्कारमानपूजार्थं तपः दम्भेन च एव यत्}
{क्रियते तत् इह प्रोक्तं राजसं चलम् अध्रुवम्}% .. 17-18

\twolineshloka
{मूढग्राहेण आत्मनः यत् पीडया क्रियते तपः}
{परस्य उत्सादनार्थं वा तत् तामसम् उदाहृतम्}% .. 17-19

\twolineshloka
{दातव्यम् इति यत् दानं दीयते अनुपकारिणे}
{देशे काले च पात्रे च तत् दानं सात्त्विकं स्मृतम्}% .. 17-20

\twolineshloka
{यत् तु प्रत्युपकारार्थं फलम् उद्दिश्य वा पुनः}
{दीयते च परिक्लिष्टं तत् दानं राजसं स्मृतम्}% .. 17-21

\twolineshloka
{अदेशकाले यत् दानम् अपात्रेभ्यः च दीयते}
{असत्कृतम् अवज्ञातं तत् तामसम् उदाहृतम्}% .. 17-22

\twolineshloka
{ओं तत् सत् इति निर्देशः ब्रह्मणः त्रिविधः स्मृतः}
{ब्राह्मणाः तेन वेदाः च यज्ञाः च विहिताः पुरा}% .. 17-23

\twolineshloka
{तस्मात् ओम् इति उदाहृत्य यज्ञदानतपःक्रियाः}
{प्रवर्तन्ते विधानोक्ताः सततं ब्रह्मवादिनाम्}% .. 17-24

\twolineshloka
{तत् इति अनभिसन्धाय फलं यज्ञतपःक्रियाः}
{दानक्रियाः च विविधाः क्रियन्ते मोक्षकाङ्क्षिभिः}% .. 17-25

\twolineshloka
{सद्भावे साधुभावे च सत् इति एतत् प्रयुज्यते}
{प्रशस्ते कर्मणि तथा सत् शब्दः पार्थ युज्यते}% .. 17-26

\twolineshloka
{यज्ञे तपसि दाने च स्थितिः सत् इति च उच्यते}
{कर्म च एव तदर्थीयं सत् इति एव अभिधीयते}% .. 17-27

\twolineshloka
{अश्रद्धया हुतं दत्तं तपः तप्तं कृतं च यत्}
{असत् इति उच्यते पार्थ न च तत् प्रेत्य न उ इह}% .. 17-28

{॥ॐ तत्सदिति श्रीमद्भगवद्गीतासूपनिषत्सु ब्रह्मविद्यायां योगशास्त्रे श्रीकृष्णार्जुनसंवादे श्रद्धात्रयविभागयोगो नाम सप्तदशोऽध्यायः॥}

\chapt{अष्टादशोऽध्यायः\textsf{---}मोक्षसन्न्यासयोगः}

\uvacha{अर्जुन उवाच}

\twolineshloka
{सन्न्यासस्य महाबाहो तत्त्वम् इच्छामि वेदितुम्}
{त्यागस्य च हृषीकेश पृथक् केशिनिषूदन}% .. 18-1

\uvacha{श्रीभगवानुवाच}

\twolineshloka
{काम्यानां कर्मणां न्यासं सन्न्यासं कवयः विदुः}
{सर्वकर्मफलत्यागं प्राहुः त्यागं विचक्षणाः}% .. 18-2

\twolineshloka
{त्याज्यं दोषवत् इति एके कर्म प्राहुः मनीषिणः}
{यज्ञदानतपःकर्म न त्याज्यम् इति च अपरे}% .. 18-3

\twolineshloka
{निश्चयं शृणु मे तत्र त्यागे भरतसत्तम}
{त्यागः हि पुरुषव्याघ्र त्रिविधः सम्प्रकीर्तितः}% .. 18-4

\twolineshloka
{यज्ञदानतपःकर्म न त्याज्यं कार्यम् एव तत्}
{यज्ञः दानं तपः च एव पावनानि मनीषिणाम्}% .. 18-5

\twolineshloka
{एतानि अपि तु कर्माणि सङ्गं त्यक्त्वा फलानि च}
{कर्तव्यानि इति मे पार्थ निश्चितं मतम् उत्तमम्}% .. 18-6

\twolineshloka
{नियतस्य तु सन्न्यासः कर्मणः न उपपद्यते}
{मोहात् तस्य परित्यागः तामसः परिकीर्तितः}% .. 18-7

\twolineshloka
{दुःखम् इति एव यत् कर्म कायक्लेशभयात् त्यजेत्}
{सः कृत्वा राजसं त्यागं न एव त्यागफलं लभेत्}% .. 18-8

\twolineshloka
{कार्यम् इति एव यत् कर्म नियतं क्रियते अर्जुन}
{सङ्गं त्यक्त्वा फलं च एव सः त्यागः सात्त्विकः मतः}% .. 18-9

\twolineshloka
{न द्वेष्टि अकुशलं कर्म कुशले न अनुषज्जते}
{त्यागी सत्त्वसमाविष्टः मेधावी छिन्नसंशयः}% .. 18-10

\twolineshloka
{न हि देहभृता शक्यं त्यक्तुं कर्माणि अशेषतः}
{यः तु कर्मफलत्यागी सः त्यागी इति अभिधीयते}% .. 18-11

\twolineshloka
{अनिष्टम् इष्टं मिश्रं च त्रिविधं कर्मणः फलम्}
{भवति अत्यागिनां प्रेत्य न तु सन्न्यासिनां क्वचित्}% .. 18-12

\twolineshloka
{पञ्च एतानि महाबाहो कारणानि निबोध मे}
{साङ्ख्ये कृतान्ते प्रोक्तानि सिद्धये सर्वकर्मणाम्}% .. 18-13

\twolineshloka
{अधिष्ठानं तथा कर्ता करणं च पृथग्विधम्}
{विविधाः च पृथक् चेष्टाः दैवं च एव अत्र पञ्चमम्}% .. 18-14

\twolineshloka
{शरीरवाङ्मनोभिः यत् कर्म प्रारभते नरः}
{न्याय्यं वा विपरीतं वा पञ्च एते तस्य हेतवः}% .. 18-15

\twolineshloka
{तत्र एवं सति कर्तारम् आत्मानं केवलं तु यः}
{पश्यति अकृतबुद्धित्वात् न सः पश्यति दुर्मतिः}% .. 18-16

\twolineshloka
{यस्य न अहङ्कृतः भावः बुद्धिः यस्य न लिप्यते}
{हत्वा अपि सः इमान् लोकान् न हन्ति न निबध्यते}% .. 18-17

\twolineshloka
{ज्ञानं ज्ञेयं परिज्ञाता त्रिविधा कर्मचोदना}
{करणं कर्म कर्ता इति त्रिविधः कर्मसङ्ग्रहः}% .. 18-18

\twolineshloka
{ज्ञानं कर्म च कर्ता च त्रिधा एव गुणभेदतः}
{प्रोच्यते गुणसङ्ख्याने यथावत् शृणु तानि अपि}% .. 18-19

\twolineshloka
{सर्वभूतेषु येन एकं भावम् अव्ययम् ईक्षते}
{अविभक्तं विभक्तेषु तत् ज्ञानं विद्धि सात्त्विकम्}% .. 18-20

\twolineshloka
{पृथक्त्वेन तु यत् ज्ञानं नानाभावान् पृथग्विधान्}
{वेत्ति सर्वेषु भूतेषु तत् ज्ञानं विद्धि राजसम्}% .. 18-21

\twolineshloka
{यत् तु कृत्स्नवत् एकस्मिन् कार्ये सक्तम् अहैतुकम्}
{अतत्त्वार्थवत् अल्पं च तत् तामसम् उदाहृतम्}% .. 18-22

\twolineshloka
{नियतं सङ्गरहितम् अरागद्वेषतः कृतम्}
{अफलप्रेप्सुना कर्म यत् तत् सात्त्विकम् उच्यते}% .. 18-23

\twolineshloka
{यत् तु कामेप्सुना कर्म साहङ्कारेण वा पुनः}
{क्रियते बहुलायासं तत् राजसम् उदाहृतम्}% .. 18-24

\twolineshloka
{अनुबन्धं क्षयं हिंसाम् अनपेक्ष्य च पौरुषम्}
{मोहात् आरभ्यते कर्म यत् तत् तामसम् उच्यते}% .. 18-25

\twolineshloka
{मुक्तसङ्गः अनहंवादी धृत्युत्साहसमन्वितः}
{सिद्ध्यसिद्ध्योः निर्विकारः कर्ता सात्त्विकः उच्यते}% .. 18-26

\twolineshloka
{रागी कर्मफलप्रेप्सुः लुब्धः हिंसात्मकः अशुचिः}
{हर्षशोकान्वितः कर्ता राजसः परिकीर्तितः}% .. 18-27

\twolineshloka
{अयुक्तः प्राकृतः स्तब्धः शठः नैष्कृतिकः अलसः}
{विषादी दीर्घसूत्री च कर्ता तामसः उच्यते}% .. 18-28

\twolineshloka
{बुद्धेः भेदं धृतेः च एव गुणतः त्रिविधं शृणु}
{प्रोच्यमानम् अशेषेण पृथक्त्वेन धनञ्जय}% .. 18-29

\twolineshloka
{प्रवृत्तिं च निवृत्तिं च कार्याकार्ये भयाभये}
{बन्धं मोक्षं च या वेत्ति बुद्धिः सा पार्थ सात्त्विकी}% .. 18-30

\twolineshloka
{यया धर्मम् अधर्मं च कार्यं च अकार्यम् एव च}
{अयथावत् प्रजानाति बुद्धिः सा पार्थ राजसी}% .. 18-31

\twolineshloka
{अधर्मं धर्मम् इति या मन्यते तमसा आवृता}
{सर्वार्थान् विपरीतान् च बुद्धिः सा पार्थ तामसी}% .. 18-32

\twolineshloka
{धृत्या यया धारयते मनःप्राणेन्द्रियक्रियाः}
{योगेन अव्यभिचारिण्या धृतिः सा पार्थ सात्त्विकी}% .. 18-33

\twolineshloka
{यया तु धर्मकामार्थान् धृत्या धारयते अर्जुन}
{प्रसङ्गेन फलाकाङ्क्षी धृतिः सा पार्थ राजसी}% .. 18-34

\twolineshloka
{यया स्वप्नं भयं शोकं विषादं मदम् एव च}
{न विमुञ्चति दुर्मेधा धृतिः सा पार्थ तामसी}% .. 18-35

\twolineshloka
{सुखं तु इदानीं त्रिविधं शृणु मे भरतर्षभ}
{अभ्यासात् रमते यत्र दुःखान्तं च निगच्छति}% .. 18-36

\twolineshloka
{यत् तत् अग्रे विषम् इव परिणामे अमृतोपमम्}
{तत् सुखं सात्त्विकं प्रोक्तम् आत्मबुद्धिप्रसादजम्}% .. 18-37

\twolineshloka
{विषयेन्द्रियसंयोगात् यत् तत् अग्रे अमृतोपमम्}
{परिणामे विषम् इव तत् सुखं राजसं स्मृतम्}% .. 18-38

\twolineshloka
{यत् अग्रे च अनुबन्धे च सुखं मोहनम् आत्मनः}
{निद्रालस्यप्रमादोत्थं तत् तामसम् उदाहृतम्}% .. 18-39

\twolineshloka
{न तत् अस्ति पृथिव्यां वा दिवि देवेषु वा पुनः}
{सत्त्वं प्रकृतिजैः मुक्तं यत् एभिः स्यात् त्रिभिः गुणैः}% .. 18-40

\twolineshloka
{ब्राह्मणक्षत्रियविशां शूद्राणां च परन्तप}
{कर्माणि प्रविभक्तानि स्वभावप्रभवैः गुणैः}% .. 18-41

\twolineshloka
{शमः दमः तपः शौचं क्षान्तिः आर्जवम् एव च}
{ज्ञानं विज्ञानम् आस्तिक्यं ब्रह्मकर्म स्वभावजम्}% .. 18-42

\twolineshloka
{शौर्यं तेजः धृतिः दाक्ष्यं युद्धे च अपि अपलायनम्}
{दानम् ईश्वरभावः च क्षात्रं कर्म स्वभावजम्}% .. 18-43

\twolineshloka
{कृषिगौरक्ष्यवाणिज्यं वैश्यकर्म स्वभावजम्}
{परिचर्यात्मकं कर्म शूद्रस्य अपि स्वभावजम्}% .. 18-44

\twolineshloka
{स्वे स्वे कर्मणि अभिरतः संसिद्धिं लभते नरः}
{स्वकर्मनिरतः सिद्धिं यथा विन्दति तत् शृणु}% .. 18-45

\twolineshloka
{यतः प्रवृत्तिः भूतानां येन सर्वम् इदं ततम्}
{स्वकर्मणा तम् अभ्यर्च्य सिद्धिं विन्दति मानवः}% .. 18-46

\twolineshloka
{श्रेयान् स्वधर्मः विगुणः परधर्मात् स्वनुष्ठितात्}
{स्वभावनियतं कर्म कुर्वन् न आप्नोति किल्बिषम्}% .. 18-47

\twolineshloka
{सहजं कर्म कौन्तेय सदोषम् अपि न त्यजेत्}
{सर्वारम्भाः हि दोषेण धूमेन अग्निः इव आवृताः}% .. 18-48

\twolineshloka
{असक्तबुद्धिः सर्वत्र जितात्मा विगतस्पृहः}
{नैष्कर्म्यसिद्धिं परमां सन्न्यासेन अधिगच्छति}% .. 18-49

\twolineshloka
{सिद्धिं प्राप्तः यथा ब्रह्म तथा आप्नोति निबोध मे}
{समासेन एव कौन्तेय निष्ठा ज्ञानस्य या परा}% .. 18-50

\twolineshloka
{बुद्ध्या विशुद्धया युक्तः धृत्या आत्मानं नियम्य च}
{शब्दादीन् विषयान् त्यक्त्वा रागद्वेषौ व्युदस्य च}% .. 18-51

\twolineshloka
{विविक्तसेवी लघ्वाशी यतवाक्कायमानसः}
{ध्यानयोगपरः नित्यं वैराग्यं समुपाश्रितः}% .. 18-52

\twolineshloka
{अहङ्कारं बलं दर्पं कामं क्रोधं परिग्रहम्}
{विमुच्य निर्ममः शान्तः ब्रह्मभूयाय कल्पते}% .. 18-53

\twolineshloka
{ब्रह्मभूतः प्रसन्नात्मा न शोचति न काङ्क्षति}
{समः सर्वेषु भूतेषु मद्भक्तिं लभते पराम्}% .. 18-54

\twolineshloka
{भक्त्या माम् अभिजानाति यावान् यः च अस्मि तत्त्वतः}
{ततः मां तत्त्वतः ज्ञात्वा विशते तत् अनन्तरम्}% .. 18-55

\twolineshloka
{सर्वकर्माणि अपि सदा कुर्वाणः मद्व्यपाश्रयः}
{मत्प्रसादात् अवाप्नोति शाश्वतं पदम् अव्ययम्}% .. 18-56

\twolineshloka
{चेतसा सर्वकर्माणि मयि सन्न्यस्य मत्परः}
{बुद्धियोगम् उपाश्रित्य मच्चित्तः सततं भव}% .. 18-57

\twolineshloka
{मच्चित्तः सर्वदुर्गाणि मत्प्रसादात् तरिष्यसि}
{अथ चेत् त्वम् अहङ्कारात् न श्रोष्यसि विनङ्क्ष्यसि}% .. 18-58

\twolineshloka
{यत् अहङ्कारम् आश्रित्य न योत्स्ये इति मन्यसे}
{मिथ्या एषः व्यवसायः ते प्रकृतिः त्वां नियोक्ष्यति}% .. 18-59

\twolineshloka
{स्वभावजेन कौन्तेय निबद्धः स्वेन कर्मणा}
{कर्तुं न इच्छसि यत् मोहात् करिष्यसि अवशः अपि तत्}% .. 18-60

\twolineshloka
{ईश्वरः सर्वभूतानां हृद्देशे अर्जुन तिष्ठति}
{भ्रामयन् सर्वभूतानि यन्त्रारूढानि मायया}% .. 18-61

\twolineshloka
{तम् एव शरणं गच्छ सर्वभावेन भारत}
{तत् प्रसादात् परां शान्तिं स्थानं प्राप्स्यसि शाश्वतम्}% .. 18-62

\twolineshloka
{इति ते ज्ञानम् आख्यातं गुह्यात् गुह्यतरं मया}
{विमृश्य एतत् अशेषेण यथा इच्छसि तथा कुरु}% .. 18-63

\twolineshloka
{सर्वगुह्यतमं भूयः शृणु मे परमं वचः}
{इष्टः असि मे दृढम् इति ततः वक्ष्यामि ते हितम्}% .. 18-64

\twolineshloka
{मन्मनाः भव मद्भक्तः मद्याजी मां नमस्कुरु}
{माम् एव एष्यसि सत्यं ते प्रतिजाने प्रियः असि मे}% .. 18-65

\twolineshloka
{सर्वधर्मान् परित्यज्य माम् एकं शरणं व्रज}
{अहं त्वा सर्वपापेभ्यः मोक्षयिष्यामि मा शुचः}% .. 18-66

\twolineshloka
{इदं ते न अतपस्काय न अभक्ताय कदाचन}
{न च अशुश्रूषवे वाच्यं न च मां यः अभ्यसूयति}% .. 18-67

\twolineshloka
{यः इदं परमं गुह्यं मद्भक्तेषु अभिधास्यति}
{भक्तिं मयि परां कृत्वा माम् एव एष्यति असंशयः}% .. 18-68

\twolineshloka
{न च तस्मात् मनुष्येषु कश्चित् मे प्रियकृत्तमः}
{भविता न च मे तस्मात् अन्यः प्रियतरः भुवि}% .. 18-69

\twolineshloka
{अध्येष्यते च यः इमं धर्म्यं संवादम् आवयोः}
{ज्ञानयज्ञेन तेन अहम् इष्टः स्याम् इति मे मतिः}% .. 18-70

\twolineshloka
{श्रद्धावान् अनसूयः च शृणुयात् अपि यः नरः}
{सः अपि मुक्तः शुभान् लोकान् प्राप्नुयात् पुण्यकर्मणाम्}% .. 18-71

\twolineshloka
{कच्चित् एतत् श्रुतं पार्थ त्वया एकाग्रेण चेतसा}
{कच्चित् अज्ञानसम्मोहः प्रनष्टः ते धनञ्जय}% .. 18-72

\uvacha{अर्जुन उवाच}

\twolineshloka
{नष्टः मोहः स्मृतिः लब्धा त्वत् प्रसादात् मया अच्युत}
{स्थितः अस्मि गतसन्देहः करिष्ये वचनं तव}% .. 18-73

\uvacha{सञ्जय उवाच}

\twolineshloka
{इति अहं वासुदेवस्य पार्थस्य च महात्मनः}
{संवादम् इमम् अश्रौषम् अद्भुतं रोमहर्षणम्}% .. 18-74

\twolineshloka
{व्यासप्रसादात् श्रुतवान् एतत् गुह्यम् अहं परम्}
{योगं योगेश्वरात् कृष्णात् साक्षात् कथयतः स्वयम्}% .. 18-75

\twolineshloka
{राजन् संस्मृत्य संस्मृत्य संवादम् इमम् अद्भुतम्}
{केशवार्जुनयोः पुण्यं हृष्यामि च मुहुः मुहुः}% .. 18-76

\twolineshloka
{तत् च संस्मृत्य संस्मृत्य रूपम् अत्यद्भुतं हरेः}
{विस्मयः मे महान् राजन् हृष्यामि च पुनः पुनः}% .. 18-77

\twolineshloka
{यत्र योगेश्वरः कृष्णः यत्र पार्थः धनुर्धरः}
{तत्र श्रीः विजयः भूतिः ध्रुवा नीतिः मतिः मम}% .. 18-78

{॥ॐ तत्सदिति श्रीमद्भगवद्गीतासूपनिषत्सु ब्रह्मविद्यायां योगशास्त्रे श्रीकृष्णार्जुनसंवादे मोक्षसन्न्यासयोगो नाम अष्टादशोऽध्यायः॥}
