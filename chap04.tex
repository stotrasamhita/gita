% !TeX program = XeLaTeX
% !TeX root = gitabook.tex
\chapt{चतुर्थोऽध्यायः}
\uvacha{श्रीभगवानुवाच}
\twolineshloka
{इमं विवस्वते योगं प्रोक्तवानहमव्ययम्}
{विवस्वान् मनवे प्राह मनुरिक्ष्वाकवेऽब्रवीत्}% .. 4-1

\twolineshloka
{एवं परम्पराप्राप्तमिमं राजर्षयो विदुः}
{स कालेनेह महता योगो नष्टः परन्तप}% .. 4-2

\twolineshloka
{स एवायं मया तेऽद्य योगः प्रोक्तः पुरातनः}
{भक्तोऽसि मे सखा चेति रहस्यं ह्येतदुत्तमम्}% .. 4-3

\uvacha{अर्जुन उवाच}
\twolineshloka
{अपरं भवतो जन्म परं जन्म विवस्वतः}
{कथमेतद्विजानीयां त्वमादौ प्रोक्तवानिति}% .. 4-4

\uvacha{श्रीभगवानुवाच}
\twolineshloka
{बहूनि मे व्यतीतानि जन्मानि तव चार्जुन}
{तान्यहं वेद सर्वाणि न त्वं वेत्थ परन्तप}% .. 4-5

\twolineshloka
{अजोऽपि सन्नव्ययात्मा भूतानामीश्वरोऽपि सन्}
{प्रकृतिं स्वामधिष्ठाय सम्भवाम्यात्ममायया}% .. 4-6

\twolineshloka
{यदा यदा हि धर्मस्य ग्लानिर्भवति भारत}
{अभ्युत्थानमधर्मस्य तदाऽऽत्मानं सृजाम्यहम्}% .. 4-7

\twolineshloka
{परित्राणाय साधूनां विनाशाय च दुष्कृताम्}
{धर्मसंस्थापनार्थाय सम्भवामि युगे युगे}% .. 4-8

\twolineshloka
{जन्म कर्म च मे दिव्यमेवं यो वेत्ति तत्त्वतः}
{त्यक्त्वा देहं पुनर्जन्म नैति मामेति सोऽर्जुन}% .. 4-9

\twolineshloka
{वीतरागभयक्रोधा मन्मया मामुपाश्रिताः}
{बहवो ज्ञानतपसा पूता मद्भावमागताः}% .. 4-10

\twolineshloka
{ये यथा मां प्रपद्यन्ते तांस्तथैव भजाम्यहम्}
{मम वर्त्मानुवर्तन्ते मनुष्याः पार्थ सर्वशः}% .. 4-11

\twolineshloka
{काङ्क्षन्तः कर्मणां सिद्धिं यजन्त इह देवताः}
{क्षिप्रं हि मानुषे लोके सिद्धिर्भवति कर्मजा}% .. 4-12

\twolineshloka
{चातुर्वर्ण्यं मया सृष्टं गुणकर्मविभागशः}
{तस्य कर्तारमपि मां विद्‌ध्यकर्तारमव्ययम्}% .. 4-13

\twolineshloka
{न मां कर्माणि लिम्पन्ति न मे कर्मफले स्पृहा}
{इति मां योऽभिजानाति कर्मभिर्न स बध्यते}% .. 4-14

\twolineshloka
{एवं ज्ञात्वा कृतं कर्म पूर्वैरपि मुमुक्षुभिः}
{कुरु कर्मैव तस्मात्त्वं पूर्वैः पूर्वतरं कृतम्}% .. 4-15

\twolineshloka
{किं कर्म किमकर्मेति कवयोऽप्यत्र मोहिताः}
{तत्ते कर्म प्रवक्ष्यामि यज्ज्ञात्वा मोक्ष्यसेऽशुभात्}% .. 4-16

\twolineshloka
{कर्मणो ह्यपि बोद्धव्यं बोद्धव्यं च विकर्मणः}
{अकर्मणश्च बोद्धव्यं गहना कर्मणो गतिः}% .. 4-17

\twolineshloka
{कर्मण्यकर्म यः पश्येदकर्मणि च कर्म यः}
{स बुद्धिमान् मनुष्येषु स युक्तः कृत्स्नकर्मकृत्}% .. 4-18

\twolineshloka
{यस्य सर्वे समारम्भाः कामसङ्कल्पवर्जिताः}
{ज्ञानाग्निदग्धकर्माणं तमाहुः पण्डितं बुधाः}% .. 4-19

\twolineshloka
{त्यक्त्वा कर्मफलासङ्गं नित्यतृप्तो निराश्रयः}
{कर्मण्यभिप्रवृत्तोऽपि नैव किञ्चित् करोति सः}% .. 4-20

\twolineshloka
{निराशीर्यतचित्तात्मा त्यक्तसर्वपरिग्रहः}
{शारीरं केवलं कर्म कुर्वन्नऽऽप्नोति किल्बिषम्}% .. 4-21

\twolineshloka
{यदृच्छालाभसन्तुष्टो द्वन्द्वातीतो विमत्सरः}
{समः सिद्धावसिद्धौ च कृत्वाऽपि न निबध्यते}% .. 4-22

\twolineshloka
{गतसङ्गस्य मुक्तस्य ज्ञानावस्थितचेतसः}
{यज्ञायऽऽचरतः कर्म समग्रं प्रविलीयते}% .. 4-23

\twolineshloka
{ब्रह्मार्पणं ब्रह्म हविर्ब्रह्माग्नौ ब्रह्मणा हुतम्}
{ब्रह्मैव तेन गन्तव्यं ब्रह्मकर्मसमाधिना}% .. 4-24

\twolineshloka
{दैवमेवापरे यज्ञं योगिनः पर्युपासते}
{ब्रह्माग्नावपरे यज्ञं यज्ञेनैवोपजुह्वति}% .. 4-25

\twolineshloka
{श्रोत्रादीनीन्द्रियाण्यन्ये संयमाग्निषु जुह्वति}
{शब्दादीन् विषयानन्य इन्द्रियाग्निषु जुह्वति}% .. 4-26

\twolineshloka
{सर्वाणीन्द्रियकर्माणि प्राणकर्माणि चापरे}
{आत्मसंयमयोगाग्नौ जुह्वति ज्ञानदीपिते}% .. 4-27

\twolineshloka
{द्रव्ययज्ञास्तपोयज्ञा योगयज्ञास्तथाऽपरे}
{स्वाध्यायज्ञानयज्ञाश्च यतयः संशितव्रताः}% .. 4-28

\twolineshloka
{अपाने जुह्वति प्राणं प्राणेऽपानं तथाऽपरे}
{प्राणापानगती रुद्‌ध्वा प्राणायामपरायणाः}% .. 4-29

\twolineshloka
{अपरे नियताहाराः प्राणान् प्राणेषु जुह्वति}
{सर्वेऽप्येते यज्ञविदो यज्ञक्षपितकल्मषाः}% .. 4-30

\twolineshloka
{यज्ञशिष्टामृतभुजो यान्ति ब्रह्म सनातनम्}
{नायं लोकोऽस्त्ययज्ञस्य कुतोऽन्यः कुरुसत्तम}% .. 4-31

\twolineshloka
{एवं बहुविधा यज्ञा वितता ब्रह्मणो मुखे}
{कर्मजान् विद्धि तान् सर्वानेवं ज्ञात्वा विमोक्ष्यसे}% .. 4-32

\twolineshloka
{श्रेयान् द्रव्यमयाद्यज्ञाज्ज्ञानयज्ञः परन्तप}
{सर्वं कर्माखिलं पार्थ ज्ञाने परिसमाप्यते}% .. 4-33

\twolineshloka
{तद्विद्धि प्रणिपातेन परिप्रश्नेन सेवया}
{उपदेक्ष्यन्ति ते ज्ञानं ज्ञानिनस्तत्त्वदर्शिनः}% .. 4-34

\twolineshloka
{यज्ज्ञात्वा न पुनर्मोहमेवं यास्यसि पाण्डव}
{येन भूतान्यशेषेण द्रक्ष्यस्यात्मन्यथो मयि}% .. 4-35

\twolineshloka
{अपि चेदसि पापेभ्यः सर्वेभ्यः पापकृत्तमः}
{सर्वं ज्ञानप्लवेनैव वृजिनं सन्तरिष्यसि}% .. 4-36

\twolineshloka
{यथैधांसि समिद्धोऽग्निर्भस्मसात् कुरुतेऽर्जुन}
{ज्ञानाग्निः सर्वकर्माणि भस्मसात् कुरुते तथा}% .. 4-37

\twolineshloka
{न हि ज्ञानेन सदृशं पवित्रमिह विद्यते}
{तत् स्वयं योगसंसिद्धः कालेनऽऽत्मनि विन्दति}% .. 4-38

\twolineshloka
{श्रद्धावाँल्लभते ज्ञानं तत्परः संयतेन्द्रियः}
{ज्ञानं लब्ध्वा परां शान्तिमचिरेणाधिगच्छति}% .. 4-39

\twolineshloka
{अज्ञश्चाश्रद्दधानश्च संशयात्मा विनश्यति}
{नायं लोकोऽस्ति न परो न सुखं संशयात्मनः}% .. 4-40

\twolineshloka
{योगसन्न्यस्तकर्माणं ज्ञानसञ्छिन्नसंशयम्}
{आत्मवन्तं न कर्माणि निबध्नन्ति धनञ्जय}% .. 4-41

\twolineshloka
{तस्मादज्ञानसम्भूतं हृत्स्थं ज्ञानासिनाऽऽत्मनः}
{छित्त्वैनं संशयं योगमातिष्ठोत्तिष्ठ भारत}% .. 4-42
{॥ॐ तत्सदिति श्रीमद्भगवद्गीतासूपनिषत्सु ब्रह्मविद्यायां योगशास्त्रे श्रीकृष्णार्जुनसंवादे ज्ञानकर्मसन्न्यासयोगो नाम चतुर्थोऽध्यायः॥}
