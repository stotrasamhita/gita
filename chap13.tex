% !TeX program = XeLaTeX
% !TeX root = gitabook.tex
\chapt{त्रयोदशोऽध्यायः}
%\uvacha{अर्जुन उवाच}
%\twolineshloka
%{प्रकृतिं पुरुषं चैव क्षेत्रं क्षेत्रज्ञमेव च}
%{एतद्वेदितुमिच्छामि ज्ञानं ज्ञेयं च केशव}% .. 13-1

\uvacha{श्रीभगवानुवाच}
\twolineshloka
{इदं शरीरं कौन्तेय क्षेत्रमित्यभिधीयते}
{एतद्यो वेत्ति तं प्राहुः क्षेत्रज्ञ इति तद्विदः}% .. 13-2

\twolineshloka
{क्षेत्रज्ञं चापि मां विद्धि सर्वक्षेत्रेषु भारत}
{क्षेत्रक्षेत्रज्ञयोर्ज्ञानं यत्तज्ज्ञानं मतं मम}% .. 13-3

\twolineshloka
{तत् क्षेत्रं यच्च यादृक्च यद्विकारि यतश्च यत्}
{स च यो यत्प्रभावश्च तत्समासेन मे शृणु}% .. 13-4

\twolineshloka
{ऋषिभिर्बहुधा गीतं छन्दोभिर्विविधैः पृथक्}
{ब्रह्मसूत्रपदैश्चैव हेतुमद्भिर्विनिश्चितैः}% .. 13-5

\twolineshloka
{महाभूतान्यहङ्कारो बुद्धिरव्यक्तमेव च}
{इन्द्रियाणि दशैकं च पञ्च चेन्द्रियगोचराः}% .. 13-6

\twolineshloka
{इच्छा द्वेषः सुखं दुःखं सङ्घातश्चेतना धृतिः}
{एतत् क्षेत्रं समासेन सविकारमुदाहृतम्}% .. 13-7

\twolineshloka
{अमानित्वमदम्भित्वमहिंसा क्षान्तिरार्जवम्}
{आचार्योपासनं शौचं स्थैर्यमात्मविनिग्रहः}% .. 13-8

\twolineshloka
{इन्द्रियार्थेषु वैराग्यमनहङ्कार एव च}
{जन्ममृत्युजराव्याधिदुःखदोषानुदर्शनम्}% .. 13-9

\twolineshloka
{असक्तिरनभिष्वङ्गः पुत्रदारगृहादिषु}
{नित्यं च समचित्तत्वमिष्टानिष्टोपपत्तिषु}% .. 13-10

\twolineshloka
{मयि चानन्ययोगेन भक्तिरव्यभिचारिणी}
{विविक्तदेशसेवित्वम् अरतिर्जनसंसदि}% .. 13-11

\twolineshloka
{अध्यात्मज्ञाननित्यत्वं तत्त्वज्ञानार्थदर्शनम्}
{एतज्ज्ञानमिति प्रोक्तमज्ञानं यदतोऽन्यथा}% .. 13-12

\twolineshloka
{ज्ञेयं यत्तत्प्रवक्ष्यामि यज्ज्ञात्वाऽमृतमश्नुते}
{अनादिमत्परं ब्रह्म न सत्तन्नासदुच्यते}% .. 13-13

\twolineshloka
{सर्वतः पाणिपादं तत् सर्वतोऽक्षिशिरोमुखम्}
{सर्वतः श्रुतिमल्लोके सर्वमावृत्य तिष्ठति}% .. 13-14

\twolineshloka
{सर्वेन्द्रियगुणाभासं सर्वेन्द्रियविवर्जितम्}
{असक्तं सर्वभृच्चैव निर्गुणं गुणभोक्तृ च}% .. 13-15

\twolineshloka
{बहिरन्तश्च भूतानामचरं चरमेव च}
{सूक्ष्मत्वात्तदविज्ञेयं दूरस्थं चान्तिके च तत्}% .. 13-16

\twolineshloka
{अविभक्तं च भूतेषु विभक्तमिव च स्थितम्}
{भूतभर्तृ च तज्ज्ञेयं ग्रसिष्णु प्रभविष्णु च}% .. 13-17

\twolineshloka
{ज्योतिषामपि तज्ज्योतिस्तमसः परमुच्यते}
{ज्ञानं ज्ञेयं ज्ञानगम्यं हृदि सर्वस्य विष्ठितम्}% .. 13-18

\twolineshloka
{इति क्षेत्रं तथा ज्ञानं ज्ञेयं चोक्तं समासतः}
{मद्भक्त एतद्विज्ञाय मद्भावायोपपद्यते}% .. 13-19

\twolineshloka
{प्रकृतिं पुरुषं चैव विद्‌ध्यनादी उभावपि}
{विकारांश्च गुणांश्चैव विद्धि प्रकृतिसम्भवान्}% .. 13-20

\twolineshloka
{कार्यकरणकर्तृत्वे हेतुः प्रकृतिरुच्यते}
{पुरुषः सुखदुःखानां भोक्तृत्वे हेतुरुच्यते}% .. 13-21

\twolineshloka
{पुरुषः प्रकृतिस्थो हि भुङ्क्ते प्रकृतिजान् गुणान्}
{कारणं गुणसङ्गोऽस्य सदसद्योनिजन्मसु}% .. 13-22

\twolineshloka
{उपद्रष्टाऽनुमन्ता च भर्ता भोक्ता महेश्वरः}
{परमात्मेति चाप्युक्तो देहेऽस्मिन् पुरुषः परः}% .. 13-23

\twolineshloka
{य एवं वेत्ति पुरुषं प्रकृतिं च गुणैः सह}
{सर्वथा वर्तमानोऽपि न स भूयोऽभिजायते}% .. 13-24

\twolineshloka
{ध्यानेनाऽऽत्मनि पश्यन्ति केचिदात्मानमात्मना}
{अन्ये साङ्ख्येन योगेन कर्मयोगेन चापरे}% .. 13-25

\twolineshloka
{अन्ये त्वेवमजानन्तः श्रुत्वाऽन्येभ्य उपासते}
{तेऽपि चातितरन्त्येव मृत्युं श्रुतिपरायणाः}% .. 13-26

\twolineshloka
{यावत् सञ्जायते किञ्चित् सत्त्वं स्थावरजङ्गमम्}
{क्षेत्रक्षेत्रज्ञसंयोगात् तद्विद्धि भरतर्षभ}% .. 13-27

\twolineshloka
{समं सर्वेषु भूतेषु तिष्ठन्तं परमेश्वरम्}
{विनश्यत्स्वविनश्यन्तं यः पश्यति स पश्यति}% .. 13-28

\twolineshloka
{समं पश्यन् हि सर्वत्र समवस्थितमीश्वरम्}
{न हिनस्त्यात्मनाऽऽत्मानं ततो याति परां गतिम्}% .. 13-29

\twolineshloka
{प्रकृत्यैव च कर्माणि क्रियमाणानि सर्वशः}
{यः पश्यति तथाऽऽत्मानमकर्तारं स पश्यति}% .. 13-30

\twolineshloka
{यदा भूतपृथग्भावमेकस्थमनुपश्यति}
{तत एव च विस्तारं ब्रह्म सम्पद्यते तदा}% .. 13-31

\twolineshloka
{अनादित्वान्निर्गुणत्वात् परमात्माऽयमव्ययः}
{शरीरस्थोऽपि कौन्तेय न करोति न लिप्यते}% .. 13-32

\twolineshloka
{यथा सर्वगतं सौक्ष्म्यादाकाशं नोपलिप्यते}
{सर्वत्रावस्थितो देहे तथाऽऽत्मा नोपलिप्यते}% .. 13-33

\twolineshloka
{यथा प्रकाशयत्येकः कृत्स्नं लोकमिमं रविः}
{क्षेत्रं क्षेत्री तथा कृत्स्नं प्रकाशयति भारत}% .. 13-34

\twolineshloka
{क्षेत्रक्षेत्रज्ञयोरेवम् अन्तरं ज्ञानचक्षुषा}
{भूतप्रकृतिमोक्षं च ये विदुर्यान्ति ते परम्}% .. 13-35
{॥ॐ तत्सदिति श्रीमद्भगवद्गीतासूपनिषत्सु ब्रह्मविद्यायां योगशास्त्रे श्रीकृष्णार्जुनसंवादे क्षेत्रक्षेत्रज्ञविभागयोगो नाम त्रयोदशोऽध्यायः॥}