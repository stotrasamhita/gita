% !TeX program = XeLaTeX
% !TeX root = gitabook.tex
\chapt{अष्टादशोऽध्यायः\textsf{---}मोक्षसन्न्यासयोगः}
\uvacha{अर्जुन उवाच}
\twolineshloka
{सन्न्यासस्य महाबाहो तत्त्वमिच्छामि वेदितुम्}
{त्यागस्य च हृषीकेश पृथक्केशिनिषूदन}% .. 18-1

\uvacha{श्रीभगवानुवाच}
\twolineshloka
{काम्यानां कर्मणां न्यासं सन्न्यासं कवयो विदुः}
{सर्वकर्मफलत्यागं प्राहुस्त्यागं विचक्षणाः}% .. 18-2

\twolineshloka
{त्याज्यं दोषवदित्येके कर्म प्राहुर्मनीषिणः}
{यज्ञदानतपःकर्म न त्याज्यमिति चापरे}% .. 18-3

\twolineshloka
{निश्चयं शृणु मे तत्र त्यागे भरतसत्तम}
{त्यागो हि पुरुषव्याघ्र त्रिविधः सम्प्रकीर्तितः}% .. 18-4

\twolineshloka
{यज्ञदानतपःकर्म न त्याज्यं कार्यमेव तत्}
{यज्ञो दानं तपश्चैव पावनानि मनीषिणाम्}% .. 18-5

\twolineshloka
{एतान्यपि तु कर्माणि सङ्गं त्यक्त्वा फलानि च}
{कर्तव्यानीति मे पार्थ निश्चितं मतमुत्तमम्}% .. 18-6

\twolineshloka
{नियतस्य तु सन्न्यासः कर्मणो नोपपद्यते}
{मोहात्तस्य परित्यागस्तामसः परिकीर्तितः}% .. 18-7

\twolineshloka
{दुःखमित्येव यत्कर्म कायक्लेशभयात् त्यजेत्}
{स कृत्वा राजसं त्यागं नैव त्यागफलं लभेत्}% .. 18-8

\twolineshloka
{कार्यमित्येव यत्कर्म नियतं क्रियतेऽर्जुन}
{सङ्गं त्यक्त्वा फलं चैव स त्यागः सात्त्विको मतः}% .. 18-9

\twolineshloka
{न द्वेष्ट्यकुशलं कर्म कुशले नानुषज्जते}
{त्यागी सत्त्वसमाविष्टो मेधावी छिन्नसंशयः}% .. 18-10

\twolineshloka
{न हि देहभृता शक्यं त्यक्तुं कर्माण्यशेषतः}
{यस्तु कर्मफलत्यागी स त्यागीत्यभिधीयते}% .. 18-11

\twolineshloka
{अनिष्टमिष्टं मिश्रं च त्रिविधं कर्मणः फलम्}
{भवत्यत्यागिनां प्रेत्य न तु सन्न्यासिनां क्वचित्}% .. 18-12

\twolineshloka
{पञ्चैतानि महाबाहो कारणानि निबोध मे}
{साङ्ख्ये कृतान्ते प्रोक्तानि सिद्धये सर्वकर्मणाम्}% .. 18-13

\twolineshloka
{अधिष्ठानं तथा कर्ता करणं च पृथग्विधम्}
{विविधाश्च पृथक्चेष्टा दैवं चैवात्र पञ्चमम्}%.. 18-14

\twolineshloka
{शरीरवाङ्मनोभिर्यत् कर्म प्रारभते नरः}
{न्याय्यं वा विपरीतं वा पञ्चैते तस्य हेतवः}% .. 18-15

\twolineshloka
{तत्रैवं सति कर्तारमात्मानं केवलं तु यः}
{पश्यत्यकृतबुद्धित्वान्न स पश्यति दुर्मतिः}% .. 18-16

\twolineshloka
{यस्य नाहं कृतो भावो बुद्धिर्यस्य न लिप्यते}
{हत्वाऽपि स इमाँल्लोकान्न हन्ति न निबध्यते}% .. 18-17

\twolineshloka
{ज्ञानं ज्ञेयं परिज्ञाता त्रिविधा कर्मचोदना}
{करणं कर्म कर्तेति त्रिविधः कर्मसङ्ग्रहः}% .. 18-18

\twolineshloka
{ज्ञानं कर्म च कर्ता च त्रिधैव गुणभेदतः}
{प्रोच्यते गुणसङ्ख्याने यथावच्छृणु तान्यपि}% .. 18-19

\twolineshloka
{सर्वभूतेषु येनैकं भावमव्ययमीक्षते}
{अविभक्तं विभक्तेषु तज्ज्ञानं विद्धि सात्त्विकम्}% .. 18-20

\twolineshloka
{पृथक्त्वेन तु यज्ज्ञानं नानाभावान् पृथग्विधान्}
{वेत्ति सर्वेषु भूतेषु तज्ज्ञानं विद्धि राजसम्}% .. 18-21

\twolineshloka
{यत्तु कृत्स्नवदेकस्मिन् कार्ये सक्तमहैतुकम्}
{अतत्त्वार्थवदल्पं च तत्तामसमुदाहृतम्}% .. 18-22

\twolineshloka
{नियतं सङ्गरहितमरागद्वेषतः कृतम्}
{अफलप्रेप्सुना कर्म यत्तत्सात्त्विकमुच्यते}% .. 18-23

\twolineshloka
{यत्तु कामेप्सुना कर्म साहङ्कारेण वा पुनः}
{क्रियते बहुलायासं तद्राजसमुदाहृतम्}% .. 18-24

\twolineshloka
{अनुबन्धं क्षयं हिंसामनपेक्ष्य च पौरुषम्}
{मोहादारभ्यते कर्म यत्तत्तामसमुच्यते}% .. 18-25

\twolineshloka
{मुक्तसङ्गोऽनहंवादी धृत्युत्साहसमन्वितः}
{सिद्‌ध्यसिद्‌ध्योर्निर्विकारः कर्ता सात्त्विक उच्यते}% .. 18-26

\twolineshloka
{रागी कर्मफलप्रेप्सुर्लुब्धो हिंसात्मकोऽशुचिः}
{हर्षशोकान्वितः कर्ता राजसः परिकीर्तितः}% .. 18-27

\twolineshloka
{अयुक्तः प्राकृतः स्तब्धः शठो नैकृतिकोऽलसः}
{विषादी दीर्घसूत्री च कर्ता तामस उच्यते}% .. 18-28

\twolineshloka
{बुद्धेर्भेदं धृतेश्चैव गुणतस्त्रिविधं शृणु}
{प्रोच्यमानमशेषेण पृथक्त्वेन धनञ्जय}% .. 18-29

\twolineshloka
{प्रवृत्तिं च निवृत्तिं च कार्याकार्ये भयाभये}
{बन्धं मोक्षं च या वेत्ति बुद्धिः सा पार्थ सात्त्विकी}%.. 18-30

\twolineshloka
{यया धर्ममधर्मं च कार्यं चाकार्यमेव च}
{अयथावत् प्रजानाति बुद्धिः सा पार्थ राजसी}% .. 18-31

\twolineshloka
{अधर्मं धर्ममिति या मन्यते तमसाऽऽवृता}
{सर्वार्थान् विपरीतांश्च बुद्धिः सा पार्थ तामसी}% .. 18-32

\twolineshloka
{धृत्या यया धारयते मनःप्राणेन्द्रियक्रियाः}
{योगेनाव्यभिचारिण्या धृतिः सा पार्थ सात्त्विकी}% .. 18-33

\twolineshloka
{यया तु धर्मकामार्थान् धृत्या धारयतेऽर्जुन}
{प्रसङ्गेन फलाकाङ्क्षी धृतिः सा पार्थ राजसी}% .. 18-34

\twolineshloka
{यया स्वप्नं भयं शोकं विषादं मदमेव च}
{न विमुञ्चति दुर्मेधा धृतिः सा पार्थ तामसी}% .. 18-35

\twolineshloka
{सुखं त्विदानीं त्रिविधं शृणु मे भरतर्षभ}
{अभ्यासाद्रमते यत्र दुःखान्तं च निगच्छति}% .. 18-36

\twolineshloka
{यत्तदग्रे विषमिव परिणामेऽमृतोपमम्}
{तत्सुखं सात्त्विकं प्रोक्तम् आत्मबुद्धिप्रसादजम्}% .. 18-37

\twolineshloka
{विषयेन्द्रियसंयोगाद्-यत्तदग्रेऽमृतोपमम्}
{परिणामे विषमिव तत्सुखं राजसं स्मृतम्}% .. 18-38

\twolineshloka
{यदग्रे चानुबन्धे च सुखं मोहनमात्मनः}
{निद्रालस्यप्रमादोत्थं तत्तामसमुदाहृतम्}% .. 18-39

\twolineshloka
{न तदस्ति पृथिव्यां वा दिवि देवेषु वा पुनः}
{सत्त्वं प्रकृतिजैर्मुक्तं यदेभिः स्यात् त्रिभिर्गुणैः}% .. 18-40

\twolineshloka
{ब्राह्मणक्षत्रियविशां शूद्राणां च परन्तप}
{कर्माणि प्रविभक्तानि स्वभावप्रभवैर्गुणैः}% .. 18-41

\twolineshloka
{शमो दमस्तपः शौचं क्षान्तिरार्जवमेव च}
{ज्ञानं विज्ञानमास्तिक्यं ब्रह्मकर्म स्वभावजम्}% .. 18-42

\twolineshloka
{शौर्यं तेजो धृतिर्दाक्ष्यं युद्धे चाप्यपलायनम्}
{दानमीश्वरभावश्च क्षात्रं कर्म स्वभावजम्}% .. 18-43

\twolineshloka
{कृषिगौरक्ष्यवाणिज्यं वैश्यकर्म स्वभावजम्}
{परिचर्यात्मकं कर्म शूद्रस्यापि स्वभावजम्}% .. 18-44

\twolineshloka
{स्वे स्वे कर्मण्यभिरतः संसिद्धिं लभते नरः}
{स्वकर्मनिरतः सिद्धिं यथा विन्दति तच्छृणु}% .. 18-45

\twolineshloka
{यतः प्रवृत्तिर्भूतानां येन सर्वमिदं ततम्}
{स्वकर्मणा तमभ्यर्च्य सिद्धिं विन्दति मानवः}% .. 18-46

\twolineshloka
{श्रेयान् स्वधर्मो विगुणः परधर्मात् स्वनुष्ठितात्}
{स्वभावनियतं कर्म कुर्वन्नाऽऽप्नोति किल्बिषम्}% .. 18-47

\twolineshloka
{सहजं कर्म कौन्तेय सदोषमपि न त्यजेत्}
{सर्वारम्भा हि दोषेण धूमेनाग्निरिवाऽऽवृताः}% .. 18-48

\twolineshloka
{असक्तबुद्धिः सर्वत्र जितात्मा विगतस्पृहः}
{नैष्कर्म्यसिद्धिं परमां सन्न्यासेनाधिगच्छति}% .. 18-49

\twolineshloka
{सिद्धिं प्राप्तो यथा ब्रह्म तथाऽऽप्नोति निबोध मे}
{समासेनैव कौन्तेय निष्ठा ज्ञानस्य या परा}% .. 18-50

\twolineshloka
{बुद्‌ध्या विशुद्धया युक्तो धृत्याऽऽत्मानं नियम्य च}
{शब्दादीन् विषयांस्त्यक्त्वा रागद्वेषौ व्युदस्य च}% .. 18-51

\twolineshloka
{विविक्तसेवी लघ्वाशी यतवाक्कायमानसः}
{ध्यानयोगपरो नित्यं वैराग्यं समुपाश्रितः}% .. 18-52

\twolineshloka
{अहङ्कारं बलं दर्पं कामं क्रोधं परिग्रहम्}
{विमुच्य निर्ममः शान्तो ब्रह्मभूयाय कल्पते}% .. 18-53

\twolineshloka
{ब्रह्मभूतः प्रसन्नात्मा न शोचति न काङ्क्षति}
{समः सर्वेषु भूतेषु मद्भक्तिं लभते पराम्}% .. 18-54

\twolineshloka
{भक्त्या मामभिजानाति यावान् यश्चास्मि तत्त्वतः}
{ततो मां तत्त्वतो ज्ञात्वा विशते तदनन्तरम्}% .. 18-55

\twolineshloka
{सर्वकर्माण्यपि सदा कुर्वाणो मद्‌व्यपाश्रयः}
{मत्प्रसादादवाप्नोति शाश्वतं पदमव्ययम्}% .. 18-56

\twolineshloka
{चेतसा सर्वकर्माणि मयि सन्न्यस्य मत्परः}
{बुद्धियोगमुपाश्रित्य मच्चित्तः सततं भव}% .. 18-57

\twolineshloka
{मच्चित्तः सर्वदुर्गाणि मत्प्रसादात् तरिष्यसि}
{अथ चेत्त्वमहङ्कारान्न श्रोष्यसि विनङ्क्ष्यसि}% .. 18-58

\twolineshloka
{यदहङ्कारमाश्रित्य न योत्स्य इति मन्यसे}
{मिथ्यैष व्यवसायस्ते प्रकृतिस्त्वां नियोक्ष्यति}% .. 18-59

\twolineshloka
{स्वभावजेन कौन्तेय निबद्धः स्वेन कर्मणा}
{कर्तुं नेच्छसि यन्मोहात् करिष्यस्यवशोऽपि तत्}% .. 18-60

\twolineshloka
{ईश्वरः सर्वभूतानां हृद्देशेऽर्जुन तिष्ठति}
{भ्रामयन् सर्वभूतानि यन्त्रारूढानि मायया}% .. 18-61

\twolineshloka
{तमेव शरणं गच्छ सर्वभावेन भारत}
{तत्प्रसादात्परां शान्तिं स्थानं प्राप्स्यसि शाश्वतम्}
{}%.. 18-62

\twolineshloka
{इति ते ज्ञानमाख्यातं गुह्याद्गुह्यतरं मया}
{विमृश्यैतदशेषेण यथेच्छसि तथा कुरु}% .. 18-63

\twolineshloka
{सर्वगुह्यतमं भूयः शृणु मे परमं वचः}
{इष्टोऽसि मे दृढमिति ततो वक्ष्यामि ते हितम्}% .. 18-64

\twolineshloka
{मन्मना भव मद्भक्तो मद्याजी मां नमस्कुरु}
{मामेवैष्यसि सत्यं ते प्रतिजाने प्रियोऽसि मे}% .. 18-65

\twolineshloka
{सर्वधर्मान् परित्यज्य मामेकं शरणं व्रज}
{अहं त्वा सर्वपापेभ्यो मोक्षयिष्यामि मा शुचः}% .. 18-66

\twolineshloka
{इदं ते नातपस्काय नाभक्ताय कदाचन}
{न चाशुश्रूषवे वाच्यं न च मां योऽभ्यसूयति}% .. 18-67

\twolineshloka
{य इदं परमं गुह्यं मद्भक्तेष्वभिधास्यति}
{भक्तिं मयि परां कृत्वा मामेवैष्यत्यसंशयः}% .. 18-68

\twolineshloka
{न च तस्मान्मनुष्येषु कश्चिन्मे प्रियकृत्तमः}
{भविता न च मे तस्मादन्यः प्रियतरो भुवि}% .. 18-69

\twolineshloka
{अध्येष्यते च य इमं धर्म्यं संवादमावयोः}
{ज्ञानयज्ञेन तेनाहमिष्टः स्यामिति मे मतिः}% .. 18-70

\twolineshloka
{श्रद्धावाननसूयश्च शृणुयादपि यो नरः}
{सोऽपि मुक्तः शुभाँल्लोकान् प्राप्नुयात् पुण्यकर्मणाम्}% .. 18-71

\twolineshloka
{कच्चिदेतच्छ्रुतं पार्थ त्वयैकाग्रेण चेतसा}
{कच्चिदज्ञानसम्मोहः प्रनष्टस्ते धनञ्जय}% .. 18-72

\uvacha{अर्जुन उवाच}
\twolineshloka
{नष्टो मोहः स्मृतिर्लब्धा त्वत्प्रसादान्मयाऽच्युत}
{स्थितोऽस्मि गतसन्देहः करिष्ये वचनं तव}% .. 18-73

\uvacha{सञ्जय उवाच}
\twolineshloka
{इत्यहं वासुदेवस्य पार्थस्य च महात्मनः}
{संवादमिममश्रौषमद्भुतं रोमहर्षणम्}% .. 18-74

\twolineshloka
{व्यासप्रसादाच्छ्रुतवान् एतद्गुह्यमहं परम्}
{योगं योगेश्वरात् कृष्णात् साक्षात् कथयतः स्वयम्}% .. 18-75

\twolineshloka
{राजन् संस्मृत्य संस्मृत्य संवादमिममद्भुतम्}
{केशवार्जुनयोः पुण्यं हृष्यामि च मुहुर्मुहुः}% .. 18-76

\twolineshloka
{तच्च संस्मृत्य संस्मृत्य रूपमत्यद्भुतं हरेः}
{विस्मयो मे महान् राजन् हृष्यामि च पुनः पुनः}% .. 18-77

\twolineshloka
{यत्र योगेश्वरः कृष्णो यत्र पार्थो धनुर्धरः}
{तत्र श्रीर्विजयो भूतिर्ध्रुवा नीतिर्मतिर्मम}% .. 18-78
{॥ॐ तत्सदिति श्रीमद्भगवद्गीतासूपनिषत्सु ब्रह्मविद्यायां योगशास्त्रे श्रीकृष्णार्जुनसंवादे मोक्षसन्न्यासयोगो नाम अष्टादशोऽध्यायः॥}

\closesection
\resetShloka
\dnsub{माहात्म्यम्}
\twolineshloka
{गीताशास्त्रमिदं पुण्यं यः पठेत् प्रयतः पुमान्}
{विष्णोः पदमवाप्नोति भय-शोकादि-वर्जितः}% ||1||

\twolineshloka
{गीताध्ययन-शीलस्य प्राणायाम-परस्य च}
{नैव सन्ति हि पापानि पूर्व-जन्म-कृतानि च}% ||2||

\twolineshloka
{मल-निर्मोचनं पुंसां जल-स्नानं दिने दिने}
{सकृद्-गीताम्भसि स्नानं संसार-मल-नाशनम्}% ||3||

\twolineshloka
{गीता सुगीता कर्तव्या किमन्यैः शास्त्र-विस्तरैः}
{या स्वयं पद्मनाभस्य मुख-पद्माद्-विनिःसृता}% ||4||

\twolineshloka
{भारतामृत-सर्वस्वं विष्णोर्वक्त्राद्-विनिःसृतम्}
{गीता-गङ्गोदकं पीत्वा पुनर्जन्म न विद्यते}% ||5||

\twolineshloka
{सर्वोपनिषदो गावो दोग्धा गोपाल-नन्दनः}
{पार्थो वत्सः सुधीर्भोक्ता दुग्धं गीतामृतं महत्}% ||6||

\fourlineindentedshloka
{एकं शास्त्रं देवकी-पुत्र-गीतम्}
{एको देवो देवकी-पुत्र एव}
{एको मन्त्रस्तस्य नामानि यानि}
{कर्माप्येकं तस्य देवस्य सेवा}% ||7||
%{॥इति गीतामाहात्म्यम्॥}
