% !TeX program = XeLaTeX
% !TeX root = gitabook.tex
\chapt{प्रथमोऽध्यायः\textsf{---}अर्जुनविषादयोगः}%\markright{अर्जुनविषादयोगः}
\uvacha{धृतराष्ट्र उवाच}
\twolineshloka
{धर्मक्षेत्रे कुरुक्षेत्रे समवेता युयुत्सवः}
{मामकाः पाण्डवाश्चैव किमकुर्वत सञ्जय}% .. 1-1

\uvacha{सञ्जय उवाच}
\twolineshloka
{दृष्ट्वा तु पाण्डवानीकं व्यूढं दुर्योधनस्तदा}
{आचार्यमुपसङ्गम्य राजा वचनमब्रवीत्}% .. 1-2

\twolineshloka
{पश्यैतां पाण्डुपुत्राणामाचार्य महतीं चमूम्}
{व्यूढां द्रुपदपुत्रेण तव शिष्येण धीमता}% .. 1-3

\twolineshloka
{अत्र शूरा महेष्वासा भीमार्जुनसमा युधि}
{युयुधानो विराटश्च द्रुपदश्च महारथः}% .. 1-4

\twolineshloka
{धृष्टकेतुश्चेकितानः काशिराजश्च वीर्यवान्}
{पुरुजित् कुन्तिभोजश्च शैब्यश्च नरपुङ्गवः}% .. 1-5

\twolineshloka
{युधामन्युश्च विक्रान्त उत्तमौजाश्च वीर्यवान्}
{सौभद्रो द्रौपदेयाश्च सर्व एव महारथाः}% .. 1-6

\twolineshloka
{अस्माकं तु विशिष्टा ये तान्निबोध द्विजोत्तम}
{नायका मम सैन्यस्य संज्ञार्थं तान् ब्रवीमि ते}% .. 1-7

\twolineshloka
{भवान् भीष्मश्च कर्णश्च कृपश्च समितिञ्जयः}
{अश्वत्थामा विकर्णश्च सौमदत्तिस्तथैव च}% .. 1-8

\twolineshloka
{अन्ये च बहवः शूरा मदर्थे त्यक्तजीविताः}
{नानाशस्त्रप्रहरणाः सर्वे युद्धविशारदाः}% .. 1-9

\twolineshloka
{अपर्याप्तं तदस्माकं बलं भीष्माभिरक्षितम्}
{पर्याप्तं त्विदमेतेषां बलं भीमाभिरक्षितम्}% .. 1-10

\twolineshloka
{अयनेषु च सर्वेषु यथाभागमवस्थिताः}
{भीष्ममेवाभिरक्षन्तु भवन्तः सर्व एव हि}% .. 1-11

\twolineshloka
{तस्य सञ्जनयन् हर्षं कुरुवृद्धः पितामहः}
{सिंहनादं विनद्योच्चैः शङ्खं दध्मौ प्रतापवान्}% .. 1-12

\twolineshloka
{ततः शङ्खाश्च भेर्यश्च पणवानकगोमुखाः}
{सहसैवाभ्यहन्यन्त स शब्दस्तुमुलोऽभवत्}% .. 1-13

\twolineshloka
{ततः श्वेतैर्हयैर्युक्ते महति स्यन्दने स्थितौ}
{माधवः पाण्डवश्चैव दिव्यौ शङ्खौ प्रदध्मतुः}% .. 1-14

\twolineshloka
{पाञ्चजन्यं हृषीकेशो देवदत्तं धनञ्जयः}
{पौण्ड्रं दध्मौ महाशङ्खं भीमकर्मा वृकोदरः}% .. 1-15

\twolineshloka
{अनन्तविजयं राजा कुन्तीपुत्रो युधिष्ठिरः}
{नकुलः सहदेवश्च सुघोषमणिपुष्पकौ}% .. 1-16

\twolineshloka
{काश्यश्च परमेष्वासः शिखण्डी च महारथः}
{धृष्टद्युम्नो विराटश्च सात्यकिश्चापराजितः}% .. 1-17

\twolineshloka
{द्रुपदो द्रौपदेयाश्च सर्वशः पृथिवीपते}
{सौभद्रश्च महाबाहुः शङ्खान् दध्मुः पृथक् पृथक्}%.. 1-18

\twolineshloka
{स घोषो धार्तराष्ट्राणां हृदयानि व्यदारयत्}
{नभश्च पृथिवीं चैव तुमुलो व्यनुनादयन्}% .. 1-19

\twolineshloka
{अथ व्यवस्थितान् दृष्ट्वा धार्तराष्ट्रान् कपिध्वजः}
{प्रवृत्ते शस्त्रसम्पाते धनुरुद्यम्य पाण्डवः}% .. 1-20..

\onelineshloka*
{हृषीकेशं तदा वाक्यमिदमाह महीपते}

\uvacha{अर्जुन उवाच}
\onelineshloka
{सेनयोरुभयोर्मध्ये रथं स्थापय मेऽच्युत}% .. 1-21

\twolineshloka
{यावदेतान्निरीक्षेऽहं योद्धुकामानवस्थितान्}
{कैर्मया सह योद्धव्यमस्मिन् रणसमुद्यमे}% .. 1-22

\twolineshloka
{योत्स्यमानानवेक्षेऽहं य एतेऽत्र समागताः}
{धार्तराष्ट्रस्य दुर्बुद्धेर्युद्धे प्रियचिकीर्षवः}% .. 1-23

\uvacha{सञ्जय उवाच}
\twolineshloka
{एवमुक्तो हृषीकेशो गुडाकेशेन भारत}
{सेनयोरुभयोर्मध्ये स्थापयित्वा रथोत्तमम्}% .. 1-24

\twolineshloka
{भीष्मद्रोणप्रमुखतः सर्वेषां च महीक्षिताम्}
{उवाच पार्थ पश्यैतान् समवेतान् कुरूनिति}% .. 1-25

\twolineshloka
{तत्रापश्यत् स्थितान् पार्थः पितॄनथ पितामहान्}
{आचार्यान् मातुलान् भ्रातॄन् पुत्रान् पौत्रान् सखींस्तथा}% .. 1-26

\twolineshloka
{श्वशुरान् सुहृदश्चैव सेनयोरुभयोरपि}
{तान् समीक्ष्य स कौन्तेयः सर्वान् बन्धूनवस्थितान्}% .. 1-27

\onelineshloka*
{कृपया परयाऽऽविष्टो विषीदन्निदमब्रवीत्}

\uvacha{अर्जुन उवाच}
\onelineshloka
{दृष्ट्वेमं स्वजनं कृष्ण युयुत्सुं समुपस्थितम्}% .. 1-28\\

\twolineshloka
{सीदन्ति मम गात्राणि मुखं च परिशुष्यति}
{वेपथुश्च शरीरे मे रोमहर्षश्च जायते}% .. 1-29

\twolineshloka
{गाण्डीवं स्रंसते हस्तात् त्वक्चैव परिदह्यते}
{न च शक्नोम्यवस्थातुं भ्रमतीव च मे मनः}% .. 1-30

\twolineshloka
{निमित्तानि च पश्यामि विपरीतानि केशव}
{न च श्रेयोऽनुपश्यामि हत्वा स्वजनमाहवे}% .. 1-31

\twolineshloka
{न काङ्क्षे विजयं कृष्ण न च राज्यं सुखानि च}
{किं नो राज्येन गोविन्द किं भोगैर्जीवितेन वा}% .. 1-32

\twolineshloka
{येषामर्थे काङ्क्षितं नो राज्यं भोगाः सुखानि च}
{त इमेऽवस्थिता युद्धे प्राणांस्त्यक्त्वा धनानि च}% .. 1-33

\twolineshloka
{आचार्याः पितरः पुत्रास्तथैव च पितामहाः}
{मातुलाः श्वशुराः पौत्राः श्यालाः सम्बन्धिनस्तथा}% .. 1-34

\twolineshloka
{एतान्न हन्तुमिच्छामि घ्नतोऽपि मधुसूदन}
{अपि त्रैलोक्यराज्यस्य हेतोः किं नु महीकृते}% .. 1-35

\twolineshloka
{निहत्य धार्तराष्ट्रान्नः का प्रीतिः स्याज्जनार्दन}
{पापमेवाऽऽश्रयेदस्मान् हत्वैतानाततायिनः}% .. 1-36

\twolineshloka
{तस्मान्नार्हा वयं हन्तुं धार्तराष्ट्रान् स्वबान्धवान्}
{स्वजनं हि कथं हत्वा सुखिनः स्याम माधव}% .. 1-37

\twolineshloka
{यद्यप्येते न पश्यन्ति लोभोपहतचेतसः}
{कुलक्षयकृतं दोषं मित्रद्रोहे च पातकम्}% .. 1-38

\twolineshloka
{कथं न ज्ञेयमस्माभिः पापादस्मान्निवर्तितुम्}
{कुलक्षयकृतं दोषं प्रपश्यद्भिर्जनार्दन}% .. 1-39

\twolineshloka
{कुलक्षये प्रणश्यन्ति कुलधर्माः सनातनाः}
{धर्मे नष्टे कुलं कृत्स्नमधर्मोऽभिभवत्युत}% .. 1-40

\twolineshloka
{अधर्माभिभवात् कृष्ण प्रदुष्यन्ति कुलस्त्रियः}
{स्त्रीषु दुष्टासु वार्ष्णेय जायते वर्णसङ्करः}% .. 1-41

\twolineshloka
{सङ्करो नरकायैव कुलघ्नानां कुलस्य च}
{पतन्ति पितरो ह्येषां लुप्तपिण्डोदकक्रियाः}% .. 1-42

\twolineshloka
{दोषैरेतैः कुलघ्नानां वर्णसङ्करकारकैः}
{उत्साद्यन्ते जातिधर्माः कुलधर्माश्च शाश्वताः}% .. 1-43

\twolineshloka
{उत्सन्नकुलधर्माणां मनुष्याणां जनार्दन}
{नरकेऽनियतं वासो भवतीत्यनुशुश्रुम}% .. 1-44

\twolineshloka
{अहो बत महत्पापं कर्तुं व्यवसिता वयम्}
{यद्राज्यसुखलोभेन हन्तुं स्वजनमुद्यताः}% .. 1-45

\twolineshloka
{यदि मामप्रतीकारम् अशस्त्रं शस्त्रपाणयः}
{धार्तराष्ट्रा रणे हन्युस्तन्मे क्षेमतरं भवेत्}% .. 1-46

\uvacha{सञ्जय उवाच}
\twolineshloka
{एवमुक्त्वाऽर्जुनः सङ्ख्ये रथोपस्थ उपाविशत्}
{विसृज्य सशरं चापं शोकसंविग्नमानसः}% .. 1-47
{॥ॐ तत्सदिति श्रीमद्भगवद्गीतासूपनिषत्सु ब्रह्मविद्यायां योगशास्त्रे श्रीकृष्णार्जुनसंवादे अर्जुनविषादयोगो नाम प्रथमोऽध्यायः॥}
