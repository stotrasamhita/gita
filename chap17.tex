% !TeX program = XeLaTeX
% !TeX root = gitabook.tex
\chapt{सप्तदशोऽध्यायः}
\uvacha{अर्जुन उवाच}
\twolineshloka
{ये शास्त्रविधिमुत्सृज्य यजन्ते श्रद्धयाऽन्विताः}
{तेषां निष्ठा तु का कृष्ण सत्त्वमाहो रजस्तमः}% .. 17-1

\uvacha{श्रीभगवानुवाच}
\twolineshloka
{त्रिविधा भवति श्रद्धा देहिनां सा स्वभावजा}
{सात्त्विकी राजसी चैव तामसी चेति तां शृणु}% .. 17-2

\twolineshloka
{सत्त्वानुरूपा सर्वस्य श्रद्धा भवति भारत}
{श्रद्धामयोऽयं पुरुषो यो यच्छ्रद्धः स एव सः}% .. 17-3

\twolineshloka
{यजन्ते सात्त्विका देवान् यक्षरक्षांसि राजसाः}
{प्रेतान् भूतगणांश्चान्ये यजन्ते तामसा जनाः}% .. 17-4

\twolineshloka
{अशास्त्रविहितं घोरं तप्यन्ते ये तपो जनाः}
{दम्भाहङ्कारसंयुक्ताः कामरागबलान्विताः}% .. 17-5

\twolineshloka
{कर्शयन्तः शरीरस्थं भूतग्राममचेतसः}
{मां चैवान्तःशरीरस्थं तान् विद्‌ध्यासुरनिश्चयान्}% .. 17-6

\twolineshloka
{आहारस्त्वपि सर्वस्य त्रिविधो भवति प्रियः}
{यज्ञस्तपस्तथा दानं तेषां भेदमिमं शृणु}% .. 17-7

\twolineshloka
{आयुःसत्त्वबलारोग्यसुखप्रीतिविवर्धनाः}
{रस्याः स्निग्धाः स्थिरा हृद्या आहाराः सात्त्विकप्रियाः}%.. 17-8

\twolineshloka
{कट्वम्ललवणात्युष्णतीक्ष्णरूक्षविदाहिनः}
{आहारा राजसस्येष्टा दुःखशोकामयप्रदाः}% .. 17-9

\twolineshloka
{यातयामं गतरसं पूति पर्युषितं च यत्}
{उच्छिष्टमपि चामेध्यं भोजनं तामसप्रियम्}% .. 17-10

\twolineshloka
{अफलाकाङ्क्षिभिर्यज्ञो विधिदृष्टो य इज्यते}
{यष्टव्यमेवेति मनः समाधाय स सात्त्विकः}% .. 17-11

\twolineshloka
{अभिसन्धाय तु फलं दम्भार्थमपि चैव यत्}
{इज्यते भरतश्रेष्ठ तं यज्ञं विद्धि राजसम्}% .. 17-12

\twolineshloka
{विधिहीनमसृष्टान्नं मन्त्रहीनमदक्षिणम्}
{श्रद्धाविरहितं यज्ञं तामसं परिचक्षते}% .. 17-13

\twolineshloka
{देवद्विजगुरुप्राज्ञपूजनं शौचमार्जवम्}
{ब्रह्मचर्यमहिंसा च शारीरं तप उच्यते}% .. 17-14

\twolineshloka
{अनुद्वेगकरं वाक्यं सत्यं प्रियहितं च यत्}
{स्वाध्यायाभ्यसनं चैव वाङ्मयं तप उच्यते}% .. 17-15

\twolineshloka
{मनः प्रसादः सौम्यत्वं मौनमात्मविनिग्रहः}
{भावसंशुद्धिरित्येतत् तपो मानसमुच्यते}% .. 17-16

\twolineshloka
{श्रद्धया परया तप्तं तपस्तत् त्रिविधं नरैः}
{अफलाकाङ्क्षिभिर्युक्तैः सात्त्विकं परिचक्षते}% .. 17-17

\twolineshloka
{सत्कारमानपूजार्थं तपो दम्भेन चैव यत्}
{क्रियते तदिह प्रोक्तं राजसं चलमध्रुवम्}% .. 17-18

\twolineshloka
{मूढग्राहेणाऽऽत्मनो यत् पीडया क्रियते तपः}
{परस्योत्सादनार्थं वा तत्तामसमुदाहृतम्}% .. 17-19

\twolineshloka
{दातव्यमिति यद्दानं दीयतेऽनुपकारिणे}
{देशे काले च पात्रे च तद्दानं सात्त्विकं स्मृतम्}% .. 17-20

\twolineshloka
{यत्तु प्रत्युपकारार्थं फलमुद्दिश्य वा पुनः}
{दीयते च परिक्लिष्टं तद्दानं राजसं स्मृतम्}% .. 17-21

\twolineshloka
{अदेशकाले यद्दानम् अपात्रेभ्यश्च दीयते}
{असत्कृतमवज्ञातं तत्तामसमुदाहृतम्}% .. 17-22

\twolineshloka
{ॐ तत्सदिति निर्देशो ब्रह्मणस्त्रिविधः स्मृतः}
{ब्राह्मणास्तेन वेदाश्च यज्ञाश्च विहिताः पुरा}% .. 17-23

\twolineshloka
{तस्मादोमित्युदाहृत्य यज्ञदानतपःक्रियाः}
{प्रवर्तन्ते विधानोक्ताः सततं ब्रह्मवादिनाम्}% .. 17-24

\twolineshloka
{तदित्यनभिसन्धाय फलं यज्ञतपःक्रियाः}
{दानक्रियाश्च विविधाः क्रियन्ते मोक्षकाङ्क्षिभिः}% .. 17-25

\twolineshloka
{सद्भावे साधुभावे च सदित्येतत् प्रयुज्यते}
{प्रशस्ते कर्मणि तथा सच्छब्दः पार्थ युज्यते}% .. 17-26

\twolineshloka
{यज्ञे तपसि दाने च स्थितिः सदिति चोच्यते}
{कर्म चैव तदर्थीयं सदित्येवाभिधीयते}% .. 17-27

\twolineshloka
{अश्रद्धया हुतं दत्तं तपस्तप्तं कृतं च यत्}
{असदित्युच्यते पार्थ न च तत् प्रेत्य नो इह}% .. 17-28
{॥ॐ तत्सदिति श्रीमद्भगवद्गीतासूपनिषत्सु ब्रह्मविद्यायां योगशास्त्रे श्रीकृष्णार्जुनसंवादे श्रद्धात्रयविभागयोगो नाम सप्तदशोऽध्यायः॥}
