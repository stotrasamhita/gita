% !TeX program = XeLaTeX
% !TeX root = gitabook.tex
\chapt{द्वादशोऽध्यायः\textsf{---}भक्तियोगः}
\uvacha{अर्जुन उवाच}
\twolineshloka
{एवं सततयुक्ता ये भक्तास्त्वां पर्युपासते}
{ये चाप्यक्षरमव्यक्तं तेषां के योगवित्तमाः}% .. 12-1

\uvacha{श्रीभगवानुवाच}
\twolineshloka
{मय्यावेश्य मनो ये मां नित्ययुक्ता उपासते}
{श्रद्धया परयोपेतास्ते मे युक्ततमा मताः}% .. 12-2

\twolineshloka
{ये त्वक्षरमनिर्देश्यम् अव्यक्तं पर्युपासते}
{सर्वत्रगमचिन्त्यं च कूटस्थम् अचलं ध्रुवम्}% .. 12-3

\twolineshloka
{सन्नियम्येन्द्रियग्रामं सर्वत्र समबुद्धयः}
{ते प्राप्नुवन्ति मामेव सर्वभूतहिते रताः}% .. 12-4

\twolineshloka
{क्लेशोऽधिकतरस्तेषाम् अव्यक्तासक्तचेतसाम्}% .. 
{अव्यक्ता हि गतिर्दुःखं देहवद्भिरवाप्यते}% .. 12-5

\twolineshloka
{ये तु सर्वाणि कर्माणि मयि सन्न्यस्य मत्पराः}
{अनन्येनैव योगेन मां ध्यायन्त उपासते}% .. 12-6

\twolineshloka
{तेषामहं समुद्धर्ता मृत्युसंसारसागरात्}
{भवामि न चिरात् पार्थ मय्यावेशितचेतसाम्}% .. 12-7

\twolineshloka
{मय्येव मन आधत्स्व मयि बुद्धिं निवेशय}
{निवसिष्यसि मय्येव अत ऊर्ध्वं न संशयः}% .. 12-8

\twolineshloka
{अथ चित्तं समाधातुं न शक्नोषि मयि स्थिरम्}
{अभ्यासयोगेन ततो मामिच्छाऽऽप्तुं धनञ्जय}% .. 12-9

\twolineshloka
{अभ्यासेऽप्यसमर्थोऽसि मत्कर्मपरमो भव}
{मदर्थमपि कर्माणि कुर्वन् सिद्धिमवाप्स्यसि}% .. 12-10

\twolineshloka
{अथैतदप्यशक्तोऽसि कर्तुं मद्योगमाश्रितः}
{सर्वकर्मफलत्यागं ततः कुरु यतात्मवान्}% .. 12-11

\twolineshloka
{श्रेयो हि ज्ञानमभ्यासाज्ज्ञानाद्‌ध्यानं विशिष्यते}
{ध्यानात् कर्मफलत्यागस्त्यागाच्छान्तिरनन्तरम्}% .. 12-12

\twolineshloka
{अद्वेष्टा सर्वभूतानां मैत्रः करुण एव च}
{निर्ममो निरहङ्कारः समदुःखसुखः क्षमी}% .. 12-13

\twolineshloka
{सन्तुष्टः सततं योगी यतात्मा दृढनिश्चयः}
{मय्यर्पितमनोबुद्धिर्यो मद्भक्तः स मे प्रियः}% .. 12-14

\twolineshloka
{यस्मान्नोद्विजते लोको लोकान्नोद्विजते च यः}
{हर्षामर्षभयोद्वेगैर्मुक्तो यः स च मे प्रियः}% .. 12-15

\twolineshloka
{अनपेक्षः शुचिर्दक्ष उदासीनो गतव्यथः}
{सर्वारम्भपरित्यागी यो मद्भक्तः स मे प्रियः}% .. 12-16

\twolineshloka
{यो न हृष्यति न द्वेष्टि न शोचति न काङ्क्षति}
{शुभाशुभपरित्यागी भक्तिमान् यः स मे प्रियः}% .. 12-17

\twolineshloka
{समः शत्रौ च मित्रे च तथा मानापमानयोः}
{शीतोष्णसुखदुःखेषु समः सङ्गविवर्जितः}% .. 12-18

\twolineshloka
{तुल्यनिन्दास्तुतिर्मौनी सन्तुष्टो येन केनचित्}
{अनिकेतः स्थिरमतिर्भक्तिमान् मे प्रियो नरः}% .. 12-19

\twolineshloka
{ये तु धर्म्यामृतमिदं यथोक्तं पर्युपासते}
{श्रद्दधाना मत्परमा भक्तास्तेऽतीव मे प्रियाः}% .. 12-20
{॥ॐ तत्सदिति श्रीमद्भगवद्गीतासूपनिषत्सु ब्रह्मविद्यायां योगशास्त्रे श्रीकृष्णार्जुनसंवादे भक्तियोगो नाम द्वादशोऽध्यायः॥}