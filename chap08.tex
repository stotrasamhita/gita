% !TeX program = XeLaTeX
% !TeX root = gitabook.tex
\chapt{अष्टमोऽध्यायः}
\uvacha{अर्जुन उवाच}
\twolineshloka
{किं तद्-ब्रह्म किमध्यात्मं किं कर्म पुरुषोत्तम}
{अधिभूतं च किं प्रोक्तमधिदैवं किमुच्यते}% .. 8-1

\twolineshloka
{अधियज्ञः कथं कोऽत्र देहेऽस्मिन् मधुसूदन}
{प्रयाणकाले च कथं ज्ञेयोऽसि नियतात्मभिः}% .. 8-2

\uvacha{श्रीभगवानुवाच}
\twolineshloka
{अक्षरं ब्रह्म परमं स्वभावोऽध्यात्ममुच्यते}
{भूतभावोद्भवकरो विसर्गः कर्मसंज्ञितः}% .. 8-3

\twolineshloka
{अधिभूतं क्षरो भावः पुरुषश्चाधिदैवतम्}
{अधियज्ञोऽहमेवात्र देहे देहभृतां वर}% .. 8-4

\twolineshloka
{अन्तकाले च मामेव स्मरन् मुक्त्वा कलेवरम्}
{यः प्रयाति स मद्भावं याति नास्त्यत्र संशयः}% .. 8-5

\twolineshloka
{यं यं वाऽपि स्मरन् भावं त्यजत्यन्ते कलेवरम्}
{तं तमेवैति कौन्तेय सदा तद्भावभावितः}% .. 8-6

\twolineshloka
{तस्मात् सर्वेषु कालेषु मामनुस्मर युध्य च}
{मय्यर्पितमनोबुद्धिर्मामेवैष्यस्यसंशयः}% .. 8-7

\twolineshloka
{अभ्यासयोगयुक्तेन चेतसा नान्यगामिना}
{परमं पुरुषं दिव्यं याति पार्थानुचिन्तयन्}% .. 8-8

\fourlineindentedshloka
{कविं पुराणम् अनुशासितारम्}
{अनोरणीयांसम् अनुस्मरेद्यः}
{सर्वस्य धातारम् अचिन्त्यरूपम्}
{आदित्यवर्णं तमसः परस्तात्}% .. 8-9

\fourlineindentedshloka
{प्रयाणकाले मनसाऽचलेन}
{भक्त्या युक्तो योगबलेन चैव}
{भ्रुवोर्मध्ये प्राणमावेश्य सम्यक्}
{स तं परं पुरुषमुपैति दिव्यम्}% .. 8-10

\fourlineindentedshloka
{यदक्षरं वेदविदो वदन्ति}
{विशन्ति यद्यतयो वीतरागाः}
{यदिच्छन्तो ब्रह्मचर्यं चरन्ति}
{तत् ते पदं सङ्ग्रहेण प्रवक्ष्ये}% .. 8-11

\twolineshloka
{सर्वद्वाराणि संयम्य मनो हृदि निरुध्य च}
{मूर्ध्न्याधायऽऽत्मनः प्राणमास्थितो योगधारणाम्}% .. 8-12

\twolineshloka
{ओमित्येकाक्षरं ब्रह्म व्याहरन् मामनुस्मरन्}
{यः प्रयाति त्यजन् देहं स याति परमां गतिम्}% .. 8-13

\twolineshloka
{अनन्यचेताः सततं यो मां स्मरति नित्यशः}
{तस्याहं सुलभः पार्थ नित्ययुक्तस्य योगिनः}% .. 8-14

\twolineshloka
{मामुपेत्य पुनर्जन्म दुःखालयमशाश्वतम्}
{नऽऽप्नुवन्ति महात्मानः संसिद्धिं परमां गताः}% .. 8-15

\twolineshloka
{आब्रह्मभुवनाल्लोकाः पुनरावर्तिनोऽर्जुन}
{मामुपेत्य तु कौन्तेय पुनर्जन्म न विद्यते}% .. 8-16

\twolineshloka
{सहस्रयुगपर्यन्तमहर्यद्-ब्रह्मणो विदुः}
{रात्रिं युगसहस्रान्तां तेऽहोरात्रविदो जनाः}% .. 8-17

\twolineshloka
{अव्यक्ताद्-व्यक्तयः सर्वाः प्रभवन्त्यहरागमे}
{रात्र्यागमे प्रलीयन्ते तत्रैवाव्यक्तसंज्ञके}% .. 8-18

\twolineshloka
{भूतग्रामः स एवायं भूत्वा भूत्वा प्रलीयते}
{रात्र्यागमेऽवशः पार्थ प्रभवत्यहरागमे}% .. 8-19

\twolineshloka
{परस्तस्मात्तु भावोऽन्योऽव्यक्तोऽव्यक्तात्सनातनः}
{यः स सर्वेषु भूतेषु नश्यत्सु न विनश्यति}% .. 8-20

\twolineshloka
{अव्यक्तोऽक्षर इत्युक्तस्तमाहुः परमां गतिम्}
{यं प्राप्य न निवर्तन्ते तद्धाम परमं मम}% .. 8-21

\twolineshloka
{पुरुषः स परः पार्थ भक्त्या लभ्यस्त्वनन्यया}
{यस्यान्तःस्थानि भूतानि येन सर्वमिदं ततम्}% .. 8-22

\twolineshloka
{यत्र काले त्वनावृत्तिमावृत्तिं चैव योगिनः}
{प्रयाता यान्ति तं कालं वक्ष्यामि भरतर्षभ}% .. 8-23

\twolineshloka
{अग्निर्ज्योतिरहः शुक्लः षण्मासा उत्तरायणम्}
{तत्र प्रयाता गच्छन्ति ब्रह्म ब्रह्मविदो जनाः}% .. 8-24

\twolineshloka
{धूमो रात्रिस्तथा कृष्णः षण्मासा दक्षिणायनम्}
{तत्र चान्द्रमसं ज्योतिर्योगी प्राप्य निवर्तते}% .. 8-25

\twolineshloka
{शुक्लकृष्णे गती ह्येते जगतः शाश्वते मते}
{एकया यात्यनावृत्तिमन्ययाऽऽवर्तते पुनः}% .. 8-26

\twolineshloka
{नैते सृती पार्थ जानन् योगी मुह्यति कश्चन}
{तस्मात् सर्वेषु कालेषु योगयुक्तो भवार्जुन}% .. 8-27

\fourlineindentedshloka
{वेदेषु यज्ञेषु तपःसु चैव}
{दानेषु यत् पुण्यफलं प्रदिष्टम्}
{अत्येति तत्सर्वमिदं विदित्वा}
{योगी परं स्थानमुपैति चऽऽद्यम्}% .. 8-28.. 
{॥ॐ तत्सदिति श्रीमद्भगवद्गीतासूपनिषत्सु ब्रह्मविद्यायां योगशास्त्रे श्रीकृष्णार्जुनसंवादे अक्षरब्रह्मयोगो नाम अष्टमोऽध्यायः॥}