% !TeX program = XeLaTeX
% !TeX root = gitabook.tex
\chapt{दशमोऽध्यायः\textsf{---}विभूतियोगः}
\uvacha{श्रीभगवानुवाच}
\twolineshloka
{भूय एव महाबाहो शृणु मे परमं वचः}
{यत्तेऽहं प्रीयमाणाय वक्ष्यामि हितकाम्यया}% .. 10-1

\twolineshloka
{न मे विदुः सुरगणाः प्रभवं न महर्षयः}
{अहमादिर्हि देवानां महर्षीणां च सर्वशः}% .. 10-2

\twolineshloka
{यो मामजमनादिं च वेत्ति लोकमहेश्वरम्}
{असम्मूढः स मर्त्येषु सर्वपापैः प्रमुच्यते}% .. 10-3

\twolineshloka
{बुद्धिर्ज्ञानमसम्मोहः क्षमा सत्यं दमः शमः}
{सुखं दुःखं भवोऽभावो भयं चाभयमेव च}% .. 10-4

\twolineshloka
{अहिंसा समता तुष्टिस्तपो दानं यशोऽयशः}
{भवन्ति भावा भूतानां मत्त एव पृथग्विधाः}% .. 10-5

\twolineshloka
{महर्षयः सप्त पूर्वे चत्वारो मनवस्तथा}
{मद्भावा मानसा जाता येषां लोक इमाः प्रजाः}% .. 10-6

\twolineshloka
{एतां विभूतिं योगं च मम यो वेत्ति तत्त्वतः}
{सोऽविकम्पेन योगेन युज्यते नात्र संशयः}% .. 10-7

\twolineshloka
{अहं सर्वस्य प्रभवो मत्तः सर्वं प्रवर्तते}
{इति मत्वा भजन्ते मां बुधा भावसमन्विताः}% .. 10-8

\twolineshloka
{मच्चित्ता मद्गतप्राणा बोधयन्तः परस्परम्}
{कथयन्तश्च मां नित्यं तुष्यन्ति च रमन्ति च}% .. 10-9

\twolineshloka
{तेषां सततयुक्तानां भजतां प्रीतिपूर्वकम्}
{ददामि बुद्धियोगं तं येन मामुपयान्ति ते}% .. 10-10

\twolineshloka
{तेषामेवानुकम्पार्थम् अहमज्ञानजं तमः}
{नाशयाम्यात्मभावस्थो ज्ञानदीपेन भास्वता}% .. 10-11

\uvacha{अर्जुन उवाच}
\twolineshloka
{परं ब्रह्म परं धाम पवित्रं परमं भवान्}
{पुरुषं शाश्वतं दिव्यमादिदेवमजं विभुम्}% .. 10-12

\twolineshloka
{आहुस्त्वामृषयः सर्वे देवर्षिर्नारदस्तथा}
{असितो देवलो व्यासः स्वयं चैव ब्रवीषि मे}% .. 10-13

\twolineshloka
{सर्वमेतदृतं मन्ये यन्मां वदसि केशव}
{न हि ते भगवन् व्यक्तिं विदुर्देवा न दानवाः}% .. 10-14

\twolineshloka
{स्वयमेवाऽऽत्मनाऽऽत्मानं वेत्थ त्वं पुरुषोत्तम}
{भूतभावन भूतेश देवदेव जगत्पते}% .. 10-15

\twolineshloka
{वक्तुमर्हस्यशेषेण दिव्या ह्यात्मविभूतयः}
{याभिर्विभूतिभिर्लोकानिमांस्त्वं व्याप्य तिष्ठसि}% .. 10-16

\twolineshloka
{कथं विद्यामहं योगिंस्त्वां सदा परिचिन्तयन्}
{केषु केषु च भावेषु चिन्त्योऽसि भगवन्मया}% .. 10-17

\twolineshloka
{विस्तरेणाऽऽत्मनो योगं विभूतिं च जनार्दन}
{भूयः कथय तृप्तिर्हि शृण्वतो नास्ति मेऽमृतम्}% .. 10-18

\uvacha{श्रीभगवानुवाच}
\twolineshloka
{हन्त ते कथयिष्यामि दिव्या ह्यात्मविभूतयः}
{प्राधान्यतः कुरुश्रेष्ठ नास्त्यन्तो विस्तरस्य मे}% .. 10-19

\twolineshloka
{अहमात्मा गुडाकेश सर्वभूताशयस्थितः}
{अहमादिश्च मध्यं च भूतानामन्त एव च}% .. 10-20

\twolineshloka
{आदित्यानामहं विष्णुर्ज्योतिषां रविरंशुमान्}
{मरीचिर्मरुतामस्मि नक्षत्राणामहं शशी}% .. 10-21

\twolineshloka
{वेदानां सामवेदोऽस्मि देवानामस्मि वासवः}
{इन्द्रियाणां मनश्चास्मि भूतानामस्मि चेतना}% .. 10-22

\twolineshloka
{रुद्राणां शङ्करश्चास्मि वित्तेशो यक्षरक्षसाम्}
{वसूनां पावकश्चास्मि मेरुः शिखरिणामहम्}% .. 10-23

\twolineshloka
{पुरोधसां च मुख्यं मां विद्धि पार्थ बृहस्पतिम्}
{सेनानीनामहं स्कन्दः सरसामस्मि सागरः}% .. 10-24

\twolineshloka
{महर्षीणां भृगुरहं गिरामस्म्येकमक्षरम्}
{यज्ञानां जपयज्ञोऽस्मि स्थावराणां हिमालयः}% .. 10-25

\twolineshloka
{अश्वत्थः सर्ववृक्षाणां देवर्षीणां च नारदः}
{गन्धर्वाणां चित्ररथः सिद्धानां कपिलो मुनिः}% .. 10-26

\twolineshloka
{उच्चैःश्रवसमश्वानां विद्धि माममृतोद्भवम्}
{ऐरावतं गजेन्द्राणां नराणां च नराधिपम्}% .. 10-27

\twolineshloka
{आयुधानामहं वज्रं धेनूनामस्मि कामधुक्}
{प्रजनश्चास्मि कन्दर्पः सर्पाणामस्मि वासुकिः}% .. 10-28

\twolineshloka
{अनन्तश्चास्मि नागानां वरुणो यादसामहम्}
{पितॄणामर्यमा चास्मि यमः संयमतामहम्}% .. 10-29

\twolineshloka
{प्रह्लादश्चास्मि दैत्यानां कालः कलयतामहम्}
{मृगाणां च मृगेन्द्रोऽहं वैनतेयश्च पक्षिणाम्}% .. 10-30

\twolineshloka
{पवनः पवतामस्मि रामः शस्त्रभृतामहम्}
{झषाणां मकरश्चास्मि स्रोतसामस्मि जाह्नवी}% .. 10-31

\twolineshloka
{सर्गाणामादिरन्तश्च मध्यं चैवाहमर्जुन}
{अध्यात्मविद्या विद्यानां वादः प्रवदतामहम्}% .. 10-32

\twolineshloka
{अक्षराणामकारोऽस्मि द्वन्द्वः सामासिकस्य च}
{अहमेवाक्षयः कालो धाताऽहं विश्वतोमुखः}% .. 10-33

\twolineshloka
{मृत्युः सर्वहरश्चाहमुद्भवश्च भविष्यताम्}
{कीर्तिः श्रीर्वाक्च नारीणां स्मृतिर्मेधा धृतिः क्षमा}%.. 10-34

\twolineshloka
{बृहत्साम तथा साम्नां गायत्री छन्दसामहम्}
{मासानां मार्गशीर्षोऽहमृतूनां कुसुमाकरः}% .. 10-35

\twolineshloka
{द्यूतं छलयतामस्मि तेजस्तेजस्विनामहम्}
{जयोऽस्मि व्यवसायोऽस्मि सत्त्वं सत्त्ववतामहम्}% .. 10-36

\twolineshloka
{वृष्णीनां वासुदेवोऽस्मि पाण्डवानां धनञ्जयः}
{मुनीनामप्यहं व्यासः कवीनामुशना कविः}% .. 10-37

\twolineshloka
{दण्डो दमयतामस्मि नीतिरस्मि जिगीषताम्}
{मौनं चैवास्मि गुह्यानां ज्ञानं ज्ञानवतामहम्}% .. 10-38

\twolineshloka
{यच्चापि सर्वभूतानां बीजं तदहमर्जुन}
{न तदस्ति विना यत्स्यान्मया भूतं चराचरम्}% .. 10-39

\twolineshloka
{नान्तोऽस्ति मम दिव्यानां विभूतीनां परन्तप}
{एष तूद्देशतः प्रोक्तो विभूतेर्विस्तरो मया}% .. 10-40

\twolineshloka
{यद्यद्विभूतिमत्सत्त्वं श्रीमदूर्जितमेव वा}
{तत्तदेवावगच्छ त्वं मम तेजोंशसम्भवम्}% .. 10-41

\twolineshloka
{अथवा बहुनैतेन किं ज्ञातेन तवार्जुन}
{विष्टभ्याहमिदं कृत्स्नमेकांशेन स्थितो जगत्}% .. 10-42
{॥ॐ तत्सदिति श्रीमद्भगवद्गीतासूपनिषत्सु ब्रह्मविद्यायां योगशास्त्रे श्रीकृष्णार्जुनसंवादे विभूतियोगो नाम दशमोऽध्यायः॥}